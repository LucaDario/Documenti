\section*{A}
\addcontentsline{toc}{section}{A}
\begin{itemize}
	\item
	\textbf{Accoppiamento Afferente}: Numero di classi esterne ad un \termine{package} che dipendono da una classe interna ad esso.
	\item
	\textbf{Accoppiamento Efferente}: Numero di classi interne ad un \termine{package} che dipendono da una classe esterna ad esso.
	\item
	\textbf{Ambienti Di Sviluppo}: Insieme di software atti a suportare il programmatore durante la scrittura del codice sorgente di un software.
	\item
	\textbf{Analisi}: Metodo di indagine basato sulla scomposizione di ciò che si presenta unitario nei suoi elementi costitutivi.
	\item
	\textbf{Analisi Statica}: Verifica della correttezza del software, eseguita non a run-time.
	\item
	\textbf{Api}: Insieme di procedure disponibili al programmatore, di solito raggruppate a formare un set di strumenti specifici per l'espletamento di un determinato compito all'interno di un certo programma.
	\item
	\textbf{Applicazione}: .
	\item
	\textbf{Architettura Di Dettaglio}: Risultato della Progettazione in Dettaglio. Essa rappresenta l'architettura del sistema in tutte le sue funzionalità.
	\item
	\textbf{Atmosphere}: Sito web da cui è possibile cercare librerie compatibili con il \termine{framework} \termine{Meteor.js}.
\end{itemize}
\newpage