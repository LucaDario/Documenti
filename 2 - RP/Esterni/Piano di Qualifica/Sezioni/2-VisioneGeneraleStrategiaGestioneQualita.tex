\section{Visione Generale della Strategia di Gestione della Qualità}
\subsection{Obiettivi di Qualità}
\subsubsection{Modello per la Qualità di Processo}
La qualità del prodotto è conseguenza anche della qualità dei processi che lo definiscono. Il \termine{gruppo} però, dopo una attenta valutazione, ha constatato che, essendo una qualità verificabile soltanto a lungo termine, non vi è il tempo materiale per assicurare una qualità di processo definita dall'\termine{ISO}. Il \termine{gruppo}, consapevole della scelta, cercherà comunque di prendere spunto, il più possibile, dallo standard ISO/IEC 15504.

\subsubsection{Standard per la Qualità di Prodotto}
\gruppo\ si impegna a seguire lo standard ISO/IEC 9126 redatto con lo scopo di descrivere obiettivi qualitativi e delineare delle metriche capaci di misurare il raggiungimento di tali obiettivi.

\subsubsection{Soluzioni attuate per il controllo della Qualità di Processo}

L'attuazione del metodo di gestione \termine{PDCA} aiuterà un maggior avvicinamento allo standard ISO/IEC 15504 ed assicurerà, quindi, una maggior qualità di processo. Con il ciclo \termine{PDCA} è possibile infatti garantire un miglioramento continuo dei processi, inclusa la verifica, ed un utilizzo ottimale delle risorse, ottenendo di conseguenza il miglioramento dei prodotti risultanti.
Per avere controllo sulla qualità è necessario che: 
\begin{itemize}
\item
I processi siano pianificati nel dettaglio;
\item
Nella pianificazione siano ripartite in modo chiaro le risorse;
\item
I processi vengano costantemente monitorati.
\end{itemize}

L'attuazione di tali punti è descritta dettagliatamente nel \PdP.
La qualità dei processi viene inoltre monitorata mediante l'analisi costante della qualità del prodotto e quantificata utilizzando le varie metriche che saranno descritte in seguito all'interno di questo documento.

\subsubsection{Soluzioni attuate per il controllo della Qualità di Prodotto}

Al fine di garantire un controllo sistematico della qualità del prodotto, il \termine{gruppo} seguirà le seguenti linee guida:
\begin{itemize}
\item
\textbf{\termine{Quality Assurance}}: insieme di attività realizzate per garantire il raggiungimento degli obiettivi di qualità. Prevede l'attuazione di tecniche di analisi statica e dinamica;
\item
\textbf{\termine{Verifica}}: processo che determina se il lavoro di un determinato periodo è consistente, completo e corretto. La verifica andrà eseguita costantemente durante l'intera durata del progetto;
\item
\textbf{\termine{Validazione}}: conferma in modo oggettivo che il sistema soddisfi correttamente i requisiti. \\ Anch'essa come la \termine{verifica} andrà eseguita costantemente durante l'intera durata del progetto.
\end{itemize}

\subsection{Scadenze Temporali}
Al fine di perseguire l'obbiettivo di rispettare le scadenze fissate nel Piano di Progetto è necessario che l'attività di verifica della documentazione sia sistematica e ben organizzata. Solo così, infatti, l'individuazione e la correzione di eventuali errori avverrà il prima possibile, impedendo la compromissione dell'intero progetto. \\
Ogni attività di redazione dei documenti e di codifica dovrà essere preceduta da uno studio preliminare sulla struttura e sui contenuti degli stessi, con lo scopo di ridurre la possibilità di commettere imprecisioni di natura concettuale e tecnica.

\subsubsection{Responsabilità}

Per ottenere un maggior livello di efficacia ed efficienza nell'attività di verifica verranno attribuite delle responsabilità a specifici ruoli di progetto.
La responsabilità, per l'attività di \termine{verifica} e \termine{validazione}, sarà a carico dei membri del \termine{gruppo} che al momento di eseguire tale attività saranno in carica dei ruoli di \Pm\ e dei \VerP.

\newpage