\section*{J}
\addcontentsline{toc}{section}{J}
\begin{itemize}
	\item
	\textbf{Javascript}: Linguaggio di scripting orientato agli oggetti e agli eventi, comunemente utilizzato nella programmazione Web lato client per la creazione, in siti web e applicazioni web, di effetti dinamici interattivi tramite funzioni di script invocate da eventi innescati a loro volta in vari modi dall'utente sulla pagina web in uso.
	\item
	\textbf{Jenkins}: Strumento open source di continuous integration, scritto in linguaggio Java e che fornisce dei servizi di integrazione continua per lo sviluppo del software.
	\item
	\textbf{Jolie}: Linguaggio di programmazione orientato ai micro servizi.
	\item
	\textbf{Jquery}: Libreria JavaScript per applicazioni web. Nasce con l'obiettivo di semplificare la selezione, la manipolazione, la gestione degli eventi e l'animazione di elementi DOM in pagine HTML, nonché per implementare funzionalità AJAX.
	\item
	\textbf{Jshint}: Strumento di analisi statica del codice volto al linguaggio \termine{JavaScript}.
\end{itemize}
\newpage