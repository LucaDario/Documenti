\section*{M}
\addcontentsline{toc}{section}{M}
\begin{itemize}
	\item
	\textbf{Manuale Utente}: Documento destinato all'utilizzatore finale del prodotto, contenente la documentazione del prodotto.
	\item
	\textbf{Markdown}: linguaggio di markup con una sintassi del testo semplice progettata in modo che possa essere convertita in HTML e in molti altri formati usando uno strumento omonimo.
	\item
	\textbf{Markup}: Sequenza di caratteri con cui si marcano gli elementi di un file di testo per assegnare loro determinate caratteristiche o funzioni.
	\item
	\textbf{Meteor.Js}: \termine{Framework} per creare web app usando il linguaggio di programmazione \termine{javascript}.
	\item
	\textbf{Microservizio}: Parte di un sistema che abbraccia l'approccio di implementazione a micro-servizi, ha la caratteristica di essere indipendente dagli altri moduli in modo tale da riuscire a sviluppare in modo agile sistemi distribuiti.
	\item
	\textbf{Milestone}: \textit{eng. Pietra miliare}. Indica importanti traguardi intermedi nello svolgimento del progetto.
	\item
	\textbf{Mocha}: \termine{Framework} per eseguire test automatici su codice \termine{Javascript}.
	\item
	\textbf{Mock}: Serve per indicare che una o più componenti software sono state simulate durante l'esecuzione di test automatici.
	\item
	\textbf{Model-View-Controller}: Pattern architetturale che gestisce direttamente i dati, la logica e le regole dell'applicazione.
	\item
	\textbf{Model-View-Presenter}: Pattern architetturale che permette la separazione concettuale tra parte logica e grafica di una applicazione, mediante l'utilizzo di una classe intermedia tra le due.
	\item
	\textbf{Modello Incrementale}: Per modello incrementale o modello iterativo si intende, nell'ambito dell'ingegneria informatica, un modello di sviluppo di un progetto software basato sulla successione dei seguenti passi principali: pianificazione, analisi dei requisiti, progetto, implementazione, prove, valutazione.
	\item
	\textbf{Mongodb}: \termine{DBMS} non relazionale orientato ai documenti. Classificato come un database di tipo NoSQL, \termine{MongoDB} si allontana dalla struttura tradizionale basata su tabelle dei database relazionali in favore di documenti in stile JSON con schema dinamico (BSON), rendendo l'integrazione di dati di alcuni tipi di applicazioni più facile e veloce.
	\item
	\textbf{Monolith}: \termine{SDK} che permette agli sviluppatori che operano su piattaforma \termine{Rocket.chat} di creare varie tipologie di bolle differenti oppure di utilizzare quelle preesistenti.
\end{itemize}
\newpage