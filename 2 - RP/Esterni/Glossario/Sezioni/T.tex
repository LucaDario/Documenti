\section*{T}
\addcontentsline{toc}{section}{T}
\begin{itemize}
	\item
	\textbf{Tab}: \textit{abbr. Tabulazione}.
	\item
	\textbf{Tailoring}: Significa adattare i requisiti e le specifiche di un progetto alle attuali esigenze operative di un'organizzazione attraverso la revisione, la modifica e l'integrazione di nuovi dati al progetto.
	\item
	\textbf{Task}: \textit{eng. Compito}.
	\item
	\textbf{Team}: \textit{eng. \termine{Gruppo}}.
	\item
	\textbf{Telegram}: Servizio di messaggistica istantanea basato su cloud ed erogato senza fini di lucro dalla società \textit{Telegram LLC}.
	\item
	\textbf{Template}: Documento o programma nel quale, come in un foglio semicompilato cartaceo, su una struttura generica o standard esistono spazi temporaneamente "bianchi" da riempire successivamente.
	\item
	\textbf{Tempo Di Slack}: Periodo di tempo durante il quale un'attività può essere ritardata senza ritardare l'intero progetto di cui fa parte.
	\item
	\textbf{Testing}: Fase del ciclo di vita di un software all'interno della quale il codice prodotto dalla fase di programmazione viene sottoposto a dei controllo che ne assicurano l'\termine{efficacia} e l'\termine{efficienza}.
	\item
	\textbf{Texmaker}: Editor gratuito e cross-platform per LaTeX.
	\item
	\textbf{Twitter Bootstrap}: \termine{Framework} per sviluppare interfacce web compatibili con i dispositivi mobili.
\end{itemize}
\newpage