\subsubsection{Test di Sistema}
Con questa tipologia di test si vuole verificare il corretto funzionamento delle componenti architetturali.\\
I test di sistema saranno descritti nel modo seguente:
\begin{center}
\textbf{TS}[\textit{TipoRequisito}][\textit{ImportanzaRequisito}][\textit{IdRequisito}]
\end{center}
dove:
\begin{itemize}
\item \textbf{TipoRequisito} può assumere valori tra:
\begin{itemize}
\item \textit{F} per i requisiti funzionali;
\item \textit{Q} per i requisiti di qualità;
\item \textit{V} per i requisiti di vincolo;
\item \textit{P} per i requisiti prestazionali.
\end{itemize}
\item \textbf{ImportanzaRequisito} può assumere valori tra:
\begin{itemize}
\item \textit{D} per i requisiti desiderabili;
\item \textit{O} per i requisiti di obbligatori;
\item \textit{F} per i requisiti di facoltativi.
\end{itemize}
\item \textbf{IdRequisito} può assumere un valore gerarchico che identifica il singolo requisito.
\end{itemize}
