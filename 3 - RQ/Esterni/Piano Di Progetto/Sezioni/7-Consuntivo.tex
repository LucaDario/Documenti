\newpage
\section{Consuntivi di periodo}

In base ai consuntivi di periodo dei periodi completati sono state effettuate le diverse modifiche alla pianificazione per rientrare nel budget preventivato al momento della consegna per aggiudicarci il capitolato.

\subsection{\PD\ e \COD\ da \RP\ fino alla \RQ}
La differenza di costi tra quelli preventivati nella sezione 5 di questo documento e quello effettivamente speso è la seguente:

\begin{table}[h]
	\begin{center}
		\begin{tabular}{|c|c|c|c|c|c|}
			\hline
			\textbf{Ruolo}	& \textbf{Ore Preventivate} & \textbf{Costo preventivato} &  \textbf{Differenza ore} & \textbf{Differenza costo} \\
			\hline
			\Pm &	11 & 330 & +2 & +60\\
			\hline
			\Am	&	6 & 120 & -1 & -20\\
			\hline
			\Prog	&	21 & 462 & +21 & +462\\
			\hline
			\Progr	&	193 & 2055 & +7 & +105\\
			\hline
			\Ver	&	91 & 1155 & +6 & +90\\
			\hline
			\textbf{totale}	&	\textbf{322} & \textbf{4122} & \textbf{+35} & \textbf{+697}\\
			\hline
		\end{tabular}
	\end{center}
	\caption{Consuntivo di periodo \PD\ e \COD\ da \RP\ fino alla \RQ}
\end{table}

Dal consuntivo di periodo emerge un aumento di \textbf{697€} nel preventivo a finire (che ora ammonta a \textbf{14318€}), rispetto ai \textbf{13621€} preventivati nella pianificazione ed al momento dell'aggiudicazione del capitolato.\\

Nonostante questo discostamento di 697€ risultante dal consuntivo di periodo, il gruppo si impegna a rispettare l'impegno economico preso al momento dell'aggiudicazione del capitolato.\\
Infatti il gruppo è anche consapevole di aver superato le difficoltà di maggior peso e di aver già compiuto la maggior parte del lavoro richiesto per lo svolgimento del capitolato. Motivo per cui nell'ultimo periodo si prevede di rientrare nei costi preventivati di \textbf{13621€}. \\

\subsubsection{Preventivo a finire a seguito del consuntivo di periodo}
A seguito di una attenta analisi del consuntivo di periodo descritto appena sopra, la pianificazione dell'ultimo periodo rimanente (\VV) sarà modificata come segue:

\paragraph{Prospetto economico a seguito del consuntivo di periodo, \VV}
Nel periodo riguardante la \VV\ le ore tra i ruoli saranno divise nel seguente modo:
\begin{table}[h]
	\begin{center}
		\begin{tabular}{|l|c|c|}
			\hline
			\textbf{Ruolo}	& \textbf{Ore} &	\textbf{Costo}	 \\
			\hline
			\textit{\Pm}	&	8	&	240		\\
			\hline
			\textit{\Am}	&	3	&	60		\\ 
			\hline
			\textit{\Prog}	&	6	&	132	\\
			\hline
			\textit{\Progr}	&	8	&	120	\\ 
			\hline
			\textit{\Ver}	&	53	&	795	\\
			\hline
			\textbf{Totale}	&	\textbf{78}	&	\textbf{1347}	\\
			\hline
						
		\end{tabular}
	\end{center}
	\caption{Incidenza ore su costo per ruolo a seguito del consutivo di periodo, \VV}
\end{table}

\paragraph{Prospetto economico a seguito del consuntivo di periodo, Totale}
A seguito di questa ri-pianificazione, le ore totali, previste per la realizzazione dell'intero progetto, comprese le ore di investimento e autoapprendimento, sono riportate nella tabella seguente:

\begin{table}[h]
	\begin{center}
		\begin{tabular}{|l|c|c|c|}
			\hline
			\textbf{Ruolo}	& \textbf{Ore} &	\textbf{Ore remunerabili}	 &\textbf{Costo} \\
			\hline
			\textit{\Pm}	&	57	&	37	&	1110	\\
			\hline
			\textit{\Am}	&	44	&	25	&	500	\\
			\hline
			\textit{\An}	&	105	&	30	&	750	\\
			\hline
			\textit{\Prog}	&	233	&	233	&	5126	\\
			\hline
			\textit{\Progr}	&	208	&	152	&	2280	\\
			\hline
			\textit{\Ver}	&	306	&	257	&	3855	\\
			\hline
			\textbf{Totale}	&	\textbf{953} & \textbf{734} & \textbf{13621}	\\
			\hline
		\end{tabular}
	\end{center}
	\caption{Costo totale per ruolo in seguito a ripianificazione}
\end{table}

Il gruppo quindi, seguendo questa ri-pianificazione per l'ultimo periodo rimanente, riesce a rientrare e a rispettare il preventivo di \textbf{13621€} fissato al momento dell'aggiudicazione del capitolato.