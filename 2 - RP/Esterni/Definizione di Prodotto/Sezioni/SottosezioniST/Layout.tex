\subsubsection{BaseLayout}
\begin{itemize}
\item \textbf{Descrizione:} Classe astratta che implementa l'interfaccia BaseComponent che rappresenta un oggetto layout per la disposizione di BaseComponent nelle bolle di Monolith.
\item \textbf{Utilizzo:} Classe utilizzata ed estesa ogni qualvolta uno sviluppatore intende creare un nuovo layout da inserire in una bolla.
\item \textbf{Attributi:}
\begin{itemize}
\item \textit{private items:List<BaseComponent>}\\
Rappresenta la lista di oggetti BaseComponent contenuti nel layout.
\end{itemize}
\item \textbf{Metodi:}
\begin{itemize}
\item \textit{public addItem(component:BaseComponent):void}\\
Aggiunge un oggetto BaseComponent al layout.
\item{\textbf{Parametri}: \begin{itemize}
\item \textit{component:BaseComponent}\\
Oggetto che rappresenta il BaseComponent da aggiungere al layout.
\end{itemize}}
\end{itemize}
\end{itemize}

\subsubsection{VerticalLayoutView}
\begin{itemize}
\item \textbf{Descrizione:} Classe concreta che estende BaseLayout, destinata alla creazione di layout per la disposizione verticale di BaseComponent.
\item \textbf{Utilizzo:} Classe utilizzata ogni qualvolta uno sviluppatore intende creare un nuovo layout verticale da inserire in una bolla.
\item \textbf{Attributi:}
\item \textbf{Metodi:}
\begin{itemize}
\item \textit{public addItem(component:BaseComponent):void}\\
Aggiunge un oggetto BaseComponent al layout verticale.
\item{\textbf{Parametri}: \begin{itemize}
\item \textit{component:BaseComponent}\\
Oggetto che rappresenta il BaseComponent da aggiungere al layout verticale.
\end{itemize}}
\item \textit{public renderView():string}\\
Genera il codice HTML, CSS e JavaScript necessario per visualizzare BaseComponent disposti nel layout verticale.
\end{itemize}
\end{itemize}

\subsubsection{HorizontalLayoutView}
\begin{itemize}
\item \textbf{Descrizione:} Classe concreta che estende BaseLayout, destinata alla creazione di layout  per la disposizione orizzontale di BaseComponent.
\item \textbf{Utilizzo:} Classe utilizzata ogni qualvolta uno sviluppatore intende creare un nuovo layout verticale da inserire in una bolla.
\item \textbf{Attributi:}
\item \textbf{Metodi:}
\begin{itemize}
\item \textit{public addItem(component:BaseComponent):void}\\
Aggiunge un oggetto BaseComponent al layout orizzontale.
\item{\textbf{Parametri}: \begin{itemize}
\item \textit{component:BaseComponent}\\
Oggetto che rappresenta il BaseComponent da aggiungere al layout orizzontale.
\end{itemize}}
\item \textit{public renderView():string}\\
Genera il codice HTML, CSS e JavaScript necessario per visualizzare BaseComponent disposti nel layout orizzontale.
\end{itemize}
\end{itemize}