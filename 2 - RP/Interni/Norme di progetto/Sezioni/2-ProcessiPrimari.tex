\section{Processi primari}
\subsection{Studio di Fattibilità}
Alla pubblicazione dei capitolati d'appalto il \textit{\Pm} avrà il compito fissare un
numero sufficiente di riunioni volte alla discussione e al confronto tra i membri del \termine{team}.
In seguito, gli \textit{\AnP} dovranno stilare lo \textit{\SdF} in base a quanto
emerso nelle riunioni. Tale documento sarà articolato nei seguenti punti:
\begin{itemize}
	\item \textbf{Descrizione:} descrizione generale di ciò che viene richiesto
	dal capitolato.
	\item \textbf{Dominio Applicativo:} descrizione dell'ambito di utilizzo del
	prodotto richiesto e delle caratteristiche che deve avere il prodotto finale.
	\item \textbf{Dominio Tecnologico:} descrizione delle tecnologie impiegate
	nello sviluppo del progetto richiesto.
	\item \textbf{Potenziali Criticità:} elenco delle possibili problematiche che potrebbero
	sorgere durante lo sviluppo del prodotto richiesto, individuando quindi punti
	critici ed eventuali rischi.
	\item \textbf{Analisi di mercato} descrizione delle potenzialità del dominio applicativo.
	\item \textbf{Valutazione Finale:} piccolo elenco dei lati positivi e negativi del capitolato scelto.
\end{itemize}

\subsection{Processo di sviluppo}
\subsubsection{Analisi dei Requisiti}
Ultimato lo \textit{\SdF} gli \textit{\AnP} dovranno stilare l'\textit{\AdR} che dovrà 	essere strutturata nel seguente modo.
\paragraph{Classificazione dei requisiti}
Dovrà essere stilato un elenco di requisiti, emersi durante le riunioni interne
e/o esterne. Questo compito spetta agli \textit{\AnP}. I requisiti dovranno
essere classificati secondo la seguente codifica:

\begin{center}
R-[Importanza][Tipo][Identificativo]
\end{center}
\begin{itemize}
	\item \textbf{Importanza:} può assumere questi valori:
  		\begin{itemize}
    		\item \textbf{1:} indica un requisito obbligatorio.
    		\item \textbf{2:} indica un requisito desiderabile.
    		\item \textbf{3:} indica un requisito facoltativo.
  		\end{itemize}
  	\item \textbf{Tipo:} può assumere questi valori:
  		\begin{itemize}
   		 	\item \textbf{F:} indica un requisito funzionale.
    		\item \textbf{Q:} indica un requisito di qualità.
    		\item \textbf{P:} indica un requisito prestazionale.
    		\item \textbf{V:} indica un requisito di vincolo.
  		\end{itemize}
  	\item \textbf{Identificativo:} indica il codice identificativo del requisito, è univoco e deve essere indicato in forma gerarchica.
\end{itemize}
Per ogni requisito si dovranno inoltre indicare:
\begin{itemize}
  \item \textbf{Descrizione:} una breve descrizione del requisito, che chiarisca tutti i punti di esso senza lasciare spazio a possibili ambiguità.
  \item \textbf{Fonte:} la fonte può essere una delle seguenti:
  \begin{itemize}
    \item \textit{\termine{Capitolato}}: deriva direttamente dal testo del capitolato.
    \item \textit{Verbale}: deriva da un incontro verbalizzato, seguito dall'identificativo dell'incontro.
    \item \textit{Interno}: deriva da discussioni interne al \termine{team}.
    \item \textit{Caso d'uso}: deriva da uno o più casi d'uso, seguito dall'identificativo del caso o dei casi d'uso.
  \end{itemize}
\end{itemize}

\paragraph{Classificazione dei casi d'uso}
Ogni requisito sarà specificato da un diagramma di caso d'uso e sarà rappresentato in questo modo:
\begin{center}
  UC[Identificativo]
\end{center}
dove:
\begin{itemize}
  \item\textbf{Identificativo}: è il codice gerarchico univoco, rappresentato in numeri, per identificare ogni caso d'uso.
\end{itemize}
Inoltre per ogni caso d'uso dovranno essere indicati:
\begin{itemize}
  \item\textbf{Titolo}: indica il titolo del caso d'uso.
  \item\textbf{Descrizione}: breve descrizione del caso d'uso.
  \item\textbf{Attori}: indica gli attori, sia principali che secondari, coinvolti nel caso d'uso.
  \item\textbf{Precondizione}: indica la condizione che deve essere verificata prima
  dell'esecuzione del caso d'uso.
   \item\textbf{Precondizione}: indica la condizione che deve essere verificata dopo dell'esecuzione del caso d'uso.
  \item\textbf{Scenario principale}: descrizione composta dal flusso dei casi d'uso
  figli.
  \item\textbf{Scenari alternativi}: descrizione composta dai casi d'uso che non
  appartengono al flusso principale di esecuzione.
  \item\textbf{Estensioni}: indica quali sono tutte le estensioni, se presenti.
  \item\textbf{Inclusioni}: indica quali sono tutte le inclusioni, se presenti.
  \item\textbf{Generalizzazioni}: indica quali sono tutte le generalizzazioni,
  se presenti.
  \item\textbf{Postcondizione}: indica la condizione che deve essere verificata dopo
  l'esecuzione del caso d'uso.
\end{itemize}

\subsubsection{Progettazione}
\paragraph{Descrizione}
L'attività di progettazione definisce come deve essere realizzata la struttura dell'architettura software. I requisiti delineati all'interno del documento \textit{\AdR} devono essere utili a realizzare la documentazione specifica e a determinare le linee guida da seguire durante l'attività di codifica.
Tale attività deve essere svolta in maniera ottimale e precisa dai \textit{progettisti}.

\paragraph{Diagrammi}
La progettazione deve utilizzare le seguenti tipologie di diagrammi \termine{UML}:
\begin{itemize}
\item
\textbf{Diagrammi di classe}: illustrano una collezione di elementi dichiarativi di un modello come classi e tipi, assieme ai loro contenuti e alle loro relazioni.
\item
\textbf{Diagrammi dei \textit{\termine{package}}}: illustrano i vari raggruppamenti di classi in una unità di livello più alto.
\item
\textbf{Diagrammi di attività}: illustrano il flusso di operazioni relativo ad un'attività e utilizzati soprattutto per descrivere la logica di un algoritmo.
\item
\textbf{Diagrammi di sequenza}: descrivono una determinata sequenza di azioni dove tutte le scelte sono già state effettuate. In tali diagrammi non compaiono pertanto scelte né flussi alternativi.
\end{itemize}

\paragraph{Requisiti per i progettisti}
I \textit{\ProgP} sono responsabili delle attività di progettazione e sono tenuti ad avere:
\begin{itemize}
\item
Profonda conoscenza di tutto ciò che riguarda il processo di sviluppo del software.
\item
Capacità di saper anticipare i cambiamenti.
\item
Notevole inventiva per riuscire a trovare una soluzione progettuale accettabile anche in mancanza di una metodologia che sia sufficientemente espressiva.
\item
Capacità di individuare con rapidità e sicurezza le soluzioni più opportune.
\end{itemize}

\paragraph{Obiettivi della progettazione}
La fase di progettazione si pone i seguenti obiettivi:
\begin{itemize}
\item Soddisfare i requisiti obbligatori fissati dal committente.
\item Realizzare un applicativo  resistente alle modifiche, senza che l'intera struttura venga compromessa e/o nuovamente messa in discussione.
\item Progettare un software con le caratteristiche che sono state elencate e descritte nella fase di analisi dei requisiti.
\end{itemize}

\subsubsection{Codifica}
\paragraph{Descrizione}
La fase di codifica ha come obiettivo quello di passare, nel miglior modo possibile, dalla soluzione architetturale realizzata dai \textit{\ProgP} a quella finale eseguibile da un calcolatore.
In questa fase i responsabili della codifica sono i \textit{\ProgrP} e questi sono tenuti a seguire le seguenti \termine{best practice}, con lo scopo di produrre il prodotto designato nella fase di progettazione.

\paragraph{Formattazione}
Per semplificare il compito del \textit{\Prog} è fondamentale delineare uno standard di formattazione.
Esso coincide nei seguenti punti:

\begin{itemize}
\item Mantenere l'indentazione standard di \termine{IntelliJ IDEA} composta da quattro(4) spazi.

\item Inserire le parentesi di apertura in modo \termine{inline}.

\item Utilizzare spazi vuoti nel codice sorgente in modo da suddividere il file in paragrafi migliorandone lettura e comprensione.

\item Inserire sempre le parentesi di apertura e di chiusura di una istruzione condizionale anche quando è presente solo uno \termine{statement} all'interno del suo corpo. In tal modo verrà facilitata la comprensione e manutenzione del codice.

\item Suddividere logicamente il codice in diversi file.

\item Cercare di incapsulare più possibile il codice e non di creare funzioni e/o classi con una eccessiva quantità di codice.
\end{itemize}

\paragraph{Commenti}
All'interno del codice è fondamentale inserire una buona quantità di commenti al fine di facilitare la comprensione, sia ad un lettore esterno, sia allo stesso programmatore a distanza di tempo.
Di seguito sono suggeriti alcuni metodi di inserimento di commenti che saranno utilizzati dal gruppo:

\begin{itemize}
\item Cercare di inserire commenti di facile comprensione a tutti.

\item Commentare tutto ciò che non è immediato nel codice.

\item Evitare l'inserimento di commenti alla fine di una riga di codice ma preferire tale inserimento nella riga sovrastante ad essa.

\item Commentare sempre tutte le istruzioni condizionali se considerate non di facile comprensione.

\item Tenere sempre aggiornati i commenti relativi all'eventuale codice che si va a modificare, al fine di evitare problemi di inconsistenza con esso.

\item Evitare commenti superflui o umoristici in quanto in caso di \termine{commit} resteranno nella storia dello sviluppo del software.
\end{itemize}

\paragraph{Documentazione}
All'interno di un progetto \termine{open source} un'altra caratteristica di fondamentale importanza risulta essere la documentazione. \\
Di seguito vengono pertanto riportate alcune \termine{best practice} per stilare una documentazione di facile comprensione.

\begin{itemize}
\item Scrivere la documentazione in lingua inglese, in quanto il software dovrà poter essere letto anche da sviluppatori internazionali.
\item Elencare in modo preciso tutte le variabili(anche quelle opzionali) ed i loro significato nei rispettivi metodi.
\item Segnalare il tipo ed il significato del dato ritornato dai rispettivi metodi.
\item Documentare in modo preciso il contratto di tutti i metodi pubblici e protetti delle classi.
\end{itemize}

\paragraph{Nomi}
Lo schema di denominazione è uno dei supporti più determinanti per la comprensione del codice.
Di seguito sono elencate le tecniche per raggiungere un buon utilizzo dei nomi.

\begin{itemize}
\item Seguire la pratica \termine{camel case} per la denominazione delle variabili locali.
\item Seguire la pratica \termine{PascalCase} per la denominazione delle classi.
\item Seguire la pratica \termine{UPPER\textunderscore CASE} per la denominazioni delle variabili statiche costanti.
\item Utilizzare uno nome significativo per le variabili, le classi ed i metodi scritti, al fine di rendere superflua qualunque documentazione.
\item Evitare di utilizzare gli stessi nomi per elementi diversi.
\item Utilizzare nome composti da una sola lettera solamente per le variabili che identificano un indice all'interno di una istruzione ciclica.
\item Cercare di utilizzare nomi diverse per tutte le classi, anche se facenti parte di packages diversi.
\end{itemize}
\subsubsection{Documentazione dei file}
I file contenenti codice dovranno essere avere un’intestazione contenente:
\begin{itemize}
\color{ForestGreen}
\item /*! 
\item *   \textbackslash file Nome del file
\item *   \textbackslash author Autore
\item *   \textbackslash date Data di creazione
\item *   \textbackslash brief Breve descrizione del file
\item */
\end{itemize}

All'interno del file le classi dovranno contenere la seguente intestazione:
\begin{itemize}
\color{ForestGreen}
\item /*!
\item * \textbackslash class Nome della classe
\item * \textbackslash brief Breve descrizione della classe
\item */
\end{itemize}

All'interno del file i metodi dovranno contenere la seguente intestazione:
\begin{itemize}
\color{ForestGreen}
\item /*!
\item * \textbackslash brief Breve descrizione della funzione
\item * \textbackslash param Nome del primo parametro
\item *  eventuali successivi parametri
\item * \textbackslash return Valore ritornato dalla funzione
\item */
\end{itemize}
  
\subsubsection{Ricorsione}
La ricorsione va evitata il più possibile. Per ogni funzione ricorsiva sarà necessario valutare il costo in termini di occupazione della memoria e bisognerà associare ad esso una prova di correttezza.
Valutare la memoria occupata dalla funzione sarà obbligatorio in quanto le funzioni ricorsive possono facilmente non terminare occupando tutto lo stack della memoria andando quindi in \termine{stack overflow}.

\subsubsection{Sistemi Operativi}
Ogni componente del team ha scelto di usare il proprio sistema operativo preferito. Sono, dunque, utilizzate attualmente tre diversi sistemi: MacOs,  ArchLinux e Windows 10. L'uso di sistemi diversi, però, non risulta essere un problema poiché la programmazione avviene principalmente lato server rendendo cosi ininfluente la scelta del sistema operativo del client.

\subsubsection{Strumenti}
\paragraph{LaTeX}
Lo strumento che il \termine{team} ha scelto per la stesura e la manutenzione di tutti i documenti è \termine{LaTeX}. Le ragioni che hanno portato a questa scelta sono:
\begin{itemize}
\item La disponibilità di mezzi per l'automazione della maggior parte della composizione tipografica.
\item La possibilità di definire \termine{template} dei documenti da stilare.
\item La possibilità di utilizzarlo gratuitamente e ovunque grazie alla sua licenza \termine{open source}.
\item La possibilità di suddivisione del documento in file separati al fine di facilitare il lavoro in gruppo.
\item La disponibilità di strumenti di controllo ortografico per minimizzare i possibili errori tipografici.
\item La facilità d'uso e di configurazione.
\item La possibilità di conversione del testo scritto in file \termine{PDF}.
\item Gli editor usati usati per \termine{LaTeX} son \termine{Texmaker} su MacOS e Windows 10 e \termine{Gummi}.
\item Per creare il glossario è stato implementato un script automatico in \termine{python} che attraverso un file json contenente i vocaboli e le loro definizioni genera il glossario. Per far funzionare correttamente lo script è necessario che i redattori del documento marchino con il rispettivo comando "$\backslash$termine" gli elementi da inserire nel glossario.
\end{itemize}

\paragraph{IntelliJ IDEA}
L'ambiente di sviluppo integrato (\termine{IDE}) scelto per lo sviluppo e la manutenzione del progetto è \termine{IntelliJ IDEA}. Le ragioni che hanno portato a questa scelta sono le seguenti:
\begin{itemize}
\item Disponibilità di svariati \termine{plugin} per lo sviluppo, in particolare per \termine{Meteor.js}.
\item Buona conoscenza di questo strumento d parte di tutti i componenti del \termine{team} maturata durante il corso di laurea.
\item Possesso di una completa integrazione con i metodi di versionamento \termine{Git}.
\item Abilità di autocompletamento del codice \termine{JavaScript} scritto.
\item Possesso di un \termine{debugger} \termine{JavaScript}.
\end{itemize}

\paragraph{Slack}
Lo strumento che scelto per la comunicazione interna formale è \termine{Slack}.
Le ragioni che hanno portato a questa scelta sono:
\begin{itemize}
\item La possibilità di organizzare le discussioni in canali, rendendole quindi molto più organizzate ed accessibili ai membri del \termine{team}.
\item Disponibilità su qualsiasi piattaforma in modo gratuito, e pertanto usufruibile in ogni momento.
\end{itemize}

\paragraph{Telegram}
Lo strumento abbiamo scelto per la comunicazione interna informale è \termine{Telegram}.
Le ragioni che hanno portato a questa scelta sono:
\begin{itemize}
\item Disponibilità su qualsiasi sistema operativo mobile e desktop, attraverso le applicazioni ufficiali sviluppate e/o il client web.
\item Possibilità di aggiungere molteplici \termine{bot} utili in varie funzioni, come ad esempio la possibilità di creare sondaggi.
\item Immediatezza per la formulazione e la risposta a domande veloci e informali, evitando così di compromettere lo storico delle conversazioni su \termine{Slack}.
\end{itemize}

\paragraph{GitHub}
Il servizio di hosting che scelto per ospitare la \termine{repository} che andrà a contenere i documenti ed il codice del progetto è \termine{GitHub}.
Le ragioni che hanno portato a questa scelta sono:
\begin{itemize} 
\item La possibilità di creare \termine{issues} e indirizzarle a determinate persone definendo la rispettiva data di scadenza di risoluzione.
\item Richiesta specifica eseguita da parte del \termine{proponente} del progetto.
\end{itemize}

\paragraph{Git}
Come sistema di controllo di versionamento la scelta è ricaduta obbligatoriamente su \termine{Git} avendo scelto GitHub come host.

\paragraph{Wrike}
Lo strumento di pianificazione che abbiamo scelto è \termine{Wrike}.
Le ragioni che ci hanno portato a questa scelta sono:

\begin{itemize}
\item Disponibilità gratuita mediante la richiesta di una licenza per studenti.
\item Possibilità di assegnare un \termine{task} a più di una persona.
\item Possesso di una interfaccia semplice ed intuitiva.
\item Possibilità di utilizzo anche su sistemi operativi mobile quali Android e iOS mediante l'applicazione gratuita disponibile.
\item Possibilità, previa sincronizzazione, di segnalare, tra \termine{Wrike} e la \termine{repository} \termine{GitHub}, quando un determinato \termine{task} è completato attraverso un semplice \termine{commit}.
\item Possibilità di commentare e rispondere all'interno della pagina di un rispettivo \termine{task}.

\end{itemize}

\paragraph{GanttChart}
Lo strumento per la costruzione dei \termine{diagrammi di Gantt} che abbiamo scelto è \termine{GanttChart}.
Le ragioni che ci hanno portato a questa scelta sono:

\begin{itemize}
\item Facilità di utilizzo.
\item Problematiche legate all'estrapolazione del \termine{diagramma di Gantt} da \termine{Wrike}.
\end{itemize}

\paragraph{Visual Paradigm}
Lo strumento per la costruzione degli \termine{use cases} che abbiamo scelto è \termine{Visual Paradigm}.
Le ragioni che ci hanno portato a questa scelta sono:
\begin{itemize}
\item Facilità di utilizzo.
\end{itemize}
\newpage 