\begin{center}
	\begin{longtable}{|c|>{\centering}m{10cm}|c|}\hline
		Id & Descrizione & stato \\ \hline
		TSFO1 & Si verifica che sia possibile creare e aggiungere ad una bolla un widget per tipo, tra quelli presenti nell'SDK & Implementato \\ \hline
		TSFO2 & Si verifica che sia possibile creare e aggiungere ad una bolla un layout per tipo, tra quelli presenti nell'SDK & Implementato \\ \hline
		TSFO3 & Si verifica che sia possibile impostare le variabili di un widget testo formattato & Implementato \\ \hline
		TSFO4 & Si verifica che un widget testo formattato faccia visualizzare un messaggio d'errore in caso venga impostata una sua variabile in maniera errata & Implementato \\ \hline
		TSFO5 & Si verifica che sia possibile impostare le variabili di un widget immagine & Implementato \\ \hline
		TSFO6 & Si verifica che un widget immagine faccia visualizzare un messaggio d'errore in caso venga impostata una sua variabile in maniera errata & Implementato \\ \hline
		TSFO7 & Si verifica che sia possibile impostare le variabili di un widget bottone & Implementato \\ \hline
		TSFO8 & Si verifica che un widget bottone faccia visualizzare un messaggio d'errore in caso venga impostata una sua variabile in maniera errata & Implementato \\ \hline
		TSFO9 & Si verifica che sia possibile impostare le variabili di un widget checklistitem & Implementato \\ \hline
		TSFO10 & Si verifica che un widget checklistitem faccia visualizzare un messaggio d'errore in caso venga impostata una sua variabile in maniera errata & Implementato \\ \hline
		TSFO11 & Si verifica che sia possibile impostare le variabili di un widget lista & Implementato \\ \hline
		TSFO12 & Si verifica che un widget lista faccia visualizzare un messaggio d'errore in caso venga impostata una sua variabile in maniera errata & Implementato \\ \hline
		TSFO13 & Si verifica che sia possibile aggiungere un widget all'interno di un VerticalLayoutView & Implementato \\ \hline
		TSFO14 & Si verifica che sia possibile aggiungere un widget all'interno di un HorizzontalLayoutView & Implementato \\ \hline
		TSFO15 & Si verifica che sia possibile istanziare una bolla avviso e sia possibile utilizzare tutti i suoi metodi & Implementato \\ \hline
		TSFO16 & Si verifica che sia possibile istanziare una bolla markdown e sia possibile utilizzare tutti i suoi metodi & Implementato \\ \hline
		TSFO17 & Si verifica che sia possibile istanziare una bolla lista e sia possibile utilizzare tutti i suoi metodi & Implementato \\ \hline
		TSFO18 & Si verifica che sia possibile creare una bolla "lista della spesa" & Implementato \\ \hline
		TSFO19 & Si verifica che l'aggiunta di un nuovo prodotto alla lista della spesa venga eseguita correttamente & Implementato \\ \hline
		TSFF20 & Si verifica che sia possibile rimuovere, per un utente con i permessi ed al creatore, un elemento ad una bolla "lista della spesa" & Non implementato \\ \hline
		TSFO21 & Si verifica che sia possibile spuntare un elemento di una bolla "lista della spesa" & Implementato \\ \hline
		TSFO22 & Si verifica che l'inoltro a un utente di una bolla "lista della spesa" avvenga in modo corretto & Non implementato \\ \hline
		TSFF23 & Si verifica che l'inoltro a un gruppo, da perte di tutti gli utenti, di una bolla "lista della spesa" avvenga in modo corretto & Non implementato \\ \hline
		TSFO24 & Si verifica che la pubblicazione a un utente di una bolla "lista della spesa" avvenga in modo corretto & Implementato \\ \hline
		TSFO25 & Si verifica che la pubblicazione a un gruppo di una bolla "lista della spesa" avvenga in modo corretto & Implementato \\ \hline
		TSFO26 & Si verifica che l'utente con i permessi possa aggiungere elementi alla lista & Implementato \\ \hline
		TSFO27 & Si verifica che l'utente senza i permessi non possa aggiungere elementi alla lista & Implementato \\ \hline
		TSFO28 & Si verifica che il creatore della lista possa aggiungere elementi alla lista & Implementato \\ \hline
		TSFD29 & Si verifica che il creatore della lista e gli utenti con permessi possano modificare i prodotti della lista & Non implementato \\ \hline
	\end{longtable}
\end{center}
