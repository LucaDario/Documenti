\subsubsection{Widgets}

\paragraph{Usage of a default provided widget}
Inside \progettoShort\ there will be some default widgets already created, each one of them with a different usage. These will be:
\begin{itemize}
	\item \texttt{TextWidget}: shows a text which can be colored and/or formatted using the MarkDown notation;
	\item \texttt{ImageWidget}: displays the given image;
	\item \texttt{ChecklistWidget}: creates a checklist;
	\item \texttt{ButtonWidget}: creates a button which acts as you wish;
	\item \texttt{ListWidget}: represents an unnumbered list with customizable item indicators
\end{itemize}

To create a new instance of one of the following methods, there procedure is the following one. 
\begin{lstlisting}[language=JavaScript, frame=single]
// Needed to be able to create the new widget
import {TextWidget} from {BOH}

let widget = new TextWidget();
// Now you can work with this widget accessing its methods
// widget.setText("Hello world");
// widget.setTextColor("#123");
// ...
\end{lstlisting}


\textbf{Note}. We will not explain how to create each one of the widgets as the procedure is similar between all of them, with the exception that the only thing that changes from one to another is the class used to create the instance. We will also not give a full example of all the methods that a widget exposes as we think that our code documentation, which is provided when installing \progettoShort, can be sufficient.

\paragraph{Creation of a new type of widget}
\progettoShort\ gives also the possibility of creating a new kind of widget which will act as defined by the user. In order to do so, the only things which needs to be done is creating a new class that extends \texttt{BaseWidget} and implement its abstract methods.
 \begin{lstlisting}[language=JavaScript, frame=single]
// Needed to be able to create the new widget
import {BaseWidget} from {BOH}

class MyWidget extends BaseWidget {
    constructor(){
        // Needed in order to create the base class instance
        super();
    }

    renderView(){
        // This method needs to implement the view rendering of the widget
    }
}
\end{lstlisting}

Once it has been done, the new widget will be able to be added to layouts or bubbles.