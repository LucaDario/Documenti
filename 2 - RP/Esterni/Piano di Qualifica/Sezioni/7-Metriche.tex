\newpage

\section{Metriche}
Per raggiungere lo scopo di qualità prefissato, ma per non mancare in termini di conformità il gruppo, per ogni metrica descritta nelle \NdP\ ha scelto i seguenti range di accettazione per metrica:

Si informa che per la descrizione delle seguenti metriche si rimanda alle \NdP. \\


\subsection{Metriche per i documenti}

\subsubsection{L'indice di Gulpease}
Gli intervalli richiesti per ogni documento redatto sono i seguenti:
\begin{itemize}
\item Accettazione: [40 -- 100];
\item Ottimale: [50 -- 100].
\end{itemize}

\subsection{Metriche per il software}
Questa sezione, che verrà rivista e incrementata nelle prossime revisioni, è da intendere come una dichiarazione di propositi.
Per raggiungere gli obiettivi auspicati dallo standard ISO di riferimento (ISO/IEC 9126), ovvero funzionalità, affidabilità,
efficienza, usabilità e manutenibilità.


\subsubsection{Complessità ciclomatica}
Intervalli richiesti:
\begin{itemize}
\item
Accettazione: [1 -− 15];
\item
Ottimale: [1 -− 10].
\end{itemize}

\subsubsection{Numero di livelli di annidamento per metodo}
Intervalli richiesti:
\begin{itemize}
\item
Accettazione: [1 -- 5];
\item
Ottimale: [1 −- 3].
\end{itemize}

\subsubsection{Numero di attributi per classe}
Intervalli richiesti:
\begin{itemize}
\item
Accettazione: [0 −- 12];
\item
Ottimale: [0 −- 8].
\end{itemize}

\paragraph{Copertura dei test}
Intervalli richiesti:
\begin{itemize}
\item
Accettazione: [40\% −- 100\%];
\item
Ottimale: [65\% −- 100\%].
\end{itemize}

\paragraph{Linee di commento per linee di codice}
Intervalli richiesti:
\begin{itemize}
\item
Accettazione: [> 0.15];
\item
Ottimale: [>0.20].
\end{itemize}

\subsubsection{Numero di parametri per metodo}
Intervalli richiesti:
\begin{itemize}
\item
Accettazione: [0 −- 6];
\item
Ottimale: [0 −- 4].
\end{itemize}

\subsubsection{Livello di stabilità}

\begin{displaymath}
{\text{Accoppiamento Afferente}}\over{\text{Accoppiamento Afferente} + \text{Accoppiamento Efferente}}
\end{displaymath}

Intervalli richiesti:
\begin{itemize}
\item
Accettazione: [0.0 −- 0.8];
\item
Ottimale: [0.0 −- 0.3].
\end{itemize}

\subsection{Astrattezza}

Un buon valore per questa metrica trovato dal gruppo deve essere maggiore o uguale a:
\begin{displaymath}
	\frac{1}{7}
\end{displaymath}

\subsection{Distanza dalla sequenza principali}

\begin{displaymath}
{|\text{Astratezza} + \text{Livello di stabilità} - 1|}
\end{displaymath}

Intervalli richiesti:
\begin{itemize}
\item
Accettazione: [0.0 −- 0.3];
\item
Ottimale: [0.0 −- 0.1].
\end{itemize}


\subsection{Funzionalità}
Rappresenta la capacità del prodotto di fornire tutte le funzioni che sono state individuate attraverso l'\AdR.

\subsubsection{Obiettivi di qualità}
Il \termine{team} si impegnerà affinché:
\begin{itemize}
\item \textbf{Adeguatezza}: le funzionalità fornite siano conformi rispetto le aspettative;
\item \textbf{Accuratezza}: il prodotto fornisca i risultati attesi, con il livello di dettaglio richiesto;
\item \textbf{Sicurezza}: il prodotto protegga le informazioni e i dati da accessi e modifiche non autorizzati.
\end{itemize}

\subsubsection{Metriche}
\paragraph{Completezza dell'implementazione funzionale}
Indica la percentuale di requisiti funzionali coperti dall'implementazione.
\begin{itemize}
	\item \textbf{Misurazione}: 
		$$C=\left(1-\mathlarger{\frac{N_{FM}}{N_{FI}}}\right) \cdot 100$$ 
	dove $N_{FM}$ è il numero di funzionalità mancanti nell'implementazione e $N_{FI}$ è il numero di funzionalità individuate nell'attività di analisi.
	\item \textbf{Range ottimale}: 100.
	\item \textbf{Range di accettazione}: 100.
\end{itemize}

\paragraph{Accuratezza rispetto alle attese}
Indica la percentuale di risultati concordi alle attese.
\begin{itemize}
	\item \textbf{Misurazione}: 
		$$A=\left(1-\mathlarger{\frac{N_{RD}}{N_{TE}}}\right) \cdot 100$$
	dove $N_{RD}$ è il numero di test che producono risultati discordanti rispetto alle attese e $N_{TE}$ è il numero di test-case eseguiti.
	\item \textbf{Range ottimale}: 100.
	\item \textbf{Range di accettazione}: 90 -- 100.
\end{itemize}

\paragraph{Controllo degli accessi}
Indica la percentuale di operazioni illegali non bloccate.
\begin{itemize}
	\item \textbf{Misurazione}: 
		$$I=\mathlarger{\frac{N_{IE}}{N_{II}}} \cdot 100$$
	dove $N_{IE}$ è il numero di operazioni illegali effettuabili dai test e $N_{II}$ è il numero di operazioni illegali individuate.
	\item \textbf{Range ottimale}: 0.
	\item \textbf{Range di accettazione}: 0 -- 10.
\end{itemize}


\subsection{Affidabilità}
Rappresenta la capacità del prodotto software di svolgere correttamente le sue funzioni durante il suo utilizzo, anche nel caso in cui si presentino situazioni anomale.

\subsubsection{Obiettivi di qualità}
L'esecuzione del prodotto dovrà presentare le seguenti caratteristiche:
\begin{itemize}
\item \textbf{Maturità}: dovrà essere evitato che si verifichino malfunzionamenti, operazioni illegali e restituzione di risultati errati (\textit{failure}) in seguito a difetti;
\item \textbf{Tolleranza agli errori}: nel caso in cui si presentino degli errori, dovuti a guasti o ad un uso scorretto dell'applicativo, questi dovranno essere gestiti in modo da mantenere alto il livello di prestazione.
\end{itemize}

\subsubsection{Metriche}
\paragraph{Densità di \textit{failure}}
Indica la percentuale di operazioni di testing che si sono concluse in fallimenti.

\begin{itemize}
	\item \textbf{Misurazione}: 
		$$F=\mathlarger{\frac{N_{FR}}{N_{TE}}} \cdot 100$$
	dove $N_{FR}$ è il numero di fallimenti rilevati durante l'attività di testing e $N_{TE}$ è il numero di test-case eseguiti.
	\item \textbf{Range ottimale}: 0.
	\item \textbf{Range di accettazione}: 0 -- 10.
\end{itemize}

\paragraph{Blocco di operazioni non corrette}
Indica la percentuale di funzionalità in grado di gestire correttamente i \textit{fault} che potrebbero verificarsi.
\begin{itemize}
	\item \textbf{Misurazione}: 
		$$B=\mathlarger{\frac{N_{FE}}{N_{ON}}} \cdot 100$$
	dove $N_{FE}$ è il numero di \textit{failure} evitati durante i test effettuati e $N_{ON}$ è il numero di test-case eseguiti che prevedono l'esecuzione di operazioni non corrette, causa di possibili \textit{failure}.
	\item \textbf{Range ottimale}: 100.
	\item \textbf{Range di accettazione}: 80 -- 100.
\end{itemize}

\subsection{Usabilità}
Rappresenta la capacità del prodotto di essere facilmente comprensibile e attraente in ogni sua parte per qualsiasi utente che lo andrà ad utilizzare.

\subsubsection{Obiettivi di qualità}
Il prodotto dovrà puntare ai seguenti obiettivi di usabilità:
\begin{itemize}
\item \textbf{Comprensibilità)}: l'utente dovrà essere in grado di riconoscerne le funzionalità offerte dal software e dovrà comprenderne le modalità di utilizzo per riuscire a raggiungere i risultati attesi;
\item \textbf{Apprendibilità}: dovrà essere data la possibilità all'utente di imparare ad utilizzare l'applicazione senza troppo impegno;
\item \textbf{Operabilità}: le funzionalità presenti dovranno essere coerenti con le aspettative dell'utente.
\end{itemize}

\subsubsection{Metriche}
\paragraph{Comprensibilità delle funzioni offerte}
Indica la percentuale di operazioni comprese in modo immediato dall'utente, senza la consultazione del manuale.
\begin{itemize}
	\item \textbf{Misurazione}: 
		$$C=\mathlarger{\frac{N_{FC}}{N_{FO}}} \cdot 100$$
	dove $N_{FC}$ è il numero di funzionalità comprese in modo immediato dall'utente durante l'attività di testing del prodotto e $N_{FO}$ è il numero di funzionalità offerte dal sistema.
	\item \textbf{Range ottimale}: 90 -- 100.
	\item \textbf{Range di accettazione}: 80 -- 100.
\end{itemize}

\paragraph{Facilità di apprendimento delle funzionalità}
Indica il tempo medio impiegato dall'utente nell'imparare ad usare correttamente una data funzionalità.
\begin{itemize}
	\item \textbf{Misurazione}: indicatore numerico, espresso in minuti, che tiene traccia del tempo medio impiegato dall'utente nell'apprendere il corretto utilizzo di una funzionalità offerta dal sistema.
	\item \textbf{Range ottimale}: 0 -- 15.
	\item \textbf{Range di accettazione}: 0 -- 30.
\end{itemize}

\paragraph{Consistenza operazionale in uso}
Indica la percentuale di messaggi e funzionalità offerte all'utente che rispettano le sue aspettative riguardo al comportamento del software.
\begin{itemize}
	\item \textbf{Misurazione}: 
		$$C=\left(1-\mathlarger{\frac{N_{MFI}}{N_{MFO}}}\right) \cdot 100$$
	dove $N_{MFI}$ è il numero di messaggi e funzionalità che non rispettano le aspettative dell'utente e $N_{MFO}$ è il numero di messaggi e funzionalità offerti dal sistema.
	\item \textbf{Range ottimale}: 90 -- 100.
	\item \textbf{Range di accettazione}: 80 -- 100.
\end{itemize}

\subsection{Efficienza}
\label{efficienza}
Rappresenta la capacità di eseguire le funzionalità offerte dal software nel minor tempo possibile utilizzando al tempo stesso il minor numero di risorse possibili.

\subsubsection{Obiettivi di qualità}
Il prodotto dovrà essere efficiente, in particolare:
\begin{itemize}
\item \textbf{Comportamento rispetto al tempo}:  per svolgere le sue funzioni il software dovrà fornire adeguati tempi di risposta ed elaborazione;
\item \textbf{Utilizzo delle risorse}: il software quando eseguirà le sue funzionalità dovrà utilizzare un appropriato numero e tipo di risorse.
\end{itemize}

\subsubsection{Metriche}
\paragraph{Tempo di risposta}
Indica il periodo temporale medio che intercorre fra la richiesta al software di una determinata funzionalità e la restituzione del risultato all'utente.
\begin{itemize}
	\item \textbf{Misurazione}: 
		$$T_{RISP} = \mathlarger{\frac{\sum_{i=1}^{n} T_{i}}{n}}$$ 
	con $T_{RISP}$ misurato in secondi, e dove $T_{i}$ è il tempo intercorso fra la richiesta $i$ di una funzionalità ed il completamento delle operazioni necessarie a restituire un risultato a tale richiesta.
	\item \textbf{Range ottimale}: 0 -- 3.
	\item \textbf{Range di accettazione}: 0 -- 8.
\end{itemize}

\subsection{Manutenibilità}
Rappresenta la capacità del prodotto di essere modificato, tramite correzioni, miglioramenti o adattamenti del software a cambiamenti negli ambienti, nei requisiti e nelle specifiche funzionali.

\subsubsection{Obiettivi di qualità}
Le operazioni di manutenzione andranno agevolate il più possibile adottando le seguenti caratteristiche:
\begin{itemize}
\item \textbf{Analizzabilità}: il software dovrà consentire una rapida identificazione delle possibili cause di errori e malfunzionamenti;
\item \textbf{Modificabilità}: il prodotto originale dovrà permettere eventuali cambiamenti in alcune sue parti;
\item \textbf{Stabilità}: non dovranno insorgere effetti indesiderati in seguito a modifiche effettuate sul software;
\item \textbf{Testabilità}: il software dovrà poter essere facilmente testato per validare le modifiche effettuate.
\end{itemize}

\subsubsection{Metriche}
\paragraph{Capacità di analisi di \textit{failure}}
Indica la percentuale di \textit{failure} registrate delle quali sono state individuate le cause.
\begin{itemize}
	\item \textbf{Misurazione}: 
		$$I=\mathlarger{\frac{N_{FI}}{N_{FR}}} \cdot 100$$
	dove $N_{FI}$ è il numero di \textit{failure} delle quali sono state individuate le cause e $N_{FR}$ è il numero di \textit{failure} rilevate.
	\item \textbf{Range ottimale}: 80 -- 100.
	\item \textbf{Range di accettazione}: 60 -- 100.
\end{itemize}

\paragraph{Impatto delle modifiche}
Indica la percentuale di modifiche effettuate in risposta a \textit{failure} che hanno portato all'introduzione di nuove \textit{failure} in altre componenti del sistema.
\begin{itemize}
	\item \textbf{Misurazione}: 
		$$I=\mathlarger{\frac{N_{FRF}}{N_{FR}}} \cdot 100$$
	dove $N_{FRF}$ è il numero di \textit{failure} risolte con l'introduzione di nuove \textit{failure} e $N_{FR}$ è il numero di \textit{failure} risolte;
	\item \textbf{Range ottimale}: 0 -- 10.
	\item \textbf{Range di accettazione}: 0 -- 20.
\end{itemize}

\newpage