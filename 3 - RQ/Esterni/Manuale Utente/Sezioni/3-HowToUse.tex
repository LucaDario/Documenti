\section{Features Guide}
Inside this section there will be some examples of code that show how to use \progettoShort, followed by a guide explaining how to use the \app application.

\subsection{SDK}
This guide is dedicated to a generic developer which wants to use \progettoShort\ to integrate its feature inside his \termine{Meteor.js} project. 

\subsubsection{Usage of a bubble}
\paragraph{Use of a default bubble}
Inside \progettoShort\ there are three kinds of bubbles available:
\begin{itemize}
	\item \textbf{MarkdownBubble}: represents a bubble which holds a text that can be formatted using the \termine{markdown} markup language;
	\item \textbf{ToDoListBubble}: represents a bubble that holds a to-do list made using a list of check buttons;
	\item \textbf{AlertBubble}: represents a bubble that holds a message which will be displayed as an alert.
\end{itemize}

%immagine di codice bolle e come si vedono

In order to instantiate one of these bubble, all that needs to be done is the following:
\begin{lstlisting}[language=JavaScript, frame=single]
// Needed to be able to create the new bubble instance
import {AlertBubble} from {BOH}

let bubble = new AlertBubble();
// Now you can work with this bubble accessing its methods
\end{lstlisting}

\textbf{Note}. We are not describing how to use all the default bubbles as the code is always the same, except that the bubble's name.

\paragraph{Creation of a new type of bubble}

Supposing you wanted to create your own Bubble it would be necessary extend the \textit{Abstract} class \texttt{BaseBubble} and would add all the components you need with the method \texttt{addComponent} available in his relative class. You can add every widget there is in \textit{Monolith} ord add your coustom one.

%immagine del codice di una nuova bolla creata da uno sviluppatore con i propri campi



\subsubsection{Composition/chaining of Layouts or Components}

In the \termine{SDK} you can also compose the Bubble as you want. To make this clearer we'll give you some examples of adding a Widget and adding layouts.

\paragraph{Adding a widget}

%immagine

\paragraph{Adding a Layout with some widgets}

%immagine
