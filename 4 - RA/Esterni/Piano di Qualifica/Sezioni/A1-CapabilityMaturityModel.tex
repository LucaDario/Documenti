\newpage

\section{Capability Maturity Model}
Il \termine{Capability Maturity Model} (\textit{CMM}) è un modello di riferimento costituito
da pratiche consolidate in una disciplina specifica e viene utilizzato per stabilire la
capacità di una organizzazione o gruppo di operare in quella disciplina.

\subsection{I modelli CMM}
Esistono vari modelli di CMM che differiscono per:

\begin{itemize}
	\item Disciplina (software, sistemi, acquisizione, etc.);
	\item Struttura;
	\item Come viene definito il concetto di \textit{Maturity}, ossia il percorso di miglioramento del processo;
	\item Come viene definito il concetto di \textit{Capability}, ossia l'istituzionalizzazione.
\end{itemize}
Dal 1991 ad oggi sono stati sviluppati dei CMM per molte discipline; alcuni dei
più utilizzati includono modelli per l'ingegneria dei sistemi, l'ingegneria del software, l'acquisizione software, la gestione della forza-lavoro e dello sviluppo ed infine lo sviluppo
integrato di prodotti e processi. \\
L'uso di modelli differenti è risultato problematico a causa della confusione causata dall'utilizzo contemporaneo di più modelli stessi, e dei costi di integrazione di più di un modello in un unico programma coordinato di miglioramento.

\subsection{CMMI}
Per risolvere la confusione che veniva a crearsi con l'utilizzo di più modelli è nato
il progetto di \textit{Integrazione dei CMM} (\termine{Capability Maturity Model Integration -- CMMI}). \\
Nel modello CMMI si usano i livelli per descrivere un percorso di evoluzione raccomandato per un'organizzazione che voglia migliorare i processi che usa per sviluppare e mantenere i propri prodotti e servizi. I livelli presenti sono i seguenti:
\begin{itemize}
	\item \textbf{Livello 0}: Incomplete
	\item \textbf{Livello 1}: Performed
	\item \textbf{Livello 2}: Managed
	\item \textbf{Livello 3}: Defined
	\item \textbf{Livello 4}: Quantitatively Managed
	\item \textbf{Livello 5}: Optimizing
\end{itemize}

\subsubsection{Livello 0: Incomplete}
Un processo è incompleto quando non è stato eseguito oppure è stato eseguito solo parzialmente. Ciò significa che uno o più obiettivi specifici dell'area di processo non sono stati soddisfatti e/o non esistono obiettivi generici per questo livello in quanto non c'è ragione di istituzionalizzare un processo parzialmente eseguito.

\subsubsection{Livello 1: Performed}
Affinché un'area di processo sia al 1$^{\circ}$ livello di \textit{capability} bisogna mettere in pratica
le attività di base richieste per iniziare a lavorare su quella determinata area. \\
Un processo eseguito è tale quando soddisfa gli obiettivi specifici dell'area di processo.
Questo livello supporta e permette le attività necessarie per produrre i \termine{work product} e benché il processo eseguito abbia per risultato dei miglioramenti importanti, questi
potrebbero essere persi nel tempo se non vengono istituzionalizzati. L'istituzionalizzazione, ovvero il soddisfacimento delle pratiche generiche dei livelli di \textit{capability} dal 2 al 5, aiuta inoltre ad assicurare che i miglioramenti siano mantenuti.

\subsubsection{Livello 2: Managed}
Un processo viene definito gestito se viene eseguito e possiede le infrastrutture di base per
essere supportato. Ciò può verificarsi quando esso è organizzato ed eseguito in accordo ad alcune regole, impiega persone competenti che abbiano adeguate conoscenze per produrre dei risultati controllati, coinvolge gli \termine{stakeholder} più importanti ed è monitorato, controllato e revisionato. \\
Questo livello aiuta ad assicurare che le pratiche esistenti siano mantenute nel tempo.

\subsubsection{Livello 3: Defined}
Un processo definito è un processo gestito che è stato adattato in accordo con le linee guida di \termine{tailoring} dei processi standard dell'organizzazione. Esso contribuisce inoltre alla realizzazione dei \termine{work product}, alle misure ed ad altre informazioni di miglioramento. \\
I processi vengono gestiti in maniera preventiva attraverso la comprensione delle relazioni tra le attività di processo, le misure dettagliate, i \termine{work product} ed i servizi.

\subsubsection{Livello 4: Quantitatively Managed}
In questo livello il processo è un processo definito e controllato usando tecniche quantitative o statistiche. Gli obiettivi quantitativi per la qualità del processo sono stabiliti ed usati come criteri di gestione dei processi stessi. Infine, la performance del processo e della qualità è intesa in termini statistici e viene gestita attraverso tutta la vita del processo.

\subsubsection{Livello 5: Optimizing}
In questo caso si ha un livello di \textit{capability} pari a quello del livello precedente, con l'unica differenza che è stato migliorato sulla base di una comprensione delle più comuni cause di variazione del processo. Il punto focale a questo livello è un miglioramento continuo del range di esecuzione del processo attraverso miglioramenti innovativi ed incrementali.

\subsection{Standard ISO/IEC 15504}
Il team ha individuato in questo standard un modello da seguire per una misurazione più accurata del livello raggiunto in ogni processo. Esso è spiegato in maggior dettaglio nel documento \NdP.

\subsection{Scelte del team}
A causa della ridotta quantità di tempo e dalla struttura del progetto non tutti i processi individuati nelle \NdP{} possono raggiungere un alto livello di \textit{capability} e miglioramento. Il gruppo però si impegna a raggiungere un buon livello per i processi ritenuti fondamentali per la buona riuscita del progetto e per quelli ripetuti costantemente nel tempo. A tal fine, in questo documento verranno elencati i livelli di tutti i processi attuati dal gruppo e per ognuno verranno dichiarati gli obiettivi che si vogliono raggiungere.

\newpage