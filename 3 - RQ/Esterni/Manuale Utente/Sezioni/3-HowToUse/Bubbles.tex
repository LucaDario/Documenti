\subsubsection{Bubbles}

\paragraph{Instantiation of a default bubble}
Inside \progettoShort\ there are three kinds of bubbles available:
\begin{itemize}
	\item \textbf{MarkdownBubble}: represents a bubble which holds a text that can be formatted using the \termine{markdown} markup language;
	\item \textbf{ToDoListBubble}: represents a bubble that holds a to-do list made using a list of check buttons;
	\item \textbf{AlertBubble}: represents a bubble that holds a message which will be displayed as an alert.
\end{itemize}

%immagine di codice bolle e come si vedono

In order to instantiate one of these bubble, all that needs to be done is the following:
\begin{lstlisting}[language=JavaScript, frame=single]
// Needed to be able to create the new bubble instance
import {AlertBubble} from {BOH}

let bubble = new AlertBubble();
// Now you can work with this bubble accessing its methods
\end{lstlisting}

\textbf{Note}. We are not describing how to use all the default bubbles as the code is always the same, except that the bubble's name.

\paragraph{Creation of a new type of bubble}
To create a new bubble type it is necessary to extend the abstract class \texttt{BaseBubble} and define a no-arguments constructor which calls the one of \texttt{BaseBubble}. \\
\begin{lstlisting}[language=JavaScript, frame=single]
// Needed to be able to extend from the BaseBubble class
import {BaseBubble} from {BOH}

export class MyBubble extends BaseBubble {
    constructor(){
        // Necessary call
        super();
        
        // Initialize the bubble
    }
}
\end{lstlisting}

%immagine del codice di una nuova bolla creata da uno sviluppatore con i propri campi