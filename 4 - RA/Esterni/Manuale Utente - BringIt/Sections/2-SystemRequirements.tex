\section{Use requirements}

% Requisiti generali
\subsection{Requirements}
In order to use \app\ the user needs to access the internet either using a laptop or desktop computer, or using a mobile cellphone or tablet. Other than that, the device which will be used to use \app\ must have a browser which supports \termine{JavaScript} and has it enabled.

% Requisiti da desktop o laptop
\subsubsection{Desktop requirements}
To access \app\ using a desktop or laptop computer, one of the following browsers is required:
\begin{itemize}
	\item Microsoft Internet Edge 13 or above;
	\item Mozilla Firefox 45 or above;
	\item Google Chrome 56 or above;
	\item Opera 43 or above;
	\item Apple Safari 43 or above.
\end{itemize}

% Requisiti da mobile
\subsubsection{Mobile requirements}
To access \app\ using a mobile phone or table, one of the following requirement needs to be satisfied:
\begin{itemize}
	\item If the device has Android as its operative system, it needs to have Google Chrome 56 or above installed;
	\item If the device has iOS as its operative system, it needs to have either iOS 10 or above installed, or it needs to have Google Chrome 56 or above as browser.
\end{itemize}

% Sezione su come abilitare JS
\subsubsection{Javascript}
In order to enable \termine{JavaScript} inside the different browsers' versions the next steps must be followed:
\begin{itemize}
	\item \textbf{Google Chrome}
		\begin{itemize}
			\item Next to the navigation bar, click on "Customize and control Google Chrom", and then "Settings";
			\item Inside the "Settings" section, click on "Show advanced settings" at the bottom of the page;
			\item Search for the "Privacy" section and click on "Content settings";
			\item Under the "JavaScript" section inside the dialog popup, select "Allow all sites to run JavaScript (reccomended)";
			\item Click "Done";
			\item Close the "Section" tab;
			\item \textbf{Note}: If you were on the application homepage while enabling JavaScript, please reload the page to make sure the changes take action.
		\end{itemize}
		
	\item \textbf{Mozilla Firefox}
		\begin{itemize}
			\item On the address bar, type \texttt{about:config} and hit "Enter";
			\item Click on "I accept the risk!" if an alert message shows up;
			\item Inside the search bar, search for \texttt{javascript.enabled};
			\item If the value below the "value" field is set to "false", right click on "false" and click "Toggle";
			\item Close the "about:config" tab;
			\item \textbf{Note}: If you were on the application homepage while enabling JavaScript, please reload the page to make sure the changes take action.
		\end{itemize}
		
	\item \textbf{Safari}
		\begin{itemize}
			\item Click on "Safari" on the menu bar, and then select "Preferences";
			\item Inside the "Preferences" window, select on "Security";
			\item Search for the "Web contents" section, and check "Enable JavaScript"; 
			\item Close the "Preferences" menu;
			\item \textbf{Note}: If you were on the application homepage while enabling JavaScript, please reload the page to make sure the changes take action.
		\end{itemize}
\end{itemize}


%Sezione dedicata a come installare i prodotti
\subsection{Installation}
To install \app\ the only thing that you will need is an internet connection. \\
In order to create a local server that will contain \app\ you will need to:
\begin{itemize}
	\item Create a root folder (let's call it \texttt{root}) and clone the following \termine{GitHub} repository inside it:
	\begin{lstlisting}
  https://github.com/NPE-Developers/Rocket.Chat
	\end{lstlisting}
	
	\item Open a shell inside the \texttt{root} folder and execute the following commands:
	\begin{lstlisting}
  > git clone https://github.com/NPE-Developers/Monolith ./packages/monolith
  > git clone https://github.com/NPE-Developers/BringIt ./packages/bringit
	\end{lstlisting}
	
	\item Once that both the repository have been cloned successfully, type the following commands inside a shell prompt opened inside the \texttt{root} folder:
	\begin{lstlisting}
  > meteor install monolith
  > meteor install bringit
  > meteor reset
  > meteor npm install
  > meteor run
	\end{lstlisting}

\end{itemize}


\subsection{Access to the application}
To access \app\, the only things required are the following:
\begin{enumerate}
	\item Connect to the following \termine{Rocket.chat} server from the device used to start the server:
			\begin{lstlisting}
			http://localhost:3000
			\end{lstlisting}
	\item Either:
		  \begin{enumerate}
		  	\item Register a new account if you do not have one.
		  		\begin{enumerate}
		  			\item Click on "Register a new account";
		  			\item Fill all the required fields with your information;
		  			\item Click on "Register a new account".
		  		\end{enumerate}
		  		
		  	\item Login with you credentials.
		  		\begin{enumerate}
		  			\item Insert your credentials inside the required fields
		  			\item Click on "Login"
		  		\end{enumerate}
		  \end{enumerate}
\end{enumerate}

\subsection{Error and bug reporting}
If you encounter any error or bug during the normal use of the application, please report it to us by opening an issue on our GitHub repository at the following URL: \link{https://github.com/NPE-Developers/BringIt/issues}. \\
We will work to continuously improve our application and fix all the bugs and error that will be reported. 
\newpage