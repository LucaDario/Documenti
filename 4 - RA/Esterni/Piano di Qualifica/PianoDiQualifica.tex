% Questo file definisce lo stile che verrà applicato
% ad ogni pagina di contenuto
\documentclass[a4paper,11pt]{article}

\usepackage{ifthen}
\usepackage[
 a4paper,
 top=2.5cm,
 bottom=2.5cm,
 left=1.5cm,
 right=1.5cm,
 head=30pt
]{geometry}
\usepackage[italian]{babel}
\usepackage[utf8x]{inputenc}
\usepackage[T1]{fontenc}
\usepackage{fancyhdr}
\usepackage[colorlinks=true, urlcolor=black, citecolor=black, linkcolor=black]{hyperref}
\usepackage{tabularx}
\usepackage{multirow}
\usepackage{booktabs}
\usepackage{color}
\usepackage{graphicx}
\usepackage{eurosym}
\usepackage{amsmath}
\usepackage{relsize}

\usepackage[multidot]{grffile}
\usepackage{xcolor,colortbl}
\definecolor{lightblue}{HTML}{56B4E6}
\definecolor{blue}{HTML}{2953A1}
\definecolor{darkblue}{HTML}{1E396E}

\usepackage[toc,page]{appendix}
\renewcommand\appendixtocname{Appendice}
\renewcommand{\appendixpagename}{Appendice}

\newcommand\pagenumberingnoreset[1]{\gdef\thepage{\csname @#1\endcsname\c@page}}

% Cambia il font 
\renewcommand*\rmdefault{qhv}

% ***STILE PAGINA***
\pagestyle{fancy}
\fancyhf{}
\setlength{\headheight}{1cm} 
% No indentazione paragrafo
\setlength{\parindent}{0pt}

% ***INTESTAZIONE***
\newcommand\textline[4][t]{%
  \noindent\parbox[#1]{.333\textwidth}{\raisebox{-0.40\height}{#2}}%
  \parbox[#1]{.333\textwidth}{\centering #3}%
  \parbox[#1]{.333\textwidth}{\raggedleft #4}%
}

\lhead{
	\textline[t]{\includegraphics[width=1cm, keepaspectratio=true]{../../../Template/Logo/Logo.png}}{\progettoShort}{\documento}
}

\renewcommand{\headrulewidth}{0.4pt}  %Linea sotto l'intestazione

% ***PIÈ DI PAGINA***
\lfoot{\textit{\gruppoLink}\\ \footnotesize{\email}}
\rfoot{\thepage} %per le prime pagine: mostra solo il numero romano
\cfoot{}
\renewcommand{\footrulewidth}{0.4pt}   %Linea sopra il piè di pagina


% Ridefinisce command \paragraph{} andando a capo ogni dopo la parola dentro le parentesi ed ha la possibiltà di enumerazione fino a n cifre modificando il numero dentro "secnumdepth"
\usepackage{titlesec}

\setcounter{secnumdepth}{7}
\setcounter{tocdepth}{7}
%
%
%\titleformat{\paragraph}
%{\normalfont\normalsize\bfseries}{\theparagraph}{1em}{}
%\titlespacing*{\paragraph}
%{0pt}{3.25ex plus 1ex minus .2ex}{1.5ex plus .2ex}
%
%
%\titleclass{\subsubparagraph}{straight}[\subparagraph]
%\newcounter{subsubparagraph}
%
%\titleformat{\subsubparagraph}[display]
%  {\normalfont\normalsize\bf}
%  {\thesubsubparagraph.}
%  {.5em}
%  {}
%\renewcommand\thesubsubparagraph\textbf{\roman{subsubparagraph}}
%\titlespacing*{\subsubparagraph} {0pt}{4pt}{6pt}


%***LA SOTTOSEZIONE PARAGRAPH VIENE VISUALIZZATA COME UNA SECTION
\titleformat{\paragraph}{\normalfont\normalsize\bfseries}{\theparagraph}{1em}{}
\titlespacing*{\paragraph}{0pt}{3.25ex plus 1ex minus .2ex}{1.5ex plus .2ex}

\titleformat{\subparagraph}{\normalfont\normalsize\bfseries}{\thesubparagraph}{1em}{}
\titlespacing*{\subparagraph}{0pt}{3.25ex plus 1ex minus .2ex}{1.5ex plus .2ex}

\makeatletter
\newcounter{subsubparagraph}[subparagraph]
\renewcommand\thesubsubparagraph{%
  \thesubparagraph.\@arabic\c@subsubparagraph}
\newcommand\subsubparagraph{%
  \@startsection{subsubparagraph}    % counter
    {6}                              % level
    {\parindent}                     % indent
    {3.25ex \@plus 1ex \@minus .2ex} % beforeskip
    {0.75em}                           % afterskip
    {\normalfont\normalsize\bfseries}}
\newcommand\l@subsubparagraph{\@dottedtocline{6}{13em}{5.5em}} %gestione dell'indice
\newcommand{\subsubparagraphmark}[1]{}
\makeatother

\makeatletter
\newcounter{subsubsubparagraph}[subsubparagraph]
\renewcommand\thesubsubsubparagraph{%
  \thesubsubparagraph.\@arabic\c@subsubsubparagraph}
\newcommand\subsubsubparagraph{%
  \@startsection{subsubsubparagraph}    % counter
    {7}                              % level
    {\parindent}                     % indent
    {3.25ex \@plus 1ex \@minus .2ex} % beforeskip
    {0.75em}                           % afterskip
    {\normalfont\normalsize\bfseries}}
\newcommand\l@subsubsubparagraph{\@dottedtocline{7}{16em}{6.5em}} %gestione dell'indice
\newcommand{\subsubsubparagraphmark}[1]{}
\makeatother

%Generali
\newcommand{\capitolato}{C5 - Monolith: An interactive bubble provider}
\newcommand{\progettoShort}{Monolith}
\newcommand{\progetto}{Monolith: An interactive bubble provider}
\newcommand{\gruppo}{NPE Developers}
\newcommand{\gruppoLink}{\href{https://gitlab.com/npe-developers}{NpeDevelopers}}
\newcommand{\email}{\href{mailto:npe.developers@gmail.com}{\textcolor{blue}{npe.developers@gmail.com}}}
\newcommand{\password}{NP3Devel0pers}
\newcommand{\myincludegraphics}[2][]{%
	\setbox0=\hbox{\phantom{X}}%
	\vtop{
		\hbox{\phantom{X}}
		\vskip-\ht0
		\hbox{\includegraphics[#1]{#2}}}
}




%Componenti del gruppo
\newcommand{\RM}{Riccardo Montagnin}
\newcommand{\MT}{Manuel Turetta}
\newcommand{\FB}{Francesco Bazzerla}
\newcommand{\SL}{Stefano Lia}
\newcommand{\LD}{Luca Dario}
\newcommand{\DC}{Diego Cavestro}
\newcommand{\ND}{Nicolò Dovico}

%Ruoli
\newcommand{\Pm}{Project Manager}
\newcommand{\Am}{Amministratore}
\newcommand{\AmP}{Amministratori}
\newcommand{\An}{Analista}
\newcommand{\AnP}{Analisti}
\newcommand{\Dev}{Sviluppatore}
\newcommand{\DevP}{Sviluppatori}
\newcommand{\Ver}{Verificatore}
\newcommand{\VerP}{Verificatori}
\newcommand{\Progr}{Programmatore}
\newcommand{\ProgrP}{Programmatori}
\newcommand{\Prog}{Progettista}
\newcommand{\ProgP}{Progettisti}



%Firme
\newcommand{\RMFirma}{\myincludegraphics[scale = 0.5]{../../../Template/Firme/RM.png}}
\newcommand{\MTFirma}{\myincludegraphics[scale = 0.5]{../../../Template/Firme/MT.png}}
\newcommand{\FBFirma}{\myincludegraphics[scale = 0.5]{../../../Template/Firme/FB.png}}
\newcommand{\SLFirma}{\myincludegraphics[scale = 0.5]{../../../Template/Firme/SL.png}}
\newcommand{\LDFirma}{\myincludegraphics[scale = 0.5]{../../../Template/Firme/LD.png}}
\newcommand{\DCFirma}{\myincludegraphics[scale = 0.5]{../../../Template/Firme/DC.png}}
\newcommand{\NDFirma}{\myincludegraphics[scale = 0.5]{../../../Template/Firme/ND.png}}

%Professori e proponente
\newcommand{\TV}{Prof. Tullio Vardanega}
\newcommand{\RC}{Prof. Riccardo Cardin}
\newcommand{\RB}{Red Babel}
\newcommand{\proponente}{Red Babel}

%Documenti
\newcommand{\Gl}{Glossario}
\newcommand{\glossario}{\textit{\Gl\_v.1.0.0.pdf}}
\newcommand{\AdR}{Analisi dei Requisiti}
\newcommand{\analisiDeiRequisiti}{\textit{\AdR\_v.1.0.0.pdf}}
\newcommand{\AdRvDue}{AnalisiDeiRequisiti}
\newcommand{\NdP}{Norme di Progetto}
\newcommand{\normeDiProgetto}{\textit{\NdP\_v.1.0.0.pdf}}
\newcommand{\PdP}{Piano di Progetto}
\newcommand{\pianoDiProgetto}{\textit{\PdP\_v.1.0.0.pdf}}
\newcommand{\SdF}{Studio di Fattibilità}
\newcommand{\studioDiFattibilita}{\textit{\SdF\_v.1.0.0.pdf}}
\newcommand{\PdQ}{Piano di Qualifica}
\newcommand{\pianoDiQualifica}{\textit{\PdQ\_v.1.0.0.pdf}}
\newcommand{\VI}{Verbale Interno}
\newcommand{\VE}{Verbale Esterno}
\newcommand{\ST}{Specifica Tecnica}
\newcommand{\MU}{Manuale Utente}
\newcommand{\DDP}{Definizione di Prodotto}

%Periodo di progetto
\newcommand{\ARM}{Analisi dei Requisiti di Massima}
\newcommand{\ARD}{Analisi dei Requisiti in Dettaglio}
\newcommand{\PA}{Progettazione Architetturale}
\newcommand{\PD}{Progettazione di Dettaglio}
\newcommand{\COD}{Codifica}
\newcommand{\VV}{Verifica e Validazione Finale}

%Consegne
\newcommand{\RR}{Revisione dei Requisiti}
\newcommand{\RP}{Revisione di Progettazione}
\newcommand{\RQ}{Revisione di Qualifica}
\newcommand{\RA}{Revisione di Accettazione}


%Formattazione
\newcommand{\termine}[1]{\textit{#1}\small{$_G$}}
\newcommand{\link}[1]{\href{#1}{\textcolor{blue}{\texttt{#1}}}} 

% Testi ricorrenti
\newcommand{\scopoProdotto}{L'obiettivo di questo progetto è la realizzazione di un \termine{SDK} che permetta la creazione di bolle interattive, le quali, successivamente, verranno utilizzate all'interno dell'applicazione di messaggistica istantanea open source \termine{Rocket.chat}. \\
Dopo la realizzazione di tale \termine{SDK}, è proposto lo sviluppo di un'applicazione in grado di sfruttare l'\termine{SDK} per implementare un uso originale di tali bolle.
}
\newcommand{\descrizioneGlossario}{Al fine di mantenere questo documento compatto e di facile lettura è stato realizzato un glossario esterno contenente tutte le definizioni dei termini che più comunemente verranno presentati al lettore.  
Tale glossario si ritrova all'interno del file \glossario, e contiene tutti e soli i termini che vengono marcati con una \textit{G} a pedice.
}
\newcommand{\riferimentiNormativi}{
	\begin{itemize}
		\item \textbf{Norme di Progetto}: \normeDiProgetto
		\item \textbf{\termine{Capitolato} d'appalto C5: Monolith - An Interactive bubble provider} \\
			  \link{http://www.math.unipd.it/~tullio/IS-1/2016/Progetto/C5.pdf}
	\end{itemize}
}

% Comandi per generare l'intro
\newcommand{\documento}{\PdQ}
\newcommand{\versione}{3.0.2}
\newcommand{\redatori}{\FB\\ & \RM\\ & \MT\\ & \SL}
\newcommand{\revisori}{\RM}
\newcommand{\approvazione}{\FB}
\newcommand{\statoapprovazione}{Non approvato}
% Quando il documento sarà approvato, inserire all'interno del comando seguente la data nel formato GG mese AAAA dove GG è il giorno a due cifre, mese è il mese scritto per esteso con la prima lettera minuscola, e AAAA è l'anno a quattro cifre
\newcommand{\dataApprovazione}{}
\newcommand{\uso}{Esterno}
\newcommand{\destinatari}{\TV\\ & \RC\\ & \RB}

\newcommand{\sommario}{Questo documento si prefigge di regolamentare le operazioni di verifica e validazione del gruppo \gruppo\ necessarie ad assicurare i requisiti qualitativi per il progetto \progetto.
}
\newcommand{\modifiche}{
	Verifica sezioni 1 e 4 & \RM & \Prog & 06/03/2017 & 0.1.0 \\\midrule
	Stesura appendice 1 & \RM & \Prog & 03/03/2017 & 0.0.5 \\\midrule
	Stesura sezione 4 & \DC & \Prog & 28/02/2017 & 0.0.4 \\\midrule
	Stesura sezione 4 & \FB & \Prog & 28/02/2017 & 0.0.3 \\\midrule
	Stesura sezione 1 & \FB & \Prog & 28/02/2017 & 0.0.2 \\\midrule
    Creazione del template & \FB & \Prog & 28/02/2017 & 0.0.1 \\\midrule
}

\usepackage{caption}
\usepackage{array}
\newcolumntype{P}[1]{>{\centering\arraybackslash}p{#1}}

\begin{document}

\input{../../../Template/Intro.tex}
%Questo file si occupa di generare la tabella delle modifiche
\pagenumbering{Roman}

\begin{center}
    \Large{\textbf{Registro delle modifiche}}
    	\\\vspace{0.5cm}
    	\normalsize
    \begin{tabularx}{\textwidth}{cXXcc}
        \textbf{Versione} & \textbf{Modifica - Motivazione} & \textbf{Autore} & \textbf{Ruolo} & \textbf{Data} \\\toprule
        \modifiche
    \end{tabularx}
\end{center}

\newpage



\input{../../../Template/Indice.tex}
\setcounter{table}{0}
\listoftables
\newpage

% Sezioni
\section{Introduzione}
\subsection{Scopo del documento}
Questo documento vuole definire le strategie che il \termine{team} ha deciso di adottare per perseguire gli obiettivi di qualità di processo e di prodotto ricercati. A tal fine è necessaria una costante attività di verifica e validazione del lavoro svolto in modo da poter rilevare e correggere le anomalie che potrebbero nascere.

\subsection{Scopo del prodotto}
\scopoProdotto

\subsection{Glossario}
\descrizioneGlossario

\subsection{Riferimenti}
\subsubsection{Normativi}
\riferimentiNormativi

\subsubsection{Informativi}
\begin{itemize}
	\item \textbf{\AdR}: \analisiDeiRequisiti;
	\item \textbf{\PdP}: \pianoDiProgetto;
	\item \textbf{\textit{Slide} dell'insegnamento di Ingegneria del Software}: \\
		  \link{http://www.math.unipd.it/~tullio/IS-1/2016/}
	\item \textbf{\textit{Standard} ISO/IEC 9126}: Product quality \\
	 	  \link{https://en.wikipedia.org/wiki/ISO/IEC\_9126}
	\item \textbf{\textit{Standard} tecnici ISO/IEC 15504}: Software process assessment \\
		  \link{https://en.wikipedia.org/wiki/ISO/IEC\_15504}
	\item \textbf{Ciclo di Deming (\termine{PDCA})}: Miglioramento dei processi \\
		  \link{https://en.wikipedia.org/wiki/PDCA}
\end{itemize}

\newpage
\section{Visione Generale della Strategia di Gestione della Qualità}
\subsection{Obiettivi di Qualità}
\subsubsection{Modello per la Qualità di Processo}
La qualità del prodotto è conseguenza anche della qualità dei processi che lo definiscono. Il \termine{gruppo} però, dopo una attenta valutazione, ha constatato che, essendo una qualità verificabile soltanto a lungo termine, non vi è il tempo materiale per assicurare una qualità di processo definita dall'\termine{ISO}. Il \termine{gruppo}, consapevole della scelta, cercherà comunque di prendere spunto, il più possibile, dallo standard ISO/IEC 15504.

\subsubsection{Standard per la Qualità di Prodotto}
\gruppo\ si impegna a seguire lo standard ISO/IEC 9126 redatto con lo scopo di descrivere obiettivi qualitativi e delineare delle metriche capaci di misurare il raggiungimento di tali obiettivi.

\subsubsection{Soluzioni attuate per il controllo della Qualità di Processo}

L'attuazione del metodo di gestione \termine{PDCA} aiuterà un maggior avvicinamento allo standard ISO/IEC 15504 ed assicurerà, quindi, una maggior qualità di processo. Con il ciclo \termine{PDCA} è possibile infatti garantire un miglioramento continuo dei processi, inclusa la verifica, ed un utilizzo ottimale delle risorse, ottenendo di conseguenza il miglioramento dei prodotti risultanti.
Per avere controllo sulla qualità è necessario che: 
\begin{itemize}
\item
I processi siano pianificati nel dettaglio;
\item
Nella pianificazione siano ripartite in modo chiaro le risorse;
\item
I processi vengano costantemente monitorati.
\end{itemize}

L'attuazione di tali punti è descritta dettagliatamente nel \PdP.
La qualità dei processi viene inoltre monitorata mediante l'analisi costante della qualità del prodotto e quantificata utilizzando le varie metriche che saranno descritte in seguito all'interno di questo documento.

\subsubsection{Soluzioni attuate per il controllo della Qualità di Prodotto}

Al fine di garantire un controllo sistematico della qualità del prodotto, il \termine{gruppo} seguirà le seguenti linee guida:
\begin{itemize}
\item
\textbf{\termine{Quality Assurance}}: insieme di attività realizzate per garantire il raggiungimento degli obiettivi di qualità. Prevede l'attuazione di tecniche di analisi statica e dinamica;
\item
\textbf{\termine{Verifica}}: processo che determina se il lavoro di un determinato periodo è consistente, completo e corretto. La verifica andrà eseguita costantemente durante l'intera durata del progetto;
\item
\textbf{\termine{Validazione}}: conferma in modo oggettivo che il sistema soddisfi correttamente i requisiti. \\ Anch'essa come la \termine{verifica} andrà eseguita costantemente durante l'intera durata del progetto.
\end{itemize}

\subsection{Scadenze Temporali}
Al fine di perseguire l'obbiettivo di rispettare le scadenze fissate nel Piano di Progetto è necessario che l'attività di verifica della documentazione sia sistematica e ben organizzata. Solo così, infatti, l'individuazione e la correzione di eventuali errori avverrà il prima possibile, impedendo la compromissione dell'intero progetto. \\
Ogni attività di redazione dei documenti e di codifica dovrà essere preceduta da uno studio preliminare sulla struttura e sui contenuti degli stessi, con lo scopo di ridurre la possibilità di commettere imprecisioni di natura concettuale e tecnica.

\subsubsection{Responsabilità}

Per ottenere un maggior livello di efficacia ed efficienza nell'attività di verifica verranno attribuite delle responsabilità a specifici ruoli di progetto.
La responsabilità, per l'attività di \termine{verifica} e \termine{validazione}, sarà a carico dei membri del \termine{gruppo} che al momento di eseguire tale attività saranno in carica dei ruoli di \Pm\ e dei \VerP.

\newpage
\section{Strategia di Gestione della Qualità in Dettaglio}
\subsection{Risorse}
Per raggiungere gli obiettivi qualitativi prefissati è necessario, oltre alle risorse umane, utilizzare la potenza e l'affidabilità delle risorse tecnologiche. Infatti, per agevolare il lavoro dei \VerP, verranno impiegati numerosi strumenti automatici che eseguiranno controlli sistematici sui prodotti generati. \\
Le risorse tecniche e tecnologiche consistono in tutti quegli strumenti software e hardware che il \gruppo\ intende utilizzare per le attività di verifica su processi e prodotti; per la descrizione di tali strumenti si rimanda al documento \NdP.

\subsection{Misure e metriche}
Allo scopo di rendere quantificabile il processo di verifica verranno adottate delle misure basate su metriche stabilite a priori. Le metriche incerte, qualora ve ne fossero, verranno migliorate in modo incrementale. \\
Le varie misure che verranno rilevate saranno analizzate confrontandole con due categorie di misurazione:
\begin{itemize}
\item
\textbf{Accettazione}: intervallo di valori minimi entro i quali il prodotto sarà accettato;
\item
\textbf{Ottimale}: intervallo di valori ottimali entro i quali il prodotto risulta soddisfare pienamente, o quasi, i requisiti richiesti. I valori all'interno di tali intervalli devono essere quindi intesi come consigliati ma non vincolanti.
\end{itemize}

\subsubsection{Metriche per i documenti}
Per i documenti redatti si è scelto di utilizzare l'indice di leggibilità per la lingua italiana: l'indice di Gulpease.

\paragraph{L'indice di Gulpease}
L'indice Gulpease rispetto ad altri indici di leggibilità possiede il vantaggio di utilizzare la lunghezza delle parole in lettere anziché in sillabe, semplificando il calcolo dell'indice stesso. Esso considera due variabili linguistiche:
\begin{itemize}
\item La lunghezza della parola;
\item La lunghezza della frase rispetto al numero delle lettere.
\end{itemize}

I valori dell'indice sono compresi tra 0 e 100, dove il valore 100 indica la leggibilità più alta e 0 la leggibilità più bassa.
Gli intervalli richiesti per ogni documento redatto sono i seguenti:
\begin{itemize}
\item Accettazione: [40 -- 100];
\item Ottimale: [50 -- 100].
\end{itemize}

\subsubsection{Metriche per il software}

Questa sezione, che verrà rivista e incrementata nelle prossime revisioni, è da intendere come una dichiarazione di propositi.
Per raggiungere gli obiettivi auspicati dallo standard ISO di riferimento (ISO/IEC 9126), ovvero funzionalità, affidabilità,
efficienza, usabilità e manutenibilità, saranno applicate le seguenti metriche:

\paragraph{Numero di livelli di annidamento per metodo}
Rappresenta il numero di strutture di controllo, annidate tra loro, internamente ad un metodo.
Un valore elevato per questa metrica potrebbe essere indice di una complessità troppo elevate del metodo stesso, e di un basso livello di astrazione del codice.
Intervalli richiesti:
\begin{itemize}
\item
Accettazione: [1 -- 5];
\item
Ottimale: [1 −- 3].
\end{itemize}

\paragraph{Numero di parametri per metodo}
Rappresenta il numero di parametri da passare per la chiamata di un metodo.
Un numero elevato per un dato metodo potrebbe evidenziare la necessità di ridurne le funzionalità associate e/o suddividerle in altri metodi ausiliari.
Un alto valore di questo indice potrebbe evidenziare pertanto un possibile errore di progettazione.
Intervalli richiesti:
\begin{itemize}
\item
Accettazione: [0 −- 6];
\item
Ottimale: [0 −- 4].
\end{itemize}

\paragraph{Complessità ciclomatica}
Questo indice viene utilizzato per misurare la complessità di un programma. Esso misura direttamente il numero di cammini linearmente indipendenti attraverso il \termine{grafo di controllo di flusso}. I nodi del grafo corrispondono a gruppi indivisibili di istruzioni, mentre gli archi connettono due nodi se il secondo gruppo di istruzioni può essere eseguito immediatamente dopo il primo gruppo.
Alti valori di \termine{complessità ciclomatica} implicano una ridotta manutenibilità del codice. Valori bassi potrebbero però determinare  una scarsa efficienza dei metodi. McCabe, ideatore di questa metrica, raccomandava che i programmatori contassero la complessità dei moduli in sviluppo, e li dividessero in moduli più piccoli, qualora tale complessità superi 10 ma osservando che in certe circostanze può essere appropriato rilassare tale restrizione e permettere moduli con una complessità anche di 15.
Intervalli richiesti:
\begin{itemize}
\item
Accettazione: [1 -− 15];
\item
Ottimale: [1 -− 10].
\end{itemize}

\paragraph{Numero di attributi per classe}
Questa metrica prevede di valutare la qualità del software in base al numero di attributi presenti in una classe.
Un numero elevato di attributi potrebbe evidenziare un possibile errore di progettazione con conseguente necessità di suddividere la classe in più classi in relazione tra loro seguendo il principio dell'incapsulamento.
Intervalli richiesti:
\begin{itemize}
\item
Accettazione: [0 −- 12];
\item
Ottimale: [0 −- 8].
\end{itemize}

\paragraph{Copertura dei test}
Questo indice indica la percentuale di istruzioni del prodotto che vengono eseguite durante i test.
Un valore percentuale alto indica una maggiore copertura dei test e quindi una maggiore probabilità che le componenti abbiano una ridotta quantità di errori.
Tale indice può però essere abbassato da metodi molto semplici che non richiedono testing come ad esempio metodi \texttt{getter} e/o \texttt{setter}.
Intervalli richiesti:
\begin{itemize}
\item
Accettazione: [40\% −- 100\%];
\item
Ottimale: [65\% −- 100\%].
\end{itemize}

\paragraph{Linee di commento per linee di codice}
Questo indice indica il rapporto tra linee di commento e linee di codice ed è utile per stimare la manutenibilità e la comprensibilità del codice. 
Intervalli richiesti:
\begin{itemize}
\item
Accettazione: [> 0.15];
\item
Ottimale: [>0.20].
\end{itemize}

\paragraph{Livello di stabilità}
Per comprendere questa metrica è necessario dare una semplice spiegazione di \termine{accoppiamento afferente} e di \termine{accoppiamento efferente}.
\begin{itemize}
\item
\textbf{Accoppiamento afferente}: indica il numero di classi esterne ad un \termine{package} che dipendono da classi interne ad esso.
Un alto valore implica un alto grado di dipendenza del resto del software dal \termine{package}. Un valore eccessivamente basso, invece, potrebbe evidenziare che un \termine{package} fornisce poche funzionalità.
\item
\textbf{Accoppiamento efferente}: indica il numero di classi interne al \termine{package} che dipendono da classi esterne ad esso.
Mantenendo il valore di tale indice basso è possibile garantire funzionalità di base indipendentemente dal resto del sistema.
\end{itemize}

La stabilità di un \termine{package} indica la possibilità di effettuare modifiche a tale \termine{package} senza influenzarne altri all'interno dell'applicazione. Tale indice è strettamente legato all'accoppiamento efferente ed afferente e viene calcolato dalla seguente formula:

\begin{displaymath}
{\text{Accoppiamento Afferente}}\over{\text{Accoppiamento Afferente} + \text{Accoppiamento Efferente}}
\end{displaymath}

Intervalli richiesti:
\begin{itemize}
\item
Accettazione: [0.0 −- 0.8];
\item
Ottimale: [0.0 −- 0.3].
\end{itemize}

\newpage
\input{Sezioni/4-QualitaProcesso.tex}
\input{Sezioni/5-QualitaProdotto.tex}
\input{Sezioni/6-PianificazioneDeiTest.tex}

\newpage
\begin{appendices}
\input{Sezioni/A1-CapabilityMaturityModel.tex}
\input{Sezioni/A2-PDCA.tex}
\input{Sezioni/A3-StandardISO-IEC9126.tex}
\section{Resoconto delle attività di verifica}

In questa sezione vengono inserite tutte le misurazioni delle metriche trovate dal gruppo \gruppo.
Il team si impegna a garantire almeno il soddisfacimento del range di accettazione per ogni metrica.

Per quanto riguarda le misurazioni può essere calcolato un risultato singolo, se la metrica si riferisce ad una caratteristica singola, oppure può essere calcolato un risultato massimo, ovvero una metrica che si riferisce a più componenti, ad esempio le classi. In tal caso verrà presa la misurazione peggiore e confrontata con i valori scelti dal team precedentemente.

Alcune metriche, infine, sono state misurate più volte nel tempo e per queste verrà illustrato un diagramma cartesiano, in cui: l'asse delle ascisse rappresenterà in giorni la durata del periodo fino alla consegna, mentre l'asse delle ordinate rappresenterà i valori assunti al momento delle misurazioni. \\
Verranno comunque riportati i risultati finali di tali metriche nell'apposita tabella.


\subsection{Revisione dei Requisiti}
In questa sezione vengono inseriti i risultati relativi al periodo di Revisione dei Requisiti e le metriche relative ad esso.

\subsubsection{Analisi statica dei documenti}
L'analisi statica dei documenti è stata fatta mediante \termine{Walkthrough} ed ha portato all'individuazione di alcuni errori. Tra gli errori individuati quelli più frequenti sono stati:
		\begin{itemize}
			\item Errori nei concetti esposti.
			\item Aggettivi o verbi utilizzati in modo scorretto.
			\item Periodi troppo lunghi o complessi da capire ed interpretare.
		\end{itemize}

\subsubsection{Esiti verifiche automatizzate}
		
\paragraph{Indice di Gulpease}

\begin{table}[h]
	\begin{center}
		\begin{tabular}{|c|c|c|c|}
			\hline
			\textbf{Documento}	& \textbf{Risultato} & \textbf{Esito} & \textbf{Valore} \\
			\hline
		 \termine{Analisi} dei Requisiti v1.0.0 & 90 & Superato & Ottimale	\\
			\hline
			Glossario v1.0.0 & 56 & Superato & Ottimale	\\
			\hline
			Norme di Progetto v1.0.0 & 47 & Superato & Accettabile \\
			\hline
			Piano di Progetto v1.0.0 & 48 & Superato & Accettabile\\
			\hline
			Piano di Qualifica v1.0.0	& 48 & Superato & Accettabile\\
			\hline
			Studio di Fattibilità v1.0.0	& 47 & Superato & Accettabile\\
			\hline
			Verbale\_Esterno\_1\_20161223 v1.0.0	& 51 & Superato & Ottimale	\\
			\hline
		\end{tabular}
	\end{center}
	\caption{RR - Risultato indice di Gulpease}
\end{table}

\subsubsection{Soddisfacimento metriche}

\paragraph{Qualità di processo}
\begin{longtable}{|>{\centering}m{5cm}|c|c|c|c|c|}
\hline
\textbf{Metrica} & \textbf{Unità di misura} & \textbf{Risultato} & \textbf{Risultato Massimo} & \textbf{Esito} & \textbf{Valore}\\
\hline
\endhead

\emph{Schedule Variance} & {Attività} & \textcolor{Green}{0} & / & Superato & Ottimale\\ \hline
\emph{Budget Variance} & {Euro} & \textcolor{Green}{15} & / & Superato & Ottimale \\ \hline
\emph{Rischi non preventivati} & {Rischi} & \textcolor{Green}{0} & / & Superato & Ottimale\\ \hline
\emph{Ottimalità delle misurazioni} & {Percentuale} & \textcolor{Green}{0.6} & / & Superato & Ottimale \\ \hline
\emph{Rischi non preventivati} & {Rischi} & \textcolor{Green}{0} & / & Superato & Ottimale\\ \hline
%\emph{Efficienza di gestione dei rischi} & {Giorni} & \textcolor{Orange}{21.3} & $\geq 20$ & $\geq 60$\\ \hline
%\emph{Requisiti obbligatori soddisfatti} & {Percentuale} & \textcolor{Green}{100} & $100$ & $100$\\ \hline
%\emph{Livello di stabilità-SDK} & {Percentuale} & / &\textcolor{Green}{1} & Accettabile\\ \hline
%\emph{Livello di stabilità-Applicazione} & {Percentuale} & / &\textcolor{Green}{1} & Accettabile\\ \hline
%\emph{Astrattezza-SDK} & {Percentuale} & / &\textcolor{Green}{0.8} & Accettabile\\ \hline
%\emph{Astrattezza-Applicazione} & {Percentuale} & / &\textcolor{Green}{0.5} & Ottimale\\ \hline
%\emph{Distanza dalla sequenza principale-SDK} & {Percentuale} & / &\textcolor{Green}{1} & Accettabile\\ \hline
%\emph{Distanza dalla sequenza principale-Applicazione} & {Percentuale} & / &\textcolor{Green}{0.7} & Accettabile\\ \hline
%\emph{Numero di metodi per classe} & {Metodi} & / &\textcolor{Green}{11} & Accettabile\\ \hline
%\emph{Numero di attributi per classe} & {Attributi} & / &\textcolor{Green}{7} & Ottimale\\ \hline
%\emph{Numero di parametri per metodo} & {Parametri} & / &\textcolor{Green}{7} & Ottimale\\ \hline
%\emph{Produttività di codifica} & {Linee} & \textcolor{Orange}{9.1} & $\geq 3$ & $\geq 10$\\ \hline
%\emph{Complessità Ciclomatica media} & {Cammini} & \textcolor{Green}{0} & $1 - 15$ & $1 - 10$\\ \hline
%\emph{Livelli di annidamento medi} & {Chiamate} & \textcolor{Green}{1} & $1 - 6$ & $1 - 3$\\ \hline
%\emph{Linee di codice per linee di commento} & {Percentuale} & \textcolor{Green}{31} & $\geq 25$ & $\geq 30$\\ \hline
%\emph{Variabili inutilizzate} & {Variabili} & \textcolor{Green}{0} & $0$ & $0$\\ \hline
%\emph{Dipendenze} & {Chiamate require} & \textcolor{Green}{2.2} & $0 - 10$ & $0 - 5$\\ \hline
%\emph{Halstead Difficulty media} & {Percentuale} & \textcolor{Green}{0} & $0 - 25$ & $0 - 15$\\ \hline
%\emph{Halstead Volume media} & {Percentuale} & \textcolor{Green}{20} & $20 - 1500$ & $20 - 1000$\\ \hline
%\emph{Halstead Effort media} & {Percentuale} & \textcolor{Green}{0} & $0 - 400$ & $0 - 300$\\ \hline
%\emph{Indice di manutenibilità} & {Percentuale} & \textcolor{Orange}{5.14} & $100 - 171$ & $120 - 171$\\ \hline
%\emph{Componenti integrate} & {Percentuale} & \textcolor{Green}{100} & $100$ & $100$\\ \hline
%\emph{Test di Unità eseguiti} & {Percentuale} & \textcolor{Red}{36.6} & $90 - 100$ & $100$\\ \hline
%\emph{Test di Integrazione eseguiti} & {Percentuale} & \textcolor{Red}{0} & $60 - 100$ & $70 - 100$\\ \hline
%\emph{Test di Sistema eseguiti} & {Percentuale} & \textcolor{Red}{0} & $70 - 100$ & $80 - 100$\\ \hline
%\emph{Test di \termine{Validazione} eseguiti} & {Percentuale} & \textcolor{Red}{0} & $100$ & $100$\\ \hline
%\emph{Test superati} & {Percentuale} & \textcolor{Green}{100} & $90 - 100$ & $100$\\ \hline
%\emph{Branch Coverage} & {Percentuale} & \textcolor{Orange}{73.3} & $70 - 100$ & $80 - 100$\\ \hline
%\emph{Code Coverage} & {Percentuale} & \textcolor{Green}{76.57} & $60 - 100$ & $70 - 100$\\ \hline
\caption{RR-Metriche di qualità di processo}
\end{longtable}

\newpage

\subsubsection{Esiti delle metriche ripetute nel tempo}

\begin{figure}[H]
	\centering 
	\includegraphics[scale=0.7]{Sezioni/Immagini/ScheduleVariance-RR}
	\caption{Schedule variance - RR}
\end{figure}

\begin{figure}[H]
	\centering 
	\includegraphics[scale=0.7]{Sezioni/Immagini/BudgetVariance-RR}
	\caption{Budget variance - RR}
\end{figure}

\subsubsection{Livello dei processi}
\begin{longtable}{|>{\centering}m{6cm}|c|c|c|c|c|}
\hline
\textbf{Processo} & \textbf{Livello} & \textbf{Esito} & \textbf{Valore}\\
\hline
\endhead
\emph{Processo di fornitura} & \textcolor{Green}{1} & Superato & Accettabile\\ \hline
\emph{Processo di sviluppo} & \textcolor{Green}{2}* & Superato & Accettabile\\ \hline
\emph{Processo di documentazione} & \textcolor{Green}{2} & Superato & Accettabile\\ 
\hline
\emph{Processo di Configurazione} & \textcolor{Green}{1} & Superato & Ottimale\\ 
\hline
\emph{Processo di garanzia di qualità del Prodotto} & * & / & /\\ 
\hline
\emph{Processo di Verifica} & \textcolor{Green}{1} & Superato & Ottimale\\ 
\hline
\emph{Processo di Validazione} & * & / & /\\ 
\hline
\emph{Processo di Risoluzione dei problemi} & \textcolor{Green}{1} & Superato & Ottimale\\ 
\hline
\emph{Processo di Coordinamento} & \textcolor{Green}{1} & Superato & Ottimale\\ 
\hline
\emph{Processo di Pianificazione} & \textcolor{Green}{1} & Superato & Ottimale\\ 
\hline
\emph{Processo di Formazione} & \textcolor{Green}{1} & Superato & Ottimale\\ 
\hline
\caption{RR-Livello dei processi}
\end{longtable}

Per i processi con segnatura \texttt{Voto*} vengono considerate solo le attività inerenti al lavoro che deve essere svolto per la \textit{Revisione di progettazione}. Per quelli, invece, con solo \texttt{*} significa che nessuna attività di quel processo era necessaria per il raggiungimento della milestone esterna.

\newpage

\subsection{Revisione di Progettazione}

\subsubsection{Analisi statica dei documenti}
L'analisi statica dei documenti è stata fatta mediante \termine{Walkthrough} ed ha portato all'individuazione di alcuni errori. Tra gli errori individuati quelli più frequenti sono stati:
		\begin{itemize}
			\item Errori ortografici.
			\item Parole con lettere mancanti o invertite.
			\item Periodi troppo lunghi o complessi da capire ed interpretare.
		\end{itemize}

\subsection{Metriche per i documenti}

\subsubsection{Indice di Gulpease}

\begin{table}[h]
	\begin{center}
		\begin{tabular}{|c|c|c|c|c|}
			\hline
			\textbf{Documento}	& \textbf{Risultato} & \textbf{Esito} & \textbf{Valore}\\
			\hline
		 \termine{Analisi} dei Requisiti v2.0.0 &	90 & Superato & Ottimale\\
			\hline
			Glossario v2.0.0 &	54 & Superato & Ottimale\\
			\hline
			Norme di Progetto v2.0.0 &	52 & Superato & Ottimale\\
			\hline
			Piano di Progetto v2.0.0	&	52 & Superato & Ottimale\\
			\hline
			Piano di Qualifica v2.0.0	&	46 & Superato & Accettabile\\
			\hline
			Definizione di \termine{Prodotto} v1.0.0	&	64 & Superato & Ottimale\\
			\hline
			Verbale\_Interno\_2\_20170222 v1.0.0	&	54 & Superato & Ottimale\\
			\hline
			Verbale\_Esterno\_3\_20170224 v1.0.0	&	53 & Superato & Ottimale\\
			\hline
			Verbale\_Interno\_4\_20170226 v1.0.0	&	51 & Superato & Ottimale\\
			\hline
			Verbale\_Interno\_5\_20170228 v1.0.0	&	52 & Superato & Ottimale\\
			\hline
		\end{tabular}
	\end{center}
	\caption{RP - Risultato indice di Gulpease}
\end{table}

\subsubsection{Soddisfacimento metriche}

\paragraph{Qualità di processo}
\begin{longtable}{|>{\centering}m{5cm}|c|c|c|c|c|}
\hline
\textbf{Metrica} & \textbf{Unità di misura} & \textbf{Risultato} & \textbf{Risultato Massimo} & \textbf{Esito} & \textbf{Valore}\\
\hline
\endhead

\emph{Schedule Variance} & {Attività} & \textcolor{Green}{0} & / & Superato & Ottimale\\ \hline
\emph{Budget Variance} & {Euro} & \textcolor{Orange}{-50} & / & Non superato & /\\ \hline
\emph{Ottimalità delle misurazioni} & {Percentuale} & \textcolor{Green}{0.6} & / & Superato & Ottimale \\ \hline
\emph{Rischi non preventivati} & {Rischi} & \textcolor{Green}{2} & / & Superato & Accettabile\\ \hline
%\emph{Efficienza di gestione dei rischi} & {Giorni} & \textcolor{Orange}{21.3} & $\geq 20$ & $\geq 60$\\ \hline
%\emph{Requisiti obbligatori soddisfatti} & {Percentuale} & \textcolor{Green}{100} & $100$ & $100$\\ \hline
\emph{Livello di instabilità-SDK} & {Percentuale} & / &\textcolor{Green}{1} & Superato & Accettabile\\ \hline
\emph{Livello di instabilità-Applicazione} & {Percentuale} & / &\textcolor{Green}{1} & Superato & Accettabile\\ \hline
\emph{Astrattezza-SDK} & {Percentuale} & \textcolor{Green}{0.8} & / & Superato & Accettabile\\ \hline
\emph{Astrattezza-Applicazione} & {Percentuale} & \textcolor{Green}{0.5} & / & Superato & Ottimale\\ \hline
\emph{Distanza dalla sequenza principale-SDK} & {Percentuale} & / &\textcolor{Green}{1} & Superato & Accettabile\\ \hline
\emph{Distanza dalla sequenza principale-Applicazione} & {Percentuale} & / &\textcolor{Green}{0.7} & Superato & Accettabile\\ \hline
\emph{Numero di metodi per classe} & {Metodi} & / &\textcolor{Green}{11} & Superato & Accettabile\\ \hline
\emph{Numero di attributi per classe} & {Attributi} & / &\textcolor{Green}{7} & Superato & Ottimale\\ \hline
\emph{Numero di parametri per metodo} & {Parametri} & / &\textcolor{Green}{7} & Superato & Ottimale\\ \hline
%\emph{Produttività di codifica} & {Linee} & \textcolor{Orange}{9.1} & $\geq 3$ & $\geq 10$\\ \hline
%\emph{Complessità Ciclomatica media} & {Cammini} & \textcolor{Green}{0} & $1 - 15$ & $1 - 10$\\ \hline
%\emph{Livelli di annidamento medi} & {Chiamate} & \textcolor{Green}{1} & $1 - 6$ & $1 - 3$\\ \hline
%\emph{Linee di codice per linee di commento} & {Percentuale} & \textcolor{Green}{31} & $\geq 25$ & $\geq 30$\\ \hline
%\emph{Variabili inutilizzate} & {Variabili} & \textcolor{Green}{0} & $0$ & $0$\\ \hline
%\emph{Dipendenze} & {Chiamate require} & \textcolor{Green}{2.2} & $0 - 10$ & $0 - 5$\\ \hline
%\emph{Halstead Difficulty media} & {Percentuale} & \textcolor{Green}{0} & $0 - 25$ & $0 - 15$\\ \hline
%\emph{Halstead Volume media} & {Percentuale} & \textcolor{Green}{20} & $20 - 1500$ & $20 - 1000$\\ \hline
%\emph{Halstead Effort media} & {Percentuale} & \textcolor{Green}{0} & $0 - 400$ & $0 - 300$\\ \hline
%\emph{Indice di manutenibilità} & {Percentuale} & \textcolor{Orange}{5.14} & $100 - 171$ & $120 - 171$\\ \hline
%\emph{Componenti integrate} & {Percentuale} & \textcolor{Green}{100} & $100$ & $100$\\ \hline
%\emph{Test di Unità eseguiti} & {Percentuale} & \textcolor{Red}{36.6} & $90 - 100$ & $100$\\ \hline
%\emph{Test di Integrazione eseguiti} & {Percentuale} & \textcolor{Red}{0} & $60 - 100$ & $70 - 100$\\ \hline
%\emph{Test di Sistema eseguiti} & {Percentuale} & \textcolor{Red}{0} & $70 - 100$ & $80 - 100$\\ \hline
%\emph{Test di \termine{Validazione} eseguiti} & {Percentuale} & \textcolor{Red}{0} & $100$ & $100$\\ \hline
%\emph{Test superati} & {Percentuale} & \textcolor{Green}{100} & $90 - 100$ & $100$\\ \hline
%\emph{Branch Coverage} & {Percentuale} & \textcolor{Orange}{73.3} & $70 - 100$ & $80 - 100$\\ \hline
%\emph{Code Coverage} & {Percentuale} & \textcolor{Green}{76.57} & $60 - 100$ & $70 - 100$\\ \hline
\caption{RP-Metriche di qualità di processo}
\end{longtable}

\subsubsection{Esiti delle metriche ripetute nel tempo}

\begin{figure}[H]
	\centering 
	\includegraphics[scale=0.7]{Sezioni/Immagini/ScheduleVariance-RP}
	\caption{Schedule variance - RP}
\end{figure}

\begin{figure}[H]
	\centering 
	\includegraphics[scale=0.7]{Sezioni/Immagini/BudgetVariance-RP}
	\caption{Budget variance - RP}
\end{figure}

\begin{figure}[H]
	\centering 
	\includegraphics[scale=0.85]{Sezioni/Immagini/LivelloInstabilitaSDK-RP}
	\caption{Livello di instabilità SDK - RP}
\end{figure}

\begin{figure}[H]
	\centering 
	\includegraphics[scale=0.85]{Sezioni/Immagini/LivelloInstabilitaApp-RP}
	\caption{Livello di instabilità Applicazione - RP}
\end{figure}

\begin{figure}[H]
	\centering 
	\includegraphics[scale=0.63]{Sezioni/Immagini/DistanzaSDK-RP}
	\caption{Distanza dalla sequenza principale SDK - RP}
\end{figure}

\begin{figure}[H]
	\centering 
	\includegraphics[scale=0.63]{Sezioni/Immagini/DistanzaApp-RP}
	\caption{Distanza dalla sequenza principale Applicazione - RP}
\end{figure}

\subsubsection{Livello dei processi}
\begin{longtable}{|>{\centering}m{6cm}|c|c|c|c|c|}
\hline
\textbf{Processo} & \textbf{Livello} & \textbf{Esito} & \textbf{Valore}\\
\hline
\endhead
\emph{Processo di fornitura} & \textcolor{Green}{2} & Superato & Ottimale\\ \hline
\emph{Processo di sviluppo} & \textcolor{Green}{2}* & Superato & Accettabile\\ \hline
\emph{Processo di documentazione} & \textcolor{Green}{2} & Superato & Accettabile\\ 
\hline
\emph{Processo di Configurazione} & \textcolor{Green}{1} & Superato & Ottimale\\ 
\hline
\emph{Processo di garanzia di qualità del Prodotto} & * & / & /\\ 
\hline
\emph{Processo di Verifica} & \textcolor{Green}{1} & Superato & Ottimale\\ 
\hline
\emph{Processo di Validazione} & * & / & /\\ 
\hline
\emph{Processo di Risoluzione dei problemi} & \textcolor{Green}{1} & Superato & Ottimale\\ 
\hline
\emph{Processo di Coordinamento} & \textcolor{Green}{1} & Superato & Ottimale\\ 
\hline
\emph{Processo di Pianificazione} & \textcolor{Green}{1} & Superato & Ottimale\\ 
\hline
\emph{Processo di Formazione} & \textcolor{Green}{1} & Superato & Ottimale\\ 
\hline
\caption{RP-Livello dei processi}
\end{longtable}

Per i processi con segnatura \texttt{Voto*} vengono considerate solo le attività inerenti al lavoro che deve essere svolto per la \textit{Revisione di progettazione}. Per quelli, invece, con solo \texttt{*} significa che nessuna attività di quel processo era necessaria per il raggiungimento della milestone esterna.

\newpage

\subsection{Revisione di Qualifica}

\subsubsection{Analisi statica dei documenti}
L'analisi statica dei documenti è stata fatta mediante \termine{Walkthrough} ed ha portato all'individuazione di alcuni errori. Tra gli errori individuati quelli più frequenti sono stati:
		\begin{itemize}
			\item Errori ortografici.
			\item Frasi complesse con un basso indice di comprensibilità.
		\end{itemize}

\subsection{Metriche per i documenti}

\subsubsection{Indice di Gulpease}

\begin{table}[h]
	\begin{center}
		\begin{tabular}{|c|c|c|c|c|}
			\hline
			\textbf{Documento}	& \textbf{Risultato} & \textbf{Esito} & \textbf{Valore}\\
			\hline
		    \termine{Analisi} dei Requisiti v3.0.0 & 87 & Superato & Ottimale\\
			\hline
			Glossario v3.0.0 & 55 & Superato & Ottimale\\
			\hline
			Norme di Progetto v3.0.0 & 46 & Superato & Accettabile\\
			\hline
			Piano di Progetto v3.0.0 & 48 & Superato & Accettabile\\
			\hline
			Piano di Qualifica v3.0.0 & 40 & Superato & Accettabile\\
			\hline
		 \termine{Manuale Utente} \termine{Monolith} v1.0.0 & 93 & Superato & Ottimale\\
			\hline
		 \termine{Manuale Utente} Bringit v1.0.0 & 48 & Superato & Accettabile\\
            \hline
            Verbale\_Interno\_7\_20170327 v1.0.0 & 67 & Superato & Ottimale\\
            \hline
            Verbale\_Interno\_8\_20170402 v1.0.0 & 73 & Superato & Ottimale\\
            \hline
            Verbale\_Interno\_9\_20170407 v1.0.0 & 78 & Superato & Ottimale\\
            \hline
            Verbale\_Esterno\_10\_20170503 v1.0.0 & 73 & Superato & Ottimale\\
            \hline
		\end{tabular}
	\end{center}
	\caption{RQ - Risultato indice di Gulpease}
\end{table}


\subsubsection{Soddisfacimento metriche}
\small{
Si noti che alcune componenti e i test a loro correlati non sono ancora state sviluppate, e che la loro assenza peserà nelle misurazioni.}

\paragraph{Qualità di processo}
\begin{longtable}{|>{\centering}m{5cm}|c|c|c|c|c|}
\hline
\textbf{Metrica} & \textbf{Unità di misura} & \textbf{Risultato} & \textbf{Risultato massimo} & \textbf{Esito} & \textbf{Valore}\\
\hline
\endhead
\emph{Schedule Variance} & {Attività} & \textcolor{Green}{0} & / & Superato & Ottimale\\ \hline
\emph{Budget Variance} & {Euro} & \textcolor{Orange}{-697} & / & Non superato & /\\ \hline
\emph{Ottimalità delle misurazioni} & {Percentuale} & \textcolor{Green}{74\%} & / & Superato & Ottimale \\ \hline
\emph{Rischi non preventivati} & {Rischi} & \textcolor{Green}{2} & / & Superato & Accettabile\\ \hline
\emph{Requisiti obbligatori soddisfatti} & {Percentuale} & \textcolor{Orange}{97.6\%} & / & Non superato & /\\ \hline
\emph{Livello di instabilità-SDK} & {Percentuale} & / & \textcolor{Green}{55\%} & Superato & Ottimale\\ \hline
\emph{Livello di instabilità-Applicazione} & {Percentuale} & / & \textcolor{Green}{70\%} & Superato & Accettabile\\ \hline
\emph{Astrattezza-SDK} & {Percentuale} & \textcolor{Green}{20\%} & / & Superato & Ottimale\\ \hline
\emph{Astrattezza-Applicazione} & {Percentuale} &\textcolor{Green}{15\%} & / & Superato & Ottimale\\ \hline
\emph{Distanza dalla sequenza principale-SDK} & {Percentuale} & / & \textcolor{Green}{25\%} & Superato & Ottimale\\ \hline
\emph{Distanza dalla sequenza principale-Applicazione} & {Percentuale} & / & \textcolor{Green}{15\%} & Superato & Ottimale\\ \hline
\emph{Numero di metodi per classe (max)} & {Metodi} & / & \textcolor{Green}{12} & Superato & Accettabile\\ \hline
\emph{Numero di attributi per classe (max)} & {Attributi} & / & \textcolor{Green}{7} & Superato & Ottimale\\ \hline
\emph{Numero di parametri per metodo (max)} & {Parametri} & / & \textcolor{Green}{5} & Superato & Ottimale\\ \hline
\emph{Complessità Ciclomatica media} & {Cammini} & \textcolor{Green}{1.3} & / & Superato & Ottimale\\ \hline
\emph{Livelli di annidamento medi} & {Chiamate} & \textcolor{Green}{2.4} & / & Superato & Ottimale\\ \hline
\emph{Linee di codice per linee di commento - Monolith} & {Percentuale} & \textcolor{Green}{28.3\%} & / & Superato & Ottimale\\ \hline
\emph{Linee di codice per linee di commento - BringIt} & {Percentuale} & \textcolor{Green}{20.5\%} & / & Superato & Ottimale\\ \hline
\emph{Componenti integrate} & {Percentuale} & \textcolor{Orange}{82\%} & / & Non superato & /\\ \hline
\emph{Test di Unità eseguiti} & {Percentuale} & \textcolor{Green}{99\%} & / & Superato & Ottimale\\ \hline
\emph{Test di Integrazione eseguiti} & {Percentuale} & \textcolor{Green}{84.7\%} & / & Superato & Ottimale\\ \hline
\emph{Test di Sistema eseguiti} & {Percentuale} & \textcolor{Green}{86\%} & / & Superato & Ottimale\\ \hline
\emph{Test di \termine{Validazione} eseguiti} & {Percentuale} & \textcolor{Orange}{46\%} & / & Non superato & / \\ \hline
\emph{Test superati} & {Percentuale} & \textcolor{Green}{100\%} & / & Superato & Ottimale \\ \hline
\emph{Branch Coverage} & {Percentuale} & \textcolor{Green}{83\%} & / & Superato & Accettabile\\ \hline
\emph{Statement Coverage} & {Percentuale} & \textcolor{Green}{77.4\%} & / & Superato & Accettabile\\ \hline
\emph{Accuratezza rispetto alle attese} & {Percentuale} & \textcolor{Green}{100} & / & Superato & Ottimale\\ \hline
\emph{Completezza dell’implementazione funzionale} & {Percentuale} & \textcolor{Orange}{86} & / & Non superato & /\\ \hline
\emph{Densità di failure} & {Percentuale} & \textcolor{Green}{0} & / & Superato & Ottimale\\ \hline
\emph{Blocco di operazioni non corrette} & {Percentuale} & \textcolor{Green}{90} & / & Superato & Accettabile\\ \hline
\emph{Tempo di risposta} & {Secondi} & / & \textcolor{Green}{1.0} & Superato & Ottimale\\ \hline
\emph{Capacità di analisi di failure} & {Percentuale} & \textcolor{Green}{100} & / & Superato & Ottimale\\ \hline
\emph{Impatto delle modifiche} & {Indice} & \textcolor{Green}{2} & / & Superato & Ottimale\\ \hline
\caption{RQ-Metriche di qualità di processo}\\
\end{longtable}

\subsubsection{Esiti delle metriche ripetute nel tempo}

\begin{figure}[H]
	\centering 
	\includegraphics[scale=0.7]{Sezioni/Immagini/ScheduleVariance-RQ}
	\caption{Schedule variance - RQ}
\end{figure}

\begin{figure}[H]
	\centering 
	\includegraphics[scale=0.7]{Sezioni/Immagini/BudgetVariance-RQ}
	\caption{Budget variance - RQ}
\end{figure}

\begin{figure}[H]
	\centering 
	\includegraphics[scale=0.85]{Sezioni/Immagini/LivelloInstabilitaSDK-RQ}
	\caption{Livello di instabilità Monolith - RQ}
\end{figure}

\begin{figure}[H]
	\centering 
	\includegraphics[scale=0.85]{Sezioni/Immagini/LivelloInstabilitaApp-RQ}
	\caption{Livello di instabilità Bringit - RQ}
\end{figure}

\begin{figure}[H]
	\centering 
	\includegraphics[scale=0.85]{Sezioni/Immagini/DistanzaSDK-RQ}
	\caption{Distanza dalla sequenza principale Monolith - RQ}
\end{figure}

\begin{figure}[H]
	\centering 
	\includegraphics[scale=0.85]{Sezioni/Immagini/DistanzaApp-RQ}
	\caption{Distanza dalla sequenza principale Bringit - RQ}
\end{figure}

\begin{figure}[H]
	\centering 
	\includegraphics[scale=0.8]{Sezioni/Immagini/LineeCodiceCommentoSDK-RQ}
	\caption{Linee di codice per linee di commento Monolith - RQ}
\end{figure}

\begin{figure}[H]
	\centering 
	\includegraphics[scale=0.8]{Sezioni/Immagini/LineeCodiceCommentoApp-RQ}
	\caption{Linee di codice per linee di commento Bringit - RQ}
\end{figure}

\subsubsection{Livello dei processi}
\begin{longtable}{|>{\centering}m{6cm}|c|c|c|c|c|}
\hline
\textbf{Processo} & \textbf{Livello} & \textbf{Esito} & \textbf{Valore}\\
\hline
\endhead
\emph{Processo di fornitura} & \textcolor{Green}{2} & Superato & Ottimale\\ \hline
\emph{Processo di sviluppo} & \textcolor{Green}{2}* & Superato & Accettabile\\ \hline
\emph{Processo di documentazione} & \textcolor{Green}{2} & Superato & Accettabile\\ 
\hline
\emph{Processo di Configurazione} & \textcolor{Green}{1} & Superato & Ottimale\\ 
\hline
\emph{Processo di garanzia di qualità del Prodotto} & * & / & /\\ 
\hline
\emph{Processo di Verifica} & \textcolor{Green}{1} & Superato & Ottimale\\ 
\hline
\emph{Processo di Validazione} & * & / & /\\ 
\hline
\emph{Processo di Risoluzione dei problemi} & \textcolor{Green}{1} & Superato & Ottimale\\ 
\hline
\emph{Processo di Coordinamento} & \textcolor{Green}{1} & Superato & Ottimale\\ 
\hline
\emph{Processo di Pianificazione} & \textcolor{Green}{1} & Superato & Ottimale\\ 
\hline
\emph{Processo di Formazione} & \textcolor{Green}{1} & Superato & Ottimale\\ 
\hline
\caption{RQ-Livello dei processi}
\end{longtable}

\newpage

\subsection{Revisione di Accettazione}

\subsubsection{Analisi statica dei documenti}
L'analisi statica dei documenti è stata fatta mediante \termine{Walkthrough}, essa ha portato all'individuazione soltanto di errori di battitura o comunque non di errori gravi, causati da idee concettualmente sbagliate.

\subsection{Metriche per i documenti}

\subsubsection{Indice di Gulpease}

\begin{table}[h]
	\begin{center}
		\begin{tabular}{|c|c|c|c|c|}
			\hline
			\textbf{Documento}	& \textbf{Risultato} & \textbf{Esito} & \textbf{Valore}\\
			\hline
		    \termine{Analisi} dei Requisiti v3.0.0 &  & Superato & Ottimale\\
			\hline
			Glossario v4.0.0 &  & Superato & Ottimale\\
			\hline
			Norme di Progetto v4.0.0 &  & Superato & Accettabile\\
			\hline
			Piano di Progetto v4.0.0 &  & Superato & Accettabile\\
			\hline
			Piano di Qualifica v4.0.0 &  & Superato & Accettabile\\
			\hline
		 \termine{Manuale Utente} \termine{Monolith} v2.0.0 &  & Superato & Ottimale\\
			\hline
		 \termine{Manuale Utente} Bringit v2.0.0 &  & Superato & Accettabile\\
            \hline
		\end{tabular}
	\end{center}
	\caption{RA - Risultato indice di Gulpease}
\end{table}


\subsubsection{Soddisfacimento metriche}
\small{
Si noti che alcune componenti e i test a loro correlati non sono ancora state sviluppate, e che la loro assenza peserà nelle misurazioni.}

\paragraph{Qualità di processo}
\begin{longtable}{|>{\centering}m{5cm}|c|c|c|c|c|}
\hline
\textbf{Metrica} & \textbf{Unità di misura} & \textbf{Risultato} & \textbf{Risultato massimo} & \textbf{Esito} & \textbf{Valore}\\
\hline
\endhead
\emph{Schedule Variance} & {Attività} & \textcolor{Green}{0} & / & Superato & Ottimale\\ \hline
\emph{Budget Variance} & {Euro} & \textcolor{Orange}{-18} & / & Non superato & /\\ \hline
\emph{Ottimalità delle misurazioni} & {Percentuale} & \textcolor{Green}{81\%} & / & Superato & Ottimale \\ \hline
\emph{Rischi non preventivati} & {Rischi} & \textcolor{Green}{0} & / & Superato & Ottimale\\ \hline
\emph{Requisiti obbligatori soddisfatti} & {Percentuale} & \textcolor{Green}{100\%} & / & Superato & Ottimale\\ \hline
\emph{Livello di instabilità-SDK} & {Percentuale} & / & \textcolor{Green}{55\%} & Superato & Ottimale\\ \hline
\emph{Livello di instabilità-Applicazione} & {Percentuale} & / & \textcolor{Green}{60\%} & Superato & Accettabile\\ \hline
\emph{Astrattezza-SDK} & {Percentuale} & \textcolor{Green}{20\%} & / & Superato & Ottimale\\ \hline
\emph{Astrattezza-Applicazione} & {Percentuale} &\textcolor{Green}{15\%} & / & Superato & Ottimale\\ \hline
\emph{Distanza dalla sequenza principale-SDK} & {Percentuale} & / & \textcolor{Green}{10\%} & Superato & Ottimale\\ \hline
\emph{Distanza dalla sequenza principale-Applicazione} & {Percentuale} & / & \textcolor{Green}{25\%} & Superato & Ottimale\\ \hline
\emph{Numero di metodi per classe (max)} & {Metodi} & / & \textcolor{Green}{12} & Superato & Accettabile\\ \hline
\emph{Numero di attributi per classe (max)} & {Attributi} & / & \textcolor{Green}{7} & Superato & Ottimale\\ \hline
\emph{Numero di parametri per metodo (max)} & {Parametri} & / & \textcolor{Green}{5} & Superato & Ottimale\\ \hline
\emph{Complessità Ciclomatica media} & {Cammini} & \textcolor{Green}{1.3} & / & Superato & Ottimale\\ \hline
\emph{Livelli di annidamento medi} & {Chiamate} & \textcolor{Green}{2.4} & / & Superato & Ottimale\\ \hline
\emph{Linee di codice per linee di commento - Monolith} & {Percentuale} & \textcolor{Green}{28.3\%} & / & Superato & Ottimale\\ \hline
\emph{Linee di codice per linee di commento - BringIt} & {Percentuale} & \textcolor{Green}{20.5\%} & / & Superato & Ottimale\\ \hline
\emph{Componenti integrate} & {Percentuale} & \textcolor{Green}{100\%} & / & Superato & Ottimale\\ \hline
\emph{Test di Unità eseguiti} & {Percentuale} & \textcolor{Green}{99\%} & / & Superato & Ottimale\\ \hline
\emph{Test di Integrazione eseguiti} & {Percentuale} & \textcolor{Green}{98\%} & / & Superato & Ottimale\\ \hline
\emph{Test di Sistema eseguiti} & {Percentuale} & \textcolor{Green}{94\%} & / & Superato & Ottimale\\ \hline
\emph{Test di \termine{Validazione} eseguiti} & {Percentuale} & \textcolor{Orange}{} & / & Non superato & / \\ \hline
\emph{Test superati} & {Percentuale} & \textcolor{Green}{100\%} & / & Superato & Ottimale \\ \hline
\emph{Branch Coverage} & {Percentuale} & \textcolor{Green}{86\%} & / & Superato & Accettabile\\ \hline
\emph{Statement Coverage} & {Percentuale} & \textcolor{Green}{80.2\%} & / & Superato & Accettabile\\ \hline
\emph{Accuratezza rispetto alle attese} & {Percentuale} & \textcolor{Green}{100} & / & Superato & Ottimale\\ \hline
\emph{Completezza dell’implementazione funzionale} & {Percentuale} & \textcolor{Green}{100\%} & / & Superato & Ottimale\\ \hline
\emph{Densità di failure} & {Percentuale} & \textcolor{Green}{0} & / & Superato & Ottimale\\ \hline
\emph{Blocco di operazioni non corrette} & {Percentuale} & \textcolor{Green}{100} & / & Superato & Ottimale\\ \hline
\emph{Tempo di risposta} & {Secondi} & / & \textcolor{Green}{1.0} & Superato & Ottimale\\ \hline
\emph{Capacità di analisi di failure} & {Percentuale} & \textcolor{Green}{100} & / & Superato & Ottimale\\ \hline
\emph{Impatto delle modifiche} & {Indice} & \textcolor{Green}{2} & / & Superato & Ottimale\\ \hline
\caption{RA-Metriche di qualità di processo}\\
\end{longtable}

\subsubsection{Esiti delle metriche ripetute nel tempo}

\begin{figure}[H]
	\centering 
	\includegraphics[scale=0.75]{Sezioni/Immagini/ScheduleVariance-RA}
	\caption{Schedule variance - RA}
\end{figure}

\begin{figure}[H]
	\centering 
	\includegraphics[scale=0.75]{Sezioni/Immagini/BudgetVariance-RA}
	\caption{Budget variance - RA}
\end{figure}

\begin{figure}[H]
	\centering 
	\includegraphics[scale=0.75]{Sezioni/Immagini/LivelloInstabilitaApp-RA}
	\caption{Livello di instabilità Bringit - RA}
\end{figure}

\begin{figure}[H]
	\centering 
	\includegraphics[scale=0.65]{Sezioni/Immagini/DistanzaApp-RA}
	\caption{Distanza dalla sequenza principale Bringit - RA}
\end{figure}

\begin{figure}[H]
	\centering 
	\includegraphics[scale=0.6]{Sezioni/Immagini/LineeCodiceCommentoApp-RA}
	\caption{Linee di codice per linee di commento Bringit - RA}
\end{figure}

\subsubsection{Livello dei processi}
\begin{longtable}{|>{\centering}m{6cm}|c|c|c|c|c|}
\hline
\textbf{Processo} & \textbf{Livello} & \textbf{Esito} & \textbf{Valore}\\
\hline
\endhead
\emph{Processo di fornitura} & \textcolor{Green}{2} & Superato & Ottimale\\ \hline
\emph{Processo di sviluppo} & \textcolor{Green}{3} & Superato & Ottimale\\ \hline
\emph{Processo di documentazione} & \textcolor{Green}{2} & Superato & Accettabile\\ 
\hline
\emph{Processo di Configurazione} & \textcolor{Green}{1} & Superato & Ottimale\\ 
\hline
\emph{Processo di garanzia di qualità del Prodotto} & \textcolor{Green}{1} & Superato & Ottimale\\ 
\hline
\emph{Processo di Verifica} & \textcolor{Green}{2} & Superato & Ottimale\\ 
\hline
\emph{Processo di Validazione} & \textcolor{Green}{2} & Superato & Ottimale\\ 
\hline
\emph{Processo di Risoluzione dei problemi} & \textcolor{Green}{1} & Superato & Ottimale\\ 
\hline
\emph{Processo di Coordinamento} & \textcolor{Green}{1} & Superato & Ottimale\\ 
\hline
\emph{Processo di Pianificazione} & \textcolor{Green}{1} & Superato & Ottimale\\ 
\hline
\emph{Processo di Formazione} & \textcolor{Green}{1} & Superato & Ottimale\\ 
\hline
\caption{RA-Livello dei processi}
\end{longtable}



\end{appendices}
\end{document}