\newpage
\section{Introduzione}
\subsection{Scopo del documento}
Questo documento ha lo scopo di definire nel dettaglio la struttura di \capitolato. Tale documento fornisce una struttura dettagliata e completa che viene utilizzata dai \ProgrP per le attività di \termine{codifica}.
\subsection{Scopo del Prodotto}
\scopoProdotto

\subsection{Glossario}
\descrizioneGlossario

\subsection{Riferimenti}
\subsubsection{Normativi}
\riferimentiNormativi
\begin{itemize}
\item \textbf{\AdR}: \analisiDeiRequisiti
\end{itemize}

\subsubsection{Informativi}
\begin{itemize}
\item \textbf{Raccomandazioni HTML5}: \link{https://www.w3.org/TR/html5/}
\item \textbf{Sass: Syntactically Awesome Style Sheets}: \link{http://sass-lang.com/guide}
\item \textbf{Documentazione JavaScript}: \link{http://developer.mozilla.org/en-US/docs/Web/JavaScript/Reference}
\item \textbf{Specifiche ECMAScript 6} \link{http://es6-features.org}
\item \textbf{Documentazione Rocket.chat}: \link{https://rocket.chat/docs/}
\item \textbf{Documentazione Meteor.js}: \link{https://docs.meteor.com/}
\item \textbf{Documentazione Atmosphere.js} :\link{https://github.com/Atmosphere/atmosphere/wiki}
\item \textbf{Documentazione npm}: \link{https://docs.npmjs.com/}
\item \textbf{Documentazione della libreria Vue.js}: \link{https://vuejs.org/}
\item \textbf{Documentazione Babel.js}: \link{https://babeljs.io/learn-es2015/}
\item \textbf{Documentazione dependency-injection-es6:} \link{https://www.npmjs.com/package/dependency-injection-es6}
\item \textbf{Javascript Lint}: \link{http://javascriptlint.com/docs/index.html}
\end{itemize}

\newpage

