\newpage
\section{Qualità di prodotto}
Per garantire una buona qualità di prodotto, il \termine{team} ha individuato le qualità che ritiene più importanti nell'arco del ciclo di vita del prodotto e le ha istanziate individuando obiettivi e metriche coerenti con i livelli di qualità perseguiti. Per la descrizione delle metriche utilizzate e dei metodi utilizzati dal gruppo si rimanda al documento \NdP.

\subsection{Funzionalità}
Rappresenta la capacità del prodotto di fornire tutte le funzioni che sono state individuate attraverso l'\AdR.

\subsubsection{Obiettivi di qualità}
Il \termine{team} si impegnerà affinché:
\begin{itemize}
\item \textbf{Adeguatezza}: le funzionalità fornite siano conformi rispetto le aspettative.
\item \textbf{Accuratezza}: il prodotto fornisca i risultati attesi, con il livello di dettaglio richiesto.
\item \textbf{Sicurezza}: il prodotto protegga le informazioni e i dati da accessi e modifiche non autorizzati.
\end{itemize}

\paragraph{Completezza dell'implementazione funzionale}
\begin{itemize}
\item \textbf{Misurazione}: $C=(1-\frac{N_{FM}}{N_{FI}}) \cdot 100$, dove $N_{FM}$ è il numero di funzionalità mancanti nell'implementazione e $N_{FI}$ è il numero di funzionalità individuate nell'attività di analisi.
\end{itemize}

\begin{center}

		\begin{tabular}{|P{2.5cm}|P{2.5cm}|P{6cm}|}
		\hline
			\textbf{Valore di accettazione}	& \textbf{Valore ottimale} & \textbf{Motivazione} \\
			\hline
			$100$ & $100$ & Garantire una buona copertura dell'implementazione di tutte le funzionalità richieste. \\
			\hline
			\end{tabular}
\captionof{table}{Completezza dell'implementazione funzionale}
\end{center}

\paragraph{Accuratezza rispetto alle attese}

\textbf{Misurazione}: 
		$$A=\left(1-\mathlarger{\frac{N_{RD}}{N_{TE}}}\right) \cdot 100$$
	dove $N_{RD}$ è il numero di test che producono risultati discordanti rispetto alle attese e $N_{TE}$ è il numero di test-case eseguiti.
	
\begin{center}
	\begin{tabular}{|P{2.5cm}|P{2.5cm}|P{6cm}|}
		\hline
			\textbf{Valore di accettazione}	& \textbf{Valore ottimale} & \textbf{Motivazione} \\
			\hline
			$90 - 100$ & $100$ & Garantire una buona, se non ottima, corrispondenza tra i risultati e i valori attesi preventivati. \\
			\hline
			\end{tabular}
\captionof{table}{Accuratezza rispetto alle attese}
\end{center}	

\paragraph{Controllo degli accessi}

\begin{itemize}
\item \textbf{Misurazione}: $I=\frac{N_{IE}}{N_{II}} \cdot 100$, dove $N_{IE}$ è il numero di operazioni illegali effettuabili dai test e $N_{II}$ è il numero di operazioni illegali individuate;
\end{itemize}

\begin{center}
		\begin{tabular}{|P{2.5cm}|P{2.5cm}|P{6cm}|}
		\hline
			\textbf{Valore di accettazione}	& \textbf{Valore ottimale} & \textbf{Motivazione} \\
			\hline
			0 -- 10 & 0 &	Garantire che tutte le operazioni definite illegali dal gruppo siano bloccate almeno in gran parte. \\
			\hline
			\end{tabular}
\captionof{table}{Controllo degli accessi}
\end{center}

\subsection{Affidabilità}
Rappresenta la capacità del prodotto software di svolgere correttamente le sue funzioni durante il suo utilizzo, anche nel caso in cui si presentino situazioni anomale.

\subsubsection{Obiettivi di qualità}
L'esecuzione del prodotto dovrà presentare le seguenti caratteristiche:
\begin{itemize}
\item \textbf{Maturità}: dovrà essere evitato che si verifichino malfunzionamenti, operazioni illegali e restituzione di risultati errati (\textit{failure}) in seguito a difetti;
\item \textbf{Tolleranza agli errori}: nel caso in cui si presentino degli errori, dovuti a guasti o ad un uso scorretto dell'applicativo, questi dovranno essere gestiti in modo da mantenere alto il livello di prestazione.
\end{itemize}

\subsubsection{Metriche}

\paragraph{Densità di \textit{failure}}
Indica la percentuale di operazioni di testing che si sono concluse in fallimenti.

\begin{itemize}
	\item \textbf{Misurazione}: 
		$$F=\mathlarger{\frac{N_{FR}}{N_{TE}}} \cdot 100$$
	dove $N_{FR}$ è il numero di fallimenti rilevati durante l'attività di testing e $N_{TE}$ è il numero di test-case eseguiti.
\end{itemize}

\begin{center}
		\begin{tabular}{|P{2.5cm}|P{2.5cm}|P{6cm}|}
		\hline
			\textbf{Valore di accettazione}	& \textbf{Valore ottimale} & \textbf{Motivazione} \\
			\hline
			0 & 0 -- 10 &	Garantire il minor numero di fallimenti possibile. \\
			\hline
			\end{tabular}
\captionof{table}{Densità di \textit{failure}}
\end{center}

\paragraph{Blocco di operazioni non corrette}

\begin{itemize}
	\item \textbf{Misurazione}: 
		$$B=\mathlarger{\frac{N_{FE}}{N_{ON}}} \cdot 100$$
	dove $N_{FE}$ è il numero di \textit{failure} evitati durante i test effettuati e $N_{ON}$ è il numero di test-case eseguiti che prevedono l'esecuzione di operazioni non corrette, causa di possibili \textit{failure}.
\end{itemize}

\begin{center}
		\begin{tabular}{|P{2.5cm}|P{2.5cm}|P{6cm}|}
		\hline
			\textbf{Valore di accettazione}	& \textbf{Valore ottimale} & \textbf{Motivazione} \\
			\hline
			80 -- 100 & 100 &	Garantire che tutte le operazioni definite illegali dal gruppo siano bloccate almeno in gran parte. \\
			\hline
			\end{tabular}
\captionof{table}{Controllo degli accessi}
\end{center}


\subsection{Usabilità}
Rappresenta la capacità del prodotto di essere facilmente comprensibile e attraente in ogni sua parte per qualsiasi utente che lo andrà ad utilizzare.

\subsubsection{Obiettivi di qualità}
Il prodotto dovrà puntare ai seguenti obiettivi di usabilità:
\begin{itemize}
\item \textbf{Comprensibilità)}: l'utente dovrà essere in grado di riconoscerne le funzionalità offerte dal software e dovrà comprenderne le modalità di utilizzo per riuscire a raggiungere i risultati attesi;
\item \textbf{Apprendibilità}: dovrà essere data la possibilità all'utente di imparare ad utilizzare l'applicazione senza troppo impegno;
\item \textbf{Operabilità}: le funzionalità presenti dovranno essere coerenti con le aspettative dell'utente.
\end{itemize}

\subsubsection{Metriche}
\paragraph{Comprensibilità delle funzioni offerte}

\begin{itemize}
	\item \textbf{Misurazione}: 
		$$C=\mathlarger{\frac{N_{FC}}{N_{FO}}} \cdot 100$$
	dove $N_{FC}$ è il numero di funzionalità comprese in modo immediato dall'utente durante l'attività di testing del prodotto e $N_{FO}$ è il numero di funzionalità offerte dal sistema.
\end{itemize}

\begin{center}
		\begin{tabular}{|P{2.5cm}|P{2.5cm}|P{6cm}|}
		\hline
			\textbf{Valore di accettazione}	& \textbf{Valore ottimale} & \textbf{Motivazione} \\
			\hline
			80 -- 100 & 90 -- 100 &	Garantire una buona garanzia che l'utente comprenda tutte le funzioni offerte dal programma. \\
			\hline
			\end{tabular}
\captionof{table}{Comprensibilità delle funzioni offerte}
\end{center}


\paragraph{Facilità di apprendimento delle funzionalità}

\begin{itemize}
	\item \textbf{Misurazione}: indicatore numerico, espresso in minuti, che tiene traccia del tempo medio impiegato dall'utente nell'apprendere il corretto utilizzo di una funzionalità offerta dal sistema.

\end{itemize}

\begin{center}
		\begin{tabular}{|P{2.5cm}|P{2.5cm}|P{6cm}|}
		\hline
			\textbf{Valore di accettazione}	& \textbf{Valore ottimale} & \textbf{Motivazione} \\
			\hline
			0 -- 30 & 0 -- 15 &	Garantire che l'utente apprenda le funzionalità principali del software in un lasso di tempo non eccessivo. \\
			\hline
			\end{tabular}
\captionof{table}{Facilità di apprendimento delle funzionalità}
\end{center}


\paragraph{Consistenza operazionale in uso}
Indica la percentuale di messaggi e funzionalità offerte all'utente che rispettano le sue aspettative riguardo al comportamento del software.
\begin{itemize}
	\item \textbf{Misurazione}: 
		$$C=\left(1-\mathlarger{\frac{N_{MFI}}{N_{MFO}}}\right) \cdot 100$$
	dove $N_{MFI}$ è il numero di messaggi e funzionalità che non rispettano le aspettative dell'utente e $N_{MFO}$ è il numero di messaggi e funzionalità offerti dal sistema.
\end{itemize}

\begin{center}
		\begin{tabular}{|P{2.5cm}|P{2.5cm}|P{6cm}|}
		\hline
			\textbf{Valore di accettazione}	& \textbf{Valore ottimale} & \textbf{Motivazione} \\
			\hline
			80 -- 100 & 90 -- 100 &	Garantire che le aspettative dell'utente non siano deluse dal comportamento del software. \\
			\hline
			\end{tabular}
\captionof{table}{Consistenza operazionale in uso}
\end{center}


\subsection{Efficienza}
\label{efficienza}
Rappresenta la capacità di eseguire le funzionalità offerte dal software nel minor tempo possibile utilizzando al tempo stesso il minor numero di risorse possibili.

\subsubsection{Obiettivi di qualità}
Il prodotto dovrà essere efficiente, in particolare:
\begin{itemize}
\item \textbf{Comportamento rispetto al tempo}:  per svolgere le sue funzioni il software dovrà fornire adeguati tempi di risposta ed elaborazione;
\item \textbf{Utilizzo delle risorse}: il software quando eseguirà le sue funzionalità dovrà utilizzare un appropriato numero e tipo di risorse.
\end{itemize}

\subsubsection{Metriche}
\paragraph{Tempo di risposta}
Indica il periodo temporale medio che intercorre fra la richiesta al software di una determinata funzionalità e la restituzione del risultato all'utente.
\begin{itemize}
	\item \textbf{Misurazione}: 
		$$T_{RISP} = \mathlarger{\frac{\sum_{i=1}^{n} T_{i}}{n}}$$ 
	con $T_{RISP}$ misurato in secondi, e dove $T_{i}$ è il tempo intercorso fra la richiesta $i$ di una funzionalità ed il completamento delle operazioni necessarie a restituire un risultato a tale richiesta.
	\item \textbf{Range ottimale}: .
	\item \textbf{Range di accettazione}: .
\end{itemize}

\begin{center}
		\begin{tabular}{|P{2.5cm}|P{2.5cm}|P{6cm}|}
		\hline
			\textbf{Valore di accettazione}	& \textbf{Valore ottimale} & \textbf{Motivazione} \\
			\hline
			0 -- 8 & 0 -- 3 &	Garantire che l'utente non rimanga disorientato o deluso dal tempo di risposta del software. \\
			\hline
			\end{tabular}
\captionof{table}{Consistenza operazionale in uso}
\end{center}

\subsection{Manutenibilità}
Rappresenta la capacità del prodotto di essere modificato, tramite correzioni, miglioramenti o adattamenti del software a cambiamenti negli ambienti, nei requisiti e nelle specifiche funzionali.

\subsubsection{Obiettivi di qualità}
Le operazioni di manutenzione andranno agevolate il più possibile adottando le seguenti caratteristiche:
\begin{itemize}
\item \textbf{Analizzabilità}: il software dovrà consentire una rapida identificazione delle possibili cause di errori e malfunzionamenti;
\item \textbf{Modificabilità}: il prodotto originale dovrà permettere eventuali cambiamenti in alcune sue parti;
\item \textbf{Stabilità}: non dovranno insorgere effetti indesiderati in seguito a modifiche effettuate sul software;
\item \textbf{Testabilità}: il software dovrà poter essere facilmente testato per validare le modifiche effettuate.
\end{itemize}

\subsubsection{Metriche}
\paragraph{Capacità di analisi di \textit{failure}}

\begin{itemize}
	\item \textbf{Misurazione}: 
		$$I=\mathlarger{\frac{N_{FI}}{N_{FR}}} \cdot 100$$
	dove $N_{FI}$ è il numero di \textit{failure} delle quali sono state individuate le cause e $N_{FR}$ è il numero di \textit{failure} rilevate.
\end{itemize}

\begin{center}
		\begin{tabular}{|P{2.5cm}|P{2.5cm}|P{6cm}|}
		\hline
			\textbf{Valore di accettazione}	& \textbf{Valore ottimale} & \textbf{Motivazione} \\
			\hline
			60 -- 100 & 80 -- 100 &	Garantire che per la maggior parte delle \textit{failure} venga individuata e registrata una causa. \\
			\hline
			\end{tabular}
\captionof{table}{Capacità di analisi di \textit{failure}}
\end{center}

\paragraph{Impatto delle modifiche}

\begin{itemize}
	\item \textbf{Misurazione}: 
		$$I=\mathlarger{\frac{N_{FRF}}{N_{FR}}} \cdot 100$$
	dove $N_{FRF}$ è il numero di \textit{failure} risolte con l'introduzione di nuove \textit{failure} e $N_{FR}$ è il numero di \textit{failure} risolte;
\end{itemize}

\begin{center}
		\begin{tabular}{|P{2.5cm}|P{2.5cm}|P{6cm}|}
		\hline
			\textbf{Valore di accettazione}	& \textbf{Valore ottimale} & \textbf{Motivazione} \\
			\hline
			0 -- 20 & 0 -- 10 &	Garantire che l'introduzione di eventuali modifiche non introduca nuove failure. \\
			\hline
			\end{tabular}
\captionof{table}{Impatto delle modifiche}
\end{center}

