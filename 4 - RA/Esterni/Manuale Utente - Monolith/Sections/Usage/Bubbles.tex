\subsection{Bubbles}
\subsubsection{MarkdownBubble}
\begin{lstlisting}[language=JavaScript]
// Create the bubble
let markdownBubble = new Monolith.Bubble.MarkdownBubble("Optional text here");

// Change the bubble text
markdownBubble.setText("Text with **markdown** __inside__");

// Render the bubble
markdownBubble.renderView();
\end{lstlisting}
~\\~\\

\subsubsection{AlertBubble}
\begin{lstlisting}[language=JavaScript]
// Create the bubble
let alertBubble = new Monolith.Bubble.AlertBubble;

// Set the alert title
alertBubble.setTitle("Warning");

// Set the alert message
alertBubble.setMessage("Please check your data");

// Build the view
alertBubble.renderView();
\end{lstlisting}

\newpage
\subsubsection{ToDoListBubble}
\begin{lstlisting}[language=JavaScript]
// Create the ToDoListBubble
let toDoListBubble = new Monolith.Bubble.ToDoListBubble;

// --- GENERIC OPERATIONS ---

// Get the id of the bubble
let id = toDoListBubble.getId();

// Set the text associated with the bubble
toDoListBubble.setText("This bubble contains a lot of items that can be checked");

// Set the color of the text
toDoListBubble.setTextColor("#1A5418");

// Tell the bubble to format the text and so support the markdown notation
toDoListBubble.setFormatText(true);

// Disable text formatting
toDoListBubble.setFormatText(false);

// Set the highlight color for URLs
toDoListBubble.setUrlHighlightColor("#891C15");

// Set the bubble's text size in pixel
toDoListBubble.setTextSize(15);

// Set the message to show upon completion 
toDoListBubble.setCompletionMessage("The list has all been checked.")


// --- ITEMS OPERATIONS ---

// Add an item to the list
toDoListBubble.addItem("First item");

// You can specify a second parameter which tells if the item that is being added
// should be initially checked or not
toDoListBubble.addItem("Second item", true); 

// Set the text of an item
toDoListBubble.setItemText("New text", 0); // Changes the text of the first item

// Check an item
toDoListBubble.setChecked(true, 0); // Checks the first item

// Remove an item specifying the index of that item. Indexes start from 0
toDoListBubble.removeItem(1);


// Tell the bubble to use selection marks when ticking the options
toDoListBubble.setUseSelectionMark(true);

// Set the character to use when ticking an option
toDoListBubble.setSelectionCharacter("X");

// Tell the bubble to color the check box when ticking the options
toDoListBubble.setUseSelectionMark(false);

// Set the color used to color the check box
toDoListBubble.setSelectionColor("#1D2565");

// Set the function to call when clicking an item
// This function will be called after the click of any of the items.
// The parameter indicates the view of the item that has been clicked
toDoListBubble.setOnItemClick(function(item){
    item.setText("New text after click");
});

// Set the function to call when long-clicking an item
// This function will be called after the long click of any of the items.
// The parameter indicates the view of the item that has been long clicked
toDoListBubble.setOnItemClick(function(item){
    item.setText("New text after long click");
});

\end{lstlisting}

\newpage
\subsubsection{Create a custom bubble}
In order to create a new bubble type you have to do as follows.

\begin{enumerate}
	\item Create a new class which extends from `Monolith.Bubble.BaseBubble`.
\begin{lstlisting}[language=JavaScript]
export class CustomBubble extends Monolith.Bubble.BaseBubble {
    
    constructor(params){
        super(); // Always remember to call super!
        
        // Do something with the params
        // Setup the bubble
    }
    
    customOperation(){
        // Perform a custom operation
    }
    
    renderView(){
        super.renderView(); // Again, necessary call
        
        // Return a DOMElement object
    }
    
}
\end{lstlisting}

	\item Put the following code wherever you want. Thi can be done even inside the same file that contains the class definition, outside the definition itself.
\begin{lstlisting}[language=JavaScript]
Monolith.Bubble.addBubble("key", function(message){
    // Create the bubble
    let customBubble = new CustomBubble(params);
    
    // Perform the operations you want
    customBubble.customOperation();
    
    // Return the setup bubble
    return customBubble;
});
\end{lstlisting}
Please note that this function takes two parameters:
	\begin{enumerate}
		\item A \texttt{string} which defines the unique key that identifies your custom bubble.   \\
   		A good naming conventions for this key would be using the \url{reverse domain name notation}{https://en.wikipedia.org/wiki/Reverse_domain_name_notation} (e.g. \texttt{com.mycompany.bubble.custom}) which allows you to identify your custom bubble type among all the other bubbles types.
		\item A \texttt{function} which takes as parameter a \texttt{message} object that identifies a \url{Rocket.chat message}{https://rocket.chat/docs/developer-guides/realtime-api/the-message-object/}.  \\
   This function is the one that creates the bubble, taking data from the parameter object, performs operation on it if there's the need, and then return it.
   \end{enumerate}
\end{enumerate}   
   
\textbf{Note} \\
All of the written above should be made \textbf{only} inside the \texttt{client} directory of your Meteor project, otherwise it will make your application crash.

\newpage