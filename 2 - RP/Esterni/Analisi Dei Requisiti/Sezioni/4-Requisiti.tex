\newpage
\section{Requisiti}
Per poter svolgere al meglio le fasi di \PA\ e \PD\ dovrà essere stilato un elenco di requisiti emersi durante le riunioni interne e/o esterne. Tale compito spetta agli \textit{\AnP}. I requisiti dovranno
essere classificati secondo la seguente codifica:

\begin{center}
R-[Importanza][Tipo][Identificativo]
\end{center}
\begin{itemize}
	\item \textbf{Importanza:} può assumere questi valori:
  		\begin{itemize}
    		\item \textbf{1:} indica un requisito obbligatorio;
    		\item \textbf{2:} indica un requisito desiderabile;
    		\item \textbf{3:} indica un requisito facoltativo.
  		\end{itemize}
  	\item \textbf{Tipo:} può assumere questi valori:
  		\begin{itemize}
   		 	\item \textbf{F:} indica un requisito funzionale;
    		\item \textbf{Q:} indica un requisito di qualità;
    		\item \textbf{P:} indica un requisito prestazionale;
    		\item \textbf{V:} indica un requisito di vincolo.
  		\end{itemize}
  	\item \textbf{Identificativo:} indica il codice identificativo del requisito, è univoco e deve essere indicato in forma gerarchica.
\end{itemize}
Per ogni requisito si dovranno inoltre indicare:
\begin{itemize}
  \item \textbf{Descrizione:} una breve descrizione del requisito, che chiarisca tutti i punti di esso senza lasciare spazio a possibili ambiguità;
  \item \textbf{Fonte:} la fonte può essere una delle seguenti:
  \begin{itemize}
    \item \textit{\termine{Capitolato}}: deriva direttamente dal testo del capitolato;
    \item \textit{Verbale}: deriva da un incontro verbalizzato, seguito dall'identificativo dell'incontro;
    \item \textit{Interno}: deriva da discussioni interne al \termine{team};
    \item \textit{Caso d'uso}: deriva da uno o più casi d'uso e viene indicato tramite l'identificativo del caso o dei casi d'uso interessati.
  \end{itemize}
\end{itemize}

\input{Sezioni/TabelleRequisiti/RequisitiFunzionali.tex}
\newpage
\subsection{Requisiti Qualità}
\normalsize
\begingroup
\renewcommand\arraystretch{2}
\begin{longtable}{|c|>{\centering}m{7cm}|c|}
\hline
\textbf{Id Requisito} & \textbf{Descrizione} & \textbf{Fonti}\\
\hline
\endhead
			R1Q1 & Deve essere fornito un manuale utente & \termine{Capitolato} \\
			\hline
			R1Q2 & Il manuale utente deve essere fornito in lingua inglese & \termine{Capitolato} \\
			\hline
			R1Q3 & La documentazione utile per l'utilizzatore finale, non contenuta nel manuale utente, dovrà essere scritta in inglese e allegata al manuale  & \termine{Capitolato} \\ 
			\hline
			R1Q4 & La documentazione formale e standard (i documenti che bisogna consegnare ad ogni revisione di progetto) deve essere scritta in italiano  & Interno, Verbale di riunione del 23/12/2016\\ 
			\hline
			R1Q5 & Il manuale utente deve contenere una parte dove vengono specificati e spiegati  i vari errori che si possono incontrare quando si usa l'\termine{SDK} sviluppato & Interno \\ 
			\hline
			R2Q6 & Il manuale utente deve spiegare in maniera semplice e concisa come installare \progettoShort\ in un server contenete \termine{Rocket.chat} & Interno \\ 
			\hline
			R1Q6 & L'applicazione sviluppata come demo deve essere documentata & \termine{Capitolato} \\ 
			\hline
\caption[Requisiti di Qualità]{Requisiti di Qualità}
\label{tabella: Requisiti Funzionali}
\end{longtable}
\endgroup
\clearpage
\newpage
\subsection{Requisiti Vincolo}
\normalsize
\begingroup
\renewcommand\arraystretch{2}
\begin{longtable}{|c|>{\centering}m{7cm}|c|}
\hline
\textbf{Id Requisito} & \textbf{Descrizione} & \textbf{Fonti}\\
\hline
\endhead
			R1V1 & Deve essere creato un \termine{SDK} per permettere agli sviluppatori di creare nuove bolle & \termine{Capitolato} \\
			\hline
			R1V2 & Le bolle interattive create tramite \progettoShort\ devono funzionare dentro una istanza del \termine{server} \termine{web chat} \termine{Rocket.chat} & \termine{Capitolato}   \\
			\hline
			R2V3 & Durante lo sviluppo deve essere disponibile un server sul quale vi è installata una istanza di \termine{Rocket.chat}  & Interno \\ 
			\hline
			R2V4 & Le bolle create devono ricadere in una delle seguenti tre tipologie:  \textit{Rich media bubble}, \textit{Self-updating bubble} o \textit{Editing bubble} & Capitolato\\ 
			\hline
			R1V5 & \progettoShort\ deve essere in grado di fornire bolle predefinite già pronte ad essere utilizzate da parte dell'utente finale  & \termine{Capitolato} \\ 
			\hline
			R1V6 & \progettoShort\ deve provvedere delle \termine{API} per permettere agli sviluppatori futuri di creare delle nuove bolle secondo i loro requisiti  & \termine{Capitolato} \\ 
			\hline
			R1V7 & Dovrà essere creata una demo significativa che mostri un interessante uso delle bolle all'interno di \termine{Rocket.chat} & \termine{Capitolato} \\ 
			\hline
			R1V8 & L'\termine{SDK} e la demo dovranno essere sviluppati in \termine{javascript} 6th edition usando il "\termine{promise centric approach}" ed evitando il più possibile le \termine{callback}
			 & \termine{Capitolato} \\ 
			\hline
			R1V9 & Le \termine{callback} qualora venissero usate devono essere giustificate in maniera corretta
			 & \termine{Capitolato} \\ 
			 \hline
			 R1V10 & Per scrivere la demo e l'\termine{SDK} bisogna seguire il più possibile le \termine{12 Factors app guidelines}
			 & \termine{Capitolato} \\ 
			\hline
			R2V11 & E' consigliato l'uso di un \termine{framework frontend}
			 & \termine{Capitolato} \\ 
			\hline
			R2V12 & E' consigliato l'uso di \termine{SCSS}
			 & \termine{Capitolato} \\ 
			\hline
			R1V12 & La demo dovrà essere usufruibile tramite \termine{Heroku} sotto forma di pacchetto indipendente
			 & \termine{Capitolato} \\ 
			\hline
			R1V13 & Il codice sorgente di \progettoShort\ dovrà essere caricato e versionato su \termine{GitHub} o  \termine{Bitbucket}
			 & \termine{Capitolato} \\ 
			\hline
\caption[Requisiti di Vincolo]{Requisiti di Vincolo}
\label{tabella: Requisiti di Vincolo}
\end{longtable}
\endgroup
\clearpage
\subsection{Tracciamento Fonti-Requisiti}
\normalsize
\begin{longtable}{|>{\centering}m{5cm}|m{5cm}<{\centering}|}
\hline 
\textbf{Fonte} & \textbf{Id Requisiti}\\
\hline
\endhead
{Capitolato}&{R1F0}\\
&{R1F4}\\
&{R1Q1}\\
&{R1Q2}\\
&{R1Q3}\\
&{R1Q6}\\
&{R1V1}\\
&{R2V3}\\
&{R1V4}\\
&{R1V5}\\
&{R1V6}\\
&{R1V7}\\
&{R1V8}\\
&{R2V9}\\
&{R2V10}\\
&{R1V11}\\
&{R1V12}\\ \hline
{Interno} & {R1F1}\\
&{R1F1.1}\\
&{R1F1.2}\\
&{R1F1.3}\\
&{R1F1.4}\\
&{R1F1.5}\\
&{R2F1.1.1}\\
&{R1F1.1.2}\\
&{R2F1.1.3}\\
&{R1F1.1.4}\\
&{R1F1.1.5}\\
&{R1F1.1.6}\\
&{R2F1.1.7}\\
&{R1F1.1.8}\\
&{R1F1.1.9}\\
&{R1F1.1.10}\\
&{R1F1.1.11}\\

&{R1F1.2.1}\\
&{R1F1.2.2}\\
&{R1F1.2.3}\\
&{R1F1.2.4}\\

&{R1F1.3.1}\\
&{R1F1.3.2}\\
&{R1F1.3.3}\\
&{R3F1.3.1.1}\\
&{R1F1.3.1.2}\\
&{R2F1.3.1.3}\\
&{R1F1.3.1.4}\\
&{R1F1.3.1.5}\\
&{R1F1.3.1.6}\\
&{R2F1.3.2.3.1}\\
&{R1F1.3.2.3.2}\\
&{R2F1.3.2.3.3}\\
&{R3F1.3.2.3.4}\\
&{R2F1.3.2.3.5}\\
&{R1F1.3.3.1}\\
&{R2F1.3.3.2}\\
&{R3F1.3.3.3}\\
&{R1F1.3.3.4}\\

&{R1F1.4.1}\\
&{R1F1.4.2}\\
&{R1F1.4.3}\\
&{R3F1.4.1.1}\\
&{R1F1.4.1.2}\\
&{R2F1.4.1.3}\\

&{R1F1.5.1}\\
&{R1F1.5.2}\\
&{R1F1.5.3}\\
&{R1F1.5.1.1}\\
&{R1F1.5.1.2}\\
&{R1F1.5.1.3}\\
&{R1F1.5.1.4}\\
&{R1F1.5.1.5}\\
&{R2F1.5.1.6}\\
&{R1F1.5.1.7}\\
&{R1F1.5.1.8}\\
&{R1F1.5.1.9}\\
&{R1F1.5.1.10}\\
&{R1F1.5.1.11}\\

&{R1F2}\\
&{R1F2.1}\\
&{R1F2.2}\\
&{R1F2.3}\\

&{R1F3}\\

&{R1Q4}\\
&{R1Q5}\\
&{R2Q6}\\ 

&{R2V2}\\
&{R1V13}\\

&{R1F4}\\
&{R1F4.1}\\
&{R1F4.1.1}\\
&{R1F4.1.2}\\
&{R1F4.1.3}\\
&{R1F4.1.4}\\
&{R1F4.1.5}\\
&{R1F4.2.1}\\
&{R1F4.2.2}\\
&{R1F4.2.3}\\
&{R1F4.2.3.1}\\
&{R1F4.2.3.2}\\
&{R1F4.2.3.3}\\
&{R1F4.2.3.4}\\
&{R1F4.2.3.5}\\
&{R1F4.2.3.6}\\
&{R1F4.2.3.7}\\
&{R1F4.2.3.8}\\
&{R1F4.2.3.9}\\
&{R1F4.2.4}\\
&{R1F4.2.5}\\
&{R1F4.2.5.1}\\
&{R1F4.2.5.2}\\
&{R1F4.2.5.3}\\
&{R1F4.2.5.4}\\
&{R1F4.2.5.5}\\
&{R1F4.2.5.6}\\
&{R1F4.2.5.7}\\
&{R1F4.2.5.8}\\
&{R1F4.2.5.9}\\
&{R1F4.3.1.1}\\
&{R1F4.3.1.2}\\
&{R1F4.3.1.3}\\
&{R1F4.3.1.4}\\
&{R1F4.3.1.5}\\
&{R1F4.3.1.6}\\
&{R1F4.3.2}\\ \hline

{Verbale 2016-12-23}&{R1Q4}\\ \hline
{Verbale\_2\_E\_2017-02-24}&{R1V13}\\ \hline
{UC1}&{R1F1}\\
&{R1F1.1}\\
&{R1F1.2}\\
&{R1F1.3}\\
&{R1F1.4}\\
&{R1F1.5}\\ \hline
{UC1.1}&{R1F1.1}\\
&{R2F1.1.1}\\
&{R1F1.1.2}\\
&{R2F1.1.3}\\
&{R1F1.1.4}\\
&{R1F1.1.5}\\
&{R1F1.1.6}\\
&{R2F1.1.7}\\
&{R1F1.1.8}\\
&{R1F1.1.9}\\
&{R1F1.1.10}\\
&{R1F1.1.11}\\ \hline
{UC1.2}&{R1F1.2}\\
&{R1F1.2.1}\\
&{R1F1.2.2}\\
&{R2F1.2.3}\\
&{R1F1.2.4}\\ \hline
{UC1.3}&{R1F1.3}\\
&{R1F1.3.1}\\
&{R1F1.3.2}\\
&{R1F1.3.3}\\
&{R1F1.3}\\ \hline
{UC1.4}&{R1F1.4}\\
&{R1F1.4.1}\\
&{R1F1.4.2}\\ 
&{R1F1.4.3}\\ \hline
{UC1.5}&{R1F1.5}\\ \hline
{UC1.6}&{R1F1.6}\\
&{R1F1.1}\\
&{R1F1.2}\\
&{R1F1.3}\\
&{R1F1.4}\\
&{R1F1.5}\\ \hline
{UC1.1.1}&{R2F1.1.1}\\
&{R1F1.5.1.3}\\ \hline
{UC1.1.2}&{R1F1.1.2}\\
&{R1F1.5.1.4}\\ \hline
{UC1.1.3}&{R2F1.1.3}\\
&{R1F1.5.1.1}\\ \hline
{UC1.1.4}&{R1F1.1.4}\\
&{R1F1.5.1.5}\\ \hline
{UC1.1.5}&{R1F1.1.5}\\
&{R2F1.5.1.6}\\ \hline
{UC1.1.6}&{R1F1.1.6}\\
&{R1F1.5.1.7}\\ \hline
{UC1.1.7}&{R2F1.1.7}\\
&{R1F1.5.1.2}\\ \hline
{UC1.1.8}&{R1F1.1.8}\\
&{R1F1.5.1.8}\\ \hline
{UC1.1.9}&{R1F1.1.9}\\
&{R1F1.5.1.9}\\ \hline
{UC1.1.10}&{R1F1.1.10}\\
&{R1F1.5.1}\\
&{R1F1.5.1.10}\\ \hline
{UC1.1.11}&{R1F1.1.11}\\
&{R1F1.5.1.11}\\ \hline

{UC1.2.1}&{R1F1.2.1}\\
&{R1F1.5.2}\\ \hline
{UC1.2.2}&{R1F1.2.2}\\
&{R1F1.5.3}\\ \hline
{UC1.2.3}&{R1F1.2.3}\\ \hline

{UC1.3.1}&{R1F1.3.1}\\
&{R1F1.3.1.1}\\
&{R1F1.3.1.2}\\
&{R2F1.3.1.3}\\
&{R1F1.3.1.4}\\
&{R3F1.3.1.5}\\
&{R1F1.3.1.6}\\ \hline
{UC1.3.2}&{R1F1.3.2}\\
&{R1F1.3.2.1}\\
&{R1F1.3.2.2}\\
&{R1F1.3.2.3}\\ \hline
{UC1.3.3}&{R1F1.3.3}\\
&{R1F1.3.3.1}\\
&{R2F1.3.2.2}\\
&{R1F1.3.2.3}\\
&{R3F1.3.2.4}\\ \hline
{UC1.3.1.1}&{R3F1.3.1.1}\\ \hline
{UC1.3.1.2}&{R1F1.3.1.2}\\ \hline
{UC1.3.1.3}&{R2F1.3.1.3}\\ \hline
{UC1.3.1.4}&{R1F1.3.1.4}\\ \hline
{UC1.3.1.5}&{R3F1.3.1.5}\\ \hline
{UC1.3.1.6}&{R1F1.3.1.6}\\ \hline
{UC1.3.2.1}&{R1F1.3.2.1}\\ \hline
{UC1.3.2.2}&{R1F1.3.2.2}\\ \hline
{UC1.3.2.3}&{R1F1.3.2.3}\\
&{R2F1.3.2.3.1}\\
&{R1F1.3.2.3.1}\\
&{R2F1.3.2.3.1}\\
&{R3F1.3.2.3.1}\\
&{R2F1.3.2.3.1}\\ \hline
{UC1.3.3.1}&{R1F1.3.3.1}\\ \hline
{UC1.3.3.2}&{R2F1.3.3.2}\\ \hline
{UC1.3.3.3}&{R1F1.3.3.3}\\ \hline
{UC1.3.3.4}&{R3F1.3.3.4}\\ \hline

{UC1.4.1}&{R1F1.4.1}\\
&{R3F1.4.1.1}\\
&{R1F1.4.1.2}\\
&{R1F1.4.1.3}\\ \hline
{UC1.4.2}&{R1F1.4.2}\\ \hline
{UC1.4.3}&{R2F1.4.3}\\ \hline
{UC1.4.1.1}&{R3F1.4.1.1}\\ \hline
{UC1.4.1.1}&{R3F1.4.1.2}\\ \hline
{UC1.4.1.1}&{R3F1.4.1.3}\\ \hline

{UC2}&{R1F2}\\
&{R1F2.1}\\
&{R1F2.2}\\
&{R1F2.3}\\ \hline
{UC2.1}&{R1F2.1}\\ \hline
{UC2.2}&{R1F2.2}\\ \hline
{UC2.3}&{R1F2.3}\\ \hline

{UC3}&{R1F3}\\ \hline

\caption[Tracciamento Fonti-Requisiti]{Tracciamento Fonti-Requisiti}
\label{tabella: Tracciamento Fonti-Requisiti}
\end{longtable}


\subsection{Tracciamento Fonti-Requisiti Demo}
\normalsize
\begin{longtable}{|>{\centering}m{5cm}|m{5cm}<{\centering}|}
\hline 
\textbf{Fonte} & \textbf{Id Requisiti}\\
\hline
\endhead

{UC4}&{R1F4}\\
&{R1F4.1}\\ \hline
{UC4.1}&{R1F4.1}\\
&{R1F4.1.1}\\
&{R1F4.1.2}\\
&{R1F4.1.3}\\
&{R1F4.1.4}\\
&{R1F4.1.5}\\
{UC4.1.1}&{R1F4.1.1}\\ \hline
{UC4.1.1}&{R1F4.1.2}\\ \hline
{UC4.1.1}&{R1F4.1.3}\\ \hline
{UC4.1.1}&{R1F4.1.4}\\ \hline
{UC4.1.1}&{R1F4.1.5}\\ \hline

{UC4.2}&{R1F4.2.1}\\
&{R1F4.2.2}\\
&{R1F4.2.3}\\
&{R1F4.2.4}\\
&{R1F4.2.5}\\
&{R1F4.2.3.1}\\
&{R1F4.2.3.2}\\
&{R1F4.2.3.3}\\
&{R1F4.2.3.4}\\
&{R1F4.2.3.5}\\
&{R1F4.2.3.6}\\
&{R1F4.2.3.7}\\
&{R1F4.2.3.8}\\
&{R1F4.2.3.9}\\ \hline
{UC4.2.1}&{R1F4.2.1}\\ \hline
{UC4.2.2}&{R1F4.2.2}\\ \hline
{UC4.2.3}&{R1F4.2.3}\\
&{R1F4.2.3.1}\\
&{R1F4.2.3.2}\\
&{R1F4.2.3.3}\\
&{R1F4.2.3.4}\\
&{R1F4.2.3.5}\\
&{R1F4.2.3.6}\\
&{R1F4.2.3.7}\\
&{R1F4.2.3.8}\\
&{R1F4.2.3.9}\\ \hline
{UC4.2.4}&{R1F4.2.4}\\ \hline
{UC4.2.5}&{R1F4.2.5}\\
&{R1F4.2.5.1}\\
&{R1F4.2.5.2}\\
&{R1F4.2.5.3}\\
&{R1F4.2.5.4}\\
&{R1F4.2.5.5}\\
&{R1F4.2.5.6}\\
&{R1F4.2.5.7}\\
&{R1F4.2.5.8}\\
&{R1F4.2.5.9}\\ \hline
{UC4.2.3.1}&{R1F4.2.3.1}\\ \hline
{UC4.2.3.2}&{R1F4.2.3.2}\\ \hline
{UC4.2.3.3}&{R1F4.2.3.3}\\ \hline
{UC4.2.3.4}&{R1F4.2.3.4}\\ \hline
{UC4.2.3.5}&{R1F4.2.3.5}\\ \hline
{UC4.2.3.6}&{R1F4.2.3.6}\\ \hline
{UC4.2.3.7}&{R1F4.2.3.7}\\ \hline
{UC4.2.3.8}&{R1F4.2.3.8}\\ \hline
{UC4.2.3.9}&{R1F4.2.3.9}\\ \hline

{UC4.2.5.1}&{R1F4.2.5.1}\\ \hline
{UC4.2.5.2}&{R1F4.2.5.2}\\ \hline
{UC4.2.5.3}&{R1F4.2.5.3}\\ \hline
{UC4.2.5.4}&{R1F4.2.5.4}\\ \hline
{UC4.2.5.5}&{R1F4.2.5.5}\\ \hline
{UC4.2.5.6}&{R1F4.2.5.6}\\ \hline
{UC4.2.5.7}&{R1F4.2.5.7}\\ \hline
{UC4.2.5.8}&{R1F4.2.5.8}\\ \hline
{UC4.2.5.9}&{R1F4.2.5.9}\\ \hline

{UC4.3}&{R1F4.3.2}\\
{R1F4.3.1.1}\\
&{R1F4.3.1.2}\\
&{R1F4.3.1.3}\\
&{R1F4.3.1.4}\\
&{R1F4.3.1.5}\\
&{R1F4.3.1.6}\\ \hline
{UC4.3.1}&{R1F4.3.1.1}\\
&{R1F4.3.1.2}\\
&{R1F4.3.1.3}\\
&{R1F4.3.1.4}\\
&{R1F4.3.1.5}\\
&{R1F4.3.1.6}\\ \hline
{UC4.3.1.1}&{R1F4.3.1.1}\\ \hline
{UC4.3.1.1}&{R1F4.3.1.2}\\ \hline
{UC4.3.1.1}&{R1F4.3.1.3}\\ \hline
{UC4.3.1.1}&{R1F4.3.1.4}\\ \hline
{UC4.3.1.1}&{R1F4.3.1.5}\\ \hline
{UC4.3.1.1}&{R1F4.3.1.6}\\ \hline
{UC4.3.2}&{R1F4.3.2}\\ \hline

\caption[Tracciamento Fonti-Requisiti Demo]{Tracciamento Fonti-Requisiti Demo}
\label{tabella: Tracciamento Fonti-Requisiti Demo}
\end{longtable}
\newpage
\subsection{Riepilogo Requisiti}
\normalsize
\begingroup
\renewcommand\arraystretch{2}
\begin{longtable}{|c|c|c|c|}
\hline 
\textbf{Tipo} & \textbf{Obbligatorio} & \textbf{Desiderabile} & \textbf{Facoltativo}\\
\hline
Funzionale & 101 & 13 & 6\\ \hline
Prestazionale & 0 & 0 & 0\\ \hline
Di Qualità & 6 & 1 & 0\\ \hline
Di Vincolo & 9 & 4  & 0\\ \hline
\caption[Riepilogo Requisiti]{Riepilogo Requisiti}
\label{tabella:riepilogorequi}
\end{longtable}
\endgroup
\clearpage

%metto qui un appunto cosi se dobbiamo metterci le mani si sanno
% 	 1	2	3
%SDK 59 13  4
%APP 42 0   2


