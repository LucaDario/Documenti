% Questo file definisce lo stile che verrà applicato
% ad ogni pagina di contenuto
\documentclass[a4paper,11pt]{article}

\usepackage{ifthen}
\usepackage[
 a4paper,
 top=2.5cm,
 bottom=2.5cm,
 left=1.5cm,
 right=1.5cm,
 head=30pt
]{geometry}
\usepackage[italian]{babel}
\usepackage[utf8x]{inputenc}
\usepackage[T1]{fontenc}
\usepackage{fancyhdr}
\usepackage[colorlinks=true, urlcolor=black, citecolor=black, linkcolor=black]{hyperref}
\usepackage{tabularx}
\usepackage{multirow}
\usepackage{booktabs}
\usepackage{color}
\usepackage{graphicx}
\usepackage{eurosym}
\usepackage{amsmath}
\usepackage{relsize}

\usepackage[multidot]{grffile}
\usepackage{xcolor,colortbl}
\definecolor{lightblue}{HTML}{56B4E6}
\definecolor{blue}{HTML}{2953A1}
\definecolor{darkblue}{HTML}{1E396E}

\usepackage[toc,page]{appendix}
\renewcommand\appendixtocname{Appendice}
\renewcommand{\appendixpagename}{Appendice}

\newcommand\pagenumberingnoreset[1]{\gdef\thepage{\csname @#1\endcsname\c@page}}

% Cambia il font 
\renewcommand*\rmdefault{qhv}

% ***STILE PAGINA***
\pagestyle{fancy}
\fancyhf{}
\setlength{\headheight}{1cm} 
% No indentazione paragrafo
\setlength{\parindent}{0pt}

% ***INTESTAZIONE***
\newcommand\textline[4][t]{%
  \noindent\parbox[#1]{.333\textwidth}{\raisebox{-0.40\height}{#2}}%
  \parbox[#1]{.333\textwidth}{\centering #3}%
  \parbox[#1]{.333\textwidth}{\raggedleft #4}%
}

\lhead{
	\textline[t]{\includegraphics[width=1cm, keepaspectratio=true]{../../../Template/Logo/Logo.png}}{\progettoShort}{\documento}
}

\renewcommand{\headrulewidth}{0.4pt}  %Linea sotto l'intestazione

% ***PIÈ DI PAGINA***
\lfoot{\textit{\gruppoLink}\\ \footnotesize{\email}}
\rfoot{\thepage} %per le prime pagine: mostra solo il numero romano
\cfoot{}
\renewcommand{\footrulewidth}{0.4pt}   %Linea sopra il piè di pagina


% Ridefinisce command \paragraph{} andando a capo ogni dopo la parola dentro le parentesi ed ha la possibiltà di enumerazione fino a n cifre modificando il numero dentro "secnumdepth"
\usepackage{titlesec}

\setcounter{secnumdepth}{7}
\setcounter{tocdepth}{7}
%
%
%\titleformat{\paragraph}
%{\normalfont\normalsize\bfseries}{\theparagraph}{1em}{}
%\titlespacing*{\paragraph}
%{0pt}{3.25ex plus 1ex minus .2ex}{1.5ex plus .2ex}
%
%
%\titleclass{\subsubparagraph}{straight}[\subparagraph]
%\newcounter{subsubparagraph}
%
%\titleformat{\subsubparagraph}[display]
%  {\normalfont\normalsize\bf}
%  {\thesubsubparagraph.}
%  {.5em}
%  {}
%\renewcommand\thesubsubparagraph\textbf{\roman{subsubparagraph}}
%\titlespacing*{\subsubparagraph} {0pt}{4pt}{6pt}


%***LA SOTTOSEZIONE PARAGRAPH VIENE VISUALIZZATA COME UNA SECTION
\titleformat{\paragraph}{\normalfont\normalsize\bfseries}{\theparagraph}{1em}{}
\titlespacing*{\paragraph}{0pt}{3.25ex plus 1ex minus .2ex}{1.5ex plus .2ex}

\titleformat{\subparagraph}{\normalfont\normalsize\bfseries}{\thesubparagraph}{1em}{}
\titlespacing*{\subparagraph}{0pt}{3.25ex plus 1ex minus .2ex}{1.5ex plus .2ex}

\makeatletter
\newcounter{subsubparagraph}[subparagraph]
\renewcommand\thesubsubparagraph{%
  \thesubparagraph.\@arabic\c@subsubparagraph}
\newcommand\subsubparagraph{%
  \@startsection{subsubparagraph}    % counter
    {6}                              % level
    {\parindent}                     % indent
    {3.25ex \@plus 1ex \@minus .2ex} % beforeskip
    {0.75em}                           % afterskip
    {\normalfont\normalsize\bfseries}}
\newcommand\l@subsubparagraph{\@dottedtocline{6}{13em}{5.5em}} %gestione dell'indice
\newcommand{\subsubparagraphmark}[1]{}
\makeatother

\makeatletter
\newcounter{subsubsubparagraph}[subsubparagraph]
\renewcommand\thesubsubsubparagraph{%
  \thesubsubparagraph.\@arabic\c@subsubsubparagraph}
\newcommand\subsubsubparagraph{%
  \@startsection{subsubsubparagraph}    % counter
    {7}                              % level
    {\parindent}                     % indent
    {3.25ex \@plus 1ex \@minus .2ex} % beforeskip
    {0.75em}                           % afterskip
    {\normalfont\normalsize\bfseries}}
\newcommand\l@subsubsubparagraph{\@dottedtocline{7}{16em}{6.5em}} %gestione dell'indice
\newcommand{\subsubsubparagraphmark}[1]{}
\makeatother

%Generali
\newcommand{\capitolato}{C5 - Monolith: An interactive bubble provider}
\newcommand{\progettoShort}{Monolith}
\newcommand{\progetto}{Monolith: An interactive bubble provider}
\newcommand{\gruppo}{NPE Developers}
\newcommand{\gruppoLink}{\href{https://gitlab.com/npe-developers}{NpeDevelopers}}
\newcommand{\email}{\href{mailto:npe.developers@gmail.com}{\textcolor{blue}{npe.developers@gmail.com}}}
\newcommand{\password}{NP3Devel0pers}
\newcommand{\myincludegraphics}[2][]{%
	\setbox0=\hbox{\phantom{X}}%
	\vtop{
		\hbox{\phantom{X}}
		\vskip-\ht0
		\hbox{\includegraphics[#1]{#2}}}
}




%Componenti del gruppo
\newcommand{\RM}{Riccardo Montagnin}
\newcommand{\MT}{Manuel Turetta}
\newcommand{\FB}{Francesco Bazzerla}
\newcommand{\SL}{Stefano Lia}
\newcommand{\LD}{Luca Dario}
\newcommand{\DC}{Diego Cavestro}
\newcommand{\ND}{Nicolò Dovico}

%Ruoli
\newcommand{\Pm}{Project Manager}
\newcommand{\Am}{Amministratore}
\newcommand{\AmP}{Amministratori}
\newcommand{\An}{Analista}
\newcommand{\AnP}{Analisti}
\newcommand{\Dev}{Sviluppatore}
\newcommand{\DevP}{Sviluppatori}
\newcommand{\Ver}{Verificatore}
\newcommand{\VerP}{Verificatori}
\newcommand{\Progr}{Programmatore}
\newcommand{\ProgrP}{Programmatori}
\newcommand{\Prog}{Progettista}
\newcommand{\ProgP}{Progettisti}



%Firme
\newcommand{\RMFirma}{\myincludegraphics[scale = 0.5]{../../../Template/Firme/RM.png}}
\newcommand{\MTFirma}{\myincludegraphics[scale = 0.5]{../../../Template/Firme/MT.png}}
\newcommand{\FBFirma}{\myincludegraphics[scale = 0.5]{../../../Template/Firme/FB.png}}
\newcommand{\SLFirma}{\myincludegraphics[scale = 0.5]{../../../Template/Firme/SL.png}}
\newcommand{\LDFirma}{\myincludegraphics[scale = 0.5]{../../../Template/Firme/LD.png}}
\newcommand{\DCFirma}{\myincludegraphics[scale = 0.5]{../../../Template/Firme/DC.png}}
\newcommand{\NDFirma}{\myincludegraphics[scale = 0.5]{../../../Template/Firme/ND.png}}

%Professori e proponente
\newcommand{\TV}{Prof. Tullio Vardanega}
\newcommand{\RC}{Prof. Riccardo Cardin}
\newcommand{\RB}{Red Babel}
\newcommand{\proponente}{Red Babel}

%Documenti
\newcommand{\Gl}{Glossario}
\newcommand{\glossario}{\textit{\Gl\_v.1.0.0.pdf}}
\newcommand{\AdR}{Analisi dei Requisiti}
\newcommand{\analisiDeiRequisiti}{\textit{\AdR\_v.1.0.0.pdf}}
\newcommand{\AdRvDue}{AnalisiDeiRequisiti}
\newcommand{\NdP}{Norme di Progetto}
\newcommand{\normeDiProgetto}{\textit{\NdP\_v.1.0.0.pdf}}
\newcommand{\PdP}{Piano di Progetto}
\newcommand{\pianoDiProgetto}{\textit{\PdP\_v.1.0.0.pdf}}
\newcommand{\SdF}{Studio di Fattibilità}
\newcommand{\studioDiFattibilita}{\textit{\SdF\_v.1.0.0.pdf}}
\newcommand{\PdQ}{Piano di Qualifica}
\newcommand{\pianoDiQualifica}{\textit{\PdQ\_v.1.0.0.pdf}}
\newcommand{\VI}{Verbale Interno}
\newcommand{\VE}{Verbale Esterno}
\newcommand{\ST}{Specifica Tecnica}
\newcommand{\MU}{Manuale Utente}
\newcommand{\DDP}{Definizione di Prodotto}

%Periodo di progetto
\newcommand{\ARM}{Analisi dei Requisiti di Massima}
\newcommand{\ARD}{Analisi dei Requisiti in Dettaglio}
\newcommand{\PA}{Progettazione Architetturale}
\newcommand{\PD}{Progettazione di Dettaglio}
\newcommand{\COD}{Codifica}
\newcommand{\VV}{Verifica e Validazione Finale}

%Consegne
\newcommand{\RR}{Revisione dei Requisiti}
\newcommand{\RP}{Revisione di Progettazione}
\newcommand{\RQ}{Revisione di Qualifica}
\newcommand{\RA}{Revisione di Accettazione}


%Formattazione
\newcommand{\termine}[1]{\textit{#1}\small{$_G$}}
\newcommand{\link}[1]{\href{#1}{\textcolor{blue}{\texttt{#1}}}} 

% Testi ricorrenti
\newcommand{\scopoProdotto}{L'obiettivo di questo progetto è la realizzazione di un \termine{SDK} che permetta la creazione di bolle interattive, le quali, successivamente, verranno utilizzate all'interno dell'applicazione di messaggistica istantanea open source \termine{Rocket.chat}. \\
Dopo la realizzazione di tale \termine{SDK}, è proposto lo sviluppo di un'applicazione in grado di sfruttare l'\termine{SDK} per implementare un uso originale di tali bolle.
}
\newcommand{\descrizioneGlossario}{Al fine di mantenere questo documento compatto e di facile lettura è stato realizzato un glossario esterno contenente tutte le definizioni dei termini che più comunemente verranno presentati al lettore.  
Tale glossario si ritrova all'interno del file \glossario, e contiene tutti e soli i termini che vengono marcati con una \textit{G} a pedice.
}
\newcommand{\riferimentiNormativi}{
	\begin{itemize}
		\item \textbf{Norme di Progetto}: \normeDiProgetto
		\item \textbf{\termine{Capitolato} d'appalto C5: Monolith - An Interactive bubble provider} \\
			  \link{http://www.math.unipd.it/~tullio/IS-1/2016/Progetto/C5.pdf}
	\end{itemize}
}

% Comandi per generare l'intro
\newcommand{\documento}{\DDP}
\newcommand{\versione}{1.0.6}
\newcommand{\redatori}{\DC\\ & \LD\\ & \FB\\ & \ND\\ & \SL\\}
\newcommand{\revisori}{\RM}
\newcommand{\dataApprovazione}{** marzo 2017}
\newcommand{\approvazione}{\MT}
\newcommand{\statoapprovazione}{Approvato}
\newcommand{\uso}{Esterno}
\newcommand{\destinatari}{\RB\\ & \TV\\ & \RC}

\newcommand{\sommario}{Questo documento descrive l'\termine{architettura di dettaglio} di \capitolato.
}
\usepackage{graphicx}
\usepackage{placeins}
\usepackage{ltablex}
\usepackage{float}
\usepackage{verbatim}


\newcommand{\modifiche}{
	Verifica sezioni 1 e 4 & \RM & \Prog & 06/03/2017 & 0.1.0 \\\midrule
	Stesura appendice 1 & \RM & \Prog & 03/03/2017 & 0.0.5 \\\midrule
	Stesura sezione 4 & \DC & \Prog & 28/02/2017 & 0.0.4 \\\midrule
	Stesura sezione 4 & \FB & \Prog & 28/02/2017 & 0.0.3 \\\midrule
	Stesura sezione 1 & \FB & \Prog & 28/02/2017 & 0.0.2 \\\midrule
    Creazione del template & \FB & \Prog & 28/02/2017 & 0.0.1 \\\midrule
}

\begin{document}

\input{../../../Template/Intro.tex}
%Questo file si occupa di generare la tabella delle modifiche
\pagenumbering{Roman}

\begin{center}
    \Large{\textbf{Registro delle modifiche}}
    	\\\vspace{0.5cm}
    	\normalsize
    \begin{tabularx}{\textwidth}{cXXcc}
        \textbf{Versione} & \textbf{Modifica - Motivazione} & \textbf{Autore} & \textbf{Ruolo} & \textbf{Data} \\\toprule
        \modifiche
    \end{tabularx}
\end{center}

\newpage



\tableofcontents

\newpage

\setcounter{table}{0}
\listoftables

\newpage

\listoffigures

\newpage

\pagenumbering{arabic}


\section{Introduzione}
\subsection{Scopo del documento}
Questo documento vuole definire le strategie che il \termine{team} ha deciso di adottare per perseguire gli obiettivi di qualità di processo e di prodotto ricercati. A tal fine è necessaria una costante attività di verifica e validazione del lavoro svolto in modo da poter rilevare e correggere le anomalie che potrebbero nascere.

\subsection{Scopo del prodotto}
\scopoProdotto

\subsection{Glossario}
\descrizioneGlossario

\subsection{Riferimenti}
\subsubsection{Normativi}
\riferimentiNormativi

\subsubsection{Informativi}
\begin{itemize}
	\item \textbf{\AdR}: \analisiDeiRequisiti;
	\item \textbf{\PdP}: \pianoDiProgetto;
	\item \textbf{\textit{Slide} dell'insegnamento di Ingegneria del Software}: \\
		  \link{http://www.math.unipd.it/~tullio/IS-1/2016/}
	\item \textbf{\textit{Standard} ISO/IEC 9126}: Product quality \\
	 	  \link{https://en.wikipedia.org/wiki/ISO/IEC\_9126}
	\item \textbf{\textit{Standard} tecnici ISO/IEC 15504}: Software process assessment \\
		  \link{https://en.wikipedia.org/wiki/ISO/IEC\_15504}
	\item \textbf{Ciclo di Deming (\termine{PDCA})}: Miglioramento dei processi \\
		  \link{https://en.wikipedia.org/wiki/PDCA}
\end{itemize}

\newpage
\newpage
\section{Standard di Progetto}
\subsection{Documentazione del codice}
Gli standard per la scrittura della documentazione del codice sono definiti nelle \normeDiProgetto.

\subsection{Denominazione entità e relazioni}
Tutti gli elementi definiti, siano essi package, classi, metodi o attributi, devono avere denominazioni chiare ed autoesplicative. \\
Nel caso in cui il nome risulti essere lungo è preferibile anteporre la chiarezza alla lunghezza.
Sono ammesse abbreviazioni se:
\begin{itemize}
	\item immediatamente comprensibili;
	\item non ambigue;
	\item sufficientemente contestualizzate
\end{itemize}
Le regole tipografiche relativi ai nomi delle entità sono definiti nelle \normeDiProgetto.

\subsection{Codifica}
Gli standard di programmazione sono definiti e descritti nelle  \normeDiProgetto.

\subsection{Strumenti di lavoro}
Gli strumenti da adottare e le procedure da seguire per utilizzarli correttamente durante la realizzazione del prodotto software sono definiti nelle  \normeDiProgetto.
\newpage
\section{Tecnologie utilizzate}

\subsection{Rocket.Chat}
\termine{Rocket.Chat} è una Web Chat Server, sviluppata in JavaScript, utilizzando il \termine{framework} Meteor. Lo scopo principale del capitolato è quello di sviluppare un'applicazione integrabile con il \termine{package manager} \termine{Atmosphere} nel progetto Meteor \termine{Rocket.Chat}. L'applicazione, denominata \textit{Bring-It}, userà l'\termine{SDK} integrata nello stesso metodo. Una volta compilato il tutto, il risultato sarà la versione di \termine{Rocket.Chat} del gruppo \gruppo\ che avrà incluso le funzionalità previste, sottoforma di pacchetti.

\subsection{Meteor.js}
Meteor è una piattaforma Javascript per lo sviluppo di applicazioni web e mobile. Essa viene utilizzata come ambiente di sviluppo per l'applicazione e per l'\termine{SDK} che verranno integrate come pacchetti stand-alone in \termine{Rocket.Chat}.
Meteor permette un'ottima gestione delle componenti, permettendo lo sviluppo della parte server, della parte client e della parte per la comunicazione tra esse in un unico progetto. Per fare ciò Meteor include un set di tecnologie da utilizzare, tra le quali: Node.js, MongoDB e tutte quelle disponibili all'interno del \termine{package manager} \termine{Atmosphere}.

\subsection{MongoDB}
MongoDB è un \termine{DBMS} non relazionale, orientato ai documenti. Classificato come un database di tipo NoSQL, si allontana dalla
struttura tradizionale basata su tabelle dei database relazionali in favore di documenti in stile
JSON con schema dinamico. MongoDB è utilizzato come \termine{storage} da \termine{Rocket.Chat}, così come per qualsiasi progetto meteor. Dunque, il suo utilizzo è derivato principalmente per questo motivo.

\subsection{Node.js}
Per lo sviluppo della parte Back-End dell’applicazione si è deciso di utilizzare la piattaforma \termine{event-driven} Node.js, basata sul motore JavaScript V8. Esso permette di realizzare applicazioni
Web utilizzando il linguaggio JavaScript, tipicamente client-side, per
la scrittura server-side. 
Node.js è essenziale per lo sviluppo di Monolith e per la compilazione di \termine{Rocket.Chat}. Infatti l'installazione di Meteor comporta anche l'installazione di Node.js essenziale per il corretto funzionamento dei progetti Meteor. Il principale vantaggio dell'utilizzo di tale tecnologia è che permette di accedere alle risorse del sistema operativo in modalità \termine{event-driven} non sfruttando il modello basato sui thread concorrenti, utilizzato dai classici web servers.

\subsection{Atmosphere e npm}
\textit{Atmosphere} e \textit{npm} sono dei \termine{packages managers} di Meteor e serve per aggiungere pacchetti \termine{stand-alone} all'interno di un generico progetto, come appunto \termine{Monolith} e relativa applicazione. Viene tenuta traccia dei pacchetti integrati nel progetto tramite la directory \textit{packages/} che si crea all'interno del progetto.

\subsection{Can.js}
\textit{Can.js} è una collezione di \termine{librerie} lato client per lo sviluppo di applicazioni web. In particolare il gruppo ha utilizzato la \termine{libreria} \textit{can-stache} utilizzata per costruire dei \textit{template} HTML ed aggiornarli in modo semplice e garantire un \termine{binding} dinamico.

\subsection{Bootstrap}
\textit{Bootstrap} è un \termine{framework} per la gestione di HTML, CSS e JavaScript. Viene utilizzato per semplicità di utilizzo e ridurre il tempo necessario per l'implementazione e l'integrazione dei file CSS, HTML e CSS dell'applicazione. Inoltre, tale \termine{framework} è già integrato dentro \termine{Rocket.Chat} e viene, dunque, utilizzato anche per retrocompatibilità.

\subsection{jQuery}
\textit{jQuery} è una libreria JavaScript. Essa permette di modificare facilmente il file HTML basandosi sul \termine{paradigma DOM}. Esso viene utilizzato nel caso di operazioni semplici (come una semplice animazione) oppure per implementare o gestire funzionalità che non si possono fare tramite i \termine{frameworks} scelti.

\subsection{Ecmascript 6}
\textit{Ecmascript 6} è un linguaggio di programmazione standardizzato e mantenuto da Ecma International nell'ECMA-262 ed ISO/IEC 16262. Le implementazioni più conosciute di questo linguaggio (spesso definite come dialetti) sono JavaScript, JScript e ActionScript che sono entrati largamente in uso, inizialmente, come linguaggi client-side nel web development. L versione 6 implementa significanti cambiamenti sintattici per scrivere applicazioni più complesse, incluse le classi e i moduli. Ciò ci permette di creare una struttura ben definita e di poter utilizzarla per implementare una struttura modulare.

\subsection{Dependency-injection-es6}
\textit{Dependency-injection-es6} è una \termine{libreria} per \textit{Node.js} e l'ambiente \textit{JavaScript} dove \textit{EcmaScript 6} è supportato. Grazie a quest'ultima caratteristica, tale \termine{libreria} è stata scelta dal team per implementare il \termine{design-pattern} della \termine{dependency-injection}.

\subsection{Es6-event-emitter}
\textit{Es6-event-emitter} è una \termine{libreria} che serve per emettere e catturare eventi, generati dopo particolari azioni.

\subsection{Sass}
\textit{Sass} è preprocessore CSS. Esso serve a definire fogli di stile con una forma più semplice, completa e potente rispetto ai \termine{CSS} e a generare file \termine{CSS} ottimizzati, aggregando le strutture definite anche in modo complesso. Esso è utilizzato per generare alcuni fogli di stile particolarmente complicati e per semplificare la loro gestione.

\subsection{Marked}
\textit{Marked} è una \termine{libreria} che trasforma del testo che rispetta la sintassi \termine{Markdown} in \termine{HTML}. Tale \termine{libreria} viene utilizzata per la generazione dell'\termine{HTML} del widget Markdown per semplificarne la gestione. 

\subsection{BlazeJS}
\textit{BlazeJS} è una potente \termine{libreria} per la creazione di interfacce utente tramite template \termine{HTML}. Essa elimina l'esigenza di aggiornare la parte logica dell'applicazione che è in ascolto dei cambiamenti e di eventuali manipolazioni del \termine{DOM}.
\newpage
\section{Architettura}

\subsection{Metodo e Formalismo}
Nell’esposizione dell’architettura dell’applicazione si procederà con un approccio \termine{top-down} descrivendo l’architettura dal generale al particolare. Si procederà quindi alla descrizione dei
packages e dei componenti, per poi descrivere nel dettaglio le singole classi, specificando per
ognuna il tipo, l’obiettivo, la funzione, relazioni in ingresso e uscita e i metodi e attributi contenuti. Successivamente si procederà ad elencare i diagrammi di sequenza che descriveranno il funzionamento dell'\termine{SDK} e dell'applicazione. 


\subsection{Informazioni Generali}
Il principio di progettazione che abbiamo addottato è il Common Closure Principale per i seguenti motivi:
\begin{itemize}
\item Suddivisione più logica dei vari package e classi.
\item Aumenta la manutenibilità del codice nel tempo.
\end{itemize}

\subsection{Prodotto software}
Il prodotto che il gruppo \gruppo\ si impone a fare è composto da due parti: \termine{SDK} e Applicazione. Per cui verranno prima illustrate i packages dell'\termine{SDK} e poi quelli relativi all'applicazione.

%\begin{figure}[H]
%	\centering
%	\includegraphics[scale=0.4]{Sezioni/Packages/App/pck_application.png}
%	\caption{Package application}
%\end{figure}


\subsection{SDK}

\subsubsection{Creazione di un widget immagine}

\label{Creazione di un widget immagine}
\begin{figure}[ht]
	\centering
	\includegraphics[width=16cm, height=14cm]{Sezioni/Diagrammi/SDK/Creazione di un widget immagine.png}
	\caption{Creazione di un widget immagine}
\end{figure}

Lo sviluppatore può creare un widget di tipo immagine per aggiungerlo ad una sua ipotetica \termine{bolla}. Durante la costruzione, come si vede dallo schema, vengono invocati tre metodi. Il primo per impostare l'immagine e gli altri due per impostare rispettivamente la larghezza ed altezza di essa. Il tipo delle immagini supportate sono le stesse supportate da \termine{Rocket.Chat} ovvero:
\begin{itemize}
\item .jpeg
\item .gif
\item .png
\item .jpg
\end{itemize}
Se l'immagine inserita non appartiene ad uno di questi formati oppure non viene inserita dallo sviluppatore il metodo \texttt{setText} di \texttt{TextWidgetPresenter} lancerà un'eccezione di tipo \texttt{BadParameterException}. \\
Le frecce di ritorno dall'\texttt{ImageWidgetPresenter} all'\texttt{ImageWidget} sono state inserite poiché il cambiamento dei dati sul \termine{Presenter} ha effetto anche sulla View. \\
Si noti, infine, che i metodi invocati da \texttt{ImageWidget} vengono chiamati in quest'ordine dal suo costruttore senza parametri. Tali metodi possono anche essere chiamati singolarmente dallo sviluppatore secondo l'ordine che egli preferisce. Queste azioni non verranno ulteriormente descritte poiché ritenute ridondanti.

\newpage

\subsubsection{Creazione di un widget di testo}

\label{Creazione di un widget di testo}
\begin{figure}[H]
	\centering
	\includegraphics[width=16cm, height=14cm]{Sezioni/Diagrammi/SDK/Creazione di un widget di testo.png}
	\caption{Creazione di un widget di testo}
\end{figure}

Lo sviluppatore può creare un widget di tipo testo per aggiungerlo ad una sua ipotetica \termine{bolla}. Affinché non si verifichino errori il testo inserito nel widget deve essere valido, ovvero non deve contenere caratteri speciali se non quelli supportati dal tipo di codifica UTF-8 e non deve essere del testo vuoto. Se ciò dovesse capitare il metodo \texttt{setText} di \texttt{TextWidgetPresenter} lancerà un'eccezione di tipo \texttt{BadParameterException}. \\
Le frecce di ritorno dal \texttt{TextWidgetPresenter}  al \texttt{TextWidget} sono state inserite poiché il cambiamento dei dati sul \termine{Presenter} ha effetto anche sulla View. \\
Si noti, infine, che i metodi invocati da \texttt{TextWidget} vengono chiamati in quest'ordine dal suo costruttore senza parametri. Tali metodi possono anche essere chiamati singolarmente dallo sviluppatore secondo l'ordine che egli preferisce. Queste azioni non verranno ulteriormente descritte poiché ritenute ridondanti.

\newpage

\subsubsection{Creazione di un widget ChecklistItem}

\label{Creazione di un widget ChecklistItem}
\begin{figure}[H]
	\centering
	\includegraphics[width=16cm, height=14cm]{Sezioni/Diagrammi/SDK/Creazione di un widget ChecklistItem.png}
	\caption{Creazione di un widget ChecklistItem}
\end{figure}

Lo sviluppatore può creare un widget di tipo ChecklistItem per aggiungerlo ad una sua ipotetica \termine{bolla}. Al momento dell'invocazione del costruttore di ChecklistItemWidget verrà quindi creato un nuovo oggetto ChecklistItemWidget, che da il via alla costruzione di ChecklistItemWidgetPresenter, il quale a sua volta si occuperà della creazione di CheckOption e CheckStyle.\\
Una volta terminata l'operazione qui sopra descritta, verrà richiamato, da parte del presenter, il metodo privato \texttt{\_createOption} il quale terminerà la creazione degli elementi necessari alla visualizzazione del widget.
\newpage

\subsubsection{Creazione di un widget bottone}

\label{Creazione di un widget bottone}
\begin{figure}[H]
	\centering
	\includegraphics[width=16cm, height=14cm]{Sezioni/Diagrammi/SDK/Creazione di un widget bottone.png}
	\caption{Creazione di un widget bottone}
\end{figure}

Lo sviluppatore può creare un widget di tipo bottone per aggiungerlo ad una sua ipotetica \termine{bolla}. Se, durante la creazione del widget, il testo non dovesse essere impostato correttamente, il metodo \texttt{setText} di \texttt{ButtonWidgetPresenter} lancerà un'eccezione di tipo \texttt{BadParameterException}. \\
Le frecce di ritorno da \texttt{ButtonWidgetPresenter}  a \texttt{ButtonWidget} sono state inserite poiché il cambiamento dei dati sul \termine{Presenter} ha effetto anche sulla View. \\
Si noti, infine, che i metodi invocati da \texttt{ButtonWidget} vengono chiamati in quest'ordine dal suo costruttore senza parametri. Tali metodi possono anche essere chiamati singolarmente dallo sviluppatore secondo l'ordine che egli preferisce. Queste azioni non verranno ulteriormente descritte poiché ritenute ridondanti.

\newpage

\subsubsection{Creazione di un widget lista}

\label{Creazione di un widget lista}
\begin{figure}[H]
	\centering
	\includegraphics[width=16cm, height=14cm]{Sezioni/Diagrammi/SDK/Creazione di un listWidget.png}
	\caption{Creazione di un widget lista}
\end{figure}

Lo sviluppatore può creare un widget di tipo lista per aggiungerlo ad una sua ipotetica \termine{bolla}. Ogni elemento aggiunto logicamente dal presenter agisce anche sulla view. \\
L'azione compiuta per creare il widget può essere anche effettuata a posteriori della creazione del widget, questa però, essendo molto simile a quella appena descritta, viene omessa per ridondanza.

\newpage

\begin{comment}
\subsubsection{Creazione di una bolla aggiungendo un widget ChecklistItem}

\label{Creazione di una bolla aggiungendo un widget ChecklistItem}
\begin{figure}[H]
	\centering
	\includegraphics[width=16cm, height=14cm]{Sezioni/Diagrammi/SDK/Creazione di una bolla aggiungendo un widget checklist.png}
	\caption{Creazione di una bolla aggiungendo un widget ChecklistItem}
\end{figure}

Lo sviluppatore può aggiungere un widget alla \termine{bolla} appena creata. L'aggiunta dell'elemento avviene tramite il metodo \texttt{addComponent} che permette l'aggiunta sia di un layout che di un widget. Si noti che l'esempio è fatto con un widget specifico, ovvero il widget ChecklistItem, ma ciò vale per qualsiasi widget presente nell'\termine{SDK}. Gli altri esempi simili vengono, per questo motivo, omessi.

\newpage
\end{comment}

\subsubsection{Aggiunta ad una bolla di un Layout contenente due widget di testo}

\label{Aggiunta ad una bolla di un Layout contenente due widget di testo}
\begin{figure}[H]
	\centering
	\includegraphics[width=16cm, height=14cm]{Sezioni/Diagrammi/SDK/Aggiunta ad una bolla un Layout contenente due widget di testo.png}
	\caption{Aggiunta ad una bolla di un Layout contenente due widget di testo}
	
\end{figure}
Lo sviluppatore può aggiungere un layout contenente dei widgets alla \termine{bolla}. Anche questo rappresenta solo un generico esempio di come si possa aggiungere un layout ad una \termine{bolla}. Gli altri esempi simili saranno dunque omessi. 

\newpage
\newpage
\subsection{Applicazione}
Questi scenari rappresentano alcune operazioni comuni della lista Bringit. Si fa notare che alcune di queste non sono ancora state realizzate nel codice, e che alcune delle soluzioni trovate dal team sono state rese necessarie da Rocket.Chat, che non utilizzando un design pattern architetturale specifico ci ha messi nella difficoltà di sviluppare in un ambiente non regolato.
\subsubsection{Creazione di una lista}

\label{Creazione di una lista}
\begin{figure}[H]
	\centering
	\includegraphics[ width=\textwidth]{Sezioni/Diagrammi/App/creazionelista.png}
	\caption{Creazione di una lista}
\end{figure}

Questo scenario rappresenta l'utente che vuole creare una nuova lista nella chat di \termine{Rocket.chat}. Innanzitutto alla pressione del pulsante creazione lista verrà emesso un evento che verrà catturato dalla classe \textit{InputListInfoView} che incapsulerà i dati della lista e demanderà la gestione di essi al presenter. Quest'ultimo emetterà un evento di tipo \textit{'saveEvent'}, che verrà catturato da \textit{CreateListView}, che demanderà al presenter la creazione vera e propria della lista; quest'ultimo infatti si occuperà di contattare \textit{ChatSource} che farà visualizzare la lista su Rocket.Chat.



\subsubsection{Cancellazione di una lista}

\label{Cancellazione di una lista}
\begin{figure}[H]
	\centering
	\includegraphics[width=\textwidth]{Sezioni/Diagrammi/App/cancellarelist.png}
	\caption{Cancellazione di una lista}
\end{figure}

In questo scenario lo sviluppatore vuole cancellare la propria lista causandone l'eliminazione da tutte le chat che la contengono. L'utente quando premerà il bottone "Elimina lista" farà emettere un evento di tipo \textit{DeleteEvent}, che verrà catturato da \textit{DeleteListView}, che demanderà al presenter la gestione dell'eliminazione; quest'ultimo userà infatti \textit{ManageListUseCase} per comunicare a \textit{DataBaseSource} che la lista venga eliminata.

\subsubsection{Spunta di un oggetto della lista}

\label{Spunta di un oggetto della lista}
\begin{figure}[H]
	\centering
	\includegraphics[width=\textwidth]{Sezioni/Diagrammi/App/clickitem.png}
	\caption{Interaione con lista}
\end{figure}

In questo scenario l'utente vuole spuntare un oggetto per segnarlo come \textit{acquistato}.
Per farlo l'utente clicca sul segno del quadrato: questo scatena l'evento del widget \textit{ChecklistItemWidget}, che viene catturato dalla lista \textit{Bringit}, la quale cambia la visualizzazione della lista in \termine{Rocket.Chat} e attraverso \textit{ModifyListUseCase} fa si che la modifica di stato avvenuta venga applicata anche alla permanenza dei dati grazie a \textit{DatabaseSource}.


\subsubsection{Aggiunta di un Item}

\label{Aggiunta di un Item}
\begin{figure}[H]
	\centering
	\includegraphics[width=\textwidth]{Sezioni/Diagrammi/App/aggiuntaitem.png}
	\caption{Aggiunta di un Item}
\end{figure}

In questo scenario lo sviluppatore vuole aggiungere dei prodotti (chiamati item nella nostra applicazione) alla lista da esso creata. Oppure vuole aggiungerli a una lista nella quale ha permessi di modifica. L'utente premendo il pulsante \textit{aggiungi item} farà emettere un evento che verrà catturato dalla classe \textit{Bringit}, che aggiungerà il nuovo item a se stessa e attraverso l'uso di \textit{ModifyListUseCase} fa si che l'aggiunta dell'item avvenuta venga applicata anche alla permanenza dei dati grazie a \textit{DatabaseSource}.


\subsubsection{Modifica Item}

\label{Modifica Item}
\begin{figure}[H]
	\centering
	\includegraphics[width=\textwidth]{Sezioni/Diagrammi/App/modifica_item.jpg}
	\caption{Modifica Item}
\end{figure}

In questo scenario lo sviluppatore vuole modificare le informazioni di un item già presente nella lista. L'utente premendo il bottone per la modifica di un item attiverà l'emissione di un evento che verrà catturato dalla classe \textit{ModifyItemViewImpl} demandando la gestione al suo presenter. Quest'ultimo attraverso il suo metodo \textit{createViewForItemWithId} gli ritornerà l'HTML per la creazione della vista dedita alla modifica dell'item con i campi dati precompilati utilizzando le informazioni contenute nel \termine{database}. Successivamente richiamando il metodo \textit{showPopUp} l'utente visualizzerà a schermo le informazioni prima ottenute con la possibilità di modifica. Una volta modificate le informazioni desiderate l'utente premerà il pulsante di conferma modifica, verrà quindi emesso l'evento di salvataggio che verrà catturato dalla classe \textit{InputItemInfoViewImpl} e successivamente demandato a \textit{ModifyItemPresenter} che si occuperà di salvare le modifiche effettuate nel \termine{database}. Infine all'avvenuta modifica della lista nel \termine{database} verranno eseguiti i metodi nel presenter della bolla lista per aggiornare tutte le chat contenenti l'istanza della lista appena modificata. 

\subsubsection{Rimozione Item}

\label{Rimozione Item }
\begin{figure}[H]
	\centering
	\includegraphics[width=\textwidth]{Sezioni/Diagrammi/App/rimozioneitem.png}
	\caption{Rimozione Item}
	
\end{figure}
In questo scenario l'utente vuole eliminare un item dalla lista. Per farlo l'utente deve premere il pulsante rimuovi item attivando quindi un evento che verrà catturato dalla classe \textit{Bringit}, che analogamente a quanto fa per l'aggiunta di un item, rimuove la parte grafica dell'item che si è deciso di eliminare e utilizza \textit{ModifyListUseCase} e \textit{DatabaseSource} per salvare le modifiche effettuate alla lista Bringit.


\subsubsection{Pubblicazione a un contatto}

\label{Pubblicazione di un contatto}
\begin{figure}[H]
	\centering
	\includegraphics[width=\textwidth]{Sezioni/Diagrammi/App/publicazione_con_contatto.jpg}
	\caption{Pubblicazione di un contatto}
	
\end{figure}
In questo scenario l'utente vuole pubblicare la lista ad un altro contatto, dando il permesso di modifica e interazione con la lista. L'utente premendo il bottone \textit{pubblica lista} sceglierà un contatto, lanciando quindi un evento che verrà catturato dalla classe \textit{ShareWithContactViewImpl} e demandato al suo presenter che, tramite il metodo \textit{shareList}, attiverà i metodi utili per aggiungere i permessi necessari al interazione al utente scelto. Successivamente la classe \textit{ShareListUseCase} si occuperà di salvare le modifiche nel \termine{database} e pubblicare infine all'utente desiderato la lista tramite il metodo {sendMessageToChat}.


\subsubsection{Pubblicazione a un gruppo}

\label{Pubblicazione a un gruppo}
\begin{figure}[H]
	\centering
	\includegraphics[width=\textwidth]{Sezioni/Diagrammi/App/publicazionecongruppo.png}
	\caption{Pubblicazione a un gruppo}
	
\end{figure}

Simile allo scenario precedente in questo caso l'utente vuole condividere e dare i permessi ad utenti specifici presenti all'interno di un gruppo. L'utente premendo il bottone \textit{pubblica lista} sceglierà un gruppo, lanciando quindi un evento che verrà catturato dalla classe \textit{ShareWithGroupViewImpl} e demanderà la gestione al suo presenter. Quest'ultimo attraverso il metodo \textit{openShareWithGroup} inizializzerà un ciclo che porterà a scegliere per ogni utente presente nel gruppo la concessione o meno dei permessi. La procedura per svolgere il ruolo sopra descritto è simile allo scenario della \textit{pubblicazione con contatto} con la differenza che non verrà inviato un messaggio agli utenti con i permessi concessi. Infine terminato il ciclo verrà si passerà per \textit{ShareListUseCase}, che si occuperà di salvare le modifiche apportate alla lista nel database.

\subsubsection{Inoltro a un contatto}

\label{Inoltro a un contatto}
\begin{figure}[H]
	\centering
	\includegraphics[width=\textwidth]{Sezioni/Diagrammi/App/forward_contatto.jpg}
	\caption{Inoltro a un contatto}
	
\end{figure}

In questo scenario l'utente vuole inoltrare la lista ad un contatto come fosse un semplice messaggio di testo, senza aggiungere allo specifico contatto nessun permesso di modifica o interazione. L'utente premendo il bottone di inoltro sceglierà un contatto. Si verificherà quindi l'evento che verrà catturato dalla classe \textit{ForwardListViewImpl} demandandone la gestione al suo presenter.
Quest'ultimo con il metodo \textit{forwardToContactWithId} inoltrerà la lista al contatto desiderato attraverso la classe \textit{ChatSource}. Prima dell'avvenuto inoltro al contatto selezionato verrà chiamato il metodo \textit{showPopUp} che, interfacciandosi con la classe \textit{ForwardListUseCase}, mostrerà a schermo un messaggio di conferma per l'inoltro e la scelta per il contatto a cui inoltrare il messaggio.



\subsubsection{Inoltro a un gruppo}

\label{Inoltro a un contatto}
\begin{figure}[H]
	\centering
	\includegraphics[width=\textwidth]{Sezioni/Diagrammi/App/forward_con_gruppo.jpg}
	\caption{Inoltro a un contatto}
	
\end{figure}

In questo scenario l'utente vuole inoltrare la lista ad un gruppo come fosse un semplice messaggio di testo, senza aggiungere al specifico gruppo nessun permesso di modifica o interazione. L'utente premendo il bottone di inoltro sceglierà un gruppo. Si verificherà quindi l'evento che verrà catturato dalla classe \textit{ForwardListViewImpl} demandandone la gestione al suo presenter.
Quest'ultimo con il metodo \textit{forwardToGroupWithId} inoltrerà la lista al gruppo desiderato attraverso la classe \textit{ChatSource}. Prima dell'avvenuto inoltro al gruppo verrà chiamato il metodo \textit{showPopUp} che, interfacciandosi con la classe \textit{ForwardListUseCase}, mostrerà a schermo un messaggio di conferma per l'inoltro e la selezione del gruppo a cui si desidera inoltrare.


\subsubsection{Help}

\label{Help}
\begin{figure}[H]
	\centering
	\includegraphics[width=\textwidth]{Sezioni/Diagrammi/App/Help.jpg}
	\caption{Help}
	
\end{figure}

In questo scenario l'utente vuole chiedere delle informazioni d'aiuto per l'utilizzo della nostra applicazione. Per richiedere aiuto l'utente dovrà cliccare il bottone di help apposito. In questa maniera verrà attivato un metodo nella classe \textit{HelpViewImpl} che demanderà la gestione al suo presenter. Quest'ultimo attraverso il metodo \textit{show} farà visualizzare a schermo un popup con le informazioni utili. Per nascondere le informazioni d'aiuto avremo lo stesso flusso di utilizzo con il metodo \textit{hide} in sostituzione al metodo \textit{show} attivabile nella stessa maniera.


\newpage
\newpage
\section{Diagramma delle classi}

\subsection{SDK}

\subsection{Bubbles}
\subsubsection{MarkdownBubble}
\begin{lstlisting}[language=JavaScript]
// Create the bubble
let markdownBubble = new Monolith.Bubble.MarkdownBubble("Optional text here");

// Change the bubble text
markdownBubble.setText("Text with **markdown** __inside__");

// Render the bubble
markdownBubble.renderView();
\end{lstlisting}
~\\~\\

\subsubsection{AlertBubble}
\begin{lstlisting}[language=JavaScript]
// Create the bubble
let alertBubble = new Monolith.Bubble.AlertBubble;

// Set the alert title
alertBubble.setTitle("Warning");

// Set the alert message
alertBubble.setMessage("Please check your data");

// Build the view
alertBubble.renderView();
\end{lstlisting}

\newpage
\subsubsection{ToDoListBubble}
\begin{lstlisting}[language=JavaScript]
// Create the ToDoListBubble
let toDoListBubble = new Monolith.Bubble.ToDoListBubble;

// --- GENERIC OPERATIONS ---

// Get the id of the bubble
let id = toDoListBubble.getId();

// Set the text associated with the bubble
toDoListBubble.setText("This bubble contains a lot of items that can be checked");

// Set the color of the text
toDoListBubble.setTextColor("#1A5418");

// Tell the bubble to format the text and so support the markdown notation
toDoListBubble.setFormatText(true);

// Disable text formatting
toDoListBubble.setFormatText(false);

// Set the highlight color for URLs
toDoListBubble.setUrlHighlightColor("#891C15");

// Set the bubble's text size in pixel
toDoListBubble.setTextSize(15);

// Set the message to show upon completion 
toDoListBubble.setCompletionMessage("The list has all been checked.")


// --- ITEMS OPERATIONS ---

// Add an item to the list
toDoListBubble.addItem("First item");

// You can specify a second parameter which tells if the item that is being added
// should be initially checked or not
toDoListBubble.addItem("Second item", true); 

// Set the text of an item
toDoListBubble.setItemText("New text", 0); // Changes the text of the first item

// Check an item
toDoListBubble.setChecked(true, 0); // Checks the first item

// Remove an item specifying the index of that item. Indexes start from 0
toDoListBubble.removeItem(1);


// Tell the bubble to use selection marks when ticking the options
toDoListBubble.setUseSelectionMark(true);

// Set the character to use when ticking an option
toDoListBubble.setSelectionCharacter("X");

// Tell the bubble to color the check box when ticking the options
toDoListBubble.setUseSelectionMark(false);

// Set the color used to color the check box
toDoListBubble.setSelectionColor("#1D2565");

// Set the function to call when clicking an item
// This function will be called after the click of any of the items.
// The parameter indicates the view of the item that has been clicked
toDoListBubble.setOnItemClick(function(item){
    item.setText("New text after click");
});

// Set the function to call when long-clicking an item
// This function will be called after the long click of any of the items.
// The parameter indicates the view of the item that has been long clicked
toDoListBubble.setOnItemClick(function(item){
    item.setText("New text after long click");
});

\end{lstlisting}

\newpage
\subsubsection{Create a custom bubble}
In order to create a new bubble type you have to do as follows.

\begin{enumerate}
	\item Create a new class which extends from `Monolith.Bubble.BaseBubble`.
\begin{lstlisting}[language=JavaScript]
export class CustomBubble extends Monolith.Bubble.BaseBubble {
    
    constructor(params){
        super(); // Always remember to call super!
        
        // Do something with the params
        // Setup the bubble
    }
    
    customOperation(){
        // Perform a custom operation
    }
    
    renderView(){
        super.renderView(); // Again, necessary call
        
        // Return a DOMElement object
    }
    
}
\end{lstlisting}

	\item Put the following code wherever you want. Thi can be done even inside the same file that contains the class definition, outside the definition itself.
\begin{lstlisting}[language=JavaScript]
Monolith.Bubble.addBubble("key", function(message){
    // Create the bubble
    let customBubble = new CustomBubble(params);
    
    // Perform the operations you want
    customBubble.customOperation();
    
    // Return the setup bubble
    return customBubble;
});
\end{lstlisting}
Please note that this function takes two parameters:
	\begin{enumerate}
		\item A \texttt{string} which defines the unique key that identifies your custom bubble.   \\
   		A good naming conventions for this key would be using the \url{reverse domain name notation}{https://en.wikipedia.org/wiki/Reverse_domain_name_notation} (e.g. \texttt{com.mycompany.bubble.custom}) which allows you to identify your custom bubble type among all the other bubbles types.
		\item A \texttt{function} which takes as parameter a \texttt{message} object that identifies a \url{Rocket.chat message}{https://rocket.chat/docs/developer-guides/realtime-api/the-message-object/}.  \\
   This function is the one that creates the bubble, taking data from the parameter object, performs operation on it if there's the need, and then return it.
   \end{enumerate}
\end{enumerate}   
   
\textbf{Note} \\
All of the written above should be made \textbf{only} inside the \texttt{client} directory of your \termine{Meteor} project, otherwise it will make your application crash.

\newpage
\subsubsection{monolith::component::BaseComponent}

\label{monolith::component::BaseComponent}
\begin{figure}[ht]
	\centering
	\includegraphics[scale=0.5]{Sezioni/SottosezioniST/img/BaseComponent.png}
	\caption{monolith::component::BaseComponent}
\end{figure}

\begin{itemize}
\item \textbf{Descrizione:} Interfaccia base che rappresenta un qualsiasi oggetto che può essere inserito all'interno di un bolla.
\item \textbf{Utilizzo:} Interfaccia che viene implementata ogni qualvolta uno sviluppatore intende inserire all'interno di un bolla un widget o un layout.
\item \textbf{Attributi:}
\item \textbf{Metodi:}
\begin{itemize}
\item \textit{public renderView():string}\\
Genera il codice HTML, CSS e JavaScript necessario per visualizzare BaseComponent.
\end{itemize}
\end{itemize}
\subsubsection{monolith::client::events::ChecklistCompleteEmitter}

\label{monolith::client::events::ChecklistCompleteEmitter}
\begin{figure}[H]
	\centering
	\includegraphics[scale=0.5]{Sezioni/SottosezioniST/img/ChecklistCompleteEmitter.png}
	\caption{monolith::client::events::ChecklistCompleteEmitter}
\end{figure}

\begin{itemize}
\item \textbf{Descrizione:} Classe che contiene l'emettitore di eventi da lanciare quando una checklist viene completata. Estende Event-Emitter (es6-event-emitter).
\item \textbf{Utilizzo:} La classe viene utilizzata ogniqualvolta una checklist viene completata, lanciando l'evento corrispondente.
\item \textbf{Attributi:}
\item \textbf{Metodi:}
\begin{itemize}
\item \textit{public emitChecklistComplete(id:string):void}\\
Questo metodo emette l'evento relativo al completamento di una lista.
			\\ \textbf{Parametri}: \begin{itemize}
			\item \textit{id:string}\\
			L'identificativo della lista che è stata completata
			\end{itemize} 
\end{itemize}
\end{itemize}

\subsubsection{monolith::client::events::ChecklistUpdateEmitter}

\label{monolith::client::events::ChecklistUpdateEmitter}
\begin{figure}[H]
	\centering
	\includegraphics[scale=0.5]{Sezioni/SottosezioniST/img/ChecklistUpdateEmitter.png}
	\caption{monolith::client::events::ChecklistUpdateEmitter}
\end{figure}

\begin{itemize}
\item \textbf{Descrizione:} Classe che contiene l'emettitore di eventi da lanciare quando gli elementi di una checklist vengono cliccati. Estende Event-Emitter (es6-event-emitter).
\item \textbf{Utilizzo:} La classe viene utilizzata ogniqualvolta un elemento di una checklist viene cliccato, lanciando l'evento corrispondente.
\item \textbf{Attributi:}
\item \textbf{Metodi:}
\begin{itemize}
\item \textit{public emitOnUpdate(id:string,string:string):void}\\
Questo metodo emette l'evento relativo al click di un elemento di una lista.
			\\ \textbf{Parametri}: \begin{itemize}
			\item \textit{id:string}\\
			L'identificativo della lista della quale è stato cliccato un elemento.
			\item \textit{string:string}\\
			Stringa che contiene l'informazione relativa al tipo di click effettuato, se normale o prolungato.
			\end{itemize} 
\end{itemize}
\end{itemize}

\subsubsection{monolith::client::events::ClickButtonEventEmitter}

\label{monolith::client::events::ClickButtonEventEmitter}
\begin{figure}[H]
	\centering
	\includegraphics[scale=0.5]{Sezioni/SottosezioniST/img/ClickButtonEventEmitter.png}
	\caption{monolith::client::events::ClickButtonEventEmitter}
\end{figure}

\begin{itemize}
\item \textbf{Descrizione:} Classe che contiene l'emettitore di eventi da lanciare quando un bottone viene cliccato. Estende Event-Emitter (es6-event-emitter).
\item \textbf{Utilizzo:} La classe viene utilizzata ogniqualvolta un bottone viene cliccato.
\item \textbf{Attributi:}
\item \textbf{Metodi:}
\begin{itemize}
\item \textit{public emitClickButtonEvent():void}\\
Questo metodo emette l'evento relativo al click di un bottone.
\item \textit{public emitLongClickButtonEvent():void}\\
Questo metodo emette l'evento relativo al click prolungato di un bottone.
\end{itemize}
\end{itemize}
\subsubsection{BaseLayout}
\begin{itemize}
\item \textbf{Descrizione:} Classe astratta che implementa l'interfaccia BaseComponent che rappresenta un oggetto layout per la disposizione di BaseComponent nelle bolle di Monolith.
\item \textbf{Utilizzo:} Classe utilizzata ed estesa ogni qualvolta uno sviluppatore intende creare un nuovo layout da inserire in una bolla.
\item \textbf{Attributi:}
\begin{itemize}
\item \textit{private items:List<BaseComponent>}\\
Rappresenta la lista di oggetti BaseComponent contenuti nel layout.
\end{itemize}
\item \textbf{Metodi:}
\begin{itemize}
\item \textit{public addItem(component:BaseComponent):void}\\
Aggiunge un oggetto BaseComponent al layout.
\item{\textbf{Parametri}: \begin{itemize}
\item \textit{component:BaseComponent}\\
Oggetto che rappresenta il BaseComponent da aggiungere al layout.
\end{itemize}}
\end{itemize}
\end{itemize}

\subsubsection{VerticalLayoutView}
\begin{itemize}
\item \textbf{Descrizione:} Classe concreta che estende BaseLayout, destinata alla creazione di layout per la disposizione verticale di BaseComponent.
\item \textbf{Utilizzo:} Classe utilizzata ogni qualvolta uno sviluppatore intende creare un nuovo layout verticale da inserire in una bolla.
\item \textbf{Attributi:}
\item \textbf{Metodi:}
\begin{itemize}
\item \textit{public addItem(component:BaseComponent):void}\\
Aggiunge un oggetto BaseComponent al layout verticale.
\item{\textbf{Parametri}: \begin{itemize}
\item \textit{component:BaseComponent}\\
Oggetto che rappresenta il BaseComponent da aggiungere al layout verticale.
\end{itemize}}
\item \textit{public renderView():string}\\
Genera il codice HTML, CSS e JavaScript necessario per visualizzare BaseComponent disposti nel layout verticale.
\end{itemize}
\end{itemize}

\subsubsection{HorizontalLayoutView}
\begin{itemize}
\item \textbf{Descrizione:} Classe concreta che estende BaseLayout, destinata alla creazione di layout  per la disposizione orizzontale di BaseComponent.
\item \textbf{Utilizzo:} Classe utilizzata ogni qualvolta uno sviluppatore intende creare un nuovo layout verticale da inserire in una bolla.
\item \textbf{Attributi:}
\item \textbf{Metodi:}
\begin{itemize}
\item \textit{public addItem(component:BaseComponent):void}\\
Aggiunge un oggetto BaseComponent al layout orizzontale.
\item{\textbf{Parametri}: \begin{itemize}
\item \textit{component:BaseComponent}\\
Oggetto che rappresenta il BaseComponent da aggiungere al layout orizzontale.
\end{itemize}}
\item \textit{public renderView():string}\\
Genera il codice HTML, CSS e JavaScript necessario per visualizzare BaseComponent disposti nel layout orizzontale.
\end{itemize}
\end{itemize}
\subsection{Widgets}
\subsubsection{TextWidget}
\begin{lstlisting}[language=JavaScript]
// Create a TextWidget
let textWidget = new Monolith.Widget.TextWidget;

// Hide the widget
textWidget.setVisibility(false); // Default is true, which will show is

// Set the text. Markdown notation is also supported
textWidget.setText("Foo");
textWidget.setText("Markdown __is supported__ **too**");

// Set the text color using HEX notation (http://www.color-hex.com/)
textWidget.setTextColor("#C61A10");

// Set the text size in pixel
textWidget.setTextSize(15);

// Set the URL highlighting color
textWidget.setUrlHighligthColor("#EE42F4");

// Enable or disable the text formatting, this includes also URL highlighting
textWidget.setFormatText(true);
textWidget.setFormatText(false);
\end{lstlisting}
~\\~\\

\subsubsection{ImageWidget}
\begin{lstlisting}[language=JavaScript]
// Create the ImageWidget
let imageWidget = new Monolith.Widget.ImageWidget;

// Hide the widget
imageWidget.setVisibility(false); // Default is true, which will show is

// Set the image associated with the widget
imageWidget.setImage("path/to/image.png");

// Set the image dimensions
imageWidget.setWidth(200);
imageWidget.setHeight(50);
\end{lstlisting}

\newpage
\subsubsection{ButtonWidget}
\begin{lstlisting}[language=JavaScript]
// Create a ButtonWidget
let buttonWidget = new Monolith.Widget.ButtonWidget;

// Set the dimensions of the button
buttonWidget.setWidth(100);
buttonWidget.setHeight(50);

// Set the color of the button
buttonWidget.setBackgroundColor("#41F492");

// Set the action associated with the button
buttonWidget.setOnClickAction(function(){
    alert("The button has been clicked");
});

// Set the action associated with the button when the user long-clicks it
buttonWidget.setOnLongClickAction(function(){
    alert("The button has been long clicked");
});

// Set the milliseconds that need to pass before a click is considered a long click
buttonWidget.setOnLongClickActionTimer(500);
\end{lstlisting}
~\\~\\

\subsubsection{ListWidget}
\begin{lstlisting}[language=JavaScript]
// Create the ListWidget
let listWidget = new Monolith.Widget.ListWidget;

// Add items to the list
listWidget.addItem("First");
listWidget.addItem("Second");
listWidget.addItem("Third");

// Set the indicator of the list
listWidget.setCharacterNumber(); // Numbered list
listWidget.setCharacterCircle(); // Unnumbered list

// Set the indicator color
listWidget. setColor("#292929");
\end{lstlisting}

\newpage
\subsubsection{CheckListItemWidget}
\begin{lstlisting}[language=JavaScript]
// Create a new CheckListItemWidget
let checkListItemWidget = new Monolith.Widget.CheckListItemWidget;

// Set the text associated with the item
checkListItemWidget.setText("Click me!");

// Customize the check appereance
// Color the check box instead of using a check tick
checkListItemWidget.setUseSelectionMark(true); 
// Set the color that will be used to color the check box
checkListItemWidget.setSelectionColor("#AAAAAA"); 
// Use a check tick
checkListItemWidget.setUseSelectionMark(false); 
// Set the character used as check tick
checkListItemWidget.setSelectionCharacter("X"); 

// Check or un-check the option
checkListItemWidget.setChecked(true); // Checked
checkListItemWidget.setChecked(false); // Un-checked

// Know it the option is checked or not
let isChecked = checkListItemWidget.isChecked();
if (isChecked){
    // The option is checked
} else {
    // The option is not checked
}


// Set the action to perform on click
checkListItemWidget.setOnClick(function(item){
    // The item parameter represents the view of the item that has been clicked
    item.setText("New text after click");
});

// Set the action to perform after a long click (1000 ms)
checkListItemWidget.setOnLongClick(function(){
    // The item parameter represents the view of the item that has been clicked
   item.setText("New text after long click");
});


// Delete the item
checkListItemWidget.removeOption();
\end{lstlisting}

\newpage
\subsubsection{Create a custom widget}
In order to create a new custom widget and add it to \termine{Monolith} so than you can use it like the default ones, you have to do as follows.
\begin{enumerate}

	\item Create a new class which extends from \texttt{BaseWidget}
\begin{lstlisting}[language=JavaScript]
export class MyWidget extends Monolith.Widget.BaseWidget {

    constructor(){
        super(); // You need to call this to create the above hierarchy
        
        // Initialize your widget here
    }
    
    renderView(){
        // Renders the view of the widget and returns a DOMElement object
    }

    performOperation(){
        // Perform some operation
    }

}
\end{lstlisting}

	\item Use your widget wherever you want
\begin{lstlisting}[language=JavaScript]
// Import the widget
import {MyWidget} from '/path/to/MyWidget.js';

// Istantiate the widget
let myWidget = new MyWidget();

// Perform operations with the widget
myWidget.performOperation();

// Render the widget's view
myWidget.renderView();
\end{lstlisting}
  
\end{enumerate}  
  
\textbf{Note} \\ 
The default widget's behaviour does \textbf{not} let the user use a single widget without a bubble container that holds it. \\
If you plan to render the widget's view inside a Rocket.chat room, please create a bubble and add your widget to the bubble, so that the bubble will render it and show it to the user.

\newpage



\subsection{Applicazione demo lista-spesa}

\subsubsection{bringit::server::usecase::GetListInfoUseCase}

\label{bringit::server::usecase::GetListInfoUseCase}
\begin{figure}[H]
	\centering
	\includegraphics[scale=0.5]{Sezioni/SottosezioniST/img/app/GetListInfoUseCase.png}
	\caption{bringit::server::usecase::GetListInfoUseCase}
\end{figure}

\begin{itemize}
\item \textbf{Descrizione}: Classe di comunicazione con il database.
\item \textbf{Utilizzo}: La classe verrà utilizzata nel caso serva ricavare informazioni riguardanti una lista dal database.
\item \textbf{Attributi}: 
	\begin{itemize}
	\item \textit{private databaseSource:DatabaseSource}\\
	Il riferimento al database.
	\end{itemize}
\item \textbf{Metodi}:
	\begin{itemize}
	\item \textit{public GetListInfoUseCase():GetListInfoUseCase}\\
	Il costruttore della classe GetListInfoUseCase.
	\item \textit{public getListData(listId:Mongo.ObjectID):ListData}\\
	Questo metodo ritorna una la lista, recuperata dal database, corrispondente all'id passato come parametro.
				\\ \textbf{Parametri}: \begin{itemize}
			\item \textit{listId:Mongo.ObjectID}\\
			L'id della lista che si vuole recuperare dal database.
					\end{itemize} 
	\end{itemize}
\item \textbf{Eventi}:
\end{itemize}

\subsubsection{bringit::server::usecase::ManageListsUseCase}

\label{bringit::server::usecase::ManageListsUseCase}
\begin{figure}[H]
	\centering
	\includegraphics[scale=0.5]{Sezioni/SottosezioniST/img/app/ManageListsUseCase.png}
	\caption{bringit::server::usecase::ManageListsUseCase}
\end{figure}

\begin{itemize}
\item \textbf{Descrizione}: Classe di comunicazione con il database.
\item \textbf{Utilizzo}: La classe verrà utilizzata nel caso serva eliminare una lista dal database.
\item \textbf{Attributi}: 
	\begin{itemize}
	\item \textit{private databaseSource:DatabaseSource}\\
	Il riferimento al database.
	\end{itemize}
\item \textbf{Metodi}:
	\begin{itemize}
	\item \textit{public ManageListsUseCase():ManageListsUseCase}\\
	Il costruttore della classe ManageListsUseCase.
	\item \textit{public deleteList(listId:Mongo.ObjectID):void}\\
	Questo metodo rimuove la lista corrispondente all'id passato come parametro dal database.
				\\ \textbf{Parametri}: \begin{itemize}
			\item \textit{listId:Mongo.ObjectID}\\
			L'id della lista che si vuole rimuovere dal database.
					\end{itemize} 
	\end{itemize}
\item \textbf{Eventi}:
\end{itemize}
 
\subsubsection{bringit::server::usecase::ShareListUseCase}

\label{bringit::server::usecase::ShareListUseCase}
\begin{figure}[H]
	\centering
	\includegraphics[scale=0.5]{Sezioni/SottosezioniST/img/app/ShareListUseCase.png}
	\caption{bringit::server::usecase::ShareListUseCase}
\end{figure}

\begin{itemize}
\item \textbf{Descrizione}: Classe di comunicazione con il database.
\item \textbf{Utilizzo}: La classe verrà utilizzata nel caso una lista venga condivisa con un'utente.
\item \textbf{Attributi}: 
	\begin{itemize}
	\item \textit{private databaseSource:DatabaseSource}\\
	Il riferimento al database.
	\end{itemize}
\item \textbf{Metodi}:
	\begin{itemize}
	\item \textit{public ShareListUseCase():ShareListUseCase}\\
	Il costruttore della classe ShareListUseCase.
	\item \textit{public shareListWithContact(listId:Mongo.ObjectID,contactId:string):void}\\
	Questo metodo salva la condivisione di una specifica lista con un utente.
				\\ \textbf{Parametri}: \begin{itemize}
			\item \textit{listId:Mongo.ObjectID}\\
			L'id della lista che si vuole condividere.
			\item \textit{contactId:string}\\
			L'id dell'utente al quale si vuole condividere la lista.
					\end{itemize} 
	\end{itemize}
\item \textbf{Eventi}:
\end{itemize}

\subsubsection{bringit::server::usecase::ModifyListUseCase}

\label{bringit::server::usecase::ModifyListUseCase}
\begin{figure}[H]
	\centering
	\includegraphics[scale=0.5]{Sezioni/SottosezioniST/img/app/ModifyListUseCase.png}
	\caption{bringit::server::usecase::ModifyListUseCase}
\end{figure}

\begin{itemize}
\item \textbf{Descrizione}: Classe di comunicazione con il database.
\item \textbf{Utilizzo}: La classe verrà utilizzata nel caso una lista venga modificata per salvare i cambiamenti di quest'ultima.
\item \textbf{Attributi}: 
	\begin{itemize}
	\item \textit{private databaseSource:DatabaseSource}\\
	Il riferimento al database.
	\end{itemize}
\item \textbf{Metodi}:
	\begin{itemize}
	\item \textit{public ModifyListUseCase():ModifyListUseCase}\\
	Il costruttore della classe ModifyListUseCase.
	\item \textit{public saveList(listId:Mongo.ObjectID):void}\\
	Questo metodo salva le modifiche a una lista se l'id passato come parametro è già presente nel database, altrimenti ne crea una nuova con id corrispondente.
				\\ \textbf{Parametri}: \begin{itemize}
			\item \textit{listId:Mongo.ObjectID}\\
			L'id della lista che si vuole modificare o creare.
					\end{itemize} 
	\item \textit{public addItem(listId:Mongo.ObjectID,item:ListItem):void}\\
	Questo metodo aggiunge un item a una lista nel database.
				\\ \textbf{Parametri}: \begin{itemize}
			\item \textit{listId:Mongo.ObjectID}\\
			L'id della lista alla quale si vuole aggiungere un prodotto.
			\item \textit{item:ListItem}\\
			Il prodotto che si vuole aggiungere.
					\end{itemize}
	\item \textit{public removeItem(listId:Mongo.ObjectID,item:ListItem):void}\\
	Questo metodo rimuove un item da una lista nel database.
				\\ \textbf{Parametri}: \begin{itemize}
			\item \textit{listId:Mongo.ObjectID}\\
			L'id della lista dalla quale si vuole rimuovere un prodotto.
			\item \textit{item:ListItem}\\
			Il prodotto che si vuole rimuovere.
					\end{itemize} 
	\item \textit{public updateItemInsideList(listId:Mongo.ObjectID,item:ListItem):void}\\
	Questo metodo aggiorna i dati di un item di una lista nel database.
				\\ \textbf{Parametri}: \begin{itemize}
			\item \textit{listId:Mongo.ObjectID}\\
			L'id della lista della quale si vuole aggiornare un prodotto.
			\item \textit{item:ListItem}\\
			Il prodotto che si vuole aggiornare.
					\end{itemize} 
	\end{itemize}
\item \textbf{Eventi}:
\end{itemize}
\subsubsection{bringit::server::database::DatabaseSource}

\label{bringit::server::database::DatabaseSource}
\begin{figure}[H]
	\centering
	\includegraphics[scale=0.5]{Sezioni/SottosezioniST/img/app/DatabaseSource.png}
	\caption{bringit::server::database::DatabaseSource}
\end{figure}

\begin{itemize}
\item \textbf{Descrizione}: Interfaccia che permette la comunicazione tra il model ed il database sul quale verranno salvati tutti i dati relativi alle varie liste. I metodi da essa esposti non vengono descritti nel dettaglio in quanto l'implementazione di questa interfaccia utilizzerà i noti metodi di Meteor.js, descritti approfonditamente nella rispettiva documentazione inserita all'interno dei riferimenti normativi.
\item \textbf{Utilizzo}: Interfaccia che permette di salvare, modificare, rimuovere dati all'interno del database.
\item \textbf{Attributi}:
	\begin{itemize}
		\item \textit{private listCollection}\\
		La collezione di tutti i messaggi presenti in rocket.chat.
	\end{itemize}
\item \textbf{Metodi}:
	\begin{itemize}
	\item \textit{public DatabaseSource(userId:string):DatabaseSource}\\
	Il costruttore di DatabaseSource.
	\item \textit{}\\
	
			\\ \textbf{Parametri}: \begin{itemize}
			\item \textit{}\\
			
			\end{itemize} 
	\item \textit{public getLists():ListData[]}\\
	Questo metodo ritorna tutte le liste salvate attualmente nel database.
	\item \textit{public removeList(listId:Mongo.ObjectID):void}\\
	Questo metodo rimuove la lista corrispondente all'id passato come parametro dal database.
				\\ \textbf{Parametri}: \begin{itemize}
			\item \textit{listId:Mongo.ObjectID}\\
			L'id della lista che si vuole rimuovere dal database.
			\end{itemize} 
			
	\item \textit{public getListWithId(listId:Mongo.ObjectID):ListData}\\
	Questo metodo ritorna una la lista, recuperata dal database, corrispondente all'id passato come parametro.
				\\ \textbf{Parametri}: \begin{itemize}
			\item \textit{listId:Mongo.ObjectID}\\
			L'id della lista che si vuole recuperare dal database.
			\end{itemize} 
			
	\item \textit{public saveList(listData:ListData):void}\\
	Questo metodo salva la lista data nel database.
				\\ \textbf{Parametri}: \begin{itemize}
			\item \textit{listData:ListData}\\
			La lista che si vuole salvare.
					\end{itemize}
	\item \textit{public getItemWithId(itemId:Mongo.ObjectID):ListItem}\\
	Ritorna il listItem con id corrispondente a quello passato per parametro che è attualmente salvato nel database.
			\\ \textbf{Parametri}: \begin{itemize}
			\item \textit{itemId:Mongo.ObjectID}\\
			L'id del prodotto che si vuole recuperare dal database.
			\end{itemize}
	\item \textit{public clear():void}\\
	Pulisce tutti gli oggetti presenti in tutte le collezioni del database.
	\item \textit{private convertToListData(data:JSONObject):ListData}\\
	Converte un file json opportuno in un oggetto di tipo ListData con gli attributi richiesti.
			\\ \textbf{Parametri}: \begin{itemize}
			\item \textit{data:JSONObject}\\
			Il json che si vuole convertire in ListData.
			\end{itemize}
	\item \textit{private convertToListData(item:JSONObject):ListItem}\\
	Converte un file json opportuno in un oggetto di tipo ListItem con gli attributi richiesti.
			\\ \textbf{Parametri}: \begin{itemize}
			\item \textit{item:JSONObject}\\
			Il json che si vuole convertire in ListItem.
			\end{itemize}
	\end{itemize}
\item \textbf{Eventi}:
\end{itemize}


\newpage
\input{Sezioni/6-FlowCharts.tex}
\input{Sezioni/7-TracciamentoRequisitiClassi.tex}
\input{Sezioni/8-TracciamentoClassiRequisiti.tex}
\input{Sezioni/9-TracciamentoRequisitiPackage.tex}
\section{Tracciamento \termine{Package} - Requisiti}
\subsection{Tracciamento \termine{Package} - Requisiti}
\normalsize
%\begin{longtable}{|>{\centering}m{5cm}|m{5cm}<{\centering}|}
%\hline
%
%\textbf{Classe} & \textbf{Id Requisiti}\\
%\hline
%\endhead
%Qui va classe & Qui vanno requisiti\\
%\hline
%
%\caption[Tracciamento Classi - Requisiti]{Tracciamento Classi - Requisiti}
%\label{tabella: Tracciamento Classi - Requisiti}
%\end{longtable}
\begin{center}
	\begin{longtable}{|p{3cm}|p{10cm}|}\hline
		Package & Id Requisiti \\ \hline
		package & RC1\newline RC2\newline \\ \hline
		package1 & \\ \hline
		component & R1F1.1\newline R1F1.2\newline R1F1.3\newline R1F1.4\newline R1F1.5\newline R1F1.3.2\newline R1F1.6\newline R2F1.1.1\newline R2F1.1.2\newline R2F1.1.3\newline R2F1.1.4\newline R2F1.1.5\newline R2F1.1.6\newline R2F1.1.7\newline R2F1.1.8\newline R2F1.1.9\newline R2F1.1.10\newline R2F1.1.11\newline R1F1.5.1\newline R1F1.5.1.1\newline R1F1.5.1.2\newline R1F1.5.1.3\newline R1F1.5.1.4\newline R1F1.5.1.5\newline R2F1.5.1.6\newline R1F1.5.1.7\newline R1F1.5.1.8\newline R1F1.5.1.9\newline R1F1.5.1.10\newline R1F1.5.1.11\newline R1F1.2.1\newline R1F1.2.2\newline R1F1.2.3\newline R1F1.2.4\newline R1F1.5.2.1\newline R1F1.5.2.2\newline R1F1.3.1\newline R1F1.3.1.1\newline R1F1.3.1.2\newline R1F1.3.1.3\newline R1F1.3.1.4\newline R1F1.3.1.5\newline R1F1.3.1.6\newline R1F1.3.3.1\newline R1F1.3.3.2\newline R1F1.3.3.3\newline R1F1.3.3.4\newline R1F1.5.3.1\newline R1F1.5.3.3\newline R1F1.3.2.1\newline R1F1.3.2.2\newline R1F1.3.2.3\newline R1F1.3.2.4\newline R1F1.3.2.5\newline R1F1.3.2.3.1\newline R1F1.3.2.3.2\newline R1F1.3.2.3.3\newline R1F1.3.2.3.4\newline R1F1.3.2.3.5\newline R1F1.3.3\newline R1F1.4,1\newline R3F1.4.1.1\newline R1F1.4.1.2\newline R2F1.4.1.3\newline R1F1.4.2\newline R1F1.4.2.3\newline R1F1.5.3.2\newline R1F1.5.3.4\newline R1F3\newline \\ \hline
		layout & R1F1.1\newline R1F1.2\newline R1F1.3\newline R1F1.4\newline R1F1.5\newline R1F1.3.2\newline R1F1.6\newline \\ \hline
		widget & R2F1.1.1\newline R2F1.1.2\newline R2F1.1.3\newline R2F1.1.4\newline R2F1.1.5\newline R2F1.1.6\newline R2F1.1.7\newline R2F1.1.8\newline R2F1.1.9\newline R2F1.1.10\newline R2F1.1.11\newline R1F1.5.1\newline R1F1.5.1.1\newline R1F1.5.1.2\newline R1F1.5.1.3\newline R1F1.5.1.4\newline R1F1.5.1.5\newline R2F1.5.1.6\newline R1F1.5.1.7\newline R1F1.5.1.8\newline R1F1.5.1.9\newline R1F1.5.1.10\newline R1F1.5.1.11\newline R1F1.2.1\newline R1F1.2.2\newline R1F1.2.3\newline R1F1.2.4\newline R1F1.5.2.1\newline R1F1.5.2.2\newline R1F1.3.1\newline R1F1.3.1.1\newline R1F1.3.1.2\newline R1F1.3.1.3\newline R1F1.3.1.4\newline R1F1.3.1.5\newline R1F1.3.1.6\newline R1F1.3.3.1\newline R1F1.3.3.2\newline R1F1.3.3.3\newline R1F1.3.3.4\newline R1F1.5.3.1\newline R1F1.5.3.3\newline R1F1.3.2\newline R1F1.3.2.1\newline R1F1.3.2.2\newline R1F1.3.2.3\newline R1F1.3.2.4\newline R1F1.3.2.5\newline R1F1.3.2.3.1\newline R1F1.3.2.3.2\newline R1F1.3.2.3.3\newline R1F1.3.2.3.4\newline R1F1.3.2.3.5\newline R1F1.3.3\newline R1F1.4\newline R1F1.4,1\newline R3F1.4.1.1\newline R1F1.4.1.2\newline R2F1.4.1.3\newline R1F1.4.2\newline R1F1.4.2.3\newline R1F1.5.3.2\newline R1F1.5.3.4\newline R1F1.5\newline R1F3\newline \\ \hline
		text & R2F1.1.1\newline R2F1.1.2\newline R2F1.1.3\newline R2F1.1.4\newline R2F1.1.5\newline R2F1.1.6\newline R2F1.1.7\newline R2F1.1.8\newline R2F1.1.9\newline R2F1.1.10\newline R2F1.1.11\newline R1F1.5.1\newline R1F1.5.1.1\newline R1F1.5.1.2\newline R1F1.5.1.3\newline R1F1.5.1.4\newline R1F1.5.1.5\newline R2F1.5.1.6\newline R1F1.5.1.7\newline R1F1.5.1.8\newline R1F1.5.1.9\newline R1F1.5.1.10\newline R1F1.5.1.11\newline \\ \hline
		image & R1F1.2.1\newline R1F1.2.2\newline R1F1.2.3\newline R1F1.2.4\newline R1F1.5.2.1\newline R1F1.5.2.2\newline \\ \hline
		button & R1F1.3.1\newline R1F1.3.1.1\newline R1F1.3.1.2\newline R1F1.3.1.3\newline R1F1.3.1.4\newline R1F1.3.1.5\newline R1F1.3.1.6\newline R1F1.3.3.1\newline R1F1.3.3.2\newline R1F1.3.3.3\newline R1F1.3.3.4\newline R1F1.5.3.1\newline R1F1.5.3.3\newline \\ \hline
		checklist & R1F1.3.2\newline R1F1.3.2.1\newline R1F1.3.2.2\newline R1F1.3.2.3\newline R1F1.3.2.4\newline R1F1.3.2.5\newline R1F1.3.2.3.1\newline R1F1.3.2.3.2\newline R1F1.3.2.3.3\newline R1F1.3.2.3.4\newline R1F1.3.2.3.5\newline R1F1.3.3\newline R1F1.3.3.1\newline R1F1.3.3.2\newline R1F1.3.3.3\newline R1F1.3.3.4\newline R1F1.4\newline R1F1.4,1\newline R3F1.4.1.1\newline R1F1.4.1.2\newline R2F1.4.1.3\newline R1F1.4.2\newline R1F1.4.2.3\newline R1F1.5.3.2\newline R1F1.5.3.4\newline \\ \hline
		bubble & R1F2.1\newline R1F2.2\newline R1F2.3\newline \\ \hline
		allert & R1F2.1\newline \\ \hline
		markdown & R1F2.2\newline \\ \hline
		todolist & R1F2.3\newline \\ \hline
	\end{longtable}
\end{center}

\newpage

\newpage
\begin{appendices}
\input{Sezioni/A1-CapabilityMaturityModel.tex}
\input{Sezioni/A2-PDCA.tex}
\input{Sezioni/A3-StandardISO-IEC9126.tex}
\section{Resoconto delle attività di verifica}

In questa sezione vengono inserite tutte le misurazioni delle metriche trovate dal gruppo \gruppo.
Il team si impegna a garantire almeno il soddisfacimento del range di accettazione per ogni metrica.

Per quanto riguarda le misurazioni può essere calcolato un risultato singolo, se la metrica si riferisce ad una caratteristica singola, oppure può essere calcolato un risultato massimo, ovvero una metrica che si riferisce a più componenti, ad esempio le classi. In tal caso verrà presa la misurazione peggiore e confrontata con i valori scelti dal team precedentemente.

Alcune metriche, infine, sono state misurate più volte nel tempo e per queste verrà illustrato un diagramma cartesiano, in cui: l'asse delle ascisse rappresenterà in giorni la durata del periodo fino alla consegna, mentre l'asse delle ordinate rappresenterà i valori assunti al momento delle misurazioni. \\
Verranno comunque riportati i risultati finali di tali metriche nell'apposita tabella.


\subsection{Revisione dei Requisiti}
In questa sezione vengono inseriti i risultati relativi al periodo di Revisione dei Requisiti e le metriche relative ad esso.

\subsubsection{Analisi statica dei documenti}
L'analisi statica dei documenti è stata fatta mediante \termine{Walkthrough} ed ha portato all'individuazione di alcuni errori. Tra gli errori individuati quelli più frequenti sono stati:
		\begin{itemize}
			\item Errori nei concetti esposti.
			\item Aggettivi o verbi utilizzati in modo scorretto.
			\item Periodi troppo lunghi o complessi da capire ed interpretare.
		\end{itemize}

\subsubsection{Esiti verifiche automatizzate}
		
\paragraph{Indice di Gulpease}

\begin{table}[h]
	\begin{center}
		\begin{tabular}{|c|c|c|c|}
			\hline
			\textbf{Documento}	& \textbf{Risultato} & \textbf{Esito} & \textbf{Valore} \\
			\hline
		 \termine{Analisi} dei Requisiti v1.0.0 & 90 & Superato & Ottimale	\\
			\hline
			Glossario v1.0.0 & 56 & Superato & Ottimale	\\
			\hline
			Norme di Progetto v1.0.0 & 47 & Superato & Accettabile \\
			\hline
			Piano di Progetto v1.0.0 & 48 & Superato & Accettabile\\
			\hline
			Piano di Qualifica v1.0.0	& 48 & Superato & Accettabile\\
			\hline
			Studio di Fattibilità v1.0.0	& 47 & Superato & Accettabile\\
			\hline
			Verbale\_Esterno\_1\_20161223 v1.0.0	& 51 & Superato & Ottimale	\\
			\hline
		\end{tabular}
	\end{center}
	\caption{RR - Risultato indice di Gulpease}
\end{table}

\subsubsection{Soddisfacimento metriche}

\paragraph{Qualità di processo}
\begin{longtable}{|>{\centering}m{5cm}|c|c|c|c|c|}
\hline
\textbf{Metrica} & \textbf{Unità di misura} & \textbf{Risultato} & \textbf{Risultato Massimo} & \textbf{Esito} & \textbf{Valore}\\
\hline
\endhead

\emph{Schedule Variance} & {Attività} & \textcolor{Green}{0} & / & Superato & Ottimale\\ \hline
\emph{Budget Variance} & {Euro} & \textcolor{Green}{15} & / & Superato & Ottimale \\ \hline
\emph{Rischi non preventivati} & {Rischi} & \textcolor{Green}{0} & / & Superato & Ottimale\\ \hline
\emph{Ottimalità delle misurazioni} & {Percentuale} & \textcolor{Green}{0.6} & / & Superato & Ottimale \\ \hline
\emph{Rischi non preventivati} & {Rischi} & \textcolor{Green}{0} & / & Superato & Ottimale\\ \hline
%\emph{Efficienza di gestione dei rischi} & {Giorni} & \textcolor{Orange}{21.3} & $\geq 20$ & $\geq 60$\\ \hline
%\emph{Requisiti obbligatori soddisfatti} & {Percentuale} & \textcolor{Green}{100} & $100$ & $100$\\ \hline
%\emph{Livello di stabilità-SDK} & {Percentuale} & / &\textcolor{Green}{1} & Accettabile\\ \hline
%\emph{Livello di stabilità-Applicazione} & {Percentuale} & / &\textcolor{Green}{1} & Accettabile\\ \hline
%\emph{Astrattezza-SDK} & {Percentuale} & / &\textcolor{Green}{0.8} & Accettabile\\ \hline
%\emph{Astrattezza-Applicazione} & {Percentuale} & / &\textcolor{Green}{0.5} & Ottimale\\ \hline
%\emph{Distanza dalla sequenza principale-SDK} & {Percentuale} & / &\textcolor{Green}{1} & Accettabile\\ \hline
%\emph{Distanza dalla sequenza principale-Applicazione} & {Percentuale} & / &\textcolor{Green}{0.7} & Accettabile\\ \hline
%\emph{Numero di metodi per classe} & {Metodi} & / &\textcolor{Green}{11} & Accettabile\\ \hline
%\emph{Numero di attributi per classe} & {Attributi} & / &\textcolor{Green}{7} & Ottimale\\ \hline
%\emph{Numero di parametri per metodo} & {Parametri} & / &\textcolor{Green}{7} & Ottimale\\ \hline
%\emph{Produttività di codifica} & {Linee} & \textcolor{Orange}{9.1} & $\geq 3$ & $\geq 10$\\ \hline
%\emph{Complessità Ciclomatica media} & {Cammini} & \textcolor{Green}{0} & $1 - 15$ & $1 - 10$\\ \hline
%\emph{Livelli di annidamento medi} & {Chiamate} & \textcolor{Green}{1} & $1 - 6$ & $1 - 3$\\ \hline
%\emph{Linee di codice per linee di commento} & {Percentuale} & \textcolor{Green}{31} & $\geq 25$ & $\geq 30$\\ \hline
%\emph{Variabili inutilizzate} & {Variabili} & \textcolor{Green}{0} & $0$ & $0$\\ \hline
%\emph{Dipendenze} & {Chiamate require} & \textcolor{Green}{2.2} & $0 - 10$ & $0 - 5$\\ \hline
%\emph{Halstead Difficulty media} & {Percentuale} & \textcolor{Green}{0} & $0 - 25$ & $0 - 15$\\ \hline
%\emph{Halstead Volume media} & {Percentuale} & \textcolor{Green}{20} & $20 - 1500$ & $20 - 1000$\\ \hline
%\emph{Halstead Effort media} & {Percentuale} & \textcolor{Green}{0} & $0 - 400$ & $0 - 300$\\ \hline
%\emph{Indice di manutenibilità} & {Percentuale} & \textcolor{Orange}{5.14} & $100 - 171$ & $120 - 171$\\ \hline
%\emph{Componenti integrate} & {Percentuale} & \textcolor{Green}{100} & $100$ & $100$\\ \hline
%\emph{Test di Unità eseguiti} & {Percentuale} & \textcolor{Red}{36.6} & $90 - 100$ & $100$\\ \hline
%\emph{Test di Integrazione eseguiti} & {Percentuale} & \textcolor{Red}{0} & $60 - 100$ & $70 - 100$\\ \hline
%\emph{Test di Sistema eseguiti} & {Percentuale} & \textcolor{Red}{0} & $70 - 100$ & $80 - 100$\\ \hline
%\emph{Test di \termine{Validazione} eseguiti} & {Percentuale} & \textcolor{Red}{0} & $100$ & $100$\\ \hline
%\emph{Test superati} & {Percentuale} & \textcolor{Green}{100} & $90 - 100$ & $100$\\ \hline
%\emph{Branch Coverage} & {Percentuale} & \textcolor{Orange}{73.3} & $70 - 100$ & $80 - 100$\\ \hline
%\emph{Code Coverage} & {Percentuale} & \textcolor{Green}{76.57} & $60 - 100$ & $70 - 100$\\ \hline
\caption{RR-Metriche di qualità di processo}
\end{longtable}

\newpage

\subsubsection{Esiti delle metriche ripetute nel tempo}

\begin{figure}[H]
	\centering 
	\includegraphics[scale=0.7]{Sezioni/Immagini/ScheduleVariance-RR}
	\caption{Schedule variance - RR}
\end{figure}

\begin{figure}[H]
	\centering 
	\includegraphics[scale=0.7]{Sezioni/Immagini/BudgetVariance-RR}
	\caption{Budget variance - RR}
\end{figure}

\subsubsection{Livello dei processi}
\begin{longtable}{|>{\centering}m{6cm}|c|c|c|c|c|}
\hline
\textbf{Processo} & \textbf{Livello} & \textbf{Esito} & \textbf{Valore}\\
\hline
\endhead
\emph{Processo di fornitura} & \textcolor{Green}{1} & Superato & Accettabile\\ \hline
\emph{Processo di sviluppo} & \textcolor{Green}{2}* & Superato & Accettabile\\ \hline
\emph{Processo di documentazione} & \textcolor{Green}{2} & Superato & Accettabile\\ 
\hline
\emph{Processo di Configurazione} & \textcolor{Green}{1} & Superato & Ottimale\\ 
\hline
\emph{Processo di garanzia di qualità del Prodotto} & * & / & /\\ 
\hline
\emph{Processo di Verifica} & \textcolor{Green}{1} & Superato & Ottimale\\ 
\hline
\emph{Processo di Validazione} & * & / & /\\ 
\hline
\emph{Processo di Risoluzione dei problemi} & \textcolor{Green}{1} & Superato & Ottimale\\ 
\hline
\emph{Processo di Coordinamento} & \textcolor{Green}{1} & Superato & Ottimale\\ 
\hline
\emph{Processo di Pianificazione} & \textcolor{Green}{1} & Superato & Ottimale\\ 
\hline
\emph{Processo di Formazione} & \textcolor{Green}{1} & Superato & Ottimale\\ 
\hline
\caption{RR-Livello dei processi}
\end{longtable}

Per i processi con segnatura \texttt{Voto*} vengono considerate solo le attività inerenti al lavoro che deve essere svolto per la \textit{Revisione di progettazione}. Per quelli, invece, con solo \texttt{*} significa che nessuna attività di quel processo era necessaria per il raggiungimento della milestone esterna.

\newpage

\subsection{Revisione di Progettazione}

\subsubsection{Analisi statica dei documenti}
L'analisi statica dei documenti è stata fatta mediante \termine{Walkthrough} ed ha portato all'individuazione di alcuni errori. Tra gli errori individuati quelli più frequenti sono stati:
		\begin{itemize}
			\item Errori ortografici.
			\item Parole con lettere mancanti o invertite.
			\item Periodi troppo lunghi o complessi da capire ed interpretare.
		\end{itemize}

\subsection{Metriche per i documenti}

\subsubsection{Indice di Gulpease}

\begin{table}[h]
	\begin{center}
		\begin{tabular}{|c|c|c|c|c|}
			\hline
			\textbf{Documento}	& \textbf{Risultato} & \textbf{Esito} & \textbf{Valore}\\
			\hline
		 \termine{Analisi} dei Requisiti v2.0.0 &	90 & Superato & Ottimale\\
			\hline
			Glossario v2.0.0 &	54 & Superato & Ottimale\\
			\hline
			Norme di Progetto v2.0.0 &	52 & Superato & Ottimale\\
			\hline
			Piano di Progetto v2.0.0	&	52 & Superato & Ottimale\\
			\hline
			Piano di Qualifica v2.0.0	&	46 & Superato & Accettabile\\
			\hline
			Definizione di \termine{Prodotto} v1.0.0	&	64 & Superato & Ottimale\\
			\hline
			Verbale\_Interno\_2\_20170222 v1.0.0	&	54 & Superato & Ottimale\\
			\hline
			Verbale\_Esterno\_3\_20170224 v1.0.0	&	53 & Superato & Ottimale\\
			\hline
			Verbale\_Interno\_4\_20170226 v1.0.0	&	51 & Superato & Ottimale\\
			\hline
			Verbale\_Interno\_5\_20170228 v1.0.0	&	52 & Superato & Ottimale\\
			\hline
		\end{tabular}
	\end{center}
	\caption{RP - Risultato indice di Gulpease}
\end{table}

\subsubsection{Soddisfacimento metriche}

\paragraph{Qualità di processo}
\begin{longtable}{|>{\centering}m{5cm}|c|c|c|c|c|}
\hline
\textbf{Metrica} & \textbf{Unità di misura} & \textbf{Risultato} & \textbf{Risultato Massimo} & \textbf{Esito} & \textbf{Valore}\\
\hline
\endhead

\emph{Schedule Variance} & {Attività} & \textcolor{Green}{0} & / & Superato & Ottimale\\ \hline
\emph{Budget Variance} & {Euro} & \textcolor{Orange}{-50} & / & Non superato & /\\ \hline
\emph{Ottimalità delle misurazioni} & {Percentuale} & \textcolor{Green}{0.6} & / & Superato & Ottimale \\ \hline
\emph{Rischi non preventivati} & {Rischi} & \textcolor{Green}{2} & / & Superato & Accettabile\\ \hline
%\emph{Efficienza di gestione dei rischi} & {Giorni} & \textcolor{Orange}{21.3} & $\geq 20$ & $\geq 60$\\ \hline
%\emph{Requisiti obbligatori soddisfatti} & {Percentuale} & \textcolor{Green}{100} & $100$ & $100$\\ \hline
\emph{Livello di instabilità-SDK} & {Percentuale} & / &\textcolor{Green}{1} & Superato & Accettabile\\ \hline
\emph{Livello di instabilità-Applicazione} & {Percentuale} & / &\textcolor{Green}{1} & Superato & Accettabile\\ \hline
\emph{Astrattezza-SDK} & {Percentuale} & \textcolor{Green}{0.8} & / & Superato & Accettabile\\ \hline
\emph{Astrattezza-Applicazione} & {Percentuale} & \textcolor{Green}{0.5} & / & Superato & Ottimale\\ \hline
\emph{Distanza dalla sequenza principale-SDK} & {Percentuale} & / &\textcolor{Green}{1} & Superato & Accettabile\\ \hline
\emph{Distanza dalla sequenza principale-Applicazione} & {Percentuale} & / &\textcolor{Green}{0.7} & Superato & Accettabile\\ \hline
\emph{Numero di metodi per classe} & {Metodi} & / &\textcolor{Green}{11} & Superato & Accettabile\\ \hline
\emph{Numero di attributi per classe} & {Attributi} & / &\textcolor{Green}{7} & Superato & Ottimale\\ \hline
\emph{Numero di parametri per metodo} & {Parametri} & / &\textcolor{Green}{7} & Superato & Ottimale\\ \hline
%\emph{Produttività di codifica} & {Linee} & \textcolor{Orange}{9.1} & $\geq 3$ & $\geq 10$\\ \hline
%\emph{Complessità Ciclomatica media} & {Cammini} & \textcolor{Green}{0} & $1 - 15$ & $1 - 10$\\ \hline
%\emph{Livelli di annidamento medi} & {Chiamate} & \textcolor{Green}{1} & $1 - 6$ & $1 - 3$\\ \hline
%\emph{Linee di codice per linee di commento} & {Percentuale} & \textcolor{Green}{31} & $\geq 25$ & $\geq 30$\\ \hline
%\emph{Variabili inutilizzate} & {Variabili} & \textcolor{Green}{0} & $0$ & $0$\\ \hline
%\emph{Dipendenze} & {Chiamate require} & \textcolor{Green}{2.2} & $0 - 10$ & $0 - 5$\\ \hline
%\emph{Halstead Difficulty media} & {Percentuale} & \textcolor{Green}{0} & $0 - 25$ & $0 - 15$\\ \hline
%\emph{Halstead Volume media} & {Percentuale} & \textcolor{Green}{20} & $20 - 1500$ & $20 - 1000$\\ \hline
%\emph{Halstead Effort media} & {Percentuale} & \textcolor{Green}{0} & $0 - 400$ & $0 - 300$\\ \hline
%\emph{Indice di manutenibilità} & {Percentuale} & \textcolor{Orange}{5.14} & $100 - 171$ & $120 - 171$\\ \hline
%\emph{Componenti integrate} & {Percentuale} & \textcolor{Green}{100} & $100$ & $100$\\ \hline
%\emph{Test di Unità eseguiti} & {Percentuale} & \textcolor{Red}{36.6} & $90 - 100$ & $100$\\ \hline
%\emph{Test di Integrazione eseguiti} & {Percentuale} & \textcolor{Red}{0} & $60 - 100$ & $70 - 100$\\ \hline
%\emph{Test di Sistema eseguiti} & {Percentuale} & \textcolor{Red}{0} & $70 - 100$ & $80 - 100$\\ \hline
%\emph{Test di \termine{Validazione} eseguiti} & {Percentuale} & \textcolor{Red}{0} & $100$ & $100$\\ \hline
%\emph{Test superati} & {Percentuale} & \textcolor{Green}{100} & $90 - 100$ & $100$\\ \hline
%\emph{Branch Coverage} & {Percentuale} & \textcolor{Orange}{73.3} & $70 - 100$ & $80 - 100$\\ \hline
%\emph{Code Coverage} & {Percentuale} & \textcolor{Green}{76.57} & $60 - 100$ & $70 - 100$\\ \hline
\caption{RP-Metriche di qualità di processo}
\end{longtable}

\subsubsection{Esiti delle metriche ripetute nel tempo}

\begin{figure}[H]
	\centering 
	\includegraphics[scale=0.7]{Sezioni/Immagini/ScheduleVariance-RP}
	\caption{Schedule variance - RP}
\end{figure}

\begin{figure}[H]
	\centering 
	\includegraphics[scale=0.7]{Sezioni/Immagini/BudgetVariance-RP}
	\caption{Budget variance - RP}
\end{figure}

\begin{figure}[H]
	\centering 
	\includegraphics[scale=0.85]{Sezioni/Immagini/LivelloInstabilitaSDK-RP}
	\caption{Livello di instabilità SDK - RP}
\end{figure}

\begin{figure}[H]
	\centering 
	\includegraphics[scale=0.85]{Sezioni/Immagini/LivelloInstabilitaApp-RP}
	\caption{Livello di instabilità Applicazione - RP}
\end{figure}

\begin{figure}[H]
	\centering 
	\includegraphics[scale=0.63]{Sezioni/Immagini/DistanzaSDK-RP}
	\caption{Distanza dalla sequenza principale SDK - RP}
\end{figure}

\begin{figure}[H]
	\centering 
	\includegraphics[scale=0.63]{Sezioni/Immagini/DistanzaApp-RP}
	\caption{Distanza dalla sequenza principale Applicazione - RP}
\end{figure}

\subsubsection{Livello dei processi}
\begin{longtable}{|>{\centering}m{6cm}|c|c|c|c|c|}
\hline
\textbf{Processo} & \textbf{Livello} & \textbf{Esito} & \textbf{Valore}\\
\hline
\endhead
\emph{Processo di fornitura} & \textcolor{Green}{2} & Superato & Ottimale\\ \hline
\emph{Processo di sviluppo} & \textcolor{Green}{2}* & Superato & Accettabile\\ \hline
\emph{Processo di documentazione} & \textcolor{Green}{2} & Superato & Accettabile\\ 
\hline
\emph{Processo di Configurazione} & \textcolor{Green}{1} & Superato & Ottimale\\ 
\hline
\emph{Processo di garanzia di qualità del Prodotto} & * & / & /\\ 
\hline
\emph{Processo di Verifica} & \textcolor{Green}{1} & Superato & Ottimale\\ 
\hline
\emph{Processo di Validazione} & * & / & /\\ 
\hline
\emph{Processo di Risoluzione dei problemi} & \textcolor{Green}{1} & Superato & Ottimale\\ 
\hline
\emph{Processo di Coordinamento} & \textcolor{Green}{1} & Superato & Ottimale\\ 
\hline
\emph{Processo di Pianificazione} & \textcolor{Green}{1} & Superato & Ottimale\\ 
\hline
\emph{Processo di Formazione} & \textcolor{Green}{1} & Superato & Ottimale\\ 
\hline
\caption{RP-Livello dei processi}
\end{longtable}

Per i processi con segnatura \texttt{Voto*} vengono considerate solo le attività inerenti al lavoro che deve essere svolto per la \textit{Revisione di progettazione}. Per quelli, invece, con solo \texttt{*} significa che nessuna attività di quel processo era necessaria per il raggiungimento della milestone esterna.

\newpage

\subsection{Revisione di Qualifica}

\subsubsection{Analisi statica dei documenti}
L'analisi statica dei documenti è stata fatta mediante \termine{Walkthrough} ed ha portato all'individuazione di alcuni errori. Tra gli errori individuati quelli più frequenti sono stati:
		\begin{itemize}
			\item Errori ortografici.
			\item Frasi complesse con un basso indice di comprensibilità.
		\end{itemize}

\subsection{Metriche per i documenti}

\subsubsection{Indice di Gulpease}

\begin{table}[h]
	\begin{center}
		\begin{tabular}{|c|c|c|c|c|}
			\hline
			\textbf{Documento}	& \textbf{Risultato} & \textbf{Esito} & \textbf{Valore}\\
			\hline
		    \termine{Analisi} dei Requisiti v3.0.0 & 87 & Superato & Ottimale\\
			\hline
			Glossario v3.0.0 & 55 & Superato & Ottimale\\
			\hline
			Norme di Progetto v3.0.0 & 46 & Superato & Accettabile\\
			\hline
			Piano di Progetto v3.0.0 & 48 & Superato & Accettabile\\
			\hline
			Piano di Qualifica v3.0.0 & 40 & Superato & Accettabile\\
			\hline
		 \termine{Manuale Utente} \termine{Monolith} v1.0.0 & 93 & Superato & Ottimale\\
			\hline
		 \termine{Manuale Utente} Bringit v1.0.0 & 48 & Superato & Accettabile\\
            \hline
            Verbale\_Interno\_7\_20170327 v1.0.0 & 67 & Superato & Ottimale\\
            \hline
            Verbale\_Interno\_8\_20170402 v1.0.0 & 73 & Superato & Ottimale\\
            \hline
            Verbale\_Interno\_9\_20170407 v1.0.0 & 78 & Superato & Ottimale\\
            \hline
            Verbale\_Esterno\_10\_20170503 v1.0.0 & 73 & Superato & Ottimale\\
            \hline
		\end{tabular}
	\end{center}
	\caption{RQ - Risultato indice di Gulpease}
\end{table}


\subsubsection{Soddisfacimento metriche}
\small{
Si noti che alcune componenti e i test a loro correlati non sono ancora state sviluppate, e che la loro assenza peserà nelle misurazioni.}

\paragraph{Qualità di processo}
\begin{longtable}{|>{\centering}m{5cm}|c|c|c|c|c|}
\hline
\textbf{Metrica} & \textbf{Unità di misura} & \textbf{Risultato} & \textbf{Risultato massimo} & \textbf{Esito} & \textbf{Valore}\\
\hline
\endhead
\emph{Schedule Variance} & {Attività} & \textcolor{Green}{0} & / & Superato & Ottimale\\ \hline
\emph{Budget Variance} & {Euro} & \textcolor{Orange}{-697} & / & Non superato & /\\ \hline
\emph{Ottimalità delle misurazioni} & {Percentuale} & \textcolor{Green}{74\%} & / & Superato & Ottimale \\ \hline
\emph{Rischi non preventivati} & {Rischi} & \textcolor{Green}{2} & / & Superato & Accettabile\\ \hline
\emph{Requisiti obbligatori soddisfatti} & {Percentuale} & \textcolor{Orange}{97.6\%} & / & Non superato & /\\ \hline
\emph{Livello di instabilità-SDK} & {Percentuale} & / & \textcolor{Green}{55\%} & Superato & Ottimale\\ \hline
\emph{Livello di instabilità-Applicazione} & {Percentuale} & / & \textcolor{Green}{70\%} & Superato & Accettabile\\ \hline
\emph{Astrattezza-SDK} & {Percentuale} & \textcolor{Green}{20\%} & / & Superato & Ottimale\\ \hline
\emph{Astrattezza-Applicazione} & {Percentuale} &\textcolor{Green}{15\%} & / & Superato & Ottimale\\ \hline
\emph{Distanza dalla sequenza principale-SDK} & {Percentuale} & / & \textcolor{Green}{25\%} & Superato & Ottimale\\ \hline
\emph{Distanza dalla sequenza principale-Applicazione} & {Percentuale} & / & \textcolor{Green}{15\%} & Superato & Ottimale\\ \hline
\emph{Numero di metodi per classe (max)} & {Metodi} & / & \textcolor{Green}{12} & Superato & Accettabile\\ \hline
\emph{Numero di attributi per classe (max)} & {Attributi} & / & \textcolor{Green}{7} & Superato & Ottimale\\ \hline
\emph{Numero di parametri per metodo (max)} & {Parametri} & / & \textcolor{Green}{5} & Superato & Ottimale\\ \hline
\emph{Complessità Ciclomatica media} & {Cammini} & \textcolor{Green}{1.3} & / & Superato & Ottimale\\ \hline
\emph{Livelli di annidamento medi} & {Chiamate} & \textcolor{Green}{2.4} & / & Superato & Ottimale\\ \hline
\emph{Linee di codice per linee di commento - Monolith} & {Percentuale} & \textcolor{Green}{28.3\%} & / & Superato & Ottimale\\ \hline
\emph{Linee di codice per linee di commento - BringIt} & {Percentuale} & \textcolor{Green}{20.5\%} & / & Superato & Ottimale\\ \hline
\emph{Componenti integrate} & {Percentuale} & \textcolor{Orange}{82\%} & / & Non superato & /\\ \hline
\emph{Test di Unità eseguiti} & {Percentuale} & \textcolor{Green}{99\%} & / & Superato & Ottimale\\ \hline
\emph{Test di Integrazione eseguiti} & {Percentuale} & \textcolor{Green}{84.7\%} & / & Superato & Ottimale\\ \hline
\emph{Test di Sistema eseguiti} & {Percentuale} & \textcolor{Green}{86\%} & / & Superato & Ottimale\\ \hline
\emph{Test di \termine{Validazione} eseguiti} & {Percentuale} & \textcolor{Orange}{46\%} & / & Non superato & / \\ \hline
\emph{Test superati} & {Percentuale} & \textcolor{Green}{100\%} & / & Superato & Ottimale \\ \hline
\emph{Branch Coverage} & {Percentuale} & \textcolor{Green}{83\%} & / & Superato & Accettabile\\ \hline
\emph{Statement Coverage} & {Percentuale} & \textcolor{Green}{77.4\%} & / & Superato & Accettabile\\ \hline
\emph{Accuratezza rispetto alle attese} & {Percentuale} & \textcolor{Green}{100} & / & Superato & Ottimale\\ \hline
\emph{Completezza dell’implementazione funzionale} & {Percentuale} & \textcolor{Orange}{86} & / & Non superato & /\\ \hline
\emph{Densità di failure} & {Percentuale} & \textcolor{Green}{0} & / & Superato & Ottimale\\ \hline
\emph{Blocco di operazioni non corrette} & {Percentuale} & \textcolor{Green}{90} & / & Superato & Accettabile\\ \hline
\emph{Tempo di risposta} & {Secondi} & / & \textcolor{Green}{1.0} & Superato & Ottimale\\ \hline
\emph{Capacità di analisi di failure} & {Percentuale} & \textcolor{Green}{100} & / & Superato & Ottimale\\ \hline
\emph{Impatto delle modifiche} & {Indice} & \textcolor{Green}{2} & / & Superato & Ottimale\\ \hline
\caption{RQ-Metriche di qualità di processo}\\
\end{longtable}

\subsubsection{Esiti delle metriche ripetute nel tempo}

\begin{figure}[H]
	\centering 
	\includegraphics[scale=0.7]{Sezioni/Immagini/ScheduleVariance-RQ}
	\caption{Schedule variance - RQ}
\end{figure}

\begin{figure}[H]
	\centering 
	\includegraphics[scale=0.7]{Sezioni/Immagini/BudgetVariance-RQ}
	\caption{Budget variance - RQ}
\end{figure}

\begin{figure}[H]
	\centering 
	\includegraphics[scale=0.85]{Sezioni/Immagini/LivelloInstabilitaSDK-RQ}
	\caption{Livello di instabilità Monolith - RQ}
\end{figure}

\begin{figure}[H]
	\centering 
	\includegraphics[scale=0.85]{Sezioni/Immagini/LivelloInstabilitaApp-RQ}
	\caption{Livello di instabilità Bringit - RQ}
\end{figure}

\begin{figure}[H]
	\centering 
	\includegraphics[scale=0.85]{Sezioni/Immagini/DistanzaSDK-RQ}
	\caption{Distanza dalla sequenza principale Monolith - RQ}
\end{figure}

\begin{figure}[H]
	\centering 
	\includegraphics[scale=0.85]{Sezioni/Immagini/DistanzaApp-RQ}
	\caption{Distanza dalla sequenza principale Bringit - RQ}
\end{figure}

\begin{figure}[H]
	\centering 
	\includegraphics[scale=0.8]{Sezioni/Immagini/LineeCodiceCommentoSDK-RQ}
	\caption{Linee di codice per linee di commento Monolith - RQ}
\end{figure}

\begin{figure}[H]
	\centering 
	\includegraphics[scale=0.8]{Sezioni/Immagini/LineeCodiceCommentoApp-RQ}
	\caption{Linee di codice per linee di commento Bringit - RQ}
\end{figure}

\subsubsection{Livello dei processi}
\begin{longtable}{|>{\centering}m{6cm}|c|c|c|c|c|}
\hline
\textbf{Processo} & \textbf{Livello} & \textbf{Esito} & \textbf{Valore}\\
\hline
\endhead
\emph{Processo di fornitura} & \textcolor{Green}{2} & Superato & Ottimale\\ \hline
\emph{Processo di sviluppo} & \textcolor{Green}{2}* & Superato & Accettabile\\ \hline
\emph{Processo di documentazione} & \textcolor{Green}{2} & Superato & Accettabile\\ 
\hline
\emph{Processo di Configurazione} & \textcolor{Green}{1} & Superato & Ottimale\\ 
\hline
\emph{Processo di garanzia di qualità del Prodotto} & * & / & /\\ 
\hline
\emph{Processo di Verifica} & \textcolor{Green}{1} & Superato & Ottimale\\ 
\hline
\emph{Processo di Validazione} & * & / & /\\ 
\hline
\emph{Processo di Risoluzione dei problemi} & \textcolor{Green}{1} & Superato & Ottimale\\ 
\hline
\emph{Processo di Coordinamento} & \textcolor{Green}{1} & Superato & Ottimale\\ 
\hline
\emph{Processo di Pianificazione} & \textcolor{Green}{1} & Superato & Ottimale\\ 
\hline
\emph{Processo di Formazione} & \textcolor{Green}{1} & Superato & Ottimale\\ 
\hline
\caption{RQ-Livello dei processi}
\end{longtable}

\newpage

\subsection{Revisione di Accettazione}

\subsubsection{Analisi statica dei documenti}
L'analisi statica dei documenti è stata fatta mediante \termine{Walkthrough}, essa ha portato all'individuazione soltanto di errori di battitura o comunque non di errori gravi, causati da idee concettualmente sbagliate.

\subsection{Metriche per i documenti}

\subsubsection{Indice di Gulpease}

\begin{table}[h]
	\begin{center}
		\begin{tabular}{|c|c|c|c|c|}
			\hline
			\textbf{Documento}	& \textbf{Risultato} & \textbf{Esito} & \textbf{Valore}\\
			\hline
		    \termine{Analisi} dei Requisiti v3.0.0 &  & Superato & Ottimale\\
			\hline
			Glossario v4.0.0 &  & Superato & Ottimale\\
			\hline
			Norme di Progetto v4.0.0 &  & Superato & Accettabile\\
			\hline
			Piano di Progetto v4.0.0 &  & Superato & Accettabile\\
			\hline
			Piano di Qualifica v4.0.0 &  & Superato & Accettabile\\
			\hline
		 \termine{Manuale Utente} \termine{Monolith} v2.0.0 &  & Superato & Ottimale\\
			\hline
		 \termine{Manuale Utente} Bringit v2.0.0 &  & Superato & Accettabile\\
            \hline
		\end{tabular}
	\end{center}
	\caption{RA - Risultato indice di Gulpease}
\end{table}


\subsubsection{Soddisfacimento metriche}
\small{
Si noti che alcune componenti e i test a loro correlati non sono ancora state sviluppate, e che la loro assenza peserà nelle misurazioni.}

\paragraph{Qualità di processo}
\begin{longtable}{|>{\centering}m{5cm}|c|c|c|c|c|}
\hline
\textbf{Metrica} & \textbf{Unità di misura} & \textbf{Risultato} & \textbf{Risultato massimo} & \textbf{Esito} & \textbf{Valore}\\
\hline
\endhead
\emph{Schedule Variance} & {Attività} & \textcolor{Green}{0} & / & Superato & Ottimale\\ \hline
\emph{Budget Variance} & {Euro} & \textcolor{Orange}{-18} & / & Non superato & /\\ \hline
\emph{Ottimalità delle misurazioni} & {Percentuale} & \textcolor{Green}{81\%} & / & Superato & Ottimale \\ \hline
\emph{Rischi non preventivati} & {Rischi} & \textcolor{Green}{0} & / & Superato & Ottimale\\ \hline
\emph{Requisiti obbligatori soddisfatti} & {Percentuale} & \textcolor{Green}{100\%} & / & Superato & Ottimale\\ \hline
\emph{Livello di instabilità-SDK} & {Percentuale} & / & \textcolor{Green}{55\%} & Superato & Ottimale\\ \hline
\emph{Livello di instabilità-Applicazione} & {Percentuale} & / & \textcolor{Green}{60\%} & Superato & Accettabile\\ \hline
\emph{Astrattezza-SDK} & {Percentuale} & \textcolor{Green}{20\%} & / & Superato & Ottimale\\ \hline
\emph{Astrattezza-Applicazione} & {Percentuale} &\textcolor{Green}{15\%} & / & Superato & Ottimale\\ \hline
\emph{Distanza dalla sequenza principale-SDK} & {Percentuale} & / & \textcolor{Green}{10\%} & Superato & Ottimale\\ \hline
\emph{Distanza dalla sequenza principale-Applicazione} & {Percentuale} & / & \textcolor{Green}{25\%} & Superato & Ottimale\\ \hline
\emph{Numero di metodi per classe (max)} & {Metodi} & / & \textcolor{Green}{12} & Superato & Accettabile\\ \hline
\emph{Numero di attributi per classe (max)} & {Attributi} & / & \textcolor{Green}{7} & Superato & Ottimale\\ \hline
\emph{Numero di parametri per metodo (max)} & {Parametri} & / & \textcolor{Green}{5} & Superato & Ottimale\\ \hline
\emph{Complessità Ciclomatica media} & {Cammini} & \textcolor{Green}{1.3} & / & Superato & Ottimale\\ \hline
\emph{Livelli di annidamento medi} & {Chiamate} & \textcolor{Green}{2.4} & / & Superato & Ottimale\\ \hline
\emph{Linee di codice per linee di commento - Monolith} & {Percentuale} & \textcolor{Green}{28.3\%} & / & Superato & Ottimale\\ \hline
\emph{Linee di codice per linee di commento - BringIt} & {Percentuale} & \textcolor{Green}{20.5\%} & / & Superato & Ottimale\\ \hline
\emph{Componenti integrate} & {Percentuale} & \textcolor{Green}{100\%} & / & Superato & Ottimale\\ \hline
\emph{Test di Unità eseguiti} & {Percentuale} & \textcolor{Green}{99\%} & / & Superato & Ottimale\\ \hline
\emph{Test di Integrazione eseguiti} & {Percentuale} & \textcolor{Green}{98\%} & / & Superato & Ottimale\\ \hline
\emph{Test di Sistema eseguiti} & {Percentuale} & \textcolor{Green}{94\%} & / & Superato & Ottimale\\ \hline
\emph{Test di \termine{Validazione} eseguiti} & {Percentuale} & \textcolor{Orange}{} & / & Non superato & / \\ \hline
\emph{Test superati} & {Percentuale} & \textcolor{Green}{100\%} & / & Superato & Ottimale \\ \hline
\emph{Branch Coverage} & {Percentuale} & \textcolor{Green}{86\%} & / & Superato & Accettabile\\ \hline
\emph{Statement Coverage} & {Percentuale} & \textcolor{Green}{80.2\%} & / & Superato & Accettabile\\ \hline
\emph{Accuratezza rispetto alle attese} & {Percentuale} & \textcolor{Green}{100} & / & Superato & Ottimale\\ \hline
\emph{Completezza dell’implementazione funzionale} & {Percentuale} & \textcolor{Green}{100\%} & / & Superato & Ottimale\\ \hline
\emph{Densità di failure} & {Percentuale} & \textcolor{Green}{0} & / & Superato & Ottimale\\ \hline
\emph{Blocco di operazioni non corrette} & {Percentuale} & \textcolor{Green}{100} & / & Superato & Ottimale\\ \hline
\emph{Tempo di risposta} & {Secondi} & / & \textcolor{Green}{1.0} & Superato & Ottimale\\ \hline
\emph{Capacità di analisi di failure} & {Percentuale} & \textcolor{Green}{100} & / & Superato & Ottimale\\ \hline
\emph{Impatto delle modifiche} & {Indice} & \textcolor{Green}{2} & / & Superato & Ottimale\\ \hline
\caption{RA-Metriche di qualità di processo}\\
\end{longtable}

\subsubsection{Esiti delle metriche ripetute nel tempo}

\begin{figure}[H]
	\centering 
	\includegraphics[scale=0.75]{Sezioni/Immagini/ScheduleVariance-RA}
	\caption{Schedule variance - RA}
\end{figure}

\begin{figure}[H]
	\centering 
	\includegraphics[scale=0.75]{Sezioni/Immagini/BudgetVariance-RA}
	\caption{Budget variance - RA}
\end{figure}

\begin{figure}[H]
	\centering 
	\includegraphics[scale=0.75]{Sezioni/Immagini/LivelloInstabilitaApp-RA}
	\caption{Livello di instabilità Bringit - RA}
\end{figure}

\begin{figure}[H]
	\centering 
	\includegraphics[scale=0.65]{Sezioni/Immagini/DistanzaApp-RA}
	\caption{Distanza dalla sequenza principale Bringit - RA}
\end{figure}

\begin{figure}[H]
	\centering 
	\includegraphics[scale=0.6]{Sezioni/Immagini/LineeCodiceCommentoApp-RA}
	\caption{Linee di codice per linee di commento Bringit - RA}
\end{figure}

\subsubsection{Livello dei processi}
\begin{longtable}{|>{\centering}m{6cm}|c|c|c|c|c|}
\hline
\textbf{Processo} & \textbf{Livello} & \textbf{Esito} & \textbf{Valore}\\
\hline
\endhead
\emph{Processo di fornitura} & \textcolor{Green}{2} & Superato & Ottimale\\ \hline
\emph{Processo di sviluppo} & \textcolor{Green}{3} & Superato & Ottimale\\ \hline
\emph{Processo di documentazione} & \textcolor{Green}{2} & Superato & Accettabile\\ 
\hline
\emph{Processo di Configurazione} & \textcolor{Green}{1} & Superato & Ottimale\\ 
\hline
\emph{Processo di garanzia di qualità del Prodotto} & \textcolor{Green}{1} & Superato & Ottimale\\ 
\hline
\emph{Processo di Verifica} & \textcolor{Green}{2} & Superato & Ottimale\\ 
\hline
\emph{Processo di Validazione} & \textcolor{Green}{2} & Superato & Ottimale\\ 
\hline
\emph{Processo di Risoluzione dei problemi} & \textcolor{Green}{1} & Superato & Ottimale\\ 
\hline
\emph{Processo di Coordinamento} & \textcolor{Green}{1} & Superato & Ottimale\\ 
\hline
\emph{Processo di Pianificazione} & \textcolor{Green}{1} & Superato & Ottimale\\ 
\hline
\emph{Processo di Formazione} & \textcolor{Green}{1} & Superato & Ottimale\\ 
\hline
\caption{RA-Livello dei processi}
\end{longtable}



\end{appendices}




\end{document}