% Questo file definisce lo stile che verrà applicato
% ad ogni pagina di contenuto
\documentclass[a4paper,11pt]{article}

\usepackage{ifthen}
\usepackage[
 a4paper,
 top=2.5cm,
 bottom=2.5cm,
 left=1.5cm,
 right=1.5cm,
 head=30pt
]{geometry}
\usepackage[italian]{babel}
\usepackage[utf8x]{inputenc}
\usepackage[T1]{fontenc}
\usepackage{fancyhdr}
\usepackage[colorlinks=true, urlcolor=black, citecolor=black, linkcolor=black]{hyperref}
\usepackage{tabularx}
\usepackage{multirow}
\usepackage{booktabs}
\usepackage{color}
\usepackage{graphicx}
\usepackage{eurosym}
\usepackage{amsmath}
\usepackage{relsize}

\usepackage[multidot]{grffile}
\usepackage{xcolor,colortbl}
\definecolor{lightblue}{HTML}{56B4E6}
\definecolor{blue}{HTML}{2953A1}
\definecolor{darkblue}{HTML}{1E396E}

\usepackage[toc,page]{appendix}
\renewcommand\appendixtocname{Appendice}
\renewcommand{\appendixpagename}{Appendice}

\newcommand\pagenumberingnoreset[1]{\gdef\thepage{\csname @#1\endcsname\c@page}}

% Cambia il font 
\renewcommand*\rmdefault{qhv}

% ***STILE PAGINA***
\pagestyle{fancy}
\fancyhf{}
\setlength{\headheight}{1cm} 
% No indentazione paragrafo
\setlength{\parindent}{0pt}

% ***INTESTAZIONE***
\newcommand\textline[4][t]{%
  \noindent\parbox[#1]{.333\textwidth}{\raisebox{-0.40\height}{#2}}%
  \parbox[#1]{.333\textwidth}{\centering #3}%
  \parbox[#1]{.333\textwidth}{\raggedleft #4}%
}

\lhead{
	\textline[t]{\includegraphics[width=1cm, keepaspectratio=true]{../../../Template/Logo/Logo.png}}{\progettoShort}{\documento}
}

\renewcommand{\headrulewidth}{0.4pt}  %Linea sotto l'intestazione

% ***PIÈ DI PAGINA***
\lfoot{\textit{\gruppoLink}\\ \footnotesize{\email}}
\rfoot{\thepage} %per le prime pagine: mostra solo il numero romano
\cfoot{}
\renewcommand{\footrulewidth}{0.4pt}   %Linea sopra il piè di pagina


% Ridefinisce command \paragraph{} andando a capo ogni dopo la parola dentro le parentesi ed ha la possibiltà di enumerazione fino a n cifre modificando il numero dentro "secnumdepth"
\usepackage{titlesec}

\setcounter{secnumdepth}{7}
\setcounter{tocdepth}{7}
%
%
%\titleformat{\paragraph}
%{\normalfont\normalsize\bfseries}{\theparagraph}{1em}{}
%\titlespacing*{\paragraph}
%{0pt}{3.25ex plus 1ex minus .2ex}{1.5ex plus .2ex}
%
%
%\titleclass{\subsubparagraph}{straight}[\subparagraph]
%\newcounter{subsubparagraph}
%
%\titleformat{\subsubparagraph}[display]
%  {\normalfont\normalsize\bf}
%  {\thesubsubparagraph.}
%  {.5em}
%  {}
%\renewcommand\thesubsubparagraph\textbf{\roman{subsubparagraph}}
%\titlespacing*{\subsubparagraph} {0pt}{4pt}{6pt}


%***LA SOTTOSEZIONE PARAGRAPH VIENE VISUALIZZATA COME UNA SECTION
\titleformat{\paragraph}{\normalfont\normalsize\bfseries}{\theparagraph}{1em}{}
\titlespacing*{\paragraph}{0pt}{3.25ex plus 1ex minus .2ex}{1.5ex plus .2ex}

\titleformat{\subparagraph}{\normalfont\normalsize\bfseries}{\thesubparagraph}{1em}{}
\titlespacing*{\subparagraph}{0pt}{3.25ex plus 1ex minus .2ex}{1.5ex plus .2ex}

\makeatletter
\newcounter{subsubparagraph}[subparagraph]
\renewcommand\thesubsubparagraph{%
  \thesubparagraph.\@arabic\c@subsubparagraph}
\newcommand\subsubparagraph{%
  \@startsection{subsubparagraph}    % counter
    {6}                              % level
    {\parindent}                     % indent
    {3.25ex \@plus 1ex \@minus .2ex} % beforeskip
    {0.75em}                           % afterskip
    {\normalfont\normalsize\bfseries}}
\newcommand\l@subsubparagraph{\@dottedtocline{6}{13em}{5.5em}} %gestione dell'indice
\newcommand{\subsubparagraphmark}[1]{}
\makeatother

\makeatletter
\newcounter{subsubsubparagraph}[subsubparagraph]
\renewcommand\thesubsubsubparagraph{%
  \thesubsubparagraph.\@arabic\c@subsubsubparagraph}
\newcommand\subsubsubparagraph{%
  \@startsection{subsubsubparagraph}    % counter
    {7}                              % level
    {\parindent}                     % indent
    {3.25ex \@plus 1ex \@minus .2ex} % beforeskip
    {0.75em}                           % afterskip
    {\normalfont\normalsize\bfseries}}
\newcommand\l@subsubsubparagraph{\@dottedtocline{7}{16em}{6.5em}} %gestione dell'indice
\newcommand{\subsubsubparagraphmark}[1]{}
\makeatother

%Generali
\newcommand{\capitolato}{C5 - Monolith: An interactive bubble provider}
\newcommand{\progettoShort}{Monolith}
\newcommand{\progetto}{Monolith: An interactive bubble provider}
\newcommand{\gruppo}{NPE Developers}
\newcommand{\gruppoLink}{\href{https://gitlab.com/npe-developers}{NpeDevelopers}}
\newcommand{\email}{\href{mailto:npe.developers@gmail.com}{\textcolor{blue}{npe.developers@gmail.com}}}
\newcommand{\password}{NP3Devel0pers}
\newcommand{\myincludegraphics}[2][]{%
	\setbox0=\hbox{\phantom{X}}%
	\vtop{
		\hbox{\phantom{X}}
		\vskip-\ht0
		\hbox{\includegraphics[#1]{#2}}}
}




%Componenti del gruppo
\newcommand{\RM}{Riccardo Montagnin}
\newcommand{\MT}{Manuel Turetta}
\newcommand{\FB}{Francesco Bazzerla}
\newcommand{\SL}{Stefano Lia}
\newcommand{\LD}{Luca Dario}
\newcommand{\DC}{Diego Cavestro}
\newcommand{\ND}{Nicolò Dovico}

%Ruoli
\newcommand{\Pm}{Project Manager}
\newcommand{\Am}{Amministratore}
\newcommand{\AmP}{Amministratori}
\newcommand{\An}{Analista}
\newcommand{\AnP}{Analisti}
\newcommand{\Dev}{Sviluppatore}
\newcommand{\DevP}{Sviluppatori}
\newcommand{\Ver}{Verificatore}
\newcommand{\VerP}{Verificatori}
\newcommand{\Progr}{Programmatore}
\newcommand{\ProgrP}{Programmatori}
\newcommand{\Prog}{Progettista}
\newcommand{\ProgP}{Progettisti}



%Firme
\newcommand{\RMFirma}{\myincludegraphics[scale = 0.5]{../../../Template/Firme/RM.png}}
\newcommand{\MTFirma}{\myincludegraphics[scale = 0.5]{../../../Template/Firme/MT.png}}
\newcommand{\FBFirma}{\myincludegraphics[scale = 0.5]{../../../Template/Firme/FB.png}}
\newcommand{\SLFirma}{\myincludegraphics[scale = 0.5]{../../../Template/Firme/SL.png}}
\newcommand{\LDFirma}{\myincludegraphics[scale = 0.5]{../../../Template/Firme/LD.png}}
\newcommand{\DCFirma}{\myincludegraphics[scale = 0.5]{../../../Template/Firme/DC.png}}
\newcommand{\NDFirma}{\myincludegraphics[scale = 0.5]{../../../Template/Firme/ND.png}}

%Professori e proponente
\newcommand{\TV}{Prof. Tullio Vardanega}
\newcommand{\RC}{Prof. Riccardo Cardin}
\newcommand{\RB}{Red Babel}
\newcommand{\proponente}{Red Babel}

%Documenti
\newcommand{\Gl}{Glossario}
\newcommand{\glossario}{\textit{\Gl\_v.1.0.0.pdf}}
\newcommand{\AdR}{Analisi dei Requisiti}
\newcommand{\analisiDeiRequisiti}{\textit{\AdR\_v.1.0.0.pdf}}
\newcommand{\AdRvDue}{AnalisiDeiRequisiti}
\newcommand{\NdP}{Norme di Progetto}
\newcommand{\normeDiProgetto}{\textit{\NdP\_v.1.0.0.pdf}}
\newcommand{\PdP}{Piano di Progetto}
\newcommand{\pianoDiProgetto}{\textit{\PdP\_v.1.0.0.pdf}}
\newcommand{\SdF}{Studio di Fattibilità}
\newcommand{\studioDiFattibilita}{\textit{\SdF\_v.1.0.0.pdf}}
\newcommand{\PdQ}{Piano di Qualifica}
\newcommand{\pianoDiQualifica}{\textit{\PdQ\_v.1.0.0.pdf}}
\newcommand{\VI}{Verbale Interno}
\newcommand{\VE}{Verbale Esterno}
\newcommand{\ST}{Specifica Tecnica}
\newcommand{\MU}{Manuale Utente}
\newcommand{\DDP}{Definizione di Prodotto}

%Periodo di progetto
\newcommand{\ARM}{Analisi dei Requisiti di Massima}
\newcommand{\ARD}{Analisi dei Requisiti in Dettaglio}
\newcommand{\PA}{Progettazione Architetturale}
\newcommand{\PD}{Progettazione di Dettaglio}
\newcommand{\COD}{Codifica}
\newcommand{\VV}{Verifica e Validazione Finale}

%Consegne
\newcommand{\RR}{Revisione dei Requisiti}
\newcommand{\RP}{Revisione di Progettazione}
\newcommand{\RQ}{Revisione di Qualifica}
\newcommand{\RA}{Revisione di Accettazione}


%Formattazione
\newcommand{\termine}[1]{\textit{#1}\small{$_G$}}
\newcommand{\link}[1]{\href{#1}{\textcolor{blue}{\texttt{#1}}}} 

% Testi ricorrenti
\newcommand{\scopoProdotto}{L'obiettivo di questo progetto è la realizzazione di un \termine{SDK} che permetta la creazione di bolle interattive, le quali, successivamente, verranno utilizzate all'interno dell'applicazione di messaggistica istantanea open source \termine{Rocket.chat}. \\
Dopo la realizzazione di tale \termine{SDK}, è proposto lo sviluppo di un'applicazione in grado di sfruttare l'\termine{SDK} per implementare un uso originale di tali bolle.
}
\newcommand{\descrizioneGlossario}{Al fine di mantenere questo documento compatto e di facile lettura è stato realizzato un glossario esterno contenente tutte le definizioni dei termini che più comunemente verranno presentati al lettore.  
Tale glossario si ritrova all'interno del file \glossario, e contiene tutti e soli i termini che vengono marcati con una \textit{G} a pedice.
}
\newcommand{\riferimentiNormativi}{
	\begin{itemize}
		\item \textbf{Norme di Progetto}: \normeDiProgetto
		\item \textbf{\termine{Capitolato} d'appalto C5: Monolith - An Interactive bubble provider} \\
			  \link{http://www.math.unipd.it/~tullio/IS-1/2016/Progetto/C5.pdf}
	\end{itemize}
}

% Comandi per generare l'intro
\newcommand{\documento}{\PdQ}
\newcommand{\versione}{1.0.1}
\newcommand{\redatori}{\RM\\ & \MT\\ & \FB}
\newcommand{\revisori}{\DC\\ & \RM}
\newcommand{\approvazione}{\LD}
\newcommand{\statoapprovazione}{Da approvare}
% Quando il documento sarà approvato, inserire all'interno del comando seguente la data nel formato GG mese AAAA dove GG è il giorno a due cifre, mese è il mese scritto per esteso con la prima lettera minuscola, e AAAA è l'anno a quattro cifre
\newcommand{\dataApprovazione}{}
\newcommand{\uso}{Esterno}
\newcommand{\destinatari}{\TV\\ & \RC\\ & \RB}

\newcommand{\sommario}{Questo documento si prefigge di regolamentare le operazioni di verifica del gruppo \gruppo\ necessarie ad assicurare i requisiti qualitativi per il progetto \progetto.
}

\newcommand{\modifiche}{
	Verifica sezioni 1 e 4 & \RM & \Prog & 06/03/2017 & 0.1.0 \\\midrule
	Stesura appendice 1 & \RM & \Prog & 03/03/2017 & 0.0.5 \\\midrule
	Stesura sezione 4 & \DC & \Prog & 28/02/2017 & 0.0.4 \\\midrule
	Stesura sezione 4 & \FB & \Prog & 28/02/2017 & 0.0.3 \\\midrule
	Stesura sezione 1 & \FB & \Prog & 28/02/2017 & 0.0.2 \\\midrule
    Creazione del template & \FB & \Prog & 28/02/2017 & 0.0.1 \\\midrule
}

\begin{document}

\input{../../../Template/Intro.tex}
%Questo file si occupa di generare la tabella delle modifiche
\pagenumbering{Roman}

\begin{center}
    \Large{\textbf{Registro delle modifiche}}
    	\\\vspace{0.5cm}
    	\normalsize
    \begin{tabularx}{\textwidth}{cXXcc}
        \textbf{Versione} & \textbf{Modifica - Motivazione} & \textbf{Autore} & \textbf{Ruolo} & \textbf{Data} \\\toprule
        \modifiche
    \end{tabularx}
\end{center}

\newpage



\input{../../../Template/Indice.tex}

% Sezioni
\section{Introduzione}
\subsection{Scopo del documento}
Questo documento vuole definire le strategie che il \termine{team} ha deciso di adottare per perseguire gli obiettivi di qualità di processo e di prodotto ricercati. A tal fine è necessaria una costante attività di verifica e validazione del lavoro svolto in modo da poter rilevare e correggere le anomalie che potrebbero nascere.

\subsection{Scopo del prodotto}
\scopoProdotto

\subsection{Glossario}
\descrizioneGlossario

\subsection{Riferimenti}
\subsubsection{Normativi}
\riferimentiNormativi

\subsubsection{Informativi}
\begin{itemize}
	\item \textbf{\AdR}: \analisiDeiRequisiti;
	\item \textbf{\PdP}: \pianoDiProgetto;
	\item \textbf{\textit{Slide} dell'insegnamento di Ingegneria del Software}: \\
		  \link{http://www.math.unipd.it/~tullio/IS-1/2016/}
	\item \textbf{\textit{Standard} ISO/IEC 9126}: Product quality \\
	 	  \link{https://en.wikipedia.org/wiki/ISO/IEC\_9126}
	\item \textbf{\textit{Standard} tecnici ISO/IEC 15504}: Software process assessment \\
		  \link{https://en.wikipedia.org/wiki/ISO/IEC\_15504}
	\item \textbf{Ciclo di Deming (\termine{PDCA})}: Miglioramento dei processi \\
		  \link{https://en.wikipedia.org/wiki/PDCA}
\end{itemize}

\newpage
\section{Visione Generale della Strategia di Gestione della Qualità}
\subsection{Obiettivi di Qualità}
\subsubsection{Modello per la Qualità di Processo}
La qualità del prodotto è conseguenza anche della qualità dei processi che lo definiscono. Il \termine{gruppo} però, dopo una attenta valutazione, ha constatato che, essendo una qualità verificabile soltanto a lungo termine, non vi è il tempo materiale per assicurare una qualità di processo definita dall'\termine{ISO}. Il \termine{gruppo}, consapevole della scelta, cercherà comunque di prendere spunto, il più possibile, dallo standard ISO/IEC 15504.

\subsubsection{Standard per la Qualità di Prodotto}
\gruppo\ si impegna a seguire lo standard ISO/IEC 9126 redatto con lo scopo di descrivere obiettivi qualitativi e delineare delle metriche capaci di misurare il raggiungimento di tali obiettivi.

\subsubsection{Soluzioni attuate per il controllo della Qualità di Processo}

L'attuazione del metodo di gestione \termine{PDCA} aiuterà un maggior avvicinamento allo standard ISO/IEC 15504 ed assicurerà, quindi, una maggior qualità di processo. Con il ciclo \termine{PDCA} è possibile infatti garantire un miglioramento continuo dei processi, inclusa la verifica, ed un utilizzo ottimale delle risorse, ottenendo di conseguenza il miglioramento dei prodotti risultanti.
Per avere controllo sulla qualità è necessario che: 
\begin{itemize}
\item
I processi siano pianificati nel dettaglio;
\item
Nella pianificazione siano ripartite in modo chiaro le risorse;
\item
I processi vengano costantemente monitorati.
\end{itemize}

L'attuazione di tali punti è descritta dettagliatamente nel \PdP.
La qualità dei processi viene inoltre monitorata mediante l'analisi costante della qualità del prodotto e quantificata utilizzando le varie metriche che saranno descritte in seguito all'interno di questo documento.

\subsubsection{Soluzioni attuate per il controllo della Qualità di Prodotto}

Al fine di garantire un controllo sistematico della qualità del prodotto, il \termine{gruppo} seguirà le seguenti linee guida:
\begin{itemize}
\item
\textbf{\termine{Quality Assurance}}: insieme di attività realizzate per garantire il raggiungimento degli obiettivi di qualità. Prevede l'attuazione di tecniche di analisi statica e dinamica;
\item
\textbf{\termine{Verifica}}: processo che determina se il lavoro di un determinato periodo è consistente, completo e corretto. La verifica andrà eseguita costantemente durante l'intera durata del progetto;
\item
\textbf{\termine{Validazione}}: conferma in modo oggettivo che il sistema soddisfi correttamente i requisiti. \\ Anch'essa come la \termine{verifica} andrà eseguita costantemente durante l'intera durata del progetto.
\end{itemize}

\subsection{Scadenze Temporali}
Al fine di perseguire l'obbiettivo di rispettare le scadenze fissate nel Piano di Progetto è necessario che l'attività di verifica della documentazione sia sistematica e ben organizzata. Solo così, infatti, l'individuazione e la correzione di eventuali errori avverrà il prima possibile, impedendo la compromissione dell'intero progetto. \\
Ogni attività di redazione dei documenti e di codifica dovrà essere preceduta da uno studio preliminare sulla struttura e sui contenuti degli stessi, con lo scopo di ridurre la possibilità di commettere imprecisioni di natura concettuale e tecnica.

\subsubsection{Responsabilità}

Per ottenere un maggior livello di efficacia ed efficienza nell'attività di verifica verranno attribuite delle responsabilità a specifici ruoli di progetto.
La responsabilità, per l'attività di \termine{verifica} e \termine{validazione}, sarà a carico dei membri del \termine{gruppo} che al momento di eseguire tale attività saranno in carica dei ruoli di \Pm\ e dei \VerP.

\newpage
\newpage

\section{Capability Maturity Model}
Il \termine{Capability Maturity Model} (\textit{CMM}) è un modello di riferimento costituito
da pratiche consolidate in una disciplina specifica e viene utilizzato per stabilire la
capacità di una organizzazione o gruppo di operare in quella disciplina.

\subsection{I modelli CMM}
Esistono vari modelli di CMM che differiscono per:

\begin{itemize}
	\item Disciplina (software, sistemi, acquisizione, etc.);
	\item Struttura;
	\item Come viene definito il concetto di \textit{Maturity}, ossia il percorso di miglioramento del processo;
	\item Come viene definito il concetto di \textit{Capability}, ossia l'istituzionalizzazione.
\end{itemize}
Dal 1991 ad oggi sono stati sviluppati dei CMM per molte discipline; alcuni dei
più utilizzati includono modelli per l'ingegneria dei sistemi, l'ingegneria del software, l'acquisizione software, la gestione della forza-lavoro e dello sviluppo ed infine lo sviluppo
integrato di prodotti e processi. \\
L'uso di modelli differenti è risultato problematico a causa della confusione causata dall'utilizzo contemporaneo di più modelli stessi, e dei costi di integrazione di più di un modello in un unico programma coordinato di miglioramento.

\subsection{CMMI}
Per risolvere la confusione che veniva a crearsi con l'utilizzo di più modelli è nato
il progetto di \textit{Integrazione dei CMM} (\termine{Capability Maturity Model Integration -- CMMI}). \\
Nel modello CMMI si usano i livelli per descrivere un percorso di evoluzione raccomandato per un'organizzazione che voglia migliorare i processi che usa per sviluppare e mantenere i propri prodotti e servizi. I livelli presenti sono i seguenti:
\begin{itemize}
	\item \textbf{Livello 0}: Incomplete
	\item \textbf{Livello 1}: Performed
	\item \textbf{Livello 2}: Managed
	\item \textbf{Livello 3}: Defined
	\item \textbf{Livello 4}: Quantitatively Managed
	\item \textbf{Livello 5}: Optimizing
\end{itemize}

\subsubsection{Livello 0: Incomplete}
Un processo è incompleto quando non è stato eseguito oppure è stato eseguito solo parzialmente. Ciò significa che uno o più obiettivi specifici dell'area di processo non sono stati soddisfatti e/o non esistono obiettivi generici per questo livello in quanto non c'è ragione di istituzionalizzare un processo parzialmente eseguito.

\subsubsection{Livello 1: Performed}
Affinché un'area di processo sia al 1$^{\circ}$ livello di \textit{capability} bisogna mettere in pratica
le attività di base richieste per iniziare a lavorare su quella determinata area. \\
Un processo eseguito è tale quando soddisfa gli obiettivi specifici dell'area di processo.
Questo livello supporta e permette le attività necessarie per produrre i \termine{work product} e benché il processo eseguito abbia per risultato dei miglioramenti importanti, questi
potrebbero essere persi nel tempo se non vengono istituzionalizzati. L'istituzionalizzazione, ovvero il soddisfacimento delle pratiche generiche dei livelli di \textit{capability} dal 2 al 5, aiuta inoltre ad assicurare che i miglioramenti siano mantenuti.

\subsubsection{Livello 2: Managed}
Un processo viene definito gestito se viene eseguito e possiede le infrastrutture di base per
essere supportato. Ciò può verificarsi quando esso è organizzato ed eseguito in accordo ad alcune regole, impiega persone competenti che abbiano adeguate conoscenze per produrre dei risultati controllati, coinvolge gli \termine{stakeholder} più importanti ed è monitorato, controllato e revisionato. \\
Questo livello aiuta ad assicurare che le pratiche esistenti siano mantenute nel tempo.

\subsubsection{Livello 3: Defined}
Un processo definito è un processo gestito che è stato adattato in accordo con le linee guida di \termine{tailoring} dei processi standard dell'organizzazione. Esso contribuisce inoltre alla realizzazione dei \termine{work product}, alle misure ed ad altre informazioni di miglioramento. \\
I processi vengono gestiti in maniera preventiva attraverso la comprensione delle relazioni tra le attività di processo, le misure dettagliate, i \termine{work product} ed i servizi.

\subsubsection{Livello 4: Quantitatively Managed}
In questo livello il processo è un processo definito e controllato usando tecniche quantitative o statistiche. Gli obiettivi quantitativi per la qualità del processo sono stabiliti ed usati come criteri di gestione dei processi stessi. Infine, la performance del processo e della qualità è intesa in termini statistici e viene gestita attraverso tutta la vita del processo.

\subsubsection{Livello 5: Optimizing}
In questo caso si ha un livello di \textit{capability} pari a quello del livello precedente, con l'unica differenza che è stato migliorato sulla base di una comprensione delle più comuni cause di variazione del processo. Il punto focale a questo livello è un miglioramento continuo del range di esecuzione del processo attraverso miglioramenti innovativi ed incrementali.

\subsection{Scelte del team}
A causa della ridotta quantità di tempo messa a disposizione del \termine{team} per la realizzazione del progetto, è stato deciso unanimemente che non verrà eseguita alcuna attività con lo scopo di migliorare la qualità dei processi ma, allo stesso tempo, tutti i membri del \termine{gruppo} svolgeranno i propri compiti con l'obiettivo di provare a soddisfare sempre il livello 2 del \textit{CMMI}, senza però la garanzia di un totale raggiungimento di questo obiettivo. 

\newpage
\section{PDCA}
Il ciclo \termine{PDCA}, detto anche \termine{Ciclo di Deming}, definisce un metodo di controllo dei processi durante il loro ciclo di vita che consente di migliorarne continuamente la qualità. \\
Tale approccio è suddiviso in 4 fasi:
\begin{itemize}
\item \textbf{Plan}: fase di pianificazione, dove si individuano gli obiettivi e i processi necessari per il raggiungimento dei risultati attesi;
\item \textbf{Do}: fase di attuazione del piano individuato al passo precedente e raccolta di dati sulla qualità ottenuta;
\item \textbf{Check}: fase di verifica, dove si confrontano i risultati ottenuti (fase di Do) ed i risultati attesi (fase di Plan);
\item \textbf{Act}: fase in cui si determinano le cause delle differenze fra risultati ottenuti e risultati attesi, per decidere dove attuare eventuali azioni correttive per avere un effettivo miglioramento della qualità.
\end{itemize}

\subsection{Scelte del team}
A causa della mancanza di tempo e di risorse è stato deciso unanimemente da tutti i componenti del \termine{team} che non verrà istanziato alcun processo di automiglioramento della qualità dei processi ma, allo stesso tempo, ogni componente si impegnerà a svolgere tutti i compiti lui assegnati nel miglior modo possibile.

\newpage
\section{Standard ISO/IEC9126}
Lo standard ISO/IEC 9126 è stato redatto con lo scopo di descrivere quali sono gli obiettivi qualitativi che un prodotto software deve soddisfare. Questi vengono suddivisi in 3 aree tematiche diverse:
\begin{itemize}
	\item\textbf{Qualità esterna}: rappresenta la qualità del software nel momento in cui esso viene eseguito e testato;
	\item\textbf{Qualità interna}: rappresenta la qualità del software per quanto riguarda le sue caratteristiche implementative, durante le fasi di progettazione e codifica;
	\item\textbf{Qualità in uso}: rappresenta la qualità del software dal punto di vista del cliente che lo sta utilizzando.
\end{itemize}

Lo standard delinea sei macro-obbiettivi qualitativi, i quali sono suddivisi a loro volta in sotto caratteristiche specifiche:
\begin{itemize}
	\item\textbf{Funzionalità}: capacità del prodotto di fornire tutte le funzioni che sono state individuate attraverso l'\AdR:
	\begin{itemize}
		\item\textbf{Adeguatezza}: le funzionalità fornite devono essere conformi rispetto le aspettative;
		\item\textbf{Accuratezza}: il prodotto deve fornire i risultati attesi, con il livello di precisione richiesto;
		\item\textbf{Interoperabilità}: il prodotto deve poter interagire ed operare con uno o più sistemi specifici;
		\item\textbf{Sicurezza}: il prodotto deve proteggere le informazioni e i dati da accessi e modifiche non autorizzati;
		\item\textbf{Conformità di funzionalità}: il prodotto deve aderire a standard, regole e convenzioni inerenti alla funzionalità.
	\end{itemize}

	\item\textbf{Affidabilità}: capacità del prodotto software di svolgere correttamente le sue funzioni durante il suo utilizzo, anche nel caso in cui si presentino situazioni anomale:
	\begin{itemize}
		\item\textbf{Maturità}: il prodotto deve evitare che si verifichino malfunzionamenti o che vengano prodotti risultati non corretti;
		\item\textbf{Tolleranza agli errori}: nel caso in cui si presentino degli errori, dovuti a guasti o ad un uso scorretto dell'applicativo, questi devono essere gestiti in modo da mantenere alto il livello di prestazione;
		\item\textbf{Recuperabilità}: il prodotto deve essere in grado di ristabilire un
adeguato livello di prestazioni e di recuperare i dati rilevanti in seguito a errori o malfunzionamenti;
		\item\textbf{Conformità di affidabilità}: il prodotto deve aderire a standard, regole e convenzioni inerenti all'affidabilità.
	\end{itemize}

	\item\textbf{Usabilità}: capacità del prodotto di essere facilmente comprensibile e attraente in ogni sua parte per qualsiasi utente che lo andrà ad utilizzare:
	\begin{itemize}
		\item\textbf{Comprensibilità}: l'utente deve essere in grado di riconoscerne le funzionalità offerte dal software e deve comprenderne le modalità di utilizzo per riuscire a raggiungere i risultati attesi;
		\item\textbf{Apprendibilità}: deve essere data la possibilità all'utente di imparare ad utilizzare l'applicazione senza troppo impegno;
		\item\textbf{Operabilità}: le funzionalità presenti devono essere coerenti con le aspettative dell'utente;
		\item\textbf{Attrattiva}: il software deve essere piacevole per chi ne fa uso;
		\item\textbf{Conformità di usabilità}: il prodotto deve aderire a standard, regole e convenzioni inerenti all'usabilità.
	\end{itemize}

	\item\textbf{Efficienza}: capacità di eseguire le funzionalità offerte nel minor tempo possibile utilizzando al tempo stesso il minor numero di risorse possibili:
	\begin{itemize}
		\item\textbf{Comportamento rispetto al tempo}: per svolgere le sue funzioni il software deve fornire adeguati tempi di risposta ed elaborazione;
		\item\textbf{Utilizzo delle risorse}: il software quando esegue le sue funzionalità deve utilizzare un appropriato numero e tipo di risorse;
		\item\textbf{Conformità di efficienza}: il prodotto deve aderire a standard, regole e convenzioni inerenti all'efficienza.
	\end{itemize}

	\item\textbf{Manutenibilità}: capacità del prodotto di essere modificato, tramite correzioni, miglioramenti o adattamenti del software a cambiamenti negli ambienti, nei requisiti e nelle specifiche funzionali:
	\begin{itemize}
		\item\textbf{Analizzabilità}: il software deve consentire una rapida identificazione delle possibili cause di errori e malfunzionamenti;
		\item\textbf{Modificabilità}: il prodotto originale deve permettere eventuali cambiamenti in alcune sue parti;
		\item\textbf{Stabilità}: non devono insorgere effetti indesiderati in seguito a modifiche effettuate sul software;
		\item\textbf{Testabilità}: il software deve poter essere facilmente testato per validare le modifiche effettuate;
		\item\textbf{Conformità di manutenibilità}: il prodotto deve aderire a standard, regole e convenzioni inerenti alla manutenibilità.
	\end{itemize}

	\item\textbf{Portabilità}: capacità del software di poter essere utilizzato su diversi ambienti:
	\begin{itemize}
		\item\textbf{Adattabilità}: il prodotto deve adattarsi a tutti quegli ambienti di lavoro nei quali è stato previsto un suo utilizzo, senza dover apportare modifiche allo stesso;
		\item\textbf{Installabilità}: il software deve poter essere installato in determinati ambienti di lavoro;
		\item\textbf{Coesistenza}: il prodotto può coesistere in ambienti comuni
con altri software, condividendo risorse comuni;
		\item\textbf{Sostituibilità}: l'applicativo deve poter sostituire un altro software che ha lo stesso scopo e lavora nel medesimo ambiente;
		\item\textbf{Conformità di portabilità}: il prodotto deve aderire a standard, regole e convenzioni inerenti alla portabilità.
	\end{itemize}
\end{itemize}

\subsection{Scelte del team}
Per garantire una buona qualità di prodotto, il \termine{team} ha individuato dallo standard \textit{ISO/IEC 9126} le qualità che ritiene più importanti nell'arco del ciclo di vita del prodotto e le ha istanziate individuando obiettivi e metriche coerenti con i livelli di qualità perseguiti.

\subsection{Funzionalità}
Rappresenta la capacità del prodotto di fornire tutte le funzioni che sono state individuate attraverso l'\AdR.

\subsubsection{Obiettivi di qualità}
Il \termine{team} si impegnerà affinché:
\begin{itemize}
\item \textbf{Adeguatezza}: le funzionalità fornite siano conformi rispetto le aspettative;
\item \textbf{Accuratezza}: il prodotto fornisca i risultati attesi, con il livello di dettaglio richiesto;
\item \textbf{Sicurezza}: il prodotto protegga le informazioni e i dati da accessi e modifiche non autorizzati.
\end{itemize}

\subsubsection{Metriche}
\paragraph{Completezza dell'implementazione funzionale}
Indica la percentuale di requisiti funzionali coperti dall'implementazione.
\begin{itemize}
	\item \textbf{Misurazione}: 
		$$C=\left(1-\mathlarger{\frac{N_{FM}}{N_{FI}}}\right) \cdot 100$$ 
	dove $N_{FM}$ è il numero di funzionalità mancanti nell'implementazione e $N_{FI}$ è il numero di funzionalità individuate nell'attività di analisi.
	\item \textbf{Range ottimale}: 100.
	\item \textbf{Range di accettazione}: 100.
\end{itemize}

\paragraph{Accuratezza rispetto alle attese}
Indica la percentuale di risultati concordi alle attese.
\begin{itemize}
	\item \textbf{Misurazione}: 
		$$A=\left(1-\mathlarger{\frac{N_{RD}}{N_{TE}}}\right) \cdot 100$$
	dove $N_{RD}$ è il numero di test che producono risultati discordanti rispetto alle attese e $N_{TE}$ è il numero di test-case eseguiti.
	\item \textbf{Range ottimale}: 100.
	\item \textbf{Range di accettazione}: 90 -- 100.
\end{itemize}

\paragraph{Controllo degli accessi}
Indica la percentuale di operazioni illegali non bloccate.
\begin{itemize}
	\item \textbf{Misurazione}: 
		$$I=\mathlarger{\frac{N_{IE}}{N_{II}}} \cdot 100$$
	dove $N_{IE}$ è il numero di operazioni illegali effettuabili dai test e $N_{II}$ è il numero di operazioni illegali individuate.
	\item \textbf{Range ottimale}: 0.
	\item \textbf{Range di accettazione}: 0 -- 10.
\end{itemize}


\subsection{Affidabilità}
Rappresenta la capacità del prodotto software di svolgere correttamente le sue funzioni durante il suo utilizzo, anche nel caso in cui si presentino situazioni anomale.

\subsubsection{Obiettivi di qualità}
L'esecuzione del prodotto dovrà presentare le seguenti caratteristiche:
\begin{itemize}
\item \textbf{Maturità}: dovrà essere evitato che si verifichino malfunzionamenti, operazioni illegali e restituzione di risultati errati (\textit{failure}) in seguito a difetti;
\item \textbf{Tolleranza agli errori}: nel caso in cui si presentino degli errori, dovuti a guasti o ad un uso scorretto dell'applicativo, questi dovranno essere gestiti in modo da mantenere alto il livello di prestazione.
\end{itemize}

\subsubsection{Metriche}
\paragraph{Densità di \textit{failure}}
Indica la percentuale di operazioni di testing che si sono concluse in fallimenti.

\begin{itemize}
	\item \textbf{Misurazione}: 
		$$F=\mathlarger{\frac{N_{FR}}{N_{TE}}} \cdot 100$$
	dove $N_{FR}$ è il numero di fallimenti rilevati durante l'attività di testing e $N_{TE}$ è il numero di test-case eseguiti.
	\item \textbf{Range ottimale}: 0.
	\item \textbf{Range di accettazione}: 0 -- 10.
\end{itemize}

\paragraph{Blocco di operazioni non corrette}
Indica la percentuale di funzionalità in grado di gestire correttamente i \textit{fault} che potrebbero verificarsi.
\begin{itemize}
	\item \textbf{Misurazione}: 
		$$B=\mathlarger{\frac{N_{FE}}{N_{ON}}} \cdot 100$$
	dove $N_{FE}$ è il numero di \textit{failure} evitati durante i test effettuati e $N_{ON}$ è il numero di test-case eseguiti che prevedono l'esecuzione di operazioni non corrette, causa di possibili \textit{failure}.
	\item \textbf{Range ottimale}: 100.
	\item \textbf{Range di accettazione}: 80 -- 100.
\end{itemize}

\subsection{Usabilità}
Rappresenta la capacità del prodotto di essere facilmente comprensibile e attraente in ogni sua parte per qualsiasi utente che lo andrà ad utilizzare.

\subsubsection{Obiettivi di qualità}
Il prodotto dovrà puntare ai seguenti obiettivi di usabilità:
\begin{itemize}
\item \textbf{Comprensibilità)}: l'utente dovrà essere in grado di riconoscerne le funzionalità offerte dal software e dovrà comprenderne le modalità di utilizzo per riuscire a raggiungere i risultati attesi;
\item \textbf{Apprendibilità}: dovrà essere data la possibilità all'utente di imparare ad utilizzare l'applicazione senza troppo impegno;
\item \textbf{Operabilità}: le funzionalità presenti dovranno essere coerenti con le aspettative dell'utente.
\end{itemize}

\subsubsection{Metriche}
\paragraph{Comprensibilità delle funzioni offerte}
Indica la percentuale di operazioni comprese in modo immediato dall'utente, senza la consultazione del manuale.
\begin{itemize}
	\item \textbf{Misurazione}: 
		$$C=\mathlarger{\frac{N_{FC}}{N_{FO}}} \cdot 100$$
	dove $N_{FC}$ è il numero di funzionalità comprese in modo immediato dall'utente durante l'attività di testing del prodotto e $N_{FO}$ è il numero di funzionalità offerte dal sistema.
	\item \textbf{Range ottimale}: 90 -- 100.
	\item \textbf{Range di accettazione}: 80 -- 100.
\end{itemize}

\paragraph{Facilità di apprendimento delle funzionalità}
Indica il tempo medio impiegato dall'utente nell'imparare ad usare correttamente una data funzionalità.
\begin{itemize}
	\item \textbf{Misurazione}: indicatore numerico, espresso in minuti, che tiene traccia del tempo medio impiegato dall'utente nell'apprendere il corretto utilizzo di una funzionalità offerta dal sistema.
	\item \textbf{Range ottimale}: 0 -- 15.
	\item \textbf{Range di accettazione}: 0 -- 30.
\end{itemize}

\paragraph{Consistenza operazionale in uso}
Indica la percentuale di messaggi e funzionalità offerte all'utente che rispettano le sue aspettative riguardo al comportamento del software.
\begin{itemize}
	\item \textbf{Misurazione}: 
		$$C=\left(1-\mathlarger{\frac{N_{MFI}}{N_{MFO}}}\right) \cdot 100$$
	dove $N_{MFI}$ è il numero di messaggi e funzionalità che non rispettano le aspettative dell'utente e $N_{MFO}$ è il numero di messaggi e funzionalità offerti dal sistema.
	\item \textbf{Range ottimale}: 90 -- 100.
	\item \textbf{Range di accettazione}: 80 -- 100.
\end{itemize}

\subsection{Efficienza}
\label{efficienza}
Rappresenta la capacità di eseguire le funzionalità offerte dal software nel minor tempo possibile utilizzando al tempo stesso il minor numero di risorse possibili.

\subsubsection{Obiettivi di qualità}
Il prodotto dovrà essere efficiente, in particolare:
\begin{itemize}
\item \textbf{Comportamento rispetto al tempo}:  per svolgere le sue funzioni il software dovrà fornire adeguati tempi di risposta ed elaborazione;
\item \textbf{Utilizzo delle risorse}: il software quando eseguirà le sue funzionalità dovrà utilizzare un appropriato numero e tipo di risorse.
\end{itemize}

\subsubsection{Metriche}
\paragraph{Tempo di risposta}
Indica il periodo temporale medio che intercorre fra la richiesta al software di una determinata funzionalità e la restituzione del risultato all'utente.
\begin{itemize}
	\item \textbf{Misurazione}: 
		$$T_{RISP} = \mathlarger{\frac{\sum_{i=1}^{n} T_{i}}{n}}$$ 
	con $T_{RISP}$ misurato in secondi, e dove $T_{i}$ è il tempo intercorso fra la richiesta $i$ di una funzionalità ed il completamento delle operazioni necessarie a restituire un risultato a tale richiesta.
	\item \textbf{Range ottimale}: 0 -- 3.
	\item \textbf{Range di accettazione}: 0 -- 8.
\end{itemize}

\subsection{Manutenibilità}
Rappresenta la capacità del prodotto di essere modificato, tramite correzioni, miglioramenti o adattamenti del software a cambiamenti negli ambienti, nei requisiti e nelle specifiche funzionali.

\subsubsection{Obiettivi di qualità}
Le operazioni di manutenzione andranno agevolate il più possibile adottando le seguenti caratteristiche:
\begin{itemize}
\item \textbf{Analizzabilità}: il software dovrà consentire una rapida identificazione delle possibili cause di errori e malfunzionamenti;
\item \textbf{Modificabilità}: il prodotto originale dovrà permettere eventuali cambiamenti in alcune sue parti;
\item \textbf{Stabilità}: non dovranno insorgere effetti indesiderati in seguito a modifiche effettuate sul software;
\item \textbf{Testabilità}: il software dovrà poter essere facilmente testato per validare le modifiche effettuate.
\end{itemize}

\subsubsection{Metriche}
\paragraph{Capacità di analisi di \textit{failure}}
Indica la percentuale di \textit{failure} registrate delle quali sono state individuate le cause.
\begin{itemize}
	\item \textbf{Misurazione}: 
		$$I=\mathlarger{\frac{N_{FI}}{N_{FR}}} \cdot 100$$
	dove $N_{FI}$ è il numero di \textit{failure} delle quali sono state individuate le cause e $N_{FR}$ è il numero di \textit{failure} rilevate.
	\item \textbf{Range ottimale}: 80 -- 100.
	\item \textbf{Range di accettazione}: 60 -- 100.
\end{itemize}

\paragraph{Impatto delle modifiche}
Indica la percentuale di modifiche effettuate in risposta a \textit{failure} che hanno portato all'introduzione di nuove \textit{failure} in altre componenti del sistema.
\begin{itemize}
	\item \textbf{Misurazione}: 
		$$I=\mathlarger{\frac{N_{FRF}}{N_{FR}}} \cdot 100$$
	dove $N_{FRF}$ è il numero di \textit{failure} risolte con l'introduzione di nuove \textit{failure} e $N_{FR}$ è il numero di \textit{failure} risolte;
	\item \textbf{Range ottimale}: 0 -- 10.
	\item \textbf{Range di accettazione}: 0 -- 20.
\end{itemize}

\newpage
\section{Strategia di Gestione della Qualità in Dettaglio}
\subsection{Risorse}
Per raggiungere gli obiettivi qualitativi prefissati è necessario, oltre alle risorse umane, utilizzare la potenza e l'affidabilità delle risorse tecnologiche. Infatti, per agevolare il lavoro dei \VerP, verranno impiegati numerosi strumenti automatici che eseguiranno controlli sistematici sui prodotti generati. \\
Le risorse tecniche e tecnologiche consistono in tutti quegli strumenti software e hardware che il \gruppo\ intende utilizzare per le attività di verifica su processi e prodotti; per la descrizione di tali strumenti si rimanda al documento \NdP.

\subsection{Misure e metriche}
Allo scopo di rendere quantificabile il processo di verifica verranno adottate delle misure basate su metriche stabilite a priori. Le metriche incerte, qualora ve ne fossero, verranno migliorate in modo incrementale. \\
Le varie misure che verranno rilevate saranno analizzate confrontandole con due categorie di misurazione:
\begin{itemize}
\item
\textbf{Accettazione}: intervallo di valori minimi entro i quali il prodotto sarà accettato;
\item
\textbf{Ottimale}: intervallo di valori ottimali entro i quali il prodotto risulta soddisfare pienamente, o quasi, i requisiti richiesti. I valori all'interno di tali intervalli devono essere quindi intesi come consigliati ma non vincolanti.
\end{itemize}

\subsection{Strumenti}
Per aiutare la verifica delle metriche descritte nelle \NdP\ verranno utilizzati degli strumenti che, in maniera automatica, svolgeranno dei test e forniranno un resoconto del risultato. Gli strumenti verranno elencati di seguito.

\subsubsection{Jenkins}
\termine{Jenkins} è un software che permetterà di automatizzare una serie di operazioni che consentiranno al \termine{team} di monitorare la qualità del software in modo continuo ogni volta che verrà inviata una modifica alla \termine{repository} che conterrà il codice sorgente. \\
Per ottenere questi risultati questo software si appoggerà agli altri strumenti descritti qua di seguito.

\subsubsection{SonarQube}
\termine{SonarQube} è un software che permette di eseguire analisi statica sul codice che verrà inviato alla \termine{repository} fornendo un feedback al programmatore per fargli sapere cosa dovrà migliorare. Inoltre permetterà di controllare che le regole che il \termine{team} si è imposto per la qualità del codice sorgente vengano rispettate.

\subsubsection{Mocha}
\termine{Mocha} è un \termine{framework} che permette di eseguire test di unità e di integrazione andando a creare dei \termine{mock} che simulino il comportamento delle varie parti, con lo scopo di testare ogni dettaglio in modo indipendente dal resto. Ad esso si affiancheranno anche:
\begin{itemize}
\item \textbf{\termine{Chai}}: libreria per formulare asserzioni sull'output atteso;
\item \textbf{\termine{Sinon}}: libreria per simulare delle risposte di richieste a server remoti.
\end{itemize}
\newpage

\section{Obiettivi di qualità}
La qualità dei processi che il gruppo \gruppo{} intende perseguire verrà raggiunta tramite l'utilizzo delle metriche descritte nel documento \NdP.


\subsection{Valori obiettivo della qualità di processo}

\subsubsection{Indice di Gulpease}

\begin{center}

		\begin{tabular}{|P{2.5cm}|P{2.5cm}|P{6cm}|}
		\hline
			\textbf{Valore di accettazione}	& \textbf{Valore ottimale} & \textbf{Motivazione} \\
			\hline
			[40 -- 100] & [50--100] &	Per garantire una buona leggibilità dei documenti. Sono accettati anche valori inferiori, ma verranno accettati in un primo momento per poi essere migliorati successivamente. \\
			\hline
			\end{tabular}
\captionof{table}{Indice di Gulpease}
\end{center}

\subsection{Metriche per il software}
Questa sezione, che verrà rivista e incrementata nelle prossime revisioni, è da intendere come una dichiarazione di propositi.
Per raggiungere gli obiettivi auspicati dallo standard ISO di riferimento (ISO/IEC 9126), ovvero funzionalità, affidabilità,
efficienza, usabilità e manutenibilità.


\subsubsection{Complessità ciclomatica}

\begin{center}
		\begin{tabular}{|P{2.5cm}|P{2.5cm}|P{6cm}|}
		\hline
			\textbf{Valore di accettazione}	& \textbf{Valore ottimale} & \textbf{Motivazione} \\
			\hline
			[1 -− 15] & [1 -− 10] &	Per garantire una minore complessità durante la fase di \COD. \\
			\hline
			\end{tabular}
\captionof{table}{Complessità ciclomatica}
\end{center}

\subsubsection{Numero di livelli di annidamento per metodo}

\begin{center}
		\begin{tabular}{|P{2.5cm}|P{2.5cm}|P{6cm}|}
		\hline
			\textbf{Valore di accettazione}	& \textbf{Valore ottimale} & \textbf{Motivazione} \\
			\hline
			[1 -- 5] & [1 −- 3] &	Per garantire una minore complessità del metodo e garantire una sua facile comprensione. \\
			\hline
			\end{tabular}
\captionof{table}{Numero di livelli di annidamento per metodo}
\end{center}

\subsubsection{Numero di attributi per classe}

\begin{center}
		\begin{tabular}{|P{2.5cm}|P{2.5cm}|P{6cm}|}
		\hline
			\textbf{Valore di accettazione}	& \textbf{Valore ottimale} & \textbf{Motivazione} \\
			\hline
			[0 −- 11] & [0 −- 7] &	Per garantire un giusto equilibrio tra le classi, in modo da garantire una sufficiente coesione. \\
			\hline
			\end{tabular}
\captionof{table}{Numero di attributi per classe}
\end{center}

\paragraph{Copertura dei test}
Intervalli richiesti:
\begin{itemize}
\item
Accettazione: [40\% −- 100\%];
\item
Ottimale: [65\% −- 100\%].
\end{itemize}

\paragraph{Linee di commento per linee di codice}
Intervalli richiesti:
\begin{itemize}
\item
Accettazione: [> 0.15];
\item
Ottimale: [>0.20].
\end{itemize}

\subsubsection{Numero di parametri per metodo}
Intervalli richiesti:
\begin{itemize}
\item
Accettazione: [0 −- 6];
\item
Ottimale: [0 −- 4].
\end{itemize}

\subsubsection{Livello di stabilità}

\begin{displaymath}
{\text{Accoppiamento Afferente}}\over{\text{Accoppiamento Afferente} + \text{Accoppiamento Efferente}}
\end{displaymath}

Intervalli richiesti:
\begin{itemize}
\item
Accettazione: [0.0 −- 1];
\item
Ottimale: [0.0 −- 0.6].
\end{itemize}

\subsection{Astrattezza}

\begin{displaymath}
{\text{Numero classi astratte e interfacce}}\over{\text{Numero totale classi}}
\end{displaymath}

\begin{itemize}
\item
Accettazione: [0.0 −- 0.8];
\item
Ottimale: [0.0 −- 0.3].
\end{itemize}

\subsection{Distanza dalla sequenza principale}

\begin{displaymath}
{|\text{Astratezza} + \text{Livello di stabilità} - 1|}
\end{displaymath}

Intervalli richiesti:
\begin{itemize}
\item
Accettazione: [0.0 −- 1];
\item
Ottimale: [0.0 −- 0.4].
\end{itemize}


\subsection{Funzionalità}
Rappresenta la capacità del prodotto di fornire tutte le funzioni che sono state individuate attraverso l'\AdR.

\subsubsection{Obiettivi di qualità}
Il \termine{team} si impegnerà affinché:
\begin{itemize}
\item \textbf{Adeguatezza}: le funzionalità fornite siano conformi rispetto le aspettative;
\item \textbf{Accuratezza}: il prodotto fornisca i risultati attesi, con il livello di dettaglio richiesto;
\item \textbf{Sicurezza}: il prodotto protegga le informazioni e i dati da accessi e modifiche non autorizzati.
\end{itemize}

\subsubsection{Metriche}
\paragraph{Completezza dell'implementazione funzionale}
Indica la percentuale di requisiti funzionali coperti dall'implementazione.
\begin{itemize}
	\item \textbf{Misurazione}: 
		$$C=\left(1-\mathlarger{\frac{N_{FM}}{N_{FI}}}\right) \cdot 100$$ 
	dove $N_{FM}$ è il numero di funzionalità mancanti nell'implementazione e $N_{FI}$ è il numero di funzionalità individuate nell'attività di analisi.
	\item \textbf{Range ottimale}: 100.
	\item \textbf{Range di accettazione}: 100.
\end{itemize}

\paragraph{Accuratezza rispetto alle attese}
Indica la percentuale di risultati concordi alle attese.
\begin{itemize}
	\item \textbf{Misurazione}: 
		$$A=\left(1-\mathlarger{\frac{N_{RD}}{N_{TE}}}\right) \cdot 100$$
	dove $N_{RD}$ è il numero di test che producono risultati discordanti rispetto alle attese e $N_{TE}$ è il numero di test-case eseguiti.
	\item \textbf{Range ottimale}: 100.
	\item \textbf{Range di accettazione}: 90 -- 100.
\end{itemize}

\paragraph{Controllo degli accessi}
Indica la percentuale di operazioni illegali non bloccate.
\begin{itemize}
	\item \textbf{Misurazione}: 
		$$I=\mathlarger{\frac{N_{IE}}{N_{II}}} \cdot 100$$
	dove $N_{IE}$ è il numero di operazioni illegali effettuabili dai test e $N_{II}$ è il numero di operazioni illegali individuate.
	\item \textbf{Range ottimale}: 0.
	\item \textbf{Range di accettazione}: 0 -- 10.
\end{itemize}


\subsection{Affidabilità}
Rappresenta la capacità del prodotto software di svolgere correttamente le sue funzioni durante il suo utilizzo, anche nel caso in cui si presentino situazioni anomale.

\subsubsection{Obiettivi di qualità}
L'esecuzione del prodotto dovrà presentare le seguenti caratteristiche:
\begin{itemize}
\item \textbf{Maturità}: dovrà essere evitato che si verifichino malfunzionamenti, operazioni illegali e restituzione di risultati errati (\textit{failure}) in seguito a difetti;
\item \textbf{Tolleranza agli errori}: nel caso in cui si presentino degli errori, dovuti a guasti o ad un uso scorretto dell'applicativo, questi dovranno essere gestiti in modo da mantenere alto il livello di prestazione.
\end{itemize}

\subsubsection{Metriche}
\paragraph{Densità di \textit{failure}}
Indica la percentuale di operazioni di testing che si sono concluse in fallimenti.

\begin{itemize}
	\item \textbf{Misurazione}: 
		$$F=\mathlarger{\frac{N_{FR}}{N_{TE}}} \cdot 100$$
	dove $N_{FR}$ è il numero di fallimenti rilevati durante l'attività di testing e $N_{TE}$ è il numero di test-case eseguiti.
	\item \textbf{Range ottimale}: 0.
	\item \textbf{Range di accettazione}: 0 -- 10.
\end{itemize}

\paragraph{Blocco di operazioni non corrette}
Indica la percentuale di funzionalità in grado di gestire correttamente i \textit{fault} che potrebbero verificarsi.
\begin{itemize}
	\item \textbf{Misurazione}: 
		$$B=\mathlarger{\frac{N_{FE}}{N_{ON}}} \cdot 100$$
	dove $N_{FE}$ è il numero di \textit{failure} evitati durante i test effettuati e $N_{ON}$ è il numero di test-case eseguiti che prevedono l'esecuzione di operazioni non corrette, causa di possibili \textit{failure}.
	\item \textbf{Range ottimale}: 100.
	\item \textbf{Range di accettazione}: 80 -- 100.
\end{itemize}

\subsection{Usabilità}
Rappresenta la capacità del prodotto di essere facilmente comprensibile e attraente in ogni sua parte per qualsiasi utente che lo andrà ad utilizzare.

\subsubsection{Obiettivi di qualità}
Il prodotto dovrà puntare ai seguenti obiettivi di usabilità:
\begin{itemize}
\item \textbf{Comprensibilità)}: l'utente dovrà essere in grado di riconoscerne le funzionalità offerte dal software e dovrà comprenderne le modalità di utilizzo per riuscire a raggiungere i risultati attesi;
\item \textbf{Apprendibilità}: dovrà essere data la possibilità all'utente di imparare ad utilizzare l'applicazione senza troppo impegno;
\item \textbf{Operabilità}: le funzionalità presenti dovranno essere coerenti con le aspettative dell'utente.
\end{itemize}

\subsubsection{Metriche}
\paragraph{Comprensibilità delle funzioni offerte}
Indica la percentuale di operazioni comprese in modo immediato dall'utente, senza la consultazione del manuale.
\begin{itemize}
	\item \textbf{Misurazione}: 
		$$C=\mathlarger{\frac{N_{FC}}{N_{FO}}} \cdot 100$$
	dove $N_{FC}$ è il numero di funzionalità comprese in modo immediato dall'utente durante l'attività di testing del prodotto e $N_{FO}$ è il numero di funzionalità offerte dal sistema.
	\item \textbf{Range ottimale}: 90 -- 100.
	\item \textbf{Range di accettazione}: 80 -- 100.
\end{itemize}

\paragraph{Facilità di apprendimento delle funzionalità}
Indica il tempo medio impiegato dall'utente nell'imparare ad usare correttamente una data funzionalità.
\begin{itemize}
	\item \textbf{Misurazione}: indicatore numerico, espresso in minuti, che tiene traccia del tempo medio impiegato dall'utente nell'apprendere il corretto utilizzo di una funzionalità offerta dal sistema.
	\item \textbf{Range ottimale}: 0 -- 15.
	\item \textbf{Range di accettazione}: 0 -- 30.
\end{itemize}

\paragraph{Consistenza operazionale in uso}
Indica la percentuale di messaggi e funzionalità offerte all'utente che rispettano le sue aspettative riguardo al comportamento del software.
\begin{itemize}
	\item \textbf{Misurazione}: 
		$$C=\left(1-\mathlarger{\frac{N_{MFI}}{N_{MFO}}}\right) \cdot 100$$
	dove $N_{MFI}$ è il numero di messaggi e funzionalità che non rispettano le aspettative dell'utente e $N_{MFO}$ è il numero di messaggi e funzionalità offerti dal sistema.
	\item \textbf{Range ottimale}: 90 -- 100.
	\item \textbf{Range di accettazione}: 80 -- 100.
\end{itemize}

\subsection{Efficienza}
\label{efficienza}
Rappresenta la capacità di eseguire le funzionalità offerte dal software nel minor tempo possibile utilizzando al tempo stesso il minor numero di risorse possibili.

\subsubsection{Obiettivi di qualità}
Il prodotto dovrà essere efficiente, in particolare:
\begin{itemize}
\item \textbf{Comportamento rispetto al tempo}:  per svolgere le sue funzioni il software dovrà fornire adeguati tempi di risposta ed elaborazione;
\item \textbf{Utilizzo delle risorse}: il software quando eseguirà le sue funzionalità dovrà utilizzare un appropriato numero e tipo di risorse.
\end{itemize}

\subsubsection{Metriche}
\paragraph{Tempo di risposta}
Indica il periodo temporale medio che intercorre fra la richiesta al software di una determinata funzionalità e la restituzione del risultato all'utente.
\begin{itemize}
	\item \textbf{Misurazione}: 
		$$T_{RISP} = \mathlarger{\frac{\sum_{i=1}^{n} T_{i}}{n}}$$ 
	con $T_{RISP}$ misurato in secondi, e dove $T_{i}$ è il tempo intercorso fra la richiesta $i$ di una funzionalità ed il completamento delle operazioni necessarie a restituire un risultato a tale richiesta.
	\item \textbf{Range ottimale}: 0 -- 3.
	\item \textbf{Range di accettazione}: 0 -- 8.
\end{itemize}

\subsection{Manutenibilità}
Rappresenta la capacità del prodotto di essere modificato, tramite correzioni, miglioramenti o adattamenti del software a cambiamenti negli ambienti, nei requisiti e nelle specifiche funzionali.

\subsubsection{Obiettivi di qualità}
Le operazioni di manutenzione andranno agevolate il più possibile adottando le seguenti caratteristiche:
\begin{itemize}
\item \textbf{Analizzabilità}: il software dovrà consentire una rapida identificazione delle possibili cause di errori e malfunzionamenti;
\item \textbf{Modificabilità}: il prodotto originale dovrà permettere eventuali cambiamenti in alcune sue parti;
\item \textbf{Stabilità}: non dovranno insorgere effetti indesiderati in seguito a modifiche effettuate sul software;
\item \textbf{Testabilità}: il software dovrà poter essere facilmente testato per validare le modifiche effettuate.
\end{itemize}

\subsubsection{Metriche}
\paragraph{Capacità di analisi di \textit{failure}}
Indica la percentuale di \textit{failure} registrate delle quali sono state individuate le cause.
\begin{itemize}
	\item \textbf{Misurazione}: 
		$$I=\mathlarger{\frac{N_{FI}}{N_{FR}}} \cdot 100$$
	dove $N_{FI}$ è il numero di \textit{failure} delle quali sono state individuate le cause e $N_{FR}$ è il numero di \textit{failure} rilevate.
	\item \textbf{Range ottimale}: 80 -- 100.
	\item \textbf{Range di accettazione}: 60 -- 100.
\end{itemize}

\paragraph{Impatto delle modifiche}
Indica la percentuale di modifiche effettuate in risposta a \textit{failure} che hanno portato all'introduzione di nuove \textit{failure} in altre componenti del sistema.
\begin{itemize}
	\item \textbf{Misurazione}: 
		$$I=\mathlarger{\frac{N_{FRF}}{N_{FR}}} \cdot 100$$
	dove $N_{FRF}$ è il numero di \textit{failure} risolte con l'introduzione di nuove \textit{failure} e $N_{FR}$ è il numero di \textit{failure} risolte;
	\item \textbf{Range ottimale}: 0 -- 10.
	\item \textbf{Range di accettazione}: 0 -- 20.
\end{itemize}

\newpage
\section{Pianificazione delle verifiche delle metriche}
In questa sezione saranno esplicate come e quando verranno attuate le metriche descritte nelle \NdP\ al fine di raggiungere la piena conformità con gli obiettivi preposti in precedenza.
Per ogni metrica verranno elencati i seguenti punti:
\begin{itemize}
\item \textbf{Chi:} Colui o coloro che possono attuare la metrica.
\item \textbf{Quando:} Descrive quando la metrica dovrà essere applicata.
\item \textbf{Come:} Descrive come questa dovrà essere applicata.
\end{itemize}

Si informa che il non soddisfacimento di almeno il range di accettazione proposto comporterà il rifiuto del prodotto realizzato. Perciò, chi di dovere dovrà provvedere a sistemare gli errori segnalati in modo da riportare il prodotto almeno nel range di accettazione. 


\subsubsection{Verifiche dei documenti}
In questa sezione verranno descritte le come e quando effettuare le misurazioni incentrate sui documenti.

\paragraph{Indice di Gulpease}
\begin{itemize}
\item \textbf{Chi:} Il \Ver\ o il \Pm.
\item \textbf{Quando:} Il \Ver\ come passo successivo deve verificare che l'indice di Gulpease rispetti i parametri fissati per l'accettazione del documento oppure il \Pm\ prima dell'approvazione del documento, se ha effettuato delle modifiche, deve verificare che il documento superi di nuovo il vincolo di accettazione.
\item \textbf{Come:} Si può utilizzare un qualsiasi strumento automatico che verifichi la leggibilità del documento secondo l'indice stabilito.
\end{itemize}
%\subsubsection{Verifiche architetturali}
In questa sezione verranno descritte le come e quando effettuare le misurazioni per valutare l'architettura del software.

\paragraph{Numero di parametri per metodo}
\begin{itemize}
\item \textbf{Chi:} Il \Prog.
\item \textbf{Quando:} Prima della scrittura del metodo.
\item \textbf{Come:} Il \Prog deve controllare prima di progettare il metodo che il numero di parametri soddisfi almeno il livello di accettazione descritto per questa metrica.
\end{itemize}

\paragraph{Livello di stabilità}
\begin{itemize}
\item \textbf{Chi:} il \Ver.
\item \textbf{Quando:} Durante la fase di verifica che precede quella di approvazione.
\item \textbf{Come:} Il \Ver\ deve applicare la formula descritta nelle \NdP.
\end{itemize}
\newpage
\subsubsection{Verifiche del software}
In questa sezione verranno descritte come e quando effettuare le metriche incentrate sul software.

\paragraph{Complessità ciclomatica}
\begin{itemize}
\item \textbf{Chi:} \termine{SonarQube}.
\item \textbf{Quando:} Dopo ogni push sulla \termine{repository}.
\item \textbf{Come:} Questo test viene effettuato automaticamente da \termine{SonarQube}, se il test non viene superato il codice non verrà integrato nel ramo di produzione.
\end{itemize}

\paragraph{Numero di livelli di annidamento per metodo}
\begin{itemize}
\item \textbf{Chi:} \termine{SonarQube}.
\item \textbf{Quando:} Dopo ogni push sulla \termine{repository}.
\item \textbf{Come:} Questo test viene effettuato automaticamente da \termine{SonarQube}, se il test non viene superato il codice non verrà integrato nel ramo di produzione.
\end{itemize}

\paragraph{Numero di attributi per classe}
\begin{itemize}
\item \textbf{Chi:} \termine{SonarQube}.
\item \textbf{Quando:} Dopo ogni push sulla \termine{repository}.
\item \textbf{Come:} Questo test viene effettuato automaticamente da \termine{SonarQube}, se il test non viene superato il codice non verrà integrato nel ramo di produzione.
\end{itemize}

\paragraph{Copertura dei test}
\begin{itemize}
\item \textbf{Chi:} \termine{SonarQube}.
\item \textbf{Quando:} Dopo ogni push sulla \termine{repository}.
\item \textbf{Come:} Questo test viene effettuato automaticamente da \termine{SonarQube}, se il test non viene superato il codice non verrà integrato nel ramo di produzione.
\end{itemize}

\paragraph{Linee di commento per linee di codice}
\begin{itemize}
\item \textbf{Chi:} Il \Progr.
\item \textbf{Quando:} Dopo la scrittura di ogni metodo o funzione.
\item \textbf{Come:} Il \Progr\ deve verificare che il test soddisfi almeno il livello di accettazione descritto per poter considerare la funzione o il metodo accettabile. Nel caso in cui non sia raggiunto il minimo livello, il \Prog\ dovrà provvedere a trovare frasi più semplici e coese rispetto a quelle in precedenza. 
\end{itemize}

\paragraph{Numero di parametri per metodo}
\begin{itemize}
\item \textbf{Chi:} Il \Prog.
\item \textbf{Quando:} Prima della scrittura del metodo.
\item \textbf{Come:} Il \Prog deve controllare prima di progettare il metodo che il numero di parametri soddisfi almeno il livello di accettazione descritto per questa metrica.
\end{itemize}

\paragraph{Livello di stabilità}
\begin{itemize}
\item \textbf{Chi:} il \Ver.
\item \textbf{Quando:} Durante la fase di verifica che precede quella di approvazione.
\item \textbf{Come:} Il \Ver\ deve applicare la formula descritta nelle \NdP.
\end{itemize}
\newpage
\subsubsection{Verifiche di funzionalità}
In questa sezione verranno descritte come e quando effettuare le metriche incentrate sulla funzionalità del prodotto.

\paragraph{Completezza dell’implementazione funzionale}
\begin{itemize}
\item \textbf{Chi:} \Ver
\item \textbf{Quando:}
\item \textbf{Come:}
\end{itemize}

\paragraph{Accuratezza rispetto alle attese}
\begin{itemize}
\item \textbf{Chi:} Jenkins.
\item \textbf{Quando:} Dopo ogni push sulla \termine{repository}.
\item \textbf{Come:} Questo test viene effettuato automaticamente da Jenkins, se il test non viene superato il codice non verrà integrato nel ramo di produzione.
\end{itemize}

\paragraph{Controllo degli accessi}
\begin{itemize}
\item \textbf{Chi:} Jenkins.
\item \textbf{Quando:} Dopo ogni push sulla \termine{repository}.
\item \textbf{Come:} Questo test viene effettuato automaticamente da Jenkins, se il test non viene superato il codice non verrà integrato nel ramo di produzione.
\end{itemize}
\newpage
\subsubsection{Verifiche di affidabilità}
In questa sezione verranno descritte come e quando effettuare le misurazioni incentrate sulla affidabilità del prodotto.

\paragraph{Densità di failure}
\begin{itemize}
\item \textbf{Chi:} Jenkins.
\item \textbf{Quando:} Dopo ogni push sulla \termine{repository}.
\item \textbf{Come:} Questo test viene effettuato automaticamente da Jenkins, se il test non viene superato il codice non verrà integrato nel ramo di produzione.
\end{itemize}

\paragraph{Blocco di operazioni non corrette}
\begin{itemize}
\item \textbf{Chi:} Jenkins.
\item \textbf{Quando:} Dopo ogni push sulla \termine{repository}.
\item \textbf{Come:} Questo test viene effettuato automaticamente da Jenkins, se il test non viene superato il codice non verrà integrato nel ramo di produzione.
\end{itemize}
\newpage
\subsubsection{Verifiche di efficenza}
In questa sezione verranno descritte come e quando effettuare le metriche incentrate sul'efficenza del prodotto.

\paragraph{Tempo di risposta}
\begin{itemize}
\item \textbf{Chi:} \termine{Jenkins}.
\item \textbf{Quando:} Dopo ogni push sulla \termine{repository}.
\item \textbf{Come:} Questo test viene effettuato automaticamente da \termine{Jenkins}, se il test non viene superato il codice non verrà integrato nel ramo di produzione.
\end{itemize}

\newpage
\subsubsection{Verifiche di manutenibilità}
In questa sezione verranno descritte come e quando effettuare le misurazioni incentrate sull'efficienza del prodotto.

\paragraph{Capacità di analisi di failure}
\begin{itemize}
\item \textbf{Chi:}
\item \textbf{Quando:}
\item \textbf{Come:}
\end{itemize}

\paragraph{Impatto delle modifiche}
\begin{itemize}
\item \textbf{Chi:} \termine{Jenkins}.
\item \textbf{Quando:} Dopo ogni push sulla \termine{repository}.
\item \textbf{Come:} Questo test viene effettuato automaticamente da \termine{Jenkins}, se il test non viene superato il codice non verrà integrato nel ramo di produzione.
\end{itemize}
 Da terminare
\input{Sezioni/9-PianificazioneDeiTest.tex}
\end{document}