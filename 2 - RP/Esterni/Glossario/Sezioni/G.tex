\section*{G}
\addcontentsline{toc}{section}{G}
\begin{itemize}
	\item
	\textbf{Ganttchart}: \textit{eng. Diagramma di Gantt} Strumento di supporto alla gestione dei progetti usato principalmente nelle attività di project management e costruito partendo da un asse orizzontale -- che rappresenta dell'arco temporale totale del progetto, suddiviso in fasi incrementali (ad esempio, giorni, settimane, mesi) -- e da un asse verticale -- che rappresenta delle mansioni o attività che costituiscono il progetto.
	\item
	\textbf{Gesture}: Gesto a cui è associata una azione.
	\item
	\textbf{Git}: Software di controllo di versione distribuito utilizzabile da interfaccia a riga di comando.
	\item
	\textbf{Github}: Servizio di hosting per progetti software.
	\item
	\textbf{Gitlab}: Servizio di hosting per progetti software.
	\item
	\textbf{Google Drive}: Servizio web di archiviazione di file.
	\item
	\textbf{Grafo Di Controllo Di Flusso}: Grafo che mostra il flusso di esecuzione di un programma, usato per calcolare la \termine{Complessità Ciclomatica}.
	\item
	\textbf{Gruppo}: Insieme delle persone fisiche Diego Cavestro, Francesco Bazzerla, Luca Dario, Manuel Turetta, Nicolò Dovico, Stefano Lia e Riccardo Montagnin.
	\item
	\textbf{Gummi}: Editor per LaTeX disponibile per Linux e Windows.
\end{itemize}
\newpage