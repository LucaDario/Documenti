\section{R}
\begin{itemize}
	\item
	\textbf{Repository}: Ambiente di un sistema informativo (ad es. di tipo ERP), in cui vengono gestiti i metadati, attraverso tabelle relazionali; l'insieme di tabelle, regole e motori di calcolo tramite cui si gestiscono i metadati prende il nome di \textit{metabase}. Nell'ambiente \termine{Git} viene indicato con questo termine il luogo dove è possibile salvare in remoto del codice, eseguire dei \termine{pull}, dei \termine{push} e aprire delle \termine{issue}.
	\item
	\textbf{Rest}: \textit{eng. REpresentational State Transfer}. Interfaccia che usa il protocollo HTTP per invio, ricezione ed eliminazione dei dati.
	\item
	\textbf{Rich Media Bubble}: \termine{Bolla} che mostra delle anteprime mediante immagini di contenuti multimediali.
	\item
	\textbf{Rocket.Chat}: Applicazione di \termine{Instant Messaging}.
\end{itemize}
\newpage