\section*{P}
\addcontentsline{toc}{section}{P}
\begin{itemize}
	\item
	\textbf{Pacchetto Stand-Alone}: Pacchetto che non dipende da librerie esterne.
	\item
	\textbf{Package}: \textit{eng. Pacchetto}. Collezione di classi e interfacce correlate.
	\item
	\textbf{Package Management}: Collezione di strumenti che automatizzano il processo di installazione, aggiornamento, configurazione e rimozione dei pacchetti software.
	\item
	\textbf{Package Manager}: Software che gestisce e organizza dei pacchetti software.
	\item
	\textbf{Pascalcase}: Tecnica tipografica che prevede la scrittura di parole composte come una unica parola all'interno delle quali l'iniziale di ciascuna parola inizia con una lettera maiuscola.
	\item
	\textbf{Pattern Architetturale}: .
	\item
	\textbf{Pdca}: \textit{eng. Plan Do Check Act, sin. Ciclo di Deming}. Metodo di gestione in quattro fasi iterativo, utilizzato in attività per il controllo e il miglioramento continuo dei processi e dei prodotti.
	\item
	\textbf{Pdf}: \textit{eng. Portable Document Format, Formato di documenti portatile}. Formato di file basato su un linguaggio di descrizione di pagina sviluppato da Adobe Systems nel 1993, per rappresentare documenti in modo indipendente dall'hardware e dal software utilizzati per generarli o per visualizzarli.
	\item
	\textbf{Plugin}: Programma non autonomo che interagisce con un altro programma per ampliarne o estenderne le funzionalità originarie.
	\item
	\textbf{Png}: \textit{eng. Portable Network Graphics}. Formato di file per memorizzare immagini.
	\item
	\textbf{Presenter}: Classe intermedia che serve per far comunicare la parte logica e la parte grafica di un'applicazione.
	\item
	\textbf{Prodotto}: Nel campo del project management il prodotto rilasciato, indicato solitamente con il termine \textit{deliverable} nella letteratura tecnologica, indica un oggetto materiale o immateriale realizzato (fornito/consegnato) come risultato di una attività del progetto.
	\item
	\textbf{Progettazione Architetturale}: Progettazione "ad altissimo livello", in cui si definisce solo la struttura complessiva del sistema in termini dei principali moduli di cui esso è composto e delle relazioni macroscopiche fra di essi.
	\item
	\textbf{Progettazione Di Dettaglio}: Rappresenta una descrizione del sistema molto vicina alla codifica, ovvero che la vincola in maniera sostanziale.
	\item
	\textbf{Promise Centric Approach}: E un approccio centrato sul uso delle promesse, cioè oggetti che rappresentano il risultato di una chiamata di funzione asincrona, cioe una promessa che un risultato verrà fornito non appena disponibile.
	\item
	\textbf{Proponente}: Persona fisica che presenta una idea di progetto da realizzare.
	\item
	\textbf{Pull}: \textit{eng. Tirare}. Atto con il quale, nel mondo \termine{Git}, viene indicata l'azione di prelevare uno o più \termine{commit} da una \termine{repository} remota.
	\item
	\textbf{Push}: \textit{eng. Spingere}. Atto con il quale, nel mondo \termine{Git}, viene indicata l'azione di caricare uno o più \termine{commit} su una \termine{repository} remota.
	\item
	\textbf{Python}: Linguaggio di programmazione ad alto livello, orientato agli oggetti.
\end{itemize}
\newpage