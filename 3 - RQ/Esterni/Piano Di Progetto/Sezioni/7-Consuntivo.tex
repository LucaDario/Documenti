\newpage
\section{Consuntivi di periodo}

In base ai consuntivi di periodo dei periodi completati sono state effettuate le diverse modifiche alla pianificazione per rientrare nel budget preventivato al momento della consegna per aggiudicarci il capitolato.

\subsection{\PD\ e \COD\ da \RP\ fino alla \RQ}
La differenza di costi tra quelli preventivati nella sezione 5 di questo documento e quello effettivamente speso è la seguente:

\begin{table}[h]
	\begin{center}
		\begin{tabular}{|c|c|c|c|c|}
			\hline
			\textbf{Ruolo}	& \textbf{Ore Preventivate} & \textbf{Costo preventivato} &  \textbf{Differenza ore} & \textbf{Differenza costo}\\
			\hline
			\Pm &	11 & 330 & 0 & 0\\
			\hline
			\Am	&	6 & 120 & 0 & 0\\
			\hline
			\Prog	&	21 & 462 & 0 & 0\\
			\hline
			\Progr	&	193 & 2055 & 0 & 0\\
			\hline
			\Ver	&	91 & 1155 & 0 & 0\\
			\hline
			\textbf{totale}	&	\textbf{322} & \textbf{4122} & \textbf{0} & \textbf{0} \\
			\hline
		\end{tabular}
	\end{center}
	\caption{Consuntivo di periodo \PD\ e \COD\ da \RP\ fino alla \RQ}
\end{table}

Dal consuntivo di periodo, dunque, emerge un \textbf{...} rispetto ai costi preventivati in precedenza.
