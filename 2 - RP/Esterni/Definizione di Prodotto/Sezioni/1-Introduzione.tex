\section{Introduzione}
\subsection{Scopo del documento}
Questo documento ha lo scopo di definire nel dettaglio la struttura di \capitolato. Tale documento fornisce una struttura dettagliata e completa che viene utilizzata dai \ProgrP per le attività di \termine{codifica}.
\subsection{Scopo del Prodotto}
\scopoProdotto

\subsection{Glossario}
\descrizioneGlossario

\subsection{Riferimenti}
\subsubsection{Normativi}
\riferimentiNormativi
\begin{itemize}
\item \textbf{\AdR}: \analisiDeiRequisiti
\end{itemize}

\subsubsection{Informativi}
\begin{itemize}
\item \textbf{riot.js:} \link{http://riotjs.com/guide/}
\item \textbf{HTML5 reccomendation:} \link{https://www.w3.org/TR/html5/}
\item \textbf{SCSS:} \link{http://sass-lang.com/guide}
\item \textbf{Babel.js:} \link{https://babeljs.io/learn-es2015/}
\item \textbf{JavaScript:} \link{http://developer.mozilla.org/en-US/docs/Web/JavaScript/Reference}
\item \textbf{Rocket.chat documentation:} \link{https://rocket.chat/docs/}
\item \textbf{Meteor documentation:} \link{https://docs.meteor.com/}
\item \textbf{Atmosphere documentation:} \link{https://github.com/Atmosphere/atmosphere/wiki}
\item \textbf{npm documentation:} \link{https://docs.npmjs.com/}
\item \textbf{ECMASCRIPT6:} \link{http://es6-features.org}
\item \textbf{Javascript Linting Documentation:} \link{http://javascriptlint.com/docs/index.html}
\end{itemize}

\newpage

