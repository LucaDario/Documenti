\subsubsection{Caso d'uso UC 1.1: Utilizzo di una bolla di tipo testo formattato}
\label{Caso d'uso UC 1.1: Utilizzo di una bolla di tipo testo formattato}

\begin{figure}[ht]
\centering
	\includegraphics[scale=0.6]{Usecases/img/{UC1.1}.png}
	\caption{Caso d'uso UC 1.1: Utilizzo di una bolla di tipo testo formattato}
\end{figure}

\FloatBarrier
\begin{itemize}
\item\textbf{Attori}: Sviluppatore.
\item\textbf{Descrizione}: Lo sviluppatore vuole utilizzare una bolla di tipo testo formattato.
\item\textbf{Precondizione}: Lo sviluppatore ha accesso all'\termine{SDK}.
\item\textbf{Postcondizione}: Lo sviluppatore ha creato del codice eseguibile che permette di creare una bolla di tipo testo formattato.
\item\textbf{Scenario principale}:
	\begin{itemize}
		\item Lo sviluppatore vuole scegliere la grandezza del font del testo (UC 1.1.1).
		\item Lo sviluppatore vuole impostare parte del testo in corsivo (UC 1.1.3).
		\item Lo sviluppatore vuole aggiungere un link cliccabile (UC 1.1.4).
		\item Lo sviluppatore vuole impostare parte del testo in grassetto (UC 1.1.7).
		\item Lo sviluppatore vuole impostare un colore a parte del testo (UC 1.1.8).
		\item Lo sviluppatore vuole aggiungere del testo alla bolla (1.1.10)
	\end{itemize}

\end{itemize}
