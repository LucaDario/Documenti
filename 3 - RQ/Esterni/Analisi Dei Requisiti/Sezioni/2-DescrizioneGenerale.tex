\section{Descrizione generale}
\subsection{Obiettivo del Prodotto}
\scopoProdotto

\subsection{Funzioni del prodotto}
Un qualsiasi utente dell'\termine{SDK} potrà creare e personalizzare le bolle messe a disposizione all'interno dell'\termine{SDK} e, successivamente, le visualizzerà come se fossero dei messaggi normali. Ogni \termine{bolla}, inoltre, avrà un suo uso e funzionalità specifiche, che verranno illustrate all'utente non appena egli deciderà di utilizzarla. L'applicazione demo infatti, sfrutterà le capacità rese disponibili dall'\termine{SDK} per realizzare con una bolla che fungerà da lista della spesa condivisa o meno con altri utenti.

\subsection{Caratteristiche degli utenti}
Non sono richieste competenze particolari per poter usufruire dell'applicazione associata al progetto, che deve risultare quindi accessibile ad un ampia categoria di utenti. L'interfaccia dovrà quindi essere il più semplice e intuitiva possibile e favorirne un facile utilizzo. \\
Per quanto riguarda l'\termine{SDK}, esso verrà sviluppato con lo scopo di risultare quanto più comprensibile possibile, al fine di facilitarne l'utilizzo. \\
Verrà inoltre fornito un \MU\ per facilitare la comprensione dell'utilizzo di entrambi i prodotti realizzati per questo capitolato.

\subsection{Piattaforma di esecuzione} 
La piattaforma di esecuzione dell'applicazione che sfrutta l'\termine{SDK} è esclusivamente la \termine{web chat} \termine{Rocket.chat}. L'applicazione è utilizzabile quindi su tutti i browsers compatibili con \termine{Rocket.chat} e sulle applicazioni mobile e desktop Rocket.chat senza vincoli di sistema operativo.

\subsection{Vincoli generali}
Gli utenti che vogliono utilizzare l'applicazione creata dal gruppo \gruppo\ dovranno accedere all'istanza del server fornita dal gruppo stesso. \\
Per poter invece utilizzare l'\termine{SDK}, lo \termine{sviluppatore} dovrà includere il pacchetto \textit{Monolith} tramite il \termine{package manager} \termine{Atmosphere}. 

\newpage