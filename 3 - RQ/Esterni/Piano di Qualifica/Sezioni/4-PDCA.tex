\section{PDCA}
Il ciclo \termine{PDCA}, detto anche \termine{Ciclo di Deming}, definisce un metodo di controllo dei processi durante il loro ciclo di vita che consente di migliorarne continuamente la qualità. \\
Tale approccio è suddiviso in 4 fasi:
\begin{itemize}
\item \textbf{Plan}: fase di pianificazione, dove si individuano gli obiettivi e i processi necessari per il raggiungimento dei risultati attesi;
\item \textbf{Do}: fase di attuazione del piano individuato al passo precedente e raccolta di dati sulla qualità ottenuta;
\item \textbf{Check}: fase di verifica, dove si confrontano i risultati ottenuti (fase di Do) ed i risultati attesi (fase di Plan);
\item \textbf{Act}: fase in cui si determinano le cause delle differenze fra risultati ottenuti e risultati attesi, per decidere dove attuare eventuali azioni correttive per avere un effettivo miglioramento della qualità.
\end{itemize}

\subsection{Scelte del team}
A causa della mancanza di tempo e di risorse è stato deciso unanimemente da tutti i componenti del \termine{team} che non verrà istanziato alcun processo di automiglioramento della qualità dei processi ma, allo stesso tempo, ogni componente si impegnerà a svolgere tutti i compiti lui assegnati nel miglior modo possibile.

\newpage