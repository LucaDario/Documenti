\section*{H}
\addcontentsline{toc}{section}{H}
\begin{itemize}
	\item
	\textbf{Heroku}: Heroku è un Platform as a service (PaaS) su cloud che supporta diversi linguaggi di programmazione. Un PaaS è un'attività economica che consiste nel servizio di messa a disposizione di piattaforme di elaborazione (Computing platform) e di solution stack. Gli elementi del PaaS permettono di sviluppare, sottoporre a test, implementare e gestire le applicazioni aziendali senza i costi e la complessità associati all'acquisto, alla configurazione, all'ottimizzazione e alla gestione dell'hardware e del software di base.
	\item
	\textbf{Hosting}: Un servizio di rete che consiste nell'allocare su un server web le pagine web di un sito web o un'applicazione web, rendendolo così accessibile dalla rete Internet e ai suoi utenti.
	\item
	\textbf{Html}: \textit{eng. HyperText Markup Language, linguaggio a marcatori per ipertesti}. Linguaggio di \termine{markup} solitamente usato per la formattazione e impaginazione di documenti ipertestuali disponibili nel World Wide Web sotto forma di pagine web, nato assieme al web 1.0.
	\item
	\textbf{Html5}: Versione 5 di \termine{HTML}, pubblicato come W3C Recommendation da ottobre 2014.
\end{itemize}
\newpage