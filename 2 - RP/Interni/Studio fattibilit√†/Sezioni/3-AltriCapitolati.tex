\section{Altri capitolati}
\subsection{C1 - APIM: An API Market Platform}
\subsubsection{Valutazione generale}
Il \termine{capitolato} propone la creazione di un \termine{API} market per microservizi. Su tale market dovrà essere quindi possibile cercare, vendere, acquistare e amministrare una serie di \termine{API} di un relativo \termine{microservizio}. \\
Il gruppo non ha reputato interessante questo \termine{capitolato} in quanto il linguaggio con il quale è stato chiesto che fosse sviluppato questo progetto (\termine{Jolie}) è un linguaggio proprietario che il gruppo non ha valutato avere un ampio potenziale futuro sviluppo ed utilizzo.

\subsubsection{Fattori di rischio}
\begin{itemize}

	\item Lo sviluppo di un market online deve poter garantire degli standard di sicurezza informatica molto alta.

\end{itemize}

\subsection{C2 - AtAVi: Accoglienza tramite Assistente Virtuale}
\subsubsection{Valutazione generale}
Il secondo \termine{capitolato} propone la creazione di una applicazione che si occupi dell'accoglienza di clienti tramite l'utilizzo di un assistente virtuale. Tale assistente dovrà riconoscere ciò che l'utente chiede e rispondere ad esso tramite istruzioni oppure suggerimenti su ciò che sta cercando. \\
Il team ha deciso di non svolgere questo progetto in quanto è richiesto un alto grado di preparazione nell'ambito del mercato degli assistenti virtuali, fattore che il team non possiede.

\subsubsection{Fattori di rischio}
\begin{itemize}
	\item La scelta sbagliata dell'assistente virtuale da utilizzare durante la progettazione e il successivo sviluppo del prodotto potrebbero impedire una buona riuscita del progetto;
	\item L'assistente virtuale scelto potrebbe portare alla necessità di imparare un nuovo linguaggio di programmazione proprietario e non utile da un altro punto di vista applicativo.
\end{itemize}

\subsection{C3 - DeGeOP: A Designer and Geo-localizer \termine{Web App} for Organizational Plants}
Il \termine{capitolato} propone la creazione di una interfaccia di tipo \termine{web app} che permetta, attraverso l'utilizzo di \termine{API} già create, di analizzare i fattori di rischio che potrebbero interessare una determinata azienda le quali sedi sono localizzate in uno o più punti del territorio italiano. \\
Tale progetto non è stato reputato interessante da parte del \termine{team} in quanto le tecnologie richieste per lo sviluppo dell'interfaccia (PHP, HTML5, JavaScript) sono state già trattate durante il corso di studio, inoltre il progetto consiste solamente nella creazione della parte grafica della \termine{web app} e non degli algoritmi per il calcolo dei rischi.

\subsubsection{Fattori di rischio}
\begin{itemize}
	\item Lo sviluppo di una \termine{web app} che sia possibile interrogare anche da dispositivi mobili comporta il rischio di dover confrontare con un modello di sviluppo che preveda delle \termine{gesture} specifiche per tali dispositivi;
	\item La costrizione nell'uso di \termine{API} già pronte può causare difficoltà durante lo sviluppo del progetto in quanto non è possibile andare a modificare le chiamate per tali servizi.
\end{itemize}

\subsection{C4 - eBread: Applicazione di lettura per dislessici}
Il \termine{capitolato} propone lo sviluppo di una applicazione che permetta alle persone che hanno problemi di dislessia di leggere in modo facilitato ciò che viene presentato sullo schermo di un dispositivo, attraverso la realizzazione di una applicazione Android. \\
Tale progetto non è stato reputato interessante da parte del gruppo in quanto alcuni membri di esso possiedono già conoscenze nell'ambito dello sviluppo Android, e pertanto non avrebbe portato ad essi alcune conoscenze aggiuntive che si andavo invece cercando.

\subsubsection{Fattori di rischio}
\begin{itemize}
	\item Lo sviluppo di una applicazione che tenga conto di tutti i diversi gradi di dislessia che è possibile una persona abbia può essere molto dispendioso dal punto di vista progettuale, e richiede inoltre un grado di conoscenze mediche non indifferente.
\end{itemize}

\subsection{C6 - SWEDesigner: Editor di diagrammi UML con generazione di codice}
Il \termine{capitolato} prevede la realizzazione di un editor di diagrammi \termine{UML} in grado di generare codice JavaScript eseguibile. \\
Tale progetto è stato valutato interessante dal gruppo in quanto, se ben sviluppato, potrebbe portare ad una rivoluzione nel modo di programmare in JavaScript, e potrebbe avere inoltre un ampio impatto economico.

\subsubsection{Fattori di rischio}
\begin{itemize}
	\item Difficoltà nella creazione di un sistema completo che permetta allo sviluppatore di creare facilmente uno schema che poi generi istruzioni ad alto livello che eseguano correttamente.
\end{itemize}

\newpage