% Questo file definisce lo stile che verrà applicato
% ad ogni pagina di contenuto
\documentclass[a4paper,11pt]{article}

\usepackage{ifthen}
\usepackage[
 a4paper,
 top=2.5cm,
 bottom=2.5cm,
 left=1.5cm,
 right=1.5cm,
 head=30pt
]{geometry}
\usepackage[italian]{babel}
\usepackage[utf8x]{inputenc}
\usepackage[T1]{fontenc}
\usepackage{fancyhdr}
\usepackage[colorlinks=true, urlcolor=black, citecolor=black, linkcolor=black]{hyperref}
\usepackage{tabularx}
\usepackage{multirow}
\usepackage{booktabs}
\usepackage{color}
\usepackage{graphicx}
\usepackage{eurosym}
\usepackage{amsmath}
\usepackage{relsize}

\usepackage[multidot]{grffile}
\usepackage{xcolor,colortbl}
\definecolor{lightblue}{HTML}{56B4E6}
\definecolor{blue}{HTML}{2953A1}
\definecolor{darkblue}{HTML}{1E396E}

\usepackage[toc,page]{appendix}
\renewcommand\appendixtocname{Appendice}
\renewcommand{\appendixpagename}{Appendice}

\newcommand\pagenumberingnoreset[1]{\gdef\thepage{\csname @#1\endcsname\c@page}}

% Cambia il font 
\renewcommand*\rmdefault{qhv}

% ***STILE PAGINA***
\pagestyle{fancy}
\fancyhf{}
\setlength{\headheight}{1cm} 
% No indentazione paragrafo
\setlength{\parindent}{0pt}

% ***INTESTAZIONE***
\newcommand\textline[4][t]{%
  \noindent\parbox[#1]{.333\textwidth}{\raisebox{-0.40\height}{#2}}%
  \parbox[#1]{.333\textwidth}{\centering #3}%
  \parbox[#1]{.333\textwidth}{\raggedleft #4}%
}

\lhead{
	\textline[t]{\includegraphics[width=1cm, keepaspectratio=true]{../../../Template/Logo/Logo.png}}{\progettoShort}{\documento}
}

\renewcommand{\headrulewidth}{0.4pt}  %Linea sotto l'intestazione

% ***PIÈ DI PAGINA***
\lfoot{\textit{\gruppoLink}\\ \footnotesize{\email}}
\rfoot{\thepage} %per le prime pagine: mostra solo il numero romano
\cfoot{}
\renewcommand{\footrulewidth}{0.4pt}   %Linea sopra il piè di pagina


% Ridefinisce command \paragraph{} andando a capo ogni dopo la parola dentro le parentesi ed ha la possibiltà di enumerazione fino a n cifre modificando il numero dentro "secnumdepth"
\usepackage{titlesec}

\setcounter{secnumdepth}{7}
\setcounter{tocdepth}{7}
%
%
%\titleformat{\paragraph}
%{\normalfont\normalsize\bfseries}{\theparagraph}{1em}{}
%\titlespacing*{\paragraph}
%{0pt}{3.25ex plus 1ex minus .2ex}{1.5ex plus .2ex}
%
%
%\titleclass{\subsubparagraph}{straight}[\subparagraph]
%\newcounter{subsubparagraph}
%
%\titleformat{\subsubparagraph}[display]
%  {\normalfont\normalsize\bf}
%  {\thesubsubparagraph.}
%  {.5em}
%  {}
%\renewcommand\thesubsubparagraph\textbf{\roman{subsubparagraph}}
%\titlespacing*{\subsubparagraph} {0pt}{4pt}{6pt}


%***LA SOTTOSEZIONE PARAGRAPH VIENE VISUALIZZATA COME UNA SECTION
\titleformat{\paragraph}{\normalfont\normalsize\bfseries}{\theparagraph}{1em}{}
\titlespacing*{\paragraph}{0pt}{3.25ex plus 1ex minus .2ex}{1.5ex plus .2ex}

\titleformat{\subparagraph}{\normalfont\normalsize\bfseries}{\thesubparagraph}{1em}{}
\titlespacing*{\subparagraph}{0pt}{3.25ex plus 1ex minus .2ex}{1.5ex plus .2ex}

\makeatletter
\newcounter{subsubparagraph}[subparagraph]
\renewcommand\thesubsubparagraph{%
  \thesubparagraph.\@arabic\c@subsubparagraph}
\newcommand\subsubparagraph{%
  \@startsection{subsubparagraph}    % counter
    {6}                              % level
    {\parindent}                     % indent
    {3.25ex \@plus 1ex \@minus .2ex} % beforeskip
    {0.75em}                           % afterskip
    {\normalfont\normalsize\bfseries}}
\newcommand\l@subsubparagraph{\@dottedtocline{6}{13em}{5.5em}} %gestione dell'indice
\newcommand{\subsubparagraphmark}[1]{}
\makeatother

\makeatletter
\newcounter{subsubsubparagraph}[subsubparagraph]
\renewcommand\thesubsubsubparagraph{%
  \thesubsubparagraph.\@arabic\c@subsubsubparagraph}
\newcommand\subsubsubparagraph{%
  \@startsection{subsubsubparagraph}    % counter
    {7}                              % level
    {\parindent}                     % indent
    {3.25ex \@plus 1ex \@minus .2ex} % beforeskip
    {0.75em}                           % afterskip
    {\normalfont\normalsize\bfseries}}
\newcommand\l@subsubsubparagraph{\@dottedtocline{7}{16em}{6.5em}} %gestione dell'indice
\newcommand{\subsubsubparagraphmark}[1]{}
\makeatother
