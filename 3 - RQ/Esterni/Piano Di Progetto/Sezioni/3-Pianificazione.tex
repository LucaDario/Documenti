\newpage
\section{Pianificazione}

\subsection{Modello di sviluppo}
Il modello di sviluppo scelto per lo svolgimento del progetto è il \termine{Modello Incrementale}. Esso prevede che:
\begin{itemize}
	\item L'\termine{Analisi} e la \termine{Progettazione Architetturale} costituiscano una base solida, e non vengano quindi ripetute; i requisiti e l'architettura del sistema devono essere immediatamente fissati e diventare essenziali per la pianificazione dei cicli incrementali;
	\item La realizzazione è incrementale. La \termine{Progettazione di Dettaglio}, la \termine{Codifica} e i test vengano ripetuti più volte, in modo da garantire un miglioramento continuo ad ogni versione del prodotto.
\end{itemize}

I vantaggi attesi dalla scelta di tale modello sono i seguenti:
\begin{itemize}
	\item Le funzionalità essenziali, ovvero quelle che soddisfano i principali requisiti, sono sviluppate nei primi incrementi;
	\item Ogni incremento può produrre valore e riduce il rischio di fallimento, in quando esso consolida ulteriormente la base su cui si sta lavorando;
	\item L'esecuzione più dettagliata dei test che quindi risulteranno essere maggiormente esaustivi;
	\item I rilasci multipli e in successione, che inizialmente punteranno ad identificare e soddisfare i requisiti di primaria importanza ossia le componenti più critiche del sistema. 
	\item Il favoreggiamento della creazione di prototipi, ovvero di parti di prodotto funzionanti, che a loro volta favoriscono il dialogo con il cliente.
\end{itemize}

\subsection{Suddivisione delle attività}

I membri del gruppo, con l'obbiettivo di agevolare lo sviluppo del progetto seguendo il modello incrementale, hanno deciso congiuntamente di suddividere il carico di lavoro in 5 periodi assegnando agli stessi nomi significativi relativi all'attività predominante:
\begin{itemize}
	\item \ARM.
	\item \ARD.
	\item \PA.
	\item \PD\ e \COD.
	\item \VV.
\end{itemize}
Le attività di \PD\ e \COD\ sono essenzialmente concorrenti mentre il periodo di \termine{Verifica}, che comprende anche l'ultimo periodo di \VV\ sopracitato, è trasversale in tutti i periodi come sarà descritto in un paragrafo successivo.
Ad ognuno dei membri saranno assegnate delle attività principali da svolgere in un determinato periodo. Ognuno avrà dunque associato un \termine{diagramma di Gantt} al fine di agevolargli la visione e la comprensione delle tempistiche e delle scadenze delle varie attività assegnategli.
Ogni attività potrà, a sua volta, essere suddivisa in sotto-attività ed avere, o meno, forte dipendenze con altre attività.\\
Saranno inoltre definite delle \termine{milestone} esterne ed interne che coincideranno, rispettivamente, con le date di scadenza per la consegna dei documenti e con le scadenze per le revisioni stabilite dai membri del gruppo.
Ogni periodo, tra quelli sopraelencati, terminerà sempre con una \termine{milestone}.\\

Le diverse attività saranno rappresentate nel \termine{diagramma di Gantt} in termini temporali mediante linee blu.

Infine si informa che l'attività \ARM\ è un periodo di investimento da parte del gruppo per poter accedere al progetto didattico. Questa, dunque, non verrà rendicontata nel calcolo del costo totale che il software richiede.
\newpage
\subsubsection{\ARM}
Periodo: dal 13/12/2016 all'11/01/2017. \\

L'inizio del periodo di \ARM\ corrisponde all'inizio del progetto. In questo periodo il gruppo deve scegliere un \termine{capitolato} e cominciare a lavorare con il fine ultimo di aggiudicarselo.\\
Ciò comporta la stesura dei seguenti documenti:
 \begin{itemize}
 \item Esterni:
 	\begin{itemize}
 	 \item \AdR
 	 \item \PdP
	 \item \PdQ
 	\end{itemize}
 \item  Interni:
	\begin{itemize}
	\item \SdF
	\item \NdP
	\end{itemize} 
 \end{itemize}
 Il carico di lavoro viene suddiviso principalmente tra i ruoli di \An, \Am, \Ver\ e \Pm.
 Il periodo termina con una milestone esterna, corrispondente alla consegna dei documenti per la \RR.
 
 \begin{figure}[H]
	\centering 
	\includegraphics[scale=0.4]{Immagini/Gantt/ARM.png}
	\caption{Diagramma di Gantt, \ARM}
\end{figure}
\newpage
\subsubsection{\ARD}
Periodo: dal 12/01/2017 al 03/02/2017 \\

Questo periodo inizia subito dopo la consegna dei documenti per la \RR. Il termine corrisponde ad una \termine{milestone} interna che coincide con l'inizio del periodo successivo, la \PA.\\
Il \termine{gruppo} in questo periodo si impegnerà ad identificare, ampliare e fissare definitivamente i requisiti richiesti per lo svolgimento del progetto.\\
Verranno inoltre effettuate le modifiche necessarie, rilevate in seguito all'esito della \RR\, nei vari documenti.\\
I ruoli coinvolti maggiormente in questo periodo sono: \An, \Am, \Pm\ e \Ver.

\begin{figure}[H]
	\centering 
	\includegraphics[scale=0.45]{Immagini/Gantt/ARD.png}
	\caption{Diagramma di Gantt, \ARD}
\end{figure}
\newpage
\subsubsection{\PA}
Periodo: dal 06/02/2017 al 21/02/2017 \\

Inizia in seguito alla terminazione dell'\ARD\ ed il suo il termine prefissato coincide con una \termine{milestone} interna.
Il \termine{gruppo} si pone l'obiettivo di eseguire e fissare definitivamente la progettazione del sistema ad alto livello. In questa fase verrà pensata l'architettura generale del sistema per poi stendere quella definitiva nella fase successiva.
In quest'ultimo documento i \ProgP\ dovranno descrivere, ad alto livello, le scelte progettuali ed il \termine{design-pattern} scelti per la realizzazione dell'architettura generale del \termine{prodotto}. Vengono inoltre incrementati i documenti \NdP, \PdP, \PdQ\ e \Gl.\\
In questo periodo i ruoli maggiormente interessati sono: \Prog, \Pm, \Ver\ e \Am.

 \begin{figure}[H]
	\centering 
	\includegraphics[scale=0.5]{Immagini/Gantt/PA.png}
	\caption{Diagramma di Gantt, \PA}
\end{figure}
\newpage
\subsubsection{\PD\ e \COD}
Periodo: dal 22/02/2017 all'11/04/2017\\

Dal seguente periodo si procede con la realizzazione incrementale del prodotto, ovvero verranno realizzati prima i requisiti essenziali, poi quelli desiderabili e infine quelli facoltativi per ridurre il rischio di fallimento. 
L'attività di \PD\ inizia subito dopo l'approvazione della \PA\ mentre la \COD\ inizia solo il 14/03/2017, una volta completata la \PD\ delle funzionalità essenziali. Questo periodo si conclude con una \termine{milestone} esterna che coincide con la consegna del prodotto alla \RQ.
Il gruppo si impegna a terminare la \PD\ almeno delle funzionalità essenziali per il 06/03/2017, che coincide con una \termine{milestone} esterna per la consegna dei documenti della \RP.\\
L'obiettivo finale consiste nel consegnare un prodotto software completo, e le attività che ad ogni incremento verranno eseguite a tal fine sono le seguenti:
\begin{itemize}
\item Redigere o aggiornare il documento \DDP\ in cui i \ProgP\ devono descrivere il comportamento e le interazioni tra i vari componenti del sistema.
\item Incrementare i documenti \NdP, \PdP, \PdQ\ e \Gl.
\item I \ProgrP\ devono sviluppare il codice del prodotto software attenendosi il più possibile a quanto scritto dai \ProgP\ nel documento \DDP. L'attività di \COD\ deve essere fatta seguendo, iterativamente, i seguenti passi:
			\begin{itemize}
				\item \termine{Codifica} dell'incremento svolto dai \ProgP.
				\item \termine{Verifica} dell'incremento.
			\end{itemize}
\item Stesura o ampliamento del \MU: questo documento è destinato all'utilizzatore finale del prodotto che, tramite esso, deve essere in grado di capire le principali funzionalità del sistema e come utilizzarle.
\end{itemize}

I ruoli maggiormente interessati sono quelli di \Am, \Pm, \Prog, \Progr\ e \Ver.

 \begin{figure}[H]
	\centering 
	\includegraphics[scale=0.3]{Immagini/Gantt/COD.png}
	\caption{Diagramma di Gantt, \PD\ e \COD}
\end{figure}
\newpage
\subsubsection{Verifica}
Periodo: dal 16/12/2016 al 14/05/2017.\\

Questo periodo, come si può osservare dalla data di inizio, è trasversale a tutti i periodi. Infatti, per garantire ad ogni incremento efficacia ed efficienza del prodotto, le operazioni di \termine{testing} e di \termine{verifica} vengono eseguite durante tutto il corso del progetto.\\

\paragraph{\VV}
Periodo: dal 12/04/2017 al 15/05/2017 \\

Al termine del periodo di \PD\ e \COD\ verranno effettuati ed intensificati tutti i test necessari per garantire che il prodotto soddisfi tutti i requisiti dell'\AdR e la correzione di eventuali errori nel codice e/o nei vari documenti.
Le attività prevedono di:
\begin{itemize}
	\item Effettuare dei test di sistema.
	\item Incrementare, correggere o aggiornare i documenti di: \MU, \NdP, \PdP, \PdQ\ e \Gl.
	\item Verificare tutti i documenti sopra citati.
	\item Verificare il corretto funzionamento del prodotto correggendo eventuali errori.
\end{itemize}
In questo periodo i ruoli maggiormente interessati sono quelli di \Ver, \Prog, \Am\ e \Pm.

 \begin{figure}[H]
	\centering 
	\includegraphics[scale=0.4]{Immagini/Gantt/VV.png}
	\caption{Diagramma di Gantt, \VV}
\end{figure}