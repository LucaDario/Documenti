\paragraph{Caso d'uso UC 1.2.2: Utilizzo di una bolla di tipo checkbutton}
\label{UC 1.2.2: Utilizzo di una bolla di tipo checkbutton}
\begin{figure}[ht]
	\centering
	\includegraphics[scale=0.80]{Usecases/img/{UC1.2.2}.png}
	\caption{Caso d'uso UC 1.2.2: Utilizzo di una bolla di tipo checkbutton}
\end{figure}
\FloatBarrier
\begin{itemize}
\item\textbf{Attori}: Sviluppatore.
\item\textbf{Descrizione}: Lo sviluppatore vuole utiluzzare un bottone particolare che possiede due stati: "clicked" e "unclicked".
\item\textbf{Precondizione}: Lo sviluppatore vuole utilizzare una bolla di tipo checkbutton.
\item\textbf{Postcondizione}: Lo sviluppatore he generato codice eseguibile per creare una bolla di tipo checkbutton con i parametri scelti.
\item\textbf{Scenario principale}:
	\begin{itemize}
		\item Lo sviluppatore vuole impostare lo stato del bottone (UC 1.2.2.1).
		\item Lo sviluppatore vuole personalizzare il bottone (UC1.2.2.3).
	\end{itemize}
\item\textbf{Scenario principale}:
	\begin{itemize}
		\item Lo sviluppatore non imposta lo stato del bottone (UC 1.2.2.2).
	\end{itemize}
\end{itemize}
