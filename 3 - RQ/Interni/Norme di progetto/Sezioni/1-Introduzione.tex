\newpage
\section{Introduzione}

\subsection{Scopo del documento}
Questo documento definisce le norme interne del gruppo \gruppo\ valide per lo svolgimento del \termine{capitolato} d'appalto \progetto . Ogni membro del gruppo ha l'obbligo di seguirne rigorosamente il contenuto di esso con lo scopo di garantire l'\termine{uniformità} e la  \termine{conformità} del materiale prodotto. \\
Oltre a ciò, al fine di assicurare anche \termine{efficacia} ed \termine{efficienza} del lavoro svolto, verranno specificate norme riguardanti:
\begin{itemize}
	\item La definizione dei ruoli e l'identificazione delle relative mansioni;	
	\item Il \termine{way of working} nelle varie fasi del progetto;
	\item Le modalità di stesura dei documenti e le relative convenzioni utilizzate;
	\item La definizione degli \termine{ambienti di sviluppo};
	\item Le modalità di comunicazione e cooperazione tra i membri del gruppo.
\end{itemize}

\subsection{Scopo del prodotto}
\scopoProdotto

%Lo scopo del prodotto consiste in 2 parti distinte.
%
%\subsubsection{SDK}
%La prima parte consiste nell'implementazione di un \termine{SDK}, \progetto, che permetta agli sviluppatori di creare facilmente \termine{bolle interattive}. \progetto, come \termine{pacchetto stand-alone}, dovrà essere facilmente installabile e distribuibile in una istanza del \termine{server} \termine{Rocket.chat}.
%Le funzionalità base che \progetto dovrà fornire sono:
%\begin{itemize}
%\item
%Un insieme di bolle predefinite, pronte ad essere rese disponibili agli utenti finali;
%\item
%Un insieme di \termine{API} per gli sviluppatori per sviluppare e rilasciare nuove tipologie di bolle interattive.
%\end{itemize}
%Il SDK dovrà dunque offrire la possibilità di creare bolle appartenenti almeno a queste 3 categorie:
%\begin{itemize}
%\item
%\termine{Rich media bubble};
%\item
%Self-updating bubble;
%\item
%\termine{Editing bubble}.
%\end{itemize}
%
%\subsubsection{Demo}
%La seconda parte, per dimostrare le capacità di \progetto , consiste nella creazione di una \termine{demo} che implementi una bolla in un \termine{caso d'uso} significativo. 

\subsection{Glossario}
\descrizioneGlossario
%Per evitare ogni ambiguità, i termini tecnici, gli acronimi e le parole che necessitano di ulteriori spiegazioni, saranno presenti nel documento \glossario.  In tutti i documenti, tali voci saranno marcate con il pedice \ped{G}.

\subsection{Riferimenti}

\subsubsection{Normativi}
\riferimentiNormativi

\begin{itemize}
	\item \textbf{\termine{ISO} IEC 90003-2004}: \\
		  \link{http://www.praxiom.com/iso-90003.htm};
	\item \textbf{ISO 12207-1995}: \\
		  \link{http://www.math.unipd.it/~tullio/IS-1/2009/Approfondimenti/ISO\_12207-1995.pdf};
\end{itemize}

\subsubsection{Informativi}
\begin{itemize}
	\item \textbf{\textit{Slide} del corso di Ingegneria del Software sulla gestione delle persone}: \\
 	  \link{http://www.math.unipd.it/~tullio/IS-1/2006/Dispense/L05b.pdf};
	\item \textbf{Linee guida \textit{"The Twelve-Factor App"}}: \\
		  \link{https://12factor.net}.
\end{itemize}

\newpage