\section*{C}
\addcontentsline{toc}{section}{C}
\begin{itemize}
	\item
	\textbf{Callback}: In programmazione, un callback (o, in italiano, richiamo) è, in genere, una funzione, o un "blocco di codice" che viene passata come parametro ad un'altra funzione.
	\item
	\textbf{Camel Case}: Pratica di scrivere parole composte o frasi unendo le parole tra loro ma lasciando le loro iniziali maiuscole es: (Ciao mondo = CiaoMondo).
	\item
	\textbf{Capability Maturity Model}: Modello di riferimento costituito da pratiche consolidate in una disciplina specifica.
	\item
	\textbf{Capability Maturity Model Integration -- Cmmi}: Approccio al miglioramento dei processi il cui obiettivo è di aiutare un'organizzazione a migliorare le sue prestazioni.
	\item
	\textbf{Capitolato}: Atto allegato a un contratto d'appalto che intercorre tra il cliente ed una ditta appaltatrice in cui vengono indicate modalità, costi e tempi di realizzazione dell'opera oggetto del contratto.
	\item
	\textbf{Caso D'Uso}: Tecnica usata nei processi di ingegneria del software per effettuare in maniera esaustiva e non ambigua, la raccolta dei requisiti al fine di produrre software di qualità.
	\item
	\textbf{Chai}: E' una libreria per verificare che lo stato di una variabile o oggetto sia quello desiderato.
	\item
	\textbf{Checkbutton}: Componenete che ha due stati: selezionato e non selezionato.
	\item
	\textbf{Ciclo Di Deming}: Metodo di gestione in quattro fasi, utilizzato in attività per il controllo e il miglioramento continuo dei processi e dei prodotti.
	\item
	\textbf{Clean Architecture}: Pattern Architetturale e ideato da Robert Cecil Martin.
	\item
	\textbf{Codifica}: Periodo durante il quale uno o più programmatori scrivono del codice (\textit{codificano}) al fine di creare un prodotto software.
	\item
	\textbf{Commento A Blocco}: Pratica che consente di scrivere un commento separandolo su più linee.
	\item
	\textbf{Commit}: Comando del software \termine{git} attraverso il quale è possibile salvare localmente una o più modifiche eseguite all'interno di uno o più file, spesso allegando un messaggio identificativo di tali modifiche.
	\item
	\textbf{Committente}: Figura che commissiona un lavoro, indipendentemente dall'entità o dall'importo.
	\item
	\textbf{Complessità Ciclomatica}: Metrica software utilizzata per misurare la complessità di un programma.
	\item
	\textbf{Conformità}: Situazione in cui il software prodotto risulta essere coerente con gli obiettivi stabiliti in precedenza alla sua realizzazione.
	\item
	\textbf{Css}: \textit{eng. Cascading Style Sheets}, in italiano \textit{fogli di stile a cascata}. Linguaggio usato per definire la formattazione di documenti HTML, XHTML e XML ad esempio i siti web e relative pagine web.
	\item
	\textbf{Css3}: Versione 3 del linguaggio \termine{CSS}.
	\item
	\textbf{Csshint}: Software per rendere lo stile di scrittura dei documenti css conforme alle regole stabilite dal gruppo.
\end{itemize}
\newpage