\section*{W}
\addcontentsline{toc}{section}{W}
\begin{itemize}
	\item
	\textbf{Walkthrough}: Tecnica nella quale un membro del \termine{team} di sviluppo guida gli altri membri partecipanti attraverso la verifica di un prodotto, mentre quest'ultimi pongono domande e fanno commenti sulla eventuale presenza di errori.
	\item
	\textbf{Way Of Working}: Insieme di \termine{Best Practices}.
	\item
	\textbf{Web App}: Applicazione in grado di eseguire su un web browser.
	\item
	\textbf{Web Chat}: Applicazione di messaggistica istantanea che prevede un'interfaccia web per il suo utilizzo.
	\item
	\textbf{Webhook}: \textit{eng. Uncino web}. Metodo per aumentare o alterare il comportamento di una pagina web, o di un'applicazione web con chiamate di ritorno (callback) personalizzate.
	\item
	\textbf{Work Product}: Prodotti generati dal \termine{team} di sviluppo.
	\item
	\textbf{Wrike}: Strumento online per la collaborazione e il project management che permette ai suoi utenti di modificare progetti, classificare le attività per importanza, tenere traccia dei programmi e collaborare online con altri utenti dello stesso gruppo.
\end{itemize}
\newpage