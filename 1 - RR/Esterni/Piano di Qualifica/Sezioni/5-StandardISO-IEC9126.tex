\section{Standard ISO/IEC9126}
Lo standard ISO/IEC 9126 è stato redatto con lo scopo di descrivere quali sono gli obiettivi qualitativi che un prodotto software deve soddisfare. Questi vengono suddivisi in 3 aree tematiche diverse:
\begin{itemize}
	\item\textbf{Qualità esterna}: rappresenta la qualità del software nel momento in cui esso viene eseguito e testato;
	\item\textbf{Qualità interna}: rappresenta la qualità del software per quanto riguarda le sue caratteristiche implementative, durante le fasi di progettazione e codifica;
	\item\textbf{Qualità in uso}: rappresenta la qualità del software dal punto di vista del cliente che lo sta utilizzando.
\end{itemize}

Lo standard delinea sei macro-obbiettivi qualitativi, i quali sono suddivisi a loro volta in sotto caratteristiche specifiche:
\begin{itemize}
	\item\textbf{Funzionalità}: capacità del prodotto di fornire tutte le funzioni che sono state individuate attraverso l'\AdR:
	\begin{itemize}
		\item\textbf{Adeguatezza}: le funzionalità fornite devono essere conformi rispetto le aspettative;
		\item\textbf{Accuratezza}: il prodotto deve fornire i risultati attesi, con il livello di precisione richiesto;
		\item\textbf{Interoperabilità}: il prodotto deve poter interagire ed operare con uno o più sistemi specifici;
		\item\textbf{Sicurezza}: il prodotto deve proteggere le informazioni e i dati da accessi e modifiche non autorizzati;
		\item\textbf{Conformità di funzionalità}: il prodotto deve aderire a standard, regole e convenzioni inerenti alla funzionalità.
	\end{itemize}

	\item\textbf{Affidabilità}: capacità del prodotto software di svolgere correttamente le sue funzioni durante il suo utilizzo, anche nel caso in cui si presentino situazioni anomale:
	\begin{itemize}
		\item\textbf{Maturità}: il prodotto deve evitare che si verifichino malfunzionamenti o che vengano prodotti risultati non corretti;
		\item\textbf{Tolleranza agli errori}: nel caso in cui si presentino degli errori, dovuti a guasti o ad un uso scorretto dell'applicativo, questi devono essere gestiti in modo da mantenere alto il livello di prestazione;
		\item\textbf{Recuperabilità}: il prodotto deve essere in grado di ristabilire un
adeguato livello di prestazioni e di recuperare i dati rilevanti in seguito a errori o malfunzionamenti;
		\item\textbf{Conformità di affidabilità}: il prodotto deve aderire a standard, regole e convenzioni inerenti all'affidabilità.
	\end{itemize}

	\item\textbf{Usabilità}: capacità del prodotto di essere facilmente comprensibile e attraente in ogni sua parte per qualsiasi utente che lo andrà ad utilizzare:
	\begin{itemize}
		\item\textbf{Comprensibilità}: l'utente deve essere in grado di riconoscerne le funzionalità offerte dal software e deve comprenderne le modalità di utilizzo per riuscire a raggiungere i risultati attesi;
		\item\textbf{Apprendibilità}: deve essere data la possibilità all'utente di imparare ad utilizzare l'applicazione senza troppo impegno;
		\item\textbf{Operabilità}: le funzionalità presenti devono essere coerenti con le aspettative dell'utente;
		\item\textbf{Attrattiva}: il software deve essere piacevole per chi ne fa uso;
		\item\textbf{Conformità di usabilità}: il prodotto deve aderire a standard, regole e convenzioni inerenti all'usabilità.
	\end{itemize}

	\item\textbf{Efficienza}: capacità di eseguire le funzionalità offerte nel minor tempo possibile utilizzando al tempo stesso il minor numero di risorse possibili:
	\begin{itemize}
		\item\textbf{Comportamento rispetto al tempo}: per svolgere le sue funzioni il software deve fornire adeguati tempi di risposta ed elaborazione;
		\item\textbf{Utilizzo delle risorse}: il software quando esegue le sue funzionalità deve utilizzare un appropriato numero e tipo di risorse;
		\item\textbf{Conformità di efficienza}: il prodotto deve aderire a standard, regole e convenzioni inerenti all'efficienza.
	\end{itemize}

	\item\textbf{Manutenibilità}: capacità del prodotto di essere modificato, tramite correzioni, miglioramenti o adattamenti del software a cambiamenti negli ambienti, nei requisiti e nelle specifiche funzionali:
	\begin{itemize}
		\item\textbf{Analizzabilità}: il software deve consentire una rapida identificazione delle possibili cause di errori e malfunzionamenti;
		\item\textbf{Modificabilità}: il prodotto originale deve permettere eventuali cambiamenti in alcune sue parti;
		\item\textbf{Stabilità}: non devono insorgere effetti indesiderati in seguito a modifiche effettuate sul software;
		\item\textbf{Testabilità}: il software deve poter essere facilmente testato per validare le modifiche effettuate;
		\item\textbf{Conformità di manutenibilità}: il prodotto deve aderire a standard, regole e convenzioni inerenti alla manutenibilità.
	\end{itemize}

	\item\textbf{Portabilità}: capacità del software di poter essere utilizzato su diversi ambienti:
	\begin{itemize}
		\item\textbf{Adattabilità}: il prodotto deve adattarsi a tutti quegli ambienti di lavoro nei quali è stato previsto un suo utilizzo, senza dover apportare modifiche allo stesso;
		\item\textbf{Installabilità}: il software deve poter essere installato in determinati ambienti di lavoro;
		\item\textbf{Coesistenza}: il prodotto può coesistere in ambienti comuni
con altri software, condividendo risorse comuni;
		\item\textbf{Sostituibilità}: l'applicativo deve poter sostituire un altro software che ha lo stesso scopo e lavora nel medesimo ambiente;
		\item\textbf{Conformità di portabilità}: il prodotto deve aderire a standard, regole e convenzioni inerenti alla portabilità.
	\end{itemize}
\end{itemize}

\subsection{Scelte del team}
Per garantire una buona qualità di prodotto, il \termine{team} ha individuato dallo standard \textit{ISO/IEC 9126} le qualità che ritiene più importanti nell'arco del ciclo di vita del prodotto e le ha istanziate individuando obiettivi e metriche coerenti con i livelli di qualità perseguiti.

\subsection{Funzionalità}
Rappresenta la capacità del prodotto di fornire tutte le funzioni che sono state individuate attraverso l'\AdR.

\subsubsection{Obiettivi di qualità}
Il \termine{team} si impegnerà affinché:
\begin{itemize}
\item \textbf{Adeguatezza}: le funzionalità fornite siano conformi rispetto le aspettative;
\item \textbf{Accuratezza}: il prodotto fornisca i risultati attesi, con il livello di dettaglio richiesto;
\item \textbf{Sicurezza}: il prodotto protegga le informazioni e i dati da accessi e modifiche non autorizzati.
\end{itemize}

\subsubsection{Metriche}
\paragraph{Completezza dell'implementazione funzionale}
Indica la percentuale di requisiti funzionali coperti dall'implementazione.
\begin{itemize}
	\item \textbf{Misurazione}: 
		$$C=\left(1-\mathlarger{\frac{N_{FM}}{N_{FI}}}\right) \cdot 100$$ 
	dove $N_{FM}$ è il numero di funzionalità mancanti nell'implementazione e $N_{FI}$ è il numero di funzionalità individuate nell'attività di analisi.
	\item \textbf{Range ottimale}: 100.
	\item \textbf{Range di accettazione}: 100.
\end{itemize}

\paragraph{Accuratezza rispetto alle attese}
Indica la percentuale di risultati concordi alle attese.
\begin{itemize}
	\item \textbf{Misurazione}: 
		$$A=\left(1-\mathlarger{\frac{N_{RD}}{N_{TE}}}\right) \cdot 100$$
	dove $N_{RD}$ è il numero di test che producono risultati discordanti rispetto alle attese e $N_{TE}$ è il numero di test-case eseguiti.
	\item \textbf{Range ottimale}: 100.
	\item \textbf{Range di accettazione}: 90 -- 100.
\end{itemize}

\paragraph{Controllo degli accessi}
Indica la percentuale di operazioni illegali non bloccate.
\begin{itemize}
	\item \textbf{Misurazione}: 
		$$I=\mathlarger{\frac{N_{IE}}{N_{II}}} \cdot 100$$
	dove $N_{IE}$ è il numero di operazioni illegali effettuabili dai test e $N_{II}$ è il numero di operazioni illegali individuate.
	\item \textbf{Range ottimale}: 0.
	\item \textbf{Range di accettazione}: 0 -- 10.
\end{itemize}


\subsection{Affidabilità}
Rappresenta la capacità del prodotto software di svolgere correttamente le sue funzioni durante il suo utilizzo, anche nel caso in cui si presentino situazioni anomale.

\subsubsection{Obiettivi di qualità}
L'esecuzione del prodotto dovrà presentare le seguenti caratteristiche:
\begin{itemize}
\item \textbf{Maturità}: dovrà essere evitato che si verifichino malfunzionamenti, operazioni illegali e restituzione di risultati errati (\textit{failure}) in seguito a difetti;
\item \textbf{Tolleranza agli errori}: nel caso in cui si presentino degli errori, dovuti a guasti o ad un uso scorretto dell'applicativo, questi dovranno essere gestiti in modo da mantenere alto il livello di prestazione.
\end{itemize}

\subsubsection{Metriche}
\paragraph{Densità di \textit{failure}}
Indica la percentuale di operazioni di testing che si sono concluse in fallimenti.

\begin{itemize}
	\item \textbf{Misurazione}: 
		$$F=\mathlarger{\frac{N_{FR}}{N_{TE}}} \cdot 100$$
	dove $N_{FR}$ è il numero di fallimenti rilevati durante l'attività di testing e $N_{TE}$ è il numero di test-case eseguiti.
	\item \textbf{Range ottimale}: 0.
	\item \textbf{Range di accettazione}: 0 -- 10.
\end{itemize}

\paragraph{Blocco di operazioni non corrette}
Indica la percentuale di funzionalità in grado di gestire correttamente i \textit{fault} che potrebbero verificarsi.
\begin{itemize}
	\item \textbf{Misurazione}: 
		$$B=\mathlarger{\frac{N_{FE}}{N_{ON}}} \cdot 100$$
	dove $N_{FE}$ è il numero di \textit{failure} evitati durante i test effettuati e $N_{ON}$ è il numero di test-case eseguiti che prevedono l'esecuzione di operazioni non corrette, causa di possibili \textit{failure}.
	\item \textbf{Range ottimale}: 100.
	\item \textbf{Range di accettazione}: 80 -- 100.
\end{itemize}

\subsection{Usabilità}
Rappresenta la capacità del prodotto di essere facilmente comprensibile e attraente in ogni sua parte per qualsiasi utente che lo andrà ad utilizzare.

\subsubsection{Obiettivi di qualità}
Il prodotto dovrà puntare ai seguenti obiettivi di usabilità:
\begin{itemize}
\item \textbf{Comprensibilità)}: l'utente dovrà essere in grado di riconoscerne le funzionalità offerte dal software e dovrà comprenderne le modalità di utilizzo per riuscire a raggiungere i risultati attesi;
\item \textbf{Apprendibilità}: dovrà essere data la possibilità all'utente di imparare ad utilizzare l'applicazione senza troppo impegno;
\item \textbf{Operabilità}: le funzionalità presenti dovranno essere coerenti con le aspettative dell'utente.
\end{itemize}

\subsubsection{Metriche}
\paragraph{Comprensibilità delle funzioni offerte}
Indica la percentuale di operazioni comprese in modo immediato dall'utente, senza la consultazione del manuale.
\begin{itemize}
	\item \textbf{Misurazione}: 
		$$C=\mathlarger{\frac{N_{FC}}{N_{FO}}} \cdot 100$$
	dove $N_{FC}$ è il numero di funzionalità comprese in modo immediato dall'utente durante l'attività di testing del prodotto e $N_{FO}$ è il numero di funzionalità offerte dal sistema.
	\item \textbf{Range ottimale}: 90 -- 100.
	\item \textbf{Range di accettazione}: 80 -- 100.
\end{itemize}

\paragraph{Facilità di apprendimento delle funzionalità}
Indica il tempo medio impiegato dall'utente nell'imparare ad usare correttamente una data funzionalità.
\begin{itemize}
	\item \textbf{Misurazione}: indicatore numerico, espresso in minuti, che tiene traccia del tempo medio impiegato dall'utente nell'apprendere il corretto utilizzo di una funzionalità offerta dal sistema.
	\item \textbf{Range ottimale}: 0 -- 15.
	\item \textbf{Range di accettazione}: 0 -- 30.
\end{itemize}

\paragraph{Consistenza operazionale in uso}
Indica la percentuale di messaggi e funzionalità offerte all'utente che rispettano le sue aspettative riguardo al comportamento del software.
\begin{itemize}
	\item \textbf{Misurazione}: 
		$$C=\left(1-\mathlarger{\frac{N_{MFI}}{N_{MFO}}}\right) \cdot 100$$
	dove $N_{MFI}$ è il numero di messaggi e funzionalità che non rispettano le aspettative dell'utente e $N_{MFO}$ è il numero di messaggi e funzionalità offerti dal sistema.
	\item \textbf{Range ottimale}: 90 -- 100.
	\item \textbf{Range di accettazione}: 80 -- 100.
\end{itemize}

\subsection{Efficienza}
\label{efficienza}
Rappresenta la capacità di eseguire le funzionalità offerte dal software nel minor tempo possibile utilizzando al tempo stesso il minor numero di risorse possibili.

\subsubsection{Obiettivi di qualità}
Il prodotto dovrà essere efficiente, in particolare:
\begin{itemize}
\item \textbf{Comportamento rispetto al tempo}:  per svolgere le sue funzioni il software dovrà fornire adeguati tempi di risposta ed elaborazione;
\item \textbf{Utilizzo delle risorse}: il software quando eseguirà le sue funzionalità dovrà utilizzare un appropriato numero e tipo di risorse.
\end{itemize}

\subsubsection{Metriche}
\paragraph{Tempo di risposta}
Indica il periodo temporale medio che intercorre fra la richiesta al software di una determinata funzionalità e la restituzione del risultato all'utente.
\begin{itemize}
	\item \textbf{Misurazione}: 
		$$T_{RISP} = \mathlarger{\frac{\sum_{i=1}^{n} T_{i}}{n}}$$ 
	con $T_{RISP}$ misurato in secondi, e dove $T_{i}$ è il tempo intercorso fra la richiesta $i$ di una funzionalità ed il completamento delle operazioni necessarie a restituire un risultato a tale richiesta.
	\item \textbf{Range ottimale}: 0 -- 3.
	\item \textbf{Range di accettazione}: 0 -- 8.
\end{itemize}

\subsection{Manutenibilità}
Rappresenta la capacità del prodotto di essere modificato, tramite correzioni, miglioramenti o adattamenti del software a cambiamenti negli ambienti, nei requisiti e nelle specifiche funzionali.

\subsubsection{Obiettivi di qualità}
Le operazioni di manutenzione andranno agevolate il più possibile adottando le seguenti caratteristiche:
\begin{itemize}
\item \textbf{Analizzabilità}: il software dovrà consentire una rapida identificazione delle possibili cause di errori e malfunzionamenti;
\item \textbf{Modificabilità}: il prodotto originale dovrà permettere eventuali cambiamenti in alcune sue parti;
\item \textbf{Stabilità}: non dovranno insorgere effetti indesiderati in seguito a modifiche effettuate sul software;
\item \textbf{Testabilità}: il software dovrà poter essere facilmente testato per validare le modifiche effettuate.
\end{itemize}

\subsubsection{Metriche}
\paragraph{Capacità di analisi di \textit{failure}}
Indica la percentuale di \textit{failure} registrate delle quali sono state individuate le cause.
\begin{itemize}
	\item \textbf{Misurazione}: 
		$$I=\mathlarger{\frac{N_{FI}}{N_{FR}}} \cdot 100$$
	dove $N_{FI}$ è il numero di \textit{failure} delle quali sono state individuate le cause e $N_{FR}$ è il numero di \textit{failure} rilevate.
	\item \textbf{Range ottimale}: 80 -- 100.
	\item \textbf{Range di accettazione}: 60 -- 100.
\end{itemize}

\paragraph{Impatto delle modifiche}
Indica la percentuale di modifiche effettuate in risposta a \textit{failure} che hanno portato all'introduzione di nuove \textit{failure} in altre componenti del sistema.
\begin{itemize}
	\item \textbf{Misurazione}: 
		$$I=\mathlarger{\frac{N_{FRF}}{N_{FR}}} \cdot 100$$
	dove $N_{FRF}$ è il numero di \textit{failure} risolte con l'introduzione di nuove \textit{failure} e $N_{FR}$ è il numero di \textit{failure} risolte;
	\item \textbf{Range ottimale}: 0 -- 10.
	\item \textbf{Range di accettazione}: 0 -- 20.
\end{itemize}

\newpage