\section{Introduzione}
\subsection{Scopo del documento}
Questo documento vuole definire le strategie che il \termine{team} ha deciso di adottare per perseguire gli obiettivi di qualità di processo e di prodotto ricercati. A tal fine è necessaria una costante attività di verifica e validazione del lavoro svolto in modo da poter rilevare e correggere le anomalie che potrebbero nascere.

\subsection{Scopo del prodotto}
\scopoProdotto

\subsection{Glossario}
\descrizioneGlossario

\subsection{Riferimenti}
\subsubsection{Normativi}
\riferimentiNormativi

\subsubsection{Informativi}
\begin{itemize}
	\item \textbf{\AdR}: \analisiDeiRequisiti;
	\item \textbf{\PdP}: \pianoDiProgetto;
	\item \textbf{\textit{Slide} dell'insegnamento di Ingegneria del Software}: \\
		  \link{http://www.math.unipd.it/~tullio/IS-1/2016/}
	\item \textbf{\textit{Standard} ISO/IEC 9126}: Product quality \\
	 	  \link{https://en.wikipedia.org/wiki/ISO/IEC\_9126}
	\item \textbf{\textit{Standard} tecnici ISO/IEC 15504}: Software process assessment \\
		  \link{https://en.wikipedia.org/wiki/ISO/IEC\_15504}
	\item \textbf{Ciclo di Deming (\termine{PDCA})}: Miglioramento dei processi \\
		  \link{https://en.wikipedia.org/wiki/PDCA}
\end{itemize}

\newpage