\section*{S}
\addcontentsline{toc}{section}{S}
\begin{itemize}
	\item
	\textbf{Scripting}: Linguaggio di programmazione destinato in genere a compiti di automazione.
	\item
	\textbf{Scss}: E' una estensione del linguaggio \termine{CSS} che aggiunge nuove utili e potenti funzionalità.
	\item
	\textbf{Sdk}: \textit{eng. Software Development Kit}. Insieme di strumenti per lo sviluppo e la documentazione di software.
	\item
	\textbf{Server}: Componente o sottosistema informatico di elaborazione e gestione del traffico di informazioni che fornisce, a livello logico e fisico, un qualunque tipo di servizio ad altre componenti (tipicamente chiamate \textit{clients}, cioè clienti) che ne fanno richiesta attraverso una rete di computer, all'interno di un sistema informatico o anche direttamente in locale.
	\item
	\textbf{Sinon}: \termine{Framework} per eseguire il \termine{mock} di componenti software durante le fasi di testing.
	\item
	\textbf{Slack}: Sistema di collaborazione di gruppo \textit{cloud-based}. All'interno di questo sistema è possibile creare svariate chat all'interno delle quali si possono scrivere messaggi testuali, oppure inviare qualunque genere di file.
	\item
	\textbf{Sonarqube}: Piattaforma open source creata con lo scopo di offrire un servizio di ispezione continua della qualità del codice.
	\item
	\textbf{Stakeholder}: Indica genericamente un soggetto (o un gruppo di soggetti) influente nei confronti di un'iniziativa economica, che sia un'azienda o un progetto.
	\item
	\textbf{Stand-Alone}: Componenete software che non dipende da librerie esterne.
	\item
	\textbf{Statement}: \textit{eng. Dichiarazione}. All'interno dei linguaggi di programmazione, la più piccola unità di un programma che identifica una azione che deve essere eseguita.
	\item
	\textbf{Sviluppatore}: Programmatore che si prende cura di uno o più aspetti del ciclo di vita del software.
\end{itemize}
\newpage