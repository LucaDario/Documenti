\paragraph{Caso d'uso UC 4.2.5: Modificare un oggetto della lista}
\label {Caso d'uso UC 4.2.5: Modificare un oggetto della lista}
\begin{figure}[ht]
	\centering
	\includegraphics[scale=0.80]{Usecases/img/{UC4.2.5}.png}
	\caption {Caso d'uso UC 4.2.5: Modificare un oggetto della lista}
\end{figure}
\FloatBarrier
\begin{itemize}
\item\textbf{Attori}: Utente con permessi.
\item\textbf{Descrizione}: L'attore modifica un oggetto della lista
\item\textbf{Precondizione}: L'attore ha i permessi per la modifica di un oggetto della lista.
\item\textbf{Postcondizione}: L'oggetto della lista viene modificato.
\item\textbf{Estensioni}:
	\begin{itemize}
		\item Visualizzazione messaggio d'errore(UC 4.2.5.2).
		\item Impostazione della quantità di prodotto richiesta a default(UC 4.2.5.7).
		\item Impostazione dell'unità di misura della quantità richiesta a default(UC 4.2.5.9). 
	\end{itemize}

	
\item\textbf{Scenario principale}:
	\begin{itemize}
		\item Modifica del nome del prodotto(UC 4.2.5.1).
		\item Modifica dell'immagine del prodotto(UC 4.2.5.3).
		\item Modifica della descrizione del prodotto(UC 4.2.5.4).
		\item Modifica delle note relative al prodotto(UC 4.2.5.5).
		\item Modifica della quantità richiesta di prodotto(UC 4.2.5.6).
		\item Modifica dell'unità di misura relativa alla quantità richiesta di prodotto(UC 4.2.5.8).
	\end{itemize}
\item\textbf{Scenario alternativo}:
	\begin{itemize}
		\item L'attore visualizza un messaggio d'errore a causa del mancato inserimento del nome del prodotto(UC 4.2.5.2).
		\item La quantità di prodotto viene aggiornata a default a causa del mancato inserimento di essa(UC 4.2.5.7).
		\item L'unità di misura della quantità di prodotto viene aggiornata a default a causa del mancato inserimento di essa(UC 4.2.5.9).
	\end{itemize}

\end{itemize}


\subparagraph{Caso d'uso UC 4.2.5.2 Visualizzazione messaggio d'errore.}
	\begin{itemize}
		\item\textbf{Attori}: Creatore della lista.
		\item\textbf{Descrizione}: L' attore visualizza un messaggio d'errore a causa del mancato inserimento del nome del prodotto.
		\item\textbf{Precondizione}: L' attore non imposta il nome del prodotto che vuole modificare.
		\item\textbf{Postcondizione}: L' attore visualizza un messaggio d'errore.
		\item\textbf{Scenario principale}:
			\begin{itemize}
				\item L'attore visualizza un messaggio d'errore. 
			\end{itemize}
		
	\end{itemize}
	
\subparagraph{Caso d'uso UC 4.2.5.7 Impostazione della quantità di prodotto richiesta a default.}
	\begin{itemize}
		\item\textbf{Attori}: Creatore della lista.
		\item\textbf{Descrizione}: La quantità di prodotto richiesta viene impostata a default a causa del mancato inserimento di essa.
		\item\textbf{Precondizione}: L'attore non imposta la quantità di prodotto che vuole modificare.
		\item\textbf{Postcondizione}:Il prodotto modificato ha una quantità  impostata a default.
		\item\textbf{Scenario principale}:
			\begin{itemize}
				\item La quantità del prodotto viene impostata a default.
			\end{itemize}
		
	\end{itemize}
\subparagraph{Caso d'uso UC 4.2.5.9 Impostazione dell'unità di misura della quantità richiesta a default.}
	\begin{itemize}
		\item\textbf{Attori}: Creatore della lista.
		\item\textbf{Descrizione}: L'unità di misura della quantità di prodotto richiesta viene impostata a default a causa del mancato inserimento di essa.
		\item\textbf{Precondizione}: L'attore non imposta l'unità di misura della quantità di prodotto che vuole modificare.
		\item\textbf{Postcondizione}: Il prodotto ha un unità di misura impostata a default.
		\item\textbf{Scenario principale}:
			\begin{itemize}
				\item L'unità di misura della quantità del prodotto viene impostata a default.
			\end{itemize}
		
\end{itemize}
