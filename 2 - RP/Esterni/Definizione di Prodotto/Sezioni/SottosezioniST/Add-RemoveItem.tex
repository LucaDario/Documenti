\subsubsection{GeneralView}
\begin{itemize}
\item \textbf{Descrizione}: Interfaccia che sta alla base della gerarchia delle componenti della lista-spesa.
\item \textbf{Utilizzo}:
\item \textbf{Attributi}: 
\item \textbf{Metodi}:
	\begin{itemize}
	\item \textit{public renderView():string}\\
	Genera il codice HTML CSS JS necessario per visualizzare una componente grafica della lista-spesa.
	\end{itemize}
\item \textbf{Eventi}:
\end{itemize}

\subsubsection{RemoveItemView}
\begin{itemize}
\item \textbf{Descrizione}: Questa interfaccia rappresenta la view relativa alla rimozione di un oggetto dalla lista-spesa.
\item \textbf{Utilizzo}: L'interfaccia viene utilizzata per disaccoppiare presenter e implementazione della rimozione, visualizza i dati che gli vengono passati dal presenter.
\item \textbf{Attributi}: 
\item \textbf{Metodi}:
\item \textbf{Eventi}:
	\begin{itemize}	
	\item \textit{public onItemRemoveClicked(item:ListItem):void}\\
	Evento che rappresenta il click, da parte dell'utente, sull'oggetto visuale necessario alla rimozione di un oggetto dalla lista.
			\item{\textbf{Parametri}: \begin{itemize}
			\item \textit{item:ListItem}\\
			L'oggetto che è stato cliccato e che l'utente desidera rimuovere dalla lista.
			\end{itemize}}
	\end{itemize}
\end{itemize}

\subsubsection{RemoveItemViewImpl}
\begin{itemize}
\item \textbf{Descrizione}: Questa classe rappresenta la rimozione di un oggetto dalla lista-spesa, implementando l'interfaccia RemoveItemView.
\item \textbf{Utilizzo}: Questa classe viene utilizzata dall'utente ogniqualvolta vuole rimuovere un oggetto alla lista-spesa.
\item \textbf{Attributi}: 
	\begin{itemize}
	\item \textit{private presenter:RemoveItemViewPresenter}\\
	Il presenter associato alla rimozione di un oggetto della lista, al quale questa classe delega la gestione del comportamento dell'elemento di rimozione degli oggetti.
	\end{itemize}
\item \textbf{Metodi}:
	\begin{itemize}
	\item \textit{public renderView():string}\\
		Genera il codice HTML CSS JS necessario per visualizzare la componente grafica della lista-spesa necessaria alla rimozione di un oggetto da essa.
	\item \textit{public RemoveItemViewImpl():RemoveItemViewImpl}\\
	Costruttore della classe RemoveItemViewImpl.
	\end{itemize}
\item \textbf{Eventi}:
\end{itemize}

\subsubsection{RemoveItemViewPresenter}
\begin{itemize}
\item \textbf{Descrizione}: Questa classe rappresenta il presenter per gli elementi di rimozione degli oggetti  della lista-spesa.
\item \textbf{Utilizzo}: Il presenter fa da tramite tra l'implementazione dell'elemento di rimozione e la view, formattando i dati che verranno visualizzati nella view e manipolando gli input dell'utente per eseguire le operazioni predisposte.
\item \textbf{Attributi}: 
	\begin{itemize}
	\item \textit{private view:RemoveItemView}\\
	La view associata al presenter.
	\end{itemize}
\item \textbf{Metodi}:
	\begin{itemize}
	\item \textit{RemoveItemViewPresenter(view:RemoveItemView):RemoveItemViewPresenter}\\
	Costruttore della classe RemoveItemViewPresenter.
			\item{\textbf{Parametri}: \begin{itemize}
			\item \textit{view:RemoveItemView}\\
			La view necessaria alla costruzione del presenter.
			\end{itemize}}
	\item \textit{public removeItem(listId:string,item:ListItem):void}\\
	Questo metodo serve per rimuovere un oggetto dalla lista-spesa.
			\item{\textbf{Parametri}: \begin{itemize}
			\item \textit{listId:string}\\
			L'id della lista dalla quale bisogna rimuovere l'oggetto.
			\item \textit{item:ListItem}\\
			L'oggetto da rimuovere.
			\end{itemize}}
	\item \textit{public renderView():string}\\
	Genera il codice HTML CSS JS necessario per visualizzare la componente grafica della lista-spesa necessaria alla rimozione di un oggetto da essa.
	\end{itemize}
\item \textbf{Eventi}:
\end{itemize}

\subsubsection{ModifyListUseCase}
\begin{itemize}
\item \textbf{Descrizione}: La classe rappresenta l'interazione con il database nel caso di modifiche alla lista.
\item \textbf{Utilizzo}: Ogni modifica dei dati degli oggetti della lista o della lista stessa passa per questa classe, che fa ponte tra il presenter e il database.
\item \textbf{Attributi}: 
	\begin{itemize}
	\item \textit{private databaseSource:DatabaseSource}\\
	Questo attributo è un riferimento all'interfaccia \texttt{DatabaseSource}, che permette di interfacciarsi al database \termine{MongoDB}.
	\end{itemize}
\item \textbf{Metodi}:
	\begin{itemize}
	\item \textit{public ModifyListUseCase(source:DatabaseSource):ModifyListUseCase}\\
	Costruttore della classe ModifyListUseCase.
			\item{\textbf{Parametri}: \begin{itemize}
			\item \textit{source:DatabaseSource}\\
			Attributo necessario alla costruzione della classe, per la comunicazione con  il database.
			\end{itemize}}
	\item \textit{public changeListInfo(listId:string,newList:ListData):void}\\
	Questo metodo serve per modificare i dati di una lista.
			\item{\textbf{Parametri}: \begin{itemize}
			\item \textit{listId:string}\\
			L'id della lista che si vuole modificare.
			\item \textit{newList:ListData}\\
			La nuova lista che si andrà a sostituire alla precedente.
			\end{itemize}}
	\item \textit{public addItemToList(listId:string,item:ListItem):void}\\
	Metodo che aggiunge un oggetto a una lista-spesa.
			\item{\textbf{Parametri}: \begin{itemize}
			\item \textit{listId:string}\\
			Id della lista alla quale si vuole aggiungere un oggetto.
			\item \textit{item:ListItem}\\
			Oggetto che si vuole aggiungere alla lista.
			\end{itemize}}
	\item \textit{public removeItemFromList(listId:string):void}\\
	Metodo che rimuove un oggetto da una list-spesa.
			\item{\textbf{Parametri}: \begin{itemize}
			\item \textit{listId:string}\\
			Id della lista dalla quale si vuole rimuovere un oggetto.
			\end{itemize}}
	\item \textit{public updateItemInsideList(listId:string,item:ListItem):void}\\
	Metodo che modifica un oggetto della lista.
			\item{\textbf{Parametri}: \begin{itemize}
			\item \textit{listId:string}\\
			Id della lista della quale si vuole modificare un oggetto.
			\item \textit{item:ListItem}\\
			Oggetto della lista che si vuole modificare.
			\end{itemize}}
	\end{itemize}
\item \textbf{Eventi}:
\end{itemize}

\subsubsection{DatabaseSource}
\begin{itemize}
\item \textbf{Descrizione}: 
\item \textbf{Utilizzo}: Utilizzata ogniqualvolta
\item \textbf{Attributi}: 
\item \textbf{Metodi}:
	\begin{itemize}
	\item \textit{public createListForUser(userId:string):ListData}\\

			\item{\textbf{Parametri}: \begin{itemize}
			\item \textit{userId:string}\\

			\end{itemize}}
	\item \textit{public removeList(listId:string):void}\\

			\item{\textbf{Parametri}: \begin{itemize}
			\item \textit{listId:string}\\

			\end{itemize}}
	\item \textit{public getListWithId(id:string):ListData}\\

			\item{\textbf{Parametri}: \begin{itemize}
			\item \textit{id:string}\\

			\end{itemize}}
	\item \textit{public changeList(listId:string):void}\\

			\item{\textbf{Parametri}: \begin{itemize}
			\item \textit{listId:string}\\

			\end{itemize}}
	\end{itemize}
\item \textbf{Eventi}:
	\begin{itemize}
	\item \textit{public onListCreated(listId:string,creatorId:string):void}\\

			\item{\textbf{Parametri}: \begin{itemize}
			\item \textit{listId:string}\\

			\item \textit{creatorId:string}\\

			\end{itemize}}
	\item \textit{public onListRemoved(listId:string,creatorId:string):void}\\

			\item{\textbf{Parametri}: \begin{itemize}
			\item \textit{listId:string}\\

			\item \textit{creatorId:string}\\

			\end{itemize}}
	\item \textit{public onListChanged(listId:string):void}\\

			\item{\textbf{Parametri}: \begin{itemize}
			\item \textit{listId:string}\\

			\end{itemize}}
	\end{itemize}
\end{itemize}

\subsubsection{AddItemView}
\begin{itemize}
\item \textbf{Descrizione}: Questa interfaccia rappresenta la view relativa all'aggiunta di un oggetto alla lista-spesa.
\item \textbf{Utilizzo}: L'interfaccia viene utilizzata per disaccoppiare presenter e implementazione dell'aggiunta, visualizza i dati che gli vengono passati dal presenter.
\item \textbf{Attributi}: 
\item \textbf{Metodi}:
\item \textbf{Eventi}:
	\begin{itemize}	
	\item \textit{public onAddItemClicked():void}\\
	Evento che rappresenta il click, da parte dell'utente, sull'oggetto visuale necessario all'aggiunta di un oggetto alla lista.
	\end{itemize}
\end{itemize}

\subsubsection{AddItemViewImplementation}
\begin{itemize}
\item \textbf{Descrizione}: Questa classe rappresenta l'aggiunta di un oggetto alla lista-spesa, implementando l'interfaccia AddItemView.
\item \textbf{Utilizzo}: Questa classe viene utilizzata dall'utente ogniqualvolta vuole aggiungere un oggetto alla lista-spesa.
\item \textbf{Attributi}: 
	\begin{itemize}
	\item \textit{private presenter:AddItemViewPresenter}\\
	Il presenter associato alla rimozione di un oggetto della lista, al quale questa classe delega la gestione del comportamento dell'elemento di rimozione degli oggetti.
	\end{itemize}
\item \textbf{Metodi}:
	\begin{itemize}
	\item \textit{public addItem(listId:string,item:ListItem):void}\\
	Il metodo aggiunge un oggetto a una lista-spesa.
			\item{\textbf{Parametri}: \begin{itemize}
			\item \textit{listId:string}\\
			L'id della lista al quale si vuole aggiungere un oggetto.
			\item \textit{item:ListItem}\\
			L'oggetto che si vuole aggiungere alla lista.
			\end{itemize}}
	\item \textit{public renderView():string}\\
	Genera il codice HTML CSS JS necessario per visualizzare la componente grafica della lista-spesa necessaria all'aggiunta di un oggetto da essa.
	\item \textit{AddItemViewImplementation():AddItemViewImplementation}\\
	Il costruttore della classe AddItemViewImplementation.
	\end{itemize}
\item \textbf{Eventi}:
\end{itemize}

\subsubsection{AddItemViewPresenter}
\begin{itemize}
\item \textbf{Descrizione}: Questa classe rappresenta il presenter per gli elementi di rimozione degli oggetti  della lista-spesa.
\item \textbf{Utilizzo}: Il presenter fa da tramite tra l'implementazione dell'elemento di aggiunta e la view, formattando i dati che verranno visualizzati nella view e manipolando gli input dell'utente per eseguire le operazioni predisposte.
\item \textbf{Attributi}: 
	\begin{itemize}
	\item \textit{private view:AddItemView}\\
	La view associata al presenter.
	\item \textit{private inputItemInfoView:InputItemInfoView}\\
	Componente grafica per l'input dei dati relativi a un oggetto della lista-spesa.
	\item \textit{private modifyListUseCase:ModifyListUseCase}\\
	Componente necessaria alla comunicazione tra presenter e database.
	\end{itemize}
\item \textbf{Metodi}:
	\begin{itemize}
	\item \textit{public AddItemViewPresenter(view:AddItemView, inputView:InputItemInfoView, useCase:ModifyListUseCase):AddItemViewPresenter}	
		\item{\textbf{Parametri}: \begin{itemize}
		\item \textit{view:AddItemView}\\
			La view associata al presenter.
		\item \textit{inputView:InputItemInfoView}\\
			Componente grafica per l'input dei dati relativi a un oggetto della lista-spesa.
		\item \textit{useCase:ModifyListUseCase}\\
			Componente necessaria alla comunicazione tra presenter e database.
		\end{itemize}}
	\item \textit{private showInputItemInfoView():void}\\
	Mostra la componente grafica necessaria all'input dei dati per un oggetto della lista-spesa.
	\item \textit{public renderView():string}\\
	Genera il codice HTML CSS JS necessario per visualizzare la componente grafica della lista-spesa necessaria all'aggiunta di un oggetto da essa.
	\end{itemize}
\item \textbf{Eventi}:
\end{itemize}

\subsubsection{ShowPopupUseCase}
\begin{itemize}
\item \textbf{Descrizione}: La classe rappresenta l'interazione con un modale nella chat.
\item \textbf{Utilizzo}: La classe viene utilizzata ogniqualvolta una delle altre classi necessita di mostrare un modale nelle sue interazioni.
\item \textbf{Attributi}: 
	\begin{itemize}
	\item \textit{private chatSource:ChatSource}\\
	La chat sulla quale si vuole intervenire utilizzando questa classe.
	\end{itemize}
\item \textbf{Metodi}:
	\begin{itemize}
	\item \textit{public showPopup(htmlCode:string):void}\\
	Questo metodo mostra un modale.
			\item{\textbf{Parametri}: \begin{itemize}
			\item \textit{htmlCode:string}\\
			La stringa contiene il codice html del contenuto del modale che si vuole mostrare.
			\end{itemize}}
	\item \textit{ShowPopupUseCase(chatSource:ChatSource):ShowPopupUseCase}\\
	Costruttore della classe ShowPopupUseCase.
		\item{\textbf{Parametri}: \begin{itemize}
		\item \textit{chatSource:ChatSource}\\
		Chat necessaria alla costruzione della classe.
		\end{itemize}}
	\end{itemize}
\item \textbf{Eventi}:
\end{itemize}

\subsubsection{ChatSource}
\begin{itemize}
\item \textbf{Descrizione}: 
\item \textbf{Utilizzo}:
\item \textbf{Attributi}: 
\item \textbf{Metodi}:
	\begin{itemize}
	\item \textit{public showPopup(htmlCode:string):void}\\
	
			\item{\textbf{Parametri}: \begin{itemize}
			\item \textit{htmlCode:string}\\

			\end{itemize}}
	\end{itemize}
\item \textbf{Eventi}:
\end{itemize}

\subsubsection{InputItemInfoView}
\begin{itemize}
\item \textbf{Descrizione}: 
\item \textbf{Utilizzo}:
\item \textbf{Attributi}:
\item \textbf{Metodi}:
	\begin{itemize}
	\item \textit{public emitOnSavedItemEvent(item:ListItem):void}\\
	Metodo utilizzato per emettere l'evento OnSavedItemEvent.
			\item{\textbf{Parametri}: \begin{itemize}
			\item \textit{item:ListItem}\\
			L'oggetto che che è stato salvato.
			\end{itemize}}
	\item \textit{public createViewForItemWithId(itemId:string,listId:string):void}\\
	Il metodo crea una componente grafica per la modifca di un oggetto caricando i dati relativi a quell'oggetto dalla sua lista tramite l'id della lista e dell'oggetto stesso.
			\item{\textbf{Parametri}: \begin{itemize}
			\item \textit{itemId:string}\\
			L'id dell'oggetto.
			\item \textit{listId:string}\\
			L'id della lista.
			\end{itemize}}
	\end{itemize}
\item \textbf{Eventi}:
	\begin{itemize}
	\item \textit{public onSavedItem(item:ListItem):void}\\
	Evento che rappresenta il salvataggio delle modifiche fatte a un particolare oggetto della lista-spesa.
			\item{\textbf{Parametri}: \begin{itemize}
			\item \textit{item:ListItem}\\
			L'oggetto le cui modifiche sono state salvate.
			\end{itemize}}
	\item \textit{public onSaveClicked():void}\\
	Evento che rappresenta il click, da parte dell'utente, sulla componente grafica necessaria al salvataggio delle modifiche effettuate alla lista.
	\end{itemize}
\end{itemize}