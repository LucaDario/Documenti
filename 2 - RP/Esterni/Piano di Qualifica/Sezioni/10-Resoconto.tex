\section{Resoconto}

In questa sezione vengono inserite tutte le misurazioni delle metriche trovate dal gruppo \gruppo.
Il team si impegna a garantire almeno il soddisfacimento del range di accettazione per ogni metrica.


\subsection{Garanzie di qualità del processo}

Queste attività non sono ancora state misurate, in quanto richiedono l'entrata nella fase di \COD. I parametri risultanti dalle verifica delle metriche saranno tenuti sotto controllo costantemente, in modo da evitare durante la verifica finale delle metriche possa capitare che un range non sia rispettato. Sarà, dunque, compito del \Ver\ monitorare i parametri in modo da evitare tale situazione.

\subsection{Metriche per i documenti}

\subsubsection{Indice di Gulpease}

\begin{table}[h]
	\begin{center}
		\begin{tabular}{|c|c|}
			\hline
			\textbf{Documento}	& \textbf{Risultato} \\
			\hline
			Analisi dei Requisiti v2.0.0 &	90\\
			\hline
			Glossario v2.0.0 &	54\\
			\hline
			Norme di Progetto v2.0.0 &	48\\
			\hline
			Piano di Progetto v2.0.0	&	52\\
			\hline
			Piano di Qualifica v2.0.0	&	46\\
			\hline
			Definizione di Prodotto v1.0.0	&	64\\
			\hline
			Verbale 22-02-2017 v1.0.0	&	54\\
			\hline
			Verbale 24-02-2017 v1.0.0	&	53\\
			\hline
			Verbale 26-02-2017 v1.0.0	&	51\\
			\hline
		\end{tabular}
	\end{center}
	\caption{Risultato indice di Gulpease}
\end{table}

\subsection{Livello di stabilità}


\subsection{Astrattezza}

Il valore ottenuto per questa metrica è 
\begin{displaymath}
	\frac{15}{44}
\end{displaymath}

\subsection{Distanza dalla sequenza principali}

Tutte le altre metriche riguardano il software verranno analizzate a partire dalla fase di \COD.
