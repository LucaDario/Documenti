\newpage
\section{Analisi dei rischi}

In questa sezione del documento vengono elencati i possibili rischi che potrebbero colpire il gruppo \gruppo\ nella realizzazione del prodotto \progetto.

\subsection{Classificazione dei rischi}

\begin{enumerate}
		
\item \textbf{Identificazione fattori di Rischio}. \\ 
	  All'interno di questa fase vengono identificati tutti i possibili rischi e ne viene studiata la natura. Tali rischi possono essere di quattro tipi:
	\begin{itemize}
	\item \textbf{Rischi di Gruppo}: riguardanti i membri del gruppo; 
	\item \textbf{Rischi Organizzativi}: riguardanti l'organizzazione del gruppo;
	\item \textbf{Rischi Tecnologici}: riguardanti le tecnologie usate dal gruppo;
	\item \textbf{Rischi a livello dei Requisiti}: riguardanti i requisiti del progetto.
	\end{itemize}
				
\item \textbf{Analisi dei Rischi}. \\
	  All'interno di questa fase viene individuato il seguente insieme di informazioni per ogni rischio: 
	\begin{itemize}
	\item \textbf{Descrizione}: descrizione completa del rischio;
	\item \textbf{Probabilità di occorrenza}: probabilità che il rischio si verifichi più volte. Essa può essere: \textbf{bassa}, \textbf{media} o \textbf{alta};
	\item \textbf{Grado di pericolosità}: il grado che indica quanto quel rischio può influire sulla riuscita del progetto. Esso può essere \textbf{basso}, \textbf{medio} o \textbf{alto};
	\item \textbf{Trattamento}: come viene affrontato tale rischio da parte del gruppo.	
	\end{itemize}

\item \textbf{Attualizzazione nel periodo}. \\
	  All'interno di questa fase viene progressivamente descritto se il rischio si è verificato, in che modo il gruppo ha reagito ad esso e le conseguenze che si sono verificate in seguito a tale reazione.

\end{enumerate}

\subsection{Rischi di Gruppo}

\begin{center}

	\begin{tabular}{>{\centering\color{white}}m{4cm} >{\centering\color{white}}m{8cm} >{\centering\arraybackslash}m{0pt}@{}}
	\rowcolor{darkblue} \textbf{Nome} & \textbf{Rischi legati al modello} & \\[1ex]
	\rowcolor{blue} Descrizione & Essendo la prima volta che ogni membro del gruppo utilizza il \termine{Modello Incrementale} potrebbero nascere delle difficoltà nel suo utilizzo. I membri del gruppo, infatti, potrebbero progettare o codificare iterazioni anziché incrementi. & \\[2ex]	
	\rowcolor{lightblue} Probabilità di occorrenza & Bassa. &\\[1ex]
	\rowcolor{blue}  Fattore di rischio & Basso. & \\[1ex]
	\rowcolor{lightblue} Trattamento & Se verificato, si cercherà di individuare l'iterazione ed analizzarla per capire gli eventuali errori. Successivamente la suddetta parte verrà eventualmente progettata e codificata nel modo corretto. & \\[1ex] 
	\rowcolor{blue} Attualizzazione & Questo rischio non si è ancora verificato. & \\[1ex]
	\end{tabular}

\end{center}

\begin{center}

	\begin{tabular}{>{\centering\color{white}}m{4cm} >{\centering\color{white}}m{8cm} >{\centering\arraybackslash}m{0pt}@{}}
	\rowcolor{darkblue} \textbf{Nome} & \textbf{Inesperienza del gruppo} & \\[1ex]
	\rowcolor{blue} Descrizione & Il metodo di lavoro da applicare per questo progetto risulta nuovo a tutti i componenti del gruppo, e sono inoltre richieste capacità di pianificazione e di analisi che il gruppo non possiede e pertanto richiedono del tempo per essere apprese. & \\[2ex]	
	\rowcolor{lightblue} Probabilità di occorrenza & Alta. &\\[1ex]
	\rowcolor{blue}  Fattore di rischio & Alto. & \\[1ex]
	\rowcolor{lightblue} Trattamento & Ogni membro del gruppo dovrà studiare tutto il materiale necessario per essere in grado di far fronte in maniera ottimale a ciò che il progetto richiede. Inoltre, per ottimizzazione sarebbe opportuno che tale studio venga fatto nei periodi precedenti all'attuazione pratica della teoria. & \\[1ex] 
	\rowcolor{blue}  Attualizzazione & Questo rischio si è già verificato ed è stato risolto mediante una opportuna documentazione da parte dei membri del \termine{team} a rischio. & \\[1ex]
	\end{tabular}

\end{center}

\begin{center}

	\begin{tabular}{>{\centering\color{white}}m{4cm} >{\centering\color{white}}m{8cm} >{\centering\arraybackslash}m{0pt}@{}}
	\rowcolor{darkblue} \textbf{Nome} & \textbf{Problemi personali dei componenti del team} & \\[1ex]
	\rowcolor{blue} Descrizione & Ogni componente del gruppo ha impegni personali e necessità proprie. Ciò implica la possibilità che si verifichino ritardi riguardanti la terminazione delle attività previste per il componente in questione. & \\[2ex]	
	\rowcolor{lightblue} Probabilità di occorrenza & Media. &\\[1ex]
	\rowcolor{blue} Fattore di rischio & Medio. & \\[1ex]
	\rowcolor{lightblue} Trattamento & Il \Pm, consultando il calendario e già conoscendo a priori gli impegni fissati dei componenti del gruppo, cercherà di distribuire gli incarichi tenendo presente le necessità dei membri. & \\[1ex] 
	\rowcolor{blue}  Attualizzazione & Questo rischio si è già verificato ed è stato risolto mediante un opportuna ri-pianificazione da parte del \Pm. & \\[1ex]
	\end{tabular}

\end{center}

\begin{center}

	\begin{tabular}{>{\centering\color{white}}m{4cm} >{\centering\color{white}}m{8cm} >{\centering\arraybackslash}m{0pt}@{}}
	\rowcolor{darkblue} \textbf{Nome} & \textbf{Dissidi tra i componenti del gruppo} & \\[1ex]
	\rowcolor{blue} Descrizione & Tutti i componenti non hanno mai lavorato in un gruppo dove il numero di membri è alto. A causa di ciò è possibile che nascano incomprensioni o dissidi interni. & \\[2ex]	
	\rowcolor{lightblue} Probabilità di occorrenza & Bassa. &\\[1ex]
	\rowcolor{blue} Fattore di rischio & Alto. & \\[1ex]
	\rowcolor{lightblue} Trattamento & Si cercherà sempre di ripristinare sintonia all'interno del gruppo per rendere l'ambiente di lavoro il meno stressante possibile. In caso di forti dissidi, sarà il \Pm\ a mediare tra i componenti problematici. & \\[1ex] 
	\rowcolor{blue}  Attualizzazione & Questo rischio non si è ancora verificato. & \\[1ex]
	\end{tabular}

\end{center}

\subsection{Rischi Organizzativi}

\begin{center}

	\begin{tabular}{>{\centering\color{white}}m{4cm} >{\centering\color{white}}m{8cm} >{\centering\arraybackslash}m{0pt}@{}}
	\rowcolor{darkblue} \textbf{Nome} & \textbf{Suddivisione del Lavoro} & \\[1ex]
	\rowcolor{blue} Descrizione & Essendo la prima esperienza per ogni componente all'interno di un gruppo di lavoro così ampio potrebbero nascere dei problemi organizzativi riguardanti la suddivisione del lavoro. & \\[2ex]	
	\rowcolor{lightblue} Probabilità di occorrenza & Bassa. &\\[1ex]
	\rowcolor{blue} Fattore di rischio & Medio. & \\[1ex]
	\rowcolor{lightblue} Trattamento & Per suddividere il lavoro il \Pm dovrà organizzare a sua scelta degli incontri per pianificare in modo adeguato i compiti. Se possibile, i suddetti incontri verranno fatti di persona. & \\[1ex] 
	\rowcolor{blue}  Attualizzazione & Questo rischio non si è ancora verificato. & \\[1ex]
	\end{tabular}

\end{center}

\begin{center}

	\begin{tabular}{>{\centering\color{white}}m{4cm} >{\centering\color{white}}m{8cm} >{\centering\arraybackslash}m{0pt}@{}}
	\rowcolor{darkblue} \textbf{Nome} & \textbf{Pianificazione Errata} & \\[1ex]
	\rowcolor{blue} Descrizione & Durante la pianificazione, a causa di assunzioni sbagliate, può capitare che le previsioni sui tempi di adempimento delle attività e di consegna finale siano errati. & \\[2ex]	
	\rowcolor{lightblue} Probabilità di occorrenza & Media. &\\[1ex]
	\rowcolor{blue} Fattore di rischio & Alto. & \\[1ex]
	\rowcolor{lightblue} Trattamento & Per evitare di sottostimare il tempo necessario a svolgere tutte le attività, è opportuno dimensionare il tempo da dedicare ai vari processi con un \termine{tempo di slack} sufficientemente grande, in modo da evitare queste situazioni. & \\[1ex] 
	\rowcolor{blue}  Attualizzazione & Questo rischio non si è ancora verificato. & \\[1ex]
	\end{tabular}
	
\end{center}

\subsection{Rischi Tecnologici}

\begin{center}

	\begin{tabular}{>{\centering\color{white}}m{4cm} >{\centering\color{white}}m{8cm} >{\centering\arraybackslash}m{0pt}@{}}
	\rowcolor{darkblue} \textbf{Nome} & \textbf{Tecnologie da utilizzare} & \\[1ex]
	\rowcolor{blue} Descrizione & Le tecnologie adottate per sviluppare il prodotto sono quasi del tutto sconosciute ai componenti del gruppo, fattore che potrebbe causare dei ritardi nell'adempimento dei vari compiti. & \\[2ex]	
	\rowcolor{lightblue} Probabilità di occorrenza & Media. &\\[1ex]
	\rowcolor{blue} Fattore di rischio & Alto. & \\[1ex]
	\rowcolor{lightblue} Trattamento & Ciascun componente si impegnerà a documentarsi in maniera autonoma sulle tecnologie adottate. & \\[1ex] 
	\rowcolor{blue}  Attualizzazione & Questo rischio si è verificato. & \\[1ex]
	\end{tabular}
	
\end{center}

\begin{center}

	\begin{tabular}{>{\centering\color{white}}m{4cm} >{\centering\color{white}}m{8cm} >{\centering\arraybackslash}m{0pt}@{}}
	\rowcolor{darkblue} \textbf{Nome} & \textbf{Problematiche e guasti software} & \\[1ex]
	\rowcolor{blue} Descrizione & La strumentazione usata dal gruppo può essere soggetta a malfunzionamenti o rotture che potrebbero rallentare, se non addirittura bloccare, l'avanzamento del progetto. & \\[2ex]	
	\rowcolor{lightblue} Probabilità di occorrenza & Bassa. &\\[1ex]
	\rowcolor{blue} Fattore di rischio & Basso. & \\[1ex]
	\rowcolor{lightblue} Trattamento & Ad ogni incremento o avanzamento di una parte, prima dello spegnimento dell'attrezzatura, ogni membro del gruppo dovrà obbligatoriamente effettuare un \termine{push} sulla \termine{repository} \termine{GitHub}. Se ciò non fosse possibile a causa di mancanza di collegamento ad internet, il suddetto membro dovrà salvare i progressi fatti in un dispositivo di memoria esterno secondario. & \\[1ex] 
	\rowcolor{blue}  Attualizzazione & Questo rischio non si è ancora verificato. & \\[1ex]
	\end{tabular}
	
\end{center}

\subsection{Rischi a livello dei Requisiti}

\begin{center}

	\begin{tabular}{>{\centering\color{white}}m{4cm} >{\centering\color{white}}m{8cm} >{\centering\arraybackslash}m{0pt}@{}}
	\rowcolor{darkblue} \textbf{Nome} & \textbf{Comprensione dei Requisiti} & \\[1ex]
	\rowcolor{blue} Descrizione & È possibile che durante l'\AdR\ alcuni dei requisiti vengano tralasciati, considerati solo in parte o fraintesi. Questo può portare una realizzazione incompleta o scorretta del prodotto software. & \\[2ex]	
	\rowcolor{lightblue} Probabilità di occorrenza & Media. &\\[1ex]
	\rowcolor{blue} Fattore di rischio & Alto. & \\[1ex]
	\rowcolor{lightblue} Trattamento & Ogni qualvolta nascano dei dubbi sui requisiti il \Pm\ dovrà provvedere a contattare i proponenti e, se necessario, fissare degli incontri in modo da chiarire in maniera completa tutti i dubbi. & \\[1ex] 
	\rowcolor{blue}  Attualizzazione & Questo rischio non si è ancora verificato. & \\[1ex]
	\end{tabular}
	
\end{center}

\newpage
