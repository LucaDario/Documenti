\subsubsection{Caso d'uso UC 1.3: Utilizzo di una bolla di tipo checklist}
\label{UC 1.3: Utilizzo di una bolla di tipo checklist}
\begin{figure}[ht]
	\centering
	\includegraphics[scale=0.7]{Usecases/img/{UC1.3}.png}
	\caption{Caso d'uso UC 1.3: Utilizzo di una bolla di tipo checklist}
\end{figure}

\FloatBarrier
\begin{itemize}
\item\textbf{Attori}: Sviluppatore.
\item\textbf{Descrizione}: Lo sviluppatore vuole utilizzare una bolla che consiste in una lista formata da elementi ognuno con un campo testo e un bottone di tipo checkbutton.
\item\textbf{Precondizione}: Lo sviluppatore ha accesso all'\termine{SDK}.
\item\textbf{Postcondizione}: Lo sviluppatore generato codice eseguibile per creare una bolla di tipo checklist.
\item\textbf{Scenario principale}:
	\begin{itemize}
		\item Lo sviluppatore vuole personalizzare l'elenco puntato (UC 1.3.1).
		\item Lo sviluppatore vuole impostare un messaggio di completamento (UC1.3.2).
	\end{itemize}
\end{itemize}
