\subsubsection{Caso d'uso UC 1.1: Utilizzo di un widget testo formattato}
\label{Caso d'uso UC 1.1: Utilizzo di un widget testo formattato}

\begin{figure}[ht]
\centering
	\includegraphics[scale=0.6]{Usecases/img/{UC1.1}.png}
	\caption{Caso d'uso UC 1.1: Aggiunta di un widget testo formattato}
\end{figure}

\FloatBarrier
\begin{itemize}
\item\textbf{Attori}: Sviluppatore.
\item\textbf{Descrizione}: Lo sviluppatore vuole aggiungere a una bolla un widget testo formattato.
\item\textbf{Precondizione}: Lo sviluppatore ha accesso all'\termine{SDK}.
\item\textbf{Postcondizione}: Lo sviluppatore ha creato del codice eseguibile che permette di aggiungere a una bolla un widget testo formattato.
\item\textbf{Scenario principale}:
	\begin{itemize}
		\item Lo sviluppatore vuole scegliere la grandezza del font del testo (UC 1.1.1).
		\item Lo sviluppatore vuole impostare parte del testo in corsivo (UC 1.1.3).
		\item Lo sviluppatore vuole aggiungere un link cliccabile (UC 1.1.4).
		\item Lo sviluppatore vuole impostare parte del testo in grassetto (UC 1.1.7).
		\item Lo sviluppatore vuole impostare un colore a parte del testo (UC 1.1.8).
		\item Lo sviluppatore vuole aggiungere del testo alla bolla (1.1.10)
	\end{itemize}
	
\item\textbf{Scenario alternativo}
	\begin{itemize}
		\item Lo sviluppatore non inserisce la grandezza del font oppure la inserisce in modo non valido(UC 1.1.2).
		\item Lo sviluppatore non inserisce il colore del link cliccabile oppure lo inserisce in modo non valido(UC 1.1.6).
		\item Lo sviluppatore non imposta il colore del testo della bolla oppure lo inserisce in modo non valido(UC 1.1.9).
		\item Lo sviluppatore non inserisce del testo oppure lo inserisce in modo non valido(UC 1.1.11).
		
	\end{itemize}

\end{itemize}
