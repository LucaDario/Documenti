\newpage
\subsection{Requisiti Vincolo}
\normalsize
\begingroup
\renewcommand\arraystretch{2}
\begin{longtable}{|c|>{\centering}m{7cm}|c|}
\hline
\textbf{Id Requisito} & \textbf{Descrizione} & \textbf{Fonti}\\
\hline
\endhead
			R1V1 & Deve essere creato un \termine{SDK} per permettere agli sviluppatori di creare nuove bolle & \termine{Capitolato} \\
			\hline
			R1V2 & Le bolle interattive create tramite \progettoShort\ devono funzionare dentro una istanza del \termine{server} \termine{web chat} \termine{Rocket.chat} & \termine{Capitolato}   \\
			\hline
			R2V3 & Durante lo sviluppo deve essere disponibile un server sul quale vi è installata una istanza di \termine{Rocket.chat}  & Interno \\ 
			\hline
			R2V4 & Le bolle create devono ricadere in una delle seguenti tre tipologie:  \textit{Rich media bubble}, \textit{Self-updating bubble} o \textit{Editing bubble} & Capitolato\\ 
			\hline
			R1V5 & \progettoShort\ deve essere in grado di fornire bolle predefinite già pronte ad essere utilizzate da parte dell'utente finale  & \termine{Capitolato} \\ 
			\hline
			R1V6 & \progettoShort\ deve provvedere delle \termine{API} per permettere agli sviluppatori futuri di creare delle nuove bolle secondo i loro requisiti  & \termine{Capitolato} \\ 
			\hline
			R1V7 & Dovrà essere creata una demo significativa che mostri un interessante uso delle bolle all'interno di \termine{Rocket.chat} & \termine{Capitolato} \\ 
			\hline
			R1V8 & L'\termine{SDK} e la demo dovranno essere sviluppati in \termine{javascript} 6th edition usando il "\termine{promise centric approach}" ed evitando il più possibile le \termine{callback}
			 & \termine{Capitolato} \\ 
			\hline
			R1V9 & Le \termine{callback} qualora venissero usate devono essere giustificate in maniera corretta
			 & \termine{Capitolato} \\ 
			 \hline
			 R1V10 & Per scrivere la demo e l'\termine{SDK} bisogna seguire il più possibile le \termine{12 Factors app guidelines}
			 & \termine{Capitolato} \\ 
			\hline
			R2V11 & E' consigliato l'uso di un \termine{framework frontend}
			 & \termine{Capitolato} \\ 
			\hline
			R2V12 & E' consigliato l'uso di \termine{SCSS}
			 & \termine{Capitolato} \\ 
			\hline
			R1V12 & La demo dovrà essere usufruibile tramite \termine{Heroku} sotto forma di pacchetto indipendente
			 & \termine{Capitolato} \\ 
			\hline
			R1V13 & Il codice sorgente di \progettoShort\ dovrà essere caricato e versionato su \termine{GitHub} o  \termine{Bitbucket}
			 & \termine{Capitolato} \\ 
			\hline
\caption[Requisiti di Vincolo]{Requisiti di Vincolo}
\label{tabella: Requisiti di Vincolo}
\end{longtable}
\endgroup
\clearpage