\newpage
\subsubsection{Verifiche di manutenibilità}
In questa sezione verranno descritte come e quando effettuare le misurazioni incentrate sull'efficienza del prodotto.

\paragraph{Capacità di analisi di failure}
\begin{itemize}
\item \textbf{Chi:} Jenkins e il \Prog
\item \textbf{Quando:} Dopo ogni push o merge sul ramo della \termine{repository} contenente il codice della versione di produzione.
\item \textbf{Come:} Dopo il push o il merge Jenkis eseguira i test e mostrera al \Prog quelli falliti il quale dovrà individuare le causde delle failure.
\end{itemize}

\paragraph{Impatto delle modifiche}
\begin{itemize}
\item \textbf{Chi:} \termine{Jenkins}.
\item \textbf{Quando:} Dopo ogni push sulla \termine{repository}.
\item \textbf{Come:} Questo test viene effettuato automaticamente da \termine{Jenkins}, se il test non viene superato il codice non verrà integrato nel ramo di produzione.
\end{itemize}
