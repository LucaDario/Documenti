\subsubsection{ListData}
\begin{itemize}
\item \textbf{Descrizione}: Questa classe rappresenta la lista-spesa.
\item \textbf{Utilizzo}: Viene utilizzata da tutte le classi che hanno a che fare con la lista, ovvero che creano, modificano e rimuovono liste, o che aggiungono o rimuovono utenti e oggetti al suo interno.
\item \textbf{Attributi}: 
	\begin{itemize}
	\item \textit{private id:string}\\
	Questa stringa rappresenta l'identificativo della lista.
	\item \textit{private imagePath:string}\\
	Questa stringa rappresenta il percorso dell'immagine della lista.
	\item \textit{private name:string}\\
	Questa stringa rappresenta il nome della lista.
	\item \textit{private items:List<ListItem>}\\
	La lista degli oggetti contenuti dalla lista.
	\item \textit{private creatorId:string}\\
	Questa stringa rappresenta l'identificativo del creatore della lista.
	\item \textit{private users:List<string>}\\
	La lista degli identificativi di tutti i partecipanti alla lista-spesa.
	\end{itemize}
\item \textbf{Metodi}:
	\begin{itemize}
	\item \textit{public ListData():ListData}\\
	Il costruttore della classe ListData.
	\item \textit{public setId(id:string):void}\\
	Imposta l'id della lista.
				\item{\textbf{Parametri}: \begin{itemize}
				\item \textit{id:string}\\
				L'id che verrà impostato per la lista.
			\end{itemize}}
	\item \textit{public setImage(path:string):void}\\
	Imposta l'immagine della lista.
				\item{\textbf{Parametri}: \begin{itemize}
				\item \textit{path:string}\\
				Il percorso dell'immagine che verrà impostata per la lista.
			\end{itemize}}
	\item \textit{public setName(name:string):void}\\
	Imposta il nome della lista.
				\item{\textbf{Parametri}: \begin{itemize}
				\item \textit{name:string}\\
				Il nome che verrà impostato per la lista.
			\end{itemize}}
	\item \textit{public addItem(item:ListItem):void}\\
	Aggiunge un oggetto alla lista.
				\item{\textbf{Parametri}: \begin{itemize}
				\item \textit{item:ListItem}\\
				L'oggetto che verrò aggiunto alla lista.
			\end{itemize}}
	\item \textit{public removeItem(id:string):void}\\
	Rimuove un oggetto dalla lista.
				\item{\textbf{Parametri}: \begin{itemize}
				\item \textit{id:string}\\
				L'identificativo dell'oggetto che si desidera rimuovere dalla lista.
			\end{itemize}}
	\item \textit{public getItem(id:string):void}\\
	Ritorna un oggetto della lista.
				\item{\textbf{Parametri}: \begin{itemize}
				\item \textit{id:string}\\
				L'identificativo dell'oggetto che si desidera.
			\end{itemize}}
	\item \textit{public getId():string}\\
	Ritorna l'identificativo della lista.
	\item \textit{public getImagePath():string}\\
	Ritorna il percorso dell'immagine della lista.
	\item \textit{public getName():string}\\
	Ritorna il nome della lista.
	\item \textit{public getItems():List<ListItem>}\\
	Ritorna una lista contenente tutti gli oggetti presenti all'interno della lista.
	\item \textit{public setCreator(id:string):void}\\
	Imposta il creatore della lista.
				\item{\textbf{Parametri}: \begin{itemize}
				\item \textit{id:string}\\
				L'identificativo dell'utente che verrà impostato come creatore della lista.
			\end{itemize}}
	\item \textit{public getCreatorId():string}\\
	Ritorna l'identificativo del creatore della lista.
	\item \textit{public addUser(id:string):void}\\
	Aggiunge un utente ai partecipanti della lista.
				\item{\textbf{Parametri}: \begin{itemize}
				\item \textit{id:string}\\
				L'identificativo dell'utente che si desidera aggiungere alla lista.
			\end{itemize}}
	\item \textit{public hasUserPermissions(id:string):boolean}\\
	Ritorna true se l'utente in questione ha permessi di modifica sulla lista, false altrimenti.
				\item{\textbf{Parametri}: \begin{itemize}
				\item \textit{id:string}\\
				L'identificativo dell'utente di cui si vuole sapere i permessi.
			\end{itemize}}
	\item \textit{public getUsers():List<strings>}\\
	Ritorna una lista contenente tutti i partecipanti alla lista-spesa.
	\end{itemize}
\item \textbf{Eventi}:
\end{itemize}

\subsubsection{ListItem}
\begin{itemize}
\item \textbf{Descrizione}:
\item \textbf{Utilizzo}:
\item \textbf{Attributi}: 
	\begin{itemize}
	\item \textit{private id:string}\\
	
	\item \textit{private description:string}\\
	
	\item \textit{private imagePath:string}\\
	
	\item \textit{private measurementUnit:string}\\
	
	\item \textit{private name:string}\\
	
	\item \textit{private notes:List<string>}\\
	
	\item \textit{private quantity:int}\\
	
	\end{itemize}
\item \textbf{Metodi}:
	\begin{itemize}
	\item \textit{public ListItem():ListItem}\\
	Il costruttore della classe ListItem.
	\item \textit{public setId(id:string):void}\\
	
				\item{\textbf{Parametri}: \begin{itemize}
				\item \textit{id:string}\\

			\end{itemize}}
	\item \textit{public setDescription(description:string):void}\\
	
				\item{\textbf{Parametri}: \begin{itemize}
				\item \textit{description:string}\\

			\end{itemize}}
	\item \textit{public setImage(path:string):void}\\
	
				\item{\textbf{Parametri}: \begin{itemize}
				\item \textit{path:string}\\

			\end{itemize}}
	\item \textit{public setMeasurementsUnit(mu:string):void}\\
	
				\item{\textbf{Parametri}: \begin{itemize}
				\item \textit{mu:string}\\

			\end{itemize}}
	\item \textit{public setName(name:string):void}\\
	
				\item{\textbf{Parametri}: \begin{itemize}
				\item \textit{name:string}\\

			\end{itemize}}

	\item \textit{public addNote(note:string):void}\\
	
				\item{\textbf{Parametri}: \begin{itemize}
				\item \textit{note:string}\\

			\end{itemize}}
	\item \textit{public removeNote(position:int):void}\\
	
				\item{\textbf{Parametri}: \begin{itemize}
				\item \textit{position:int}\\

			\end{itemize}}
	\item \textit{public getNote(position:int):string}\\
	
				\item{\textbf{Parametri}: \begin{itemize}
				\item \textit{position:int}\\

			\end{itemize}}
	\item \textit{public getNotes():List<string>}\\
	
	\item \textit{public setQuantity(qty:int):void}\\
	
				\item{\textbf{Parametri}: \begin{itemize}
				\item \textit{qty:int}\\

			\end{itemize}}
	\item \textit{public getId():string}\\
	
	\item \textit{public getImagePath():string}\\
	
	\item \textit{public getName():string}\\
	
	\item \textit{public getDescription():List<ListItem>}\\
	
	\item \textit{public getMeasurementUnit():string}\\

	\item \textit{public getQuantity():int}\\
\end{itemize}
\item \textbf{Eventi}:
\end{itemize}