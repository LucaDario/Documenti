\newpage
\subsubsection{Verifiche del software}
In questa sezione verranno descritte le come e quando effettuare le metriche incentrate sul software.

\paragraph{Numero di livelli di annidamento per metodo}
\begin{itemize}
\item \textbf{Chi:} \termine{SonarQube}.
\item \textbf{Quando:} Dopo ogni push sulla \termine{repository}.
\item \textbf{Come:} Questo test viene effettuato automaticamente da \termine{SonarQube}, se il test non viene superato il codice non verrà integrato nel ramo di produzione.
\end{itemize}

\paragraph{Complessità ciclomatica}
\begin{itemize}
\item \textbf{Chi:} \termine{SonarQube}.
\item \textbf{Quando:} Dopo ogni push sulla \termine{repository}.
\item \textbf{Come:} Questo test viene effettuato automaticamente da \termine{SonarQube}, se il test non viene superato il codice non verrà integrato nel ramo di produzione.
\end{itemize}

\paragraph{Numero di attributi per classe}
\begin{itemize}
\item \textbf{Chi:} \termine{SonarQube}.
\item \textbf{Quando:} Dopo ogni push sulla \termine{repository}.
\item \textbf{Come:} Questo test viene effettuato automaticamente da \termine{SonarQube}, se il test non viene superato il codice non verrà integrato nel ramo di produzione.
\end{itemize}

\paragraph{Linee di commento per linee di codice}
\begin{itemize}
\item \textbf{Chi:} Il \Progr.
\item \textbf{Quando:} Dopo la scrittura di ogni metodo o funzione.
\item \textbf{Come:} Il \Progr\ deve verificare che il test soddisfi almeno il livello di accettazione descritto per poter considerare la funzione o il metodo accettabile. Nel caso in cui non sia raggiunto il minimo livello, il \Prog\ dovrà provvedere a trovare frasi più semplici e coese rispetto a quelle in precedenza. 
\end{itemize}

\paragraph{Copertura dei test}
\begin{itemize}
\item \textbf{Chi:} \termine{SonarQube}.
\item \textbf{Quando:} Dopo ogni push sulla \termine{repository}.
\item \textbf{Come:} Questo test viene effettuato automaticamente da \termine{SonarQube}, se il test non viene superato il codice non verrà integrato nel ramo di produzione.
\end{itemize}

\paragraph{Numero di parametri per metodo}
\begin{itemize}
\item \textbf{Chi:} Il \Prog.
\item \textbf{Quando:} Prima della scrittura del metodo.
\item \textbf{Come:} Il \Prog deve controllare prima di progettare il metodo che il numero di parametri soddisfi almeno il livello di accettazione descritto per questa metrica.
\end{itemize}

\paragraph{Livello di stabilità}
\begin{itemize}
\item \textbf{Chi:} il \Ver.
\item \textbf{Quando:} Durante la fase di verifica che precede quella di approvazione.
\item \textbf{Come:} Il \Ver\ deve applicare la formula descritta nelle \NdP.
\end{itemize}