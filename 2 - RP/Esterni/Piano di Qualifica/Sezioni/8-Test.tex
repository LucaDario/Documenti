\section{Test}
Il \termine{team} \gruppo\ al fine di creare del software che sia di qualità ha strutturato dei test atti a verificare
che il software prodotto rispecchi le funzionalità a fronte di risultati attesi.
Tutte le attività di testing prodotte devono poter essere ripetibili e devono essere deterministiche, al fine di poter
fornire delle informazioni utili a intraprendere azioni di correzione nel caso in cui i risultati ottenuti siano diversi da quelli attesi. \\
Per avere un tracciamento dei test prodotti e dei risultati ottenuti si è scelto di classificare il tutto producendo dei
\termine{log} che siano di facile consultazione e che possano fornire una precisa indicazione di quelli che sono stati
gli output di queste attività di verifica, eventuali errori o eventuali risultati che siano non coerenti con quanto in precedenza fissato.

\subsection{Tipi di test}
Sono stati individuati quattro livelli di testing:
\begin{itemize}
\item \textbf{Test di unità [TU]:} con questa tipologia di test si cerca di verificare la più piccola parte di lavoro prodotta da un programmatore. Questo si traduce tendenzialmente a verificare i metodi e le funzioni scritte;
\item \textbf{Test di integrazione [TI]:} con questa tipologia di test si cerca di verificare le componenti di sistema. Più precisamente, l'obiettivo è quello di testare il funzionamento dei vari package prodotti, sia singolarmente che nel loro insieme;
\item \textbf{Test di sistema [TS]:} con questa tipologia di test si cerca di verificare che il comportamento e il funzionamento dell'architettura siano corretti;
\item \textbf{Test di validazione [TV]:} con questa tipologia di test si vuole verificare che il lavoro prodotto soddisfi quanto richiesto dal proponente.
\end{itemize}

% INSERIMENTO DEI TEST
\subsubsection{Test di Unità}
Con questa tipologia di test si vuole verificare il corretto funzionamento delle unità individuate durante la definizione dell'architettura.
I test di unità saranno descritti nel modo seguente:
\begin{center}
\textbf{TU}[IdTest]
\end{center}
dove:
\begin{itemize}
\item IdTest rappresenta il codice identificativo crescente dell'unità considerata.
\end{itemize}

\begin{center}
	\begin{longtable}{|c|>{\centering}m{10cm}|c|c|}\hline
		Id & Descrizione & Stato & Esito\\ \hline
		TU1 & Verificare che in un widget di tipo testo venga impostata in modo corretto la grandezza del font del testo & \textcolor{Green}{Implementato} & \textcolor{Green}{Superato} \\ \hline
		TU2 & Verificare che all'instanziazione di un widget di tipo testo la grandezza del testo sia di default a 1em & \textcolor{Green}{Implementato} & \textcolor{Green}{Superato} \\ \hline
		TU3 & Verificare che all'interno di un widget di tipo testo venga impostata in corsivo la parte di testo richiesta & \textcolor{Green}{Implementato} & \textcolor{Green}{Superato} \\ \hline
		TU4 & Verificare che all'interno di un widget di tipo testo venga inserito un link cliccabile nel modo corretto & \textcolor{Green}{Implementato} & \textcolor{Green}{Superato} \\ \hline
		TU5 & Verificare che all'interno di un widget di tipo testo i link cliccabili abbiano il colore corretto & \textcolor{Green}{Implementato} & \textcolor{Green}{Superato} \\ \hline
		TU6 & Verificare che all'interno di un widget di tipo testo venga impostata in grassetto la parte di testo richiesta & \textcolor{Green}{Implementato} & \textcolor{Green}{Superato} \\ \hline
		TU7 & Verificare che all'interno di un widget di tipo testo venga impostato il colore del testo in modo corretto & \textcolor{Green}{Implementato} & \textcolor{Green}{Superato} \\ \hline
		TU8 & Verificare che all'instanziazione di un widget di tipo testo il colore del testo sia di default a nero & \textcolor{Green}{Implementato} & \textcolor{Green}{Superato} \\ \hline
		TU9 & Verificare che in un widget di tipo immagine l'immagine venga agiunta in modo corretto & \textcolor{Green}{Implementato} & \textcolor{Green}{Superato} \\ \hline
		TU10 & Verificare che in un widget di tipo immagine venga mostrato il messaggio di errore appropriato nel caso si inserisca un immagina non valida & \textcolor{Green}{Implementato} & \textcolor{Green}{Superato} \\ \hline
		TU11 & Verificare che un widget di tipo immagine venga impostata la dimensione del immagine da visualizzare in modo corretto & \textcolor{Green}{Implementato} & \textcolor{Green}{Superato} \\ \hline
		TU12 & Verificare che all'instanziazione di un widget di tipo immagine venga impostata la larghezza e l'altezza di default & \textcolor{Green}{Implementato} & \textcolor{Green}{Superato} \\ \hline
		TU13 & Verificare un widget di tipo immagine venga visualizzato con le dimensioni corrette & \textcolor{Green}{Implementato} & \textcolor{Green}{Superato} \\ \hline
		TU14 & Verificare che in un widget di tipo bottone venga impostato il testo in modo corretto & \textcolor{Green}{Implementato} & \textcolor{Green}{Superato} \\ \hline
		TU15 & Verificare che venga mostrarto in modo corretto il messaggio di errore se viene inserita una sequenza di caratteri non valida in un widget di tipo bottone & \textcolor{Green}{Implementato} & \textcolor{Green}{Superato} \\ \hline
		TU16 & Verificare che un widget di tipo bottone venga impostata la dimensione in modo corretto & \textcolor{Green}{Implementato} & \textcolor{Green}{Superato} \\ \hline
		TU17 & Verificare che all'instanziazione di un widget di tipo bottone venga impostata la larghezza e l'altezza di default & \textcolor{Green}{Implementato} & \textcolor{Green}{Superato} \\ \hline
		TU18 & Verificare che in un widget di tipo bottone venga impostato il colore di sfondo in modo corretto & \textcolor{Green}{Implementato} & \textcolor{Green}{Superato} \\ \hline
		TU19 & Verificare che all'instanziazione di un widget di tipo bottone venga impostato il colore di sfondo di default & \textcolor{Green}{Implementato} & \textcolor{Green}{Superato} \\ \hline
		TU20 & Verificare che in un widget di tipo button vengano eseguiti gli eventi associati alla pressione & \textcolor{Green}{Implementato} & \textcolor{Green}{Superato} \\ \hline
		TU21 & Verificare che in un widget di tipo button vengano eseguiti gli eventi associati alla pressione pressione prolungata & \textcolor{Green}{Implementato} & \textcolor{Green}{Superato} \\ \hline
		TU22 & Verificare che in un widget di tipo button venga impostato in modo corretto la soglia per determinare che una pressione da parte del utente è di tipo: "pressione prolungata" & \textcolor{Green}{Implementato} & \textcolor{Green}{Superato} \\ \hline
		TU23 & Verificare che un un widget di tipo checklistitem venga cambiato in modo corretto la stato da "cliccato" a "non cliccato" e viceversa & \textcolor{Green}{Implementato} & \textcolor{Green}{Superato} \\ \hline
		TU24 & Verificare che all'instanziazione di un widget di tipo checklistitem venga impostato lo stato a "non cliccato" & \textcolor{Green}{Implementato} & \textcolor{Green}{Superato} \\ \hline
		TU25 & Verificare che un widget di tipo checklistitem venga visualizzato con lo stile di spunta e colore corretto & \textcolor{Green}{Implementato} & \textcolor{Green}{Superato} \\ \hline
		TU26 & Verificare che in un widget di tipo checklistitem vengano associati gli eventi di pressione nel modo corretto & \textcolor{Green}{Implementato} & \textcolor{Green}{Superato} \\ \hline
		TU27 & Verificare che in un widget di tipo checklistitem vengano associati gli eventi di pressione prolungata nel modo corretto & \textcolor{Green}{Implementato} & \textcolor{Green}{Superato} \\ \hline
		TU28 & Verificare che in un widget di tipo checklistitem vengano eseguiti gli eventi associati alla pressione & \textcolor{Green}{Implementato} & \textcolor{Green}{Superato} \\ \hline
		TU29 & Verificare che in un widget di tipo checklistitem vengano eseguiti gli eventi associati alla pressione pressione prolungata & \textcolor{Green}{Implementato} & \textcolor{Green}{Superato} \\ \hline
		TU30 & Verificare che in un widget di tipo checklistitem venga impostato in modo corretto la soglia per determinare che una pressione da parte del utente è di tipo: "pressione prolungata" & \textcolor{Green}{Implementato} & \textcolor{Green}{Superato} \\ \hline
		TU31 & Verificare che in un widget di tipo list venga impostato correttamente il cambiamento dell'indicatore a 'pallino' & \textcolor{Green}{Implementato} & \textcolor{Green}{Superato} \\ \hline
		TU32 & Verificare che in un widget di tipo list venga impostato correttamente il cambiamento dell'indicatore a 'trattino' & \textcolor{Green}{Implementato} & \textcolor{Green}{Superato} \\ \hline
		TU33 & Verificare che in un widget di tipo list venga impostato correttamente il cambiamento dell'indicatore ad elenco numerato & \textcolor{Green}{Implementato} & \textcolor{Green}{Superato} \\ \hline
		TU34 & Verificare che in un widget di tipo list che vengano aggiunti correttamente tutti gli elementi & \textcolor{Green}{Implementato} & \textcolor{Green}{Superato} \\ \hline
		TU35 & Verificare che in un layout verticale sia possibile aggiungere un elemento & \textcolor{Green}{Implementato} & \textcolor{Green}{Superato} \\ \hline
		TU36 & Verificare che in un layout verticale non sia possibile aggiungere se stesso & \textcolor{Green}{Implementato} & \textcolor{Green}{Superato} \\ \hline
		TU37 & Verificare che in un layout orizzontale sia possibile aggiungere un elemento & \textcolor{Green}{Implementato} & \textcolor{Green}{Superato} \\ \hline
		TU38 & Verificare che in un layout orizzontale non sia possibile aggiungere se stesso & \textcolor{Green}{Implementato} & \textcolor{Green}{Superato} \\ \hline
		TU39 & Verificare che il nome della lista della spesa venga impostato nel modo corretto & \textcolor{Green}{Implementato} & \textcolor{Green}{Superato} \\ \hline
		TU40 & Verificare che all'instanziazione di una lista della spesa venga impostato il valore di default per il nome della lista & \textcolor{Green}{Implementato} & \textcolor{Green}{Superato} \\ \hline
		TU41 & Verificare che l'immagine della lista della spesa venga impostata in modo corretto & \textcolor{Orange}{Non implementato} & \textcolor{Orange}{Non superato} \\ \hline
		TU42 & Verificare che nel caso di un errore durante l'aggiunta di un prodotto alla lista della spesa venga mostrato un mesaggio di errore & \textcolor{Green}{Implementato} & \textcolor{Green}{Superato} \\ \hline
		TU43 & Verificare che l'immagine del prodotto venga aggiunta in modo corretto & \textcolor{Green}{Implementato} & \textcolor{Green}{Superato} \\ \hline
		TU44 & Verificare che la descrizione del prodotto venga aggiunta in modo corretto & \textcolor{Green}{Implementato} & \textcolor{Green}{Superato} \\ \hline
		TU45 & Verificare che le note del prodotto vengano aggiunte in modo corretto & \textcolor{Orange}{Non implementato} & \textcolor{Orange}{Non superato} \\ \hline
		TU46 & Verificare che la quantità del prodotto venga impostata in modo corretto quando inserita & \textcolor{Green}{Implementato} & \textcolor{Green}{Superato} \\ \hline
		TU47 & Verificare che alla creazione di un nuovo prodotto venga impostata la quantità di default & \textcolor{Green}{Implementato} & \textcolor{Green}{Superato} \\ \hline
		TU48 & Verificare che l'unità di misura relativa alla quantità del prodotto venga impostata in modo corretto & \textcolor{Green}{Implementato} & \textcolor{Green}{Superato} \\ \hline
		TU49 & Verificare che alla creazione di un nuovo prodotto venga impostata l'unità di misura di default & \textcolor{Green}{Implementato} & \textcolor{Green}{Superato} \\ \hline
		TU50 & Verificare che DatabaseSource ritorni la lista che corrisponde all'id inserito & \textcolor{Green}{Implementato} & \textcolor{Green}{Superato} \\ \hline
		TU51 & Verificare che DatabaseSource ritorni l'oggetto che corrisponde all'id inserito & \textcolor{Green}{Implementato} & \textcolor{Green}{Superato} \\ \hline
		TU52 & Verificare che DatabaseSource cancelli la lista che corrisponde all'id inserito & \textcolor{Green}{Implementato} & \textcolor{Green}{Superato} \\ \hline
		TU53 & Verificare che DatabaseSource salvi correttamente la lista nel database & \textcolor{Green}{Implementato} & \textcolor{Green}{Superato} \\ \hline
		TU54 & Verificare che ModifyListUseCase aggiunga correttamente un item alla lista nel database & \textcolor{Green}{Implementato} & \textcolor{Green}{Superato} \\ \hline
		TU55 & Verificare che ModifyListUseCase rimuova correttamente un item alla lista nel database & \textcolor{Green}{Implementato} & \textcolor{Green}{Superato} \\ \hline
		TU56 & Verificare che ModifyListUseCase aggiorni correttamente un item alla lista nel database & \textcolor{Green}{Implementato} & \textcolor{Green}{Superato} \\ \hline
		TU57 & Verificare che DeleteListViewPresenter funzioni correttamente senza lanciare eccezioni & \textcolor{Green}{Implementato} & \textcolor{Green}{Superato} \\ \hline
		TU58 & Verificare che InputListInfoViewPresenter crei correttamente un elemento della lista & \textcolor{Green}{Implementato} & \textcolor{Green}{Superato} \\ \hline
		TU59 & Verificare che ShareWithGroupViewPresenter funzioni correttamente senza lanciare eccezioni & \textcolor{Green}{Implementato} & \textcolor{Green}{Superato} \\ \hline
		TU60 & Verificare che ShareWithContactViewPresenter funzioni correttamente senza lanciare eccezioni & \textcolor{Green}{Implementato} & \textcolor{Green}{Superato} \\ \hline
	\end{longtable}
\end{center}

\subsubsection{Test di Integrazione}
Con questa tipologia di test si vuole determinare il corretto funzionamento delle componenti progettate durante la
definizione dell'architettura ad alto livello.

I test di integrazione saranno descritti nel modo seguente:
\begin{center}
\textbf{TI}[\textit{IdComponente}]
\end{center}
dove:
\begin{itemize}
\item
\textbf{IdComponente} rappresenta il codice identificativo crescente del componente considerato.
\end{itemize}
È stato scelto di utilizzare un approccio top-down nel determinare i test di integrazione.

\begin{center}
	\begin{longtable}{|c|>{\centering}m{10cm}|c|c|}\hline
		Id & Descrizione & Stato & Esito\\ \hline
		TI1 & Si verifica che il metodo ButtonWidget::setText passi correttamente il testo al presenter il quale eventualmente lo formatterà e modificherà correttamente l'HTML e CSS & \textcolor{Green}{Implementato} & \textcolor{Green}{Superato} \\ \hline
		TI2 & Si verifica che il metodo ButtonWidget::setHeight passi correttamente il valore dell'altezza al presenter il quale modificherà correttamente l'HTML e CSS & \textcolor{Green}{Implementato} & \textcolor{Green}{Superato} \\ \hline
		TI3 & Si verifica che il metodo ButtonWidget::setWidth passi correttamente il valore della lareghezza al presenter modificherà correttamente l'HTML e CSS & \textcolor{Green}{Implementato} & \textcolor{Green}{Superato} \\ \hline
		TI4 & Si verifica che il metodo ButtonWidget::setBackgroundColor passi correttamente il colore dello sfondo del bottone al presenter il quale modificherà correttamente l'HTML e CSS & \textcolor{Green}{Implementato} & \textcolor{Green}{Superato} \\ \hline
		TI5 & Si verifica che il metodo ButtonWidget::setOnClickAction passi correttamente al presenter la funzione che verrà eseguita quanto viene premuto il pulsante & \textcolor{Green}{Implementato} & \textcolor{Green}{Superato} \\ \hline
		TI6 & Si verifica che il metodo ButtonWidget::setOnLongClickAction passi correttamente al presenter la funzione che verrà eseguita quanto viene mantenuto premuto il pulsante & \textcolor{Green}{Implementato} & \textcolor{Green}{Superato} \\ \hline
		TI7 & Si verifica che il metodo TextWidget::setText passi correttamente al presenter il testo che verrà poi mostrato a video & \textcolor{Green}{Implementato} & \textcolor{Green}{Superato} \\ \hline
		TI8 & Si verifica che il metodo TextWidget::setTextColor passi correttamente il colore del testo al presenter il quale modificherà correttamente l'HTML e CSS, presentandolo con il colore corrispondente & \textcolor{Green}{Implementato} & \textcolor{Green}{Superato} \\ \hline
		TI9 & Si verifica che il metodo TextWidget::setFormatText passi il valore booleano (che rappresenta la scelta di formattare il testo) al presenter. Esso dovrà, poi, presentare il testo formattato seguendo la sintassi markdown oppure lasciarlo invariato in funzione al valore booleano & \textcolor{Green}{Implementato} & \textcolor{Green}{Superato} \\ \hline
		TI10 & Si verifica che il metodo TextWidget::setUrlHighlightColor passi il valore del colore al presenter il quale dovrà presentare i link evidenziati oppure non evidenziati in funzione al colore scelto & \textcolor{Green}{Implementato} & \textcolor{Green}{Superato} \\ \hline
		TI11 & Si verifica che il metodo TextWidget::setTextSize passi la dimensione del testo del presenter il quale dovrà occuparsi di presentare il testo della dimensione impostata & \textcolor{Green}{Implementato} & \textcolor{Green}{Superato} \\ \hline
		TI12 & Si verifica che il metodo ChecklistItemWidget::setUseSelectionMark imposti la scelta di utilizzare un carattere o un colore come spunta passando correttamente per il \termine{Presenter} & \textcolor{Green}{Implementato} & \textcolor{Green}{Superato} \\ \hline
		TI13 & Si verifica che il metodo ChecklistItemWidget::setSelectionColor passi al presenter il colore dello spunta il quale dovrà presentare la lista con lo stile corretto & \textcolor{Green}{Implementato} & \textcolor{Green}{Superato} \\ \hline
		TI14 & Si verifica che il metodo ChecklistItemWidget::setSelectionCharacter passi al presenter un carattere per la spunta, e il presenter dovrà presentare la lista con lo stile corretto & \textcolor{Green}{Implementato} & \textcolor{Green}{Superato} \\ \hline
		TI15 & Si verifica che il metodo ListWidget::addItem passi al presenter il testo del elemento il quale dovrà presentare l'HTML con il nuovo elemento & \textcolor{Green}{Implementato} & \textcolor{Green}{Superato} \\ \hline
		TI16 & Si verifica che la view passi al presenter il carattere con il quale mostrare gli indicatori della lista. Esso provvederà poi ad impostarli correttamente nell'HTML & \textcolor{Green}{Implementato} & \textcolor{Green}{Superato} \\ \hline
		TI17 & Si verifica che il metodo ImageWidget::setWidth passi correttamente al presenter il valore dell'altezza il quale aggiornerà la view & \textcolor{Green}{Implementato} & \textcolor{Green}{Superato} \\ \hline
		TI18 & Si verifica che il metodo ImageWidget::setHeight passi correttamente al presenter il valore della larghezza il quale aggiornerà la view & \textcolor{Green}{Implementato} & \textcolor{Green}{Superato} \\ \hline
		TI19 & Si verifica che il metodo ImageWidget::setImage passi correttamente al presenter il path del'immagine che dovrà essere presentata nella view & \textcolor{Green}{Implementato} & \textcolor{Green}{Superato} \\ \hline
		TI20 & Si verifica che l'aggiunta di un VerticalLayoutView, creato con degli elementi al suo interno, non causi errori nell'aggiunta di questo alla bolla e che, gli elementi, nel layout appena aggiunto, vengano mostrati uno sotto l'altro & \textcolor{Green}{Implementato} & \textcolor{Green}{Superato} \\ \hline
		TI21 & Si verifica che l'aggiunta di un HorizontalLayoutView, creato con degli elementi al suo interno, non causi errori nell'aggiunta di questi alla bolla e che gli elementi, nel layout appena aggiunto, vengano mostrati uno vicino all'altro & \textcolor{Green}{Implementato} & \textcolor{Green}{Superato} \\ \hline
		TI22 & Si verifica che il ModifyListUseCase esegua le operazioni di aggiunta, rimozione e aggiornamento dei dati trammite DatabaseSource in modo corretto & \textcolor{Green}{Implementato} & \textcolor{Green}{Superato} \\ \hline
		TI23 & Si verifica che ShareWithGroupViewPresenter interagisca in modo corretto con ChatSource & \textcolor{Green}{Implementato} & \textcolor{Green}{Superato} \\ \hline
		TI24 & Si verifica che CreateListViewPresenter interagisca con ChatSource in modo corretto & \textcolor{Orange}{Non implementato} & \textcolor{Orange}{Non superato} \\ \hline
		TI25 & Si verifica che ShareListUseCase interagisca in modo corretto con DatabaseSource & \textcolor{Green}{Implementato} & \textcolor{Green}{Superato} \\ \hline
		TI26 & Si verifica che ManageListUseCase interagisca con DatabaseSource rimuovendo e aggiungendo una lista dal database in modo corretto & \textcolor{Green}{Implementato} & \textcolor{Green}{Superato} \\ \hline
		TI27 & Si verifica che DeleteListViewPresenter interagisca con ShowPopupUseCase mostrando il popup di eliminazione della lista & \textcolor{Green}{Implementato} & \textcolor{Green}{Superato} \\ \hline
		TI28 & Si verifica che ForwardListUseCase interagisca in modo corretto con ChatSource al fine di inoltrare la lista selezionata & \textcolor{Orange}{Non implementato} & \textcolor{Orange}{Non superato} \\ \hline
		TI29 & Si verifica che GetListInfoUseCase esegua i fetch dei dati in modo corretto dal database tramite DatabaseSource & \textcolor{Green}{Implementato} & \textcolor{Green}{Superato} \\ \hline
		TI30 & Si verifica che ChangeListInfoViewPresenter interagisca in modo corretto con ModifyListUseCase & \textcolor{Orange}{Non implementato} & \textcolor{Orange}{Non superato} \\ \hline
		TI31 & Si verifica che ChangeListInfoViewPresenter interagisca in modo corretto con ShowPopupUseCase & \textcolor{Orange}{Non implementato} & \textcolor{Orange}{Non superato} \\ \hline
		TI32 & Si verifica che ShareWithGroupViewImpl interagisca in modo corretto con ShareWithGroupViewPresenter & \textcolor{Green}{Implementato} & \textcolor{Green}{Superato} \\ \hline
		TI33 & Si verifica che ShareWithContactViewImpl interagisca in modo corretto con ShareWithContactViewPresenter & \textcolor{Green}{Implementato} & \textcolor{Green}{Superato} \\ \hline
		TI34 & Si verifica che GetItemInfoUseCase interagisca in modo corretto con DatabaseSource & \textcolor{Green}{Implementato} & \textcolor{Green}{Superato} \\ \hline
		TI35 & Si verifica che ModifyItemPresenter interagisca in modod corretto con ModifyListUseCase & \textcolor{Orange}{Non implementato} & \textcolor{Orange}{Non superato} \\ \hline
		TI36 & Si verifica che ModifyItemPresenter interagisca in modod corretto con ShowPopupUseCase & \textcolor{Orange}{Non implementato} & \textcolor{Orange}{Non superato} \\ \hline
		TI37 & Si verifica che ModifyItemView interagisca in modod corretto con ModifyItemPresenter & \textcolor{Orange}{Non implementato} & \textcolor{Orange}{Non superato} \\ \hline
		TI38 & Si verifica che ModifyListUseCase interagisca in modo corretto con DatabaseSource & \textcolor{Green}{Implementato} & \textcolor{Green}{Superato} \\ \hline
		TI39 & Si verifica che ShareWithContactViewPresenter interagisca in modo corretto con ChatSource & \textcolor{Green}{Implementato} & \textcolor{Green}{Superato} \\ \hline
		TI40 & Si verifica che ShareWithContactViewPresenter interagisca in modo corretto con ShowPopupUseCase & \textcolor{Green}{Implementato} & \textcolor{Green}{Superato} \\ \hline
		TI41 & Si verifica che CreateListViewImpl interagisca in modo corretto con CreateListViewPresenter & \textcolor{Green}{Implementato} & \textcolor{Green}{Superato} \\ \hline
		TI42 & Si verifica che DeleteListViewImpl interagisca in modo corretto con DeleteListViewPresenter & \textcolor{Green}{Implementato} & \textcolor{Green}{Superato} \\ \hline
		TI43 & Si verifica che InputListInfoViewImpl interagisca in modo corretto con InputListInfoViewPresenter & \textcolor{Green}{Implementato} & \textcolor{Green}{Superato} \\ \hline
		TI44 & Si verifica che InputItemInfoViewImpl interagisca in modo corretto con InputItemInfoViewPresenter & \textcolor{Green}{Implementato} & \textcolor{Green}{Superato} \\ \hline
	\end{longtable}
\end{center}

\subsubsection{Test di Sistema}
Con questa tipologia di test si vuole verificare il corretto funzionamento delle componenti architetturali.\\
I test di sistema saranno descritti nel modo seguente:
\begin{center}
\textbf{TS}[\textit{TipoRequisito}][\textit{ImportanzaRequisito}][\textit{IdRequisito}]
\end{center}
dove:
\begin{itemize}
\item \textbf{TipoRequisito} può assumere valori tra:
\begin{itemize}
\item \textit{F} per i requisiti funzionali;
\item \textit{Q} per i requisiti di qualità;
\item \textit{V} per i requisiti di vincolo;
\item \textit{P} per i requisiti prestazionali.
\end{itemize}
\item \textbf{ImportanzaRequisito} può assumere valori tra:
\begin{itemize}
\item \textit{D} per i requisiti desiderabili;
\item \textit{O} per i requisiti di obbligatori;
\item \textit{F} per i requisiti di facoltativi.
\end{itemize}
\item \textbf{IdRequisito} può assumere un valore gerarchico che identifica il singolo requisito.
\end{itemize}

\subsubsection{Test di Validazione}
I test di validazione hanno lo scopo di accertare che tutte le funzionalità richieste dal proponente siano state soddisfatte.
Per questo motivo, attraverso delle macro azioni, si andrà a simulare il comportamento generale dell'applicativo e dell'utente
che interagisce con esso.
I test di validazione saranno organizzati nel modo seguente:
\begin{center}
\textbf{TV}[\textit{TipoRequisito}][\textit{ImportanzaRequisito}][\textit{IdRequisito}]
\end{center}
dove:
\begin{itemize}
\item
\textbf{TipoRequisito} può assumere valori tra:
	\begin{itemize}
	\item
	\textit{F} per i requisiti funzionali;
	\item
	\textit{Q} per i requisiti di qualità;
	\item
	\textit{V} per i requisiti di vincolo;
	\item
	\textit{P} per i requisiti prestazionali.
	\end{itemize}
\item
\textbf{ImportanzaRequisito} può assumere valori tra:
	\begin{itemize}
	\item
	\textit{D} per i requisiti desiderabili;
	\item
	\textit{O} per i requisiti di obbligatori;
	\item
	\textit{F} per i requisiti di facoltativi.
	\end{itemize}
\item
\textbf{IdRequisito} assume un valore gerarchico che identifica il singolo requisito.
\end{itemize}

\begin{center}
	\begin{longtable}{|c|>{\centering}m{10cm}|c|}\hline
		Id & Descrizione & stato \\ \hline
		TVVO1 & Viene verificato che sia possibile scaricare l'sdk trammite il gestore dei pacchetti npm & Non implementato \\ \hline
		TVFO2 & Viene verificato che l'sdk sia utilizzabile dal applicazione di demo & Non implementato \\ \hline
		TVFO3 & Viene verificato che sia possibile istanziare tutti i widget che sono inclusi nel sdk & Non implementato \\ \hline
		TVFO4 & Viene verificato che sia possibile istanziare tutti i layout che sono inclusi nel sdk & Non implementato \\ \hline
		TVFO5 & Viene verificato che sia possibile istanziare tutte le bolle che sono incluse nel sdk & Non implementato \\ \hline
		TVFO6 & Viene verificato che sia possibile comporre una bolla personalizzata mediante widget e layout & Non implementato \\ \hline
		TVFO7 & Viene verificato che un generico utente possa creare un lista della spesa e pubblicarla al'interno di un canale Rockt.chat & Non implementato \\ \hline
		TVFO8 & Viene verificato che un generico utente possa creare un lista della spesa e pubblicarla ad un singolo utente avente un account sul istanza Rocket.chat su cui è installata l'sdk & Non implementato \\ \hline
		TVFO9 & Viene verificato che se un utente ha i permessi per interagire con una bolla qust'ultimo sia in grado di visualizzare gli elementi e di spuntarli & Non implementato \\ \hline
		TVFO10 & Viene verificato che se un utente inoltra una lista della spesa qust'ultima venga inoltrata in forma testuale bloccando quindi ogni forma di interazione con l'utente & Non implementato \\ \hline
		TVVO11 & Viene  verificato che tutti i file che contengono codice scritto in javascript o coffe script passino i test di linting & Non implementato \\ \hline
	\end{longtable}
\end{center}

\newpage