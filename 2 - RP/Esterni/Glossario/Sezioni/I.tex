\section*{I}
\addcontentsline{toc}{section}{I}
\begin{itemize}
	\item
	\textbf{Ide}: \textit{eng. Integrated Development Environment, ambiente di sviluppo integrato}. Software che, in fase di programmazione, aiuta i programmatori nello sviluppo del codice sorgente di un programma.
	\item
	\textbf{Indice Di Manutenibilità}: Valore che indica quanto il codice risulta manutenibile.
	\item
	\textbf{Inline}: Mantenere una sequenza di caratteri nella stessa riga.
	\item
	\textbf{Inspection}: \textit{eng. Ispezione}.
	\item
	\textbf{Instant Messaging}: E' una categoria di sistemi di comunicazione in tempo reale in rete, tipicamente Internet o una rete locale, che permette ai suoi utilizzatori lo scambio di messaggi.
	\item
	\textbf{Intellij Idea}: Un ambiente di sviluppo integrato \termine{IDE}.
	\item
	\textbf{Inversion Of Control}: Pattern per cui un componente di livello applicativo riceve il controllo da un componente appartenente a un libreria riusabile.
	\item
	\textbf{Iso}: \textit{eng. International Organization for Standardization, Organizzazione internazionale per la normazione}. La più importante organizzazione a livello mondiale per la definizione di norme tecniche.
	\item
	\textbf{Issue}: \textit{eng. Problema}. All'interno del mondo \termine{Git}, una \termine{issue} è una metodologia utilizzata per far presente agli sviluppatori al lavoro su una determinata \termine{repository} che vi è un problema con il codice scritto in essa.
\end{itemize}
\newpage