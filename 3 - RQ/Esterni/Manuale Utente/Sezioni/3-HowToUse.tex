\section{Features Guide}
As it's mentioned in the introduction this product is divided in two parts, so the group \gruppo\ will supply two guides for the correct using of this software.

\subsection{SDK}
This guide is dedicated just for one actor: the developer. Using this Manual he can simply learn how to use \textit{Monolith}, the \termine{SDK} supplied by \gruppo. 

\subsubsection{Creation of a Bubble}
There are three kind of Bubbles available in \textit{Monolith}.
\begin{itemize}
\item \textbf{MarkdownBubble}
\item \textbf{ToDoListBubble}
\item \textbf{AlertBubble}
\end{itemize}

%immagine di codice bolle e come si vedono

Supposing you wanted to create your own Bubble it would be necessary extend the \textit{Abstract} class \texttt{BaseBubble} and would add all the components you need with the method \texttt{addComponent} available in his relative class. You can add every widget there is in \textit{Monolith} ord add your coustom one.

%immagine del codice di una nuova bolla creata da uno sviluppatore con i propri campi



\subsubsection{Composition/chaining of Layouts or Components}

In the \termine{SDK} you can also compose the Bubble as you want. To make this clearer we'll give you some examples of adding a Widget and adding layouts.

\paragraph{Adding a widget}

%immagine

\paragraph{Adding a Layout with some widgets}

%immagine
