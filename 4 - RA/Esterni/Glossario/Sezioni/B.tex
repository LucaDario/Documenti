\section*{B}
\addcontentsline{toc}{section}{B}
\begin{itemize}
	\item
	\textbf{Best Practice}: Insieme delle attività che, organizzate in modo sistematico, possono essere prese come riferimento e riprodotte per favorire il raggiungimento dei risultati migliori.
	\item
	\textbf{Binding}: In Javascript si esegue il binding di una certa funzione con un certo this, in modo tale da essere certi che quando quella funzione verrà chiamata, il this sarà quello con la quale viene 'incatenata'.
	\item
	\textbf{Bitbucket}: Servizio di hosting per progetti software.
	\item
	\textbf{Bolla}: Ogni messaggio presente all'interno di \termine{Monolith} e, più in generale di \termine{Rocket.chat}, che l'utente può visualizzare all'interno di una qualsiasi chat.
	\item
	\textbf{Bolle Interattive}: Messagio in grado di cambiare il proprio contenuto dopo essere stato inviato.
	\item
	\textbf{Bot}: Programma che accede alla rete attraverso lo stesso tipo di canali utilizzati dagli utenti umani, e che nel nostro caso invia messaggi all'interno di una chat offrendo servizi utili agli utenti.
	\item
	\textbf{Bottone Semplice}: Oggetto che genera eventi alla sua pressione.
	\item
	\textbf{Browser}: E' un'applicazione per il recupero, la presentazione e la navigazione delle risorse sul web.
	\item
	\textbf{Business}: Termine inglese che identifica in generale un'attività economica e approssimativamente può essere tradotto con il termine italiano \textit{affari}.
	\item
	\textbf{Business Logic}: Logica di Business, ovvero la parte di un'architettura software che contiene le informazioni per eseguire le operazioni.
\end{itemize}
\newpage