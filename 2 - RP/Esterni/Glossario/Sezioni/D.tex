\section*{D}
\addcontentsline{toc}{section}{D}
\begin{itemize}
	\item
	\textbf{Database}: Indica un insieme di dati, omogeneo per contenuti e per formato, memorizzati in un elaboratore elettronico e interrogabili via terminale utilizzando le chiavi di accesso previste.
	\item
	\textbf{Dbms}: \textit{eng. Database Management System}. Sistema software progettato per consentire la creazione, la manipolazione e l'interrogazione efficiente di database.
	\item
	\textbf{Debugger}: Programma/software specificatamente progettato per l'analisi e l'eliminazione dei bug (\textit{debugging}), ovvero errori di programmazione interni al codice di altri programmi.
	\item
	\textbf{Default}: Stato o risposta di un sistema qualunque in assenza (per difetto, cioè in mancanza) di interventi espliciti (ad esempio input o configurazioni dell'utente), ovverosia \textit{predefinito}.
	\item
	\textbf{Demo}: Campione dimostrativo della produzione dei programmatori. Specificatamente, risultato dell'utilizzo dell'\termine{SDK} che ne mostra alcune (se non tutte) le funzionalità.
	\item
	\textbf{Dependency Injection}: Design pattern della Programmazione orientata agli oggetti il cui scopo è quello di semplificare lo sviluppo e migliorare la testabilità di software di grandi dimensioni.
	\item
	\textbf{Dependency Inversion Principle}: Tecnica di disaccoppiamento dei moduli software, che consiste nel rovesciare la pratica tradizionale secondo cui i moduli di alto livello dipendono da quelli di basso livello.
	\item
	\textbf{Design-Pattern}: Soluzione progettuale generale ad un problema ricorrente.
	\item
	\textbf{Diagramma Di Gantt}: Strumento di supporto alla gestione dei progetti si rappresenta sotto forma di un grafico avente un asse orizzontale in cui vi è la rappresentazione dell'arco temporale totale del progetto, suddiviso in fasi incrementali (ad esempio, giorni, settimane, mesi) e da un asse verticale che mostra la rappresentazione delle mansioni o attività che costituiscono il progetto.
\end{itemize}
\newpage