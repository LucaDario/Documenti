\begin{center}
	\begin{longtable}{|c|>{\centering}m{10cm}|c|}\hline
		Id & Descrizione & Stato\\ \hline
		TI1 & Si verifica che il metodo ButtonWidget::setText passi correttamente il testo al presenter il quale eventualmente lo formatterà e modificherà correttamente l'HTML e CSS & \textcolor{Green}{Implementato}\\ \hline
		TI2 & Si verifica che il metodo ButtonWidget::setHeight passi correttamente il valore dell'altezza al presenter il quale modificherà correttamente l'HTML e CSS & \textcolor{Green}{Implementato}\\ \hline
		TI3 & Si verifica che il metodo ButtonWidget::setWidth passi correttamente il valore della lareghezza al presenter modificherà correttamente l'HTML e CSS & \textcolor{Green}{Implementato}\\ \hline
		TI4 & Si verifica che il metodo ButtonWidget::setBackgroundColor passi correttamente il colore dello sfondo del bottone al presenter il quale modificherà correttamente l'HTML e CSS & \textcolor{Green}{Implementato}\\ \hline
		TI5 & Si verifica che il metodo ButtonWidget::setOnClickAction passi correttamente al presenter la funzione che verrà eseguita quanto viene premuto il pulsante & \textcolor{Green}{Implementato}\\ \hline
		TI6 & Si verifica che il metodo ButtonWidget::setOnLongClickAction passi correttamente al presenter la funzione che verrà eseguita quanto viene mantenuto premuto il pulsante & \textcolor{Green}{Implementato}\\ \hline
		TI7 & Si verifica che il metodo TextWidget::setText passi correttamente al presenter il testo che verrà poi mostrato a video & \textcolor{Green}{Implementato}\\ \hline
		TI8 & Si verifica che il metodo TextWidget::setTextColor passi correttamente il colore del testo al presenter il quale modificherà correttamente l'HTML e CSS, presentandolo con il colore corrispondente & \textcolor{Green}{Implementato}\\ \hline
		TI9 & Si verifica che il metodo TextWidget::setFormatText passi il valore booleano (che rappresenta la scelta di formattare il testo) al presenter. Esso dovrà, poi, presentare il testo formattato seguendo la sintassi markdown oppure lasciarlo invariato in funzione al valore booleano & \textcolor{Green}{Implementato}\\ \hline
		TI10 & Si verifica che il metodo TextWidget::setUrlHighlightColor passi il valore del colore al presenter il quale dovrà presentare i link evidenziati oppure non evidenziati in funzione al colore scelto & \textcolor{Green}{Implementato}\\ \hline
		TI11 & Si verifica che il metodo TextWidget::setTextSize passi la dimensione del testo del presenter il quale dovrà occuparsi di presentare il testo della dimensione impostata & \textcolor{Green}{Implementato}\\ \hline
		TI12 & Si verifica che il metodo ChecklistItemWidget::setUseSelectionMark imposti la scelta di utilizzare un carattere o un colore come spunta passando correttamente per il Presenter & \textcolor{Green}{Implementato}\\ \hline
		TI13 & Si verifica che il metodo ChecklistItemWidget::setSelectionColor passi al presenter il colore dello spunta il quale dovrà presentare la lista con lo stile corretto & \textcolor{Green}{Implementato}\\ \hline
		TI14 & Si verifica che il metodo ChecklistItemWidget::setSelectionCharacter passi al presenter un carattere per la spunta, e il presenter dovrà presentare la lista con lo stile corretto & \textcolor{Green}{Implementato}\\ \hline
		TI15 & Si verifica che il metodo ListWidget::addItem passi al presenter il testo del elemento il quale dovrà presentare l'HTML con il nuovo elemento & \textcolor{Green}{Implementato}\\ \hline
		TI16 & Si verifica che la view passi al presenter il carattere con il quale mostrare gli indicatori della lista. Esso provvederà poi ad impostarli correttamente nell'HTML & \textcolor{Green}{Implementato}\\ \hline
		TI17 & Si verifica che il metodo ImageWidget::setWidth passi correttamente al presenter il valore dell'altezza il quale aggiornerà la view & \textcolor{Green}{Implementato}\\ \hline
		TI18 & Si verifica che il metodo ImageWidget::setHeight passi correttamente al presenter il valore della larghezza il quale aggiornerà la view & \textcolor{Green}{Implementato}\\ \hline
		TI19 & Si verifica che il metodo ImageWidget::setImage passi correttamente al presenter il path del'immagine che dovrà essere presentata nella view & \textcolor{Green}{Implementato}\\ \hline
		TI20 & Si verifica che l'aggiunta di un VerticalLayoutView, creato con degli elementi al suo interno, non causi errori nell'aggiunta di questo alla bolla e che, gli elementi, nel layout appena aggiunto, vengano mostrati uno sotto l'altro & \textcolor{Green}{Implementato}\\ \hline
		TI21 & Si verifica che l'aggiunta di un HorizontalLayoutView, creato con degli elementi al suo interno, non causi errori nell'aggiunta di questi alla bolla e che gli elementi, nel layout appena aggiunto, vengano mostrati uno vicino all'altro & \textcolor{Green}{Implementato}\\ \hline
		TI22 & Si verifica che il ModifyListUseCase esegua le operazioni di aggiunta, rimozione e aggiornamento dei dati trammite DatabaseSource in modo corretto & \textcolor{Green}{Implementato}\\ \hline
		TI23 & Si verifica che ShareWithGroupViewPresenter interagisca in modo corretto con ChatSource & \textcolor{Green}{Implementato}\\ \hline
		TI24 & Si verifica che CreateListViewPresenter interagisca con ChatSource in modo corretto & \textcolor{Green}{Implementato}\\ \hline
		TI25 & Si verifica che ShareListUseCase interagisca in modo corretto con DatabaseSource & \textcolor{Green}{Implementato}\\ \hline
		TI26 & Si verifica che ManageListUseCase interagisca con DatabaseSource rimuovendo e aggiungendo una lista dal database in modo corretto & \textcolor{Green}{Implementato}\\ \hline
		TI27 & Si verifica che DeleteListViewPresenter interagisca con ShowPopupUseCase mostrando il popup di eliminazione della lista & \textcolor{Green}{Implementato}\\ \hline
		TI28 & Si verifica che GetListInfoUseCase esegua i fetch dei dati in modo corretto dal database tramite DatabaseSource & \textcolor{Green}{Implementato}\\ \hline
		TI29 & Si verifica che ChangeListInfoViewPresenter interagisca in modo corretto con ModifyListUseCase & \textcolor{Orange}{Non implementato}\\ \hline
		TI30 & Si verifica che ChangeListInfoViewPresenter interagisca in modo corretto con ShowPopupUseCase & \textcolor{Orange}{Non implementato}\\ \hline
		TI31 & Si verifica che ShareWithGroupViewImpl interagisca in modo corretto con ShareWithGroupViewPresenter & \textcolor{Green}{Implementato}\\ \hline
		TI32 & Si verifica che ShareWithContactViewImpl interagisca in modo corretto con ShareWithContactViewPresenter & \textcolor{Green}{Implementato}\\ \hline
		TI33 & Si verifica che GetItemInfoUseCase interagisca in modo corretto con DatabaseSource & \textcolor{Green}{Implementato}\\ \hline
		TI34 & Si verifica che ModifyItemPresenter interagisca in modod corretto con ModifyListUseCase & \textcolor{Orange}{Non implementato}\\ \hline
		TI35 & Si verifica che ModifyItemPresenter interagisca in modod corretto con ShowPopupUseCase & \textcolor{Orange}{Non implementato}\\ \hline
		TI36 & Si verifica che ModifyItemView interagisca in modod corretto con ModifyItemPresenter & \textcolor{Orange}{Non implementato}\\ \hline
		TI37 & Si verifica che ModifyListUseCase interagisca in modo corretto con DatabaseSource & \textcolor{Green}{Implementato}\\ \hline
		TI38 & Si verifica che ShareWithContactViewPresenter interagisca in modo corretto con ChatSource & \textcolor{Green}{Implementato}\\ \hline
		TI39 & Si verifica che ShareWithContactViewPresenter interagisca in modo corretto con ShowPopupUseCase & \textcolor{Green}{Implementato}\\ \hline
		TI40 & Si verifica che CreateListViewImpl interagisca in modo corretto con CreateListViewPresenter & \textcolor{Green}{Implementato}\\ \hline
		TI41 & Si verifica che DeleteListViewImpl interagisca in modo corretto con DeleteListViewPresenter & \textcolor{Green}{Implementato}\\ \hline
		TI42 & Si verifica che InputListInfoViewImpl interagisca in modo corretto con InputListInfoViewPresenter & \textcolor{Green}{Implementato}\\ \hline
		TI43 & Si verifica che InputItemInfoViewImpl interagisca in modo corretto con InputItemInfoViewPresenter & \textcolor{Green}{Implementato}\\ \hline
	\end{longtable}
\end{center}
