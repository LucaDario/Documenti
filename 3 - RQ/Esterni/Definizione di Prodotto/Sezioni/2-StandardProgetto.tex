\newpage
\section{Standard di Progetto}
\subsection{Documentazione del codice}
Gli standard per la scrittura della documentazione del codice sono definiti nelle \normeDiProgetto.

\subsection{Denominazione entità e relazioni}
Tutti gli elementi definiti, siano essi package, classi, metodi o attributi, devono avere denominazioni chiare ed autoesplicative. \\
Nel caso in cui il nome risulti essere lungo è preferibile anteporre la chiarezza alla lunghezza.
Sono ammesse abbreviazioni se:
\begin{itemize}
	\item immediatamente comprensibili;
	\item non ambigue;
	\item sufficientemente contestualizzate
\end{itemize}
Le regole tipografiche relativi ai nomi delle entità sono definiti nelle \normeDiProgetto.

\subsection{Codifica}
Gli standard di programmazione sono definiti e descritti nelle  \normeDiProgetto.

\subsection{Strumenti di lavoro}
Gli strumenti da adottare e le procedure da seguire per utilizzarli correttamente durante la realizzazione del prodotto software sono definiti nelle  \normeDiProgetto.
\newpage