\subsubsection{Test di Validazione}
I test di validazione hanno lo scopo di accertare che tutte le funzionalità richieste dal proponente siano state soddisfatte.
Per questo motivo, attraverso delle macro azioni, si andrà a simulare il comportamento generale dell'applicativo e dell'utente
che interagisce con esso.
I test di validazione saranno organizzati nel modo seguente:
\begin{center}
\textbf{TV}[\textit{TipoRequisito}][\textit{ImportanzaRequisito}][\textit{IdRequisito}]
\end{center}
dove:
\begin{itemize}
\item
\textbf{TipoRequisito} può assumere valori tra:
	\begin{itemize}
	\item
	\textit{F} per i requisiti funzionali;
	\item
	\textit{Q} per i requisiti di qualità;
	\item
	\textit{V} per i requisiti di vincolo;
	\item
	\textit{P} per i requisiti prestazionali.
	\end{itemize}
\item
\textbf{ImportanzaRequisito} può assumere valori tra:
	\begin{itemize}
	\item
	\textit{D} per i requisiti desiderabili;
	\item
	\textit{O} per i requisiti di obbligatori;
	\item
	\textit{F} per i requisiti di facoltativi.
	\end{itemize}
\item
\textbf{IdRequisito} assume un valore gerarchico che identifica il singolo requisito.
\end{itemize}

\begin{center}
	\begin{longtable}{|c|>{\centering}m{10cm}|c|}\hline
		Id & Descrizione & stato \\ \hline
		TVVO1 & Viene verificato che sia possibile scaricare l'sdk trammite il gestore dei pacchetti npm & Non implementato \\ \hline
		TVFO2 & Viene verificato che l'sdk sia utilizzabile dal applicazione di demo & Non implementato \\ \hline
		TVFO3 & Viene verificato che sia possibile istanziare tutti i widget che sono inclusi nel sdk & Non implementato \\ \hline
		TVFO4 & Viene verificato che sia possibile istanziare tutti i layout che sono inclusi nel sdk & Non implementato \\ \hline
		TVFO5 & Viene verificato che sia possibile istanziare tutte le bolle che sono incluse nel sdk & Non implementato \\ \hline
		TVFO6 & Viene verificato che sia possibile comporre una bolla personalizzata mediante widget e layout & Non implementato \\ \hline
		TVFO7 & Viene verificato che un generico utente possa creare un lista della spesa e pubblicarla al'interno di un canale Rockt.chat & Non implementato \\ \hline
		TVFO8 & Viene verificato che un generico utente possa creare un lista della spesa e pubblicarla ad un singolo utente avente un account sul istanza Rocket.chat su cui è installata l'sdk & Non implementato \\ \hline
		TVFO9 & Viene verificato che se un utente ha i permessi per interagire con una bolla qust'ultimo sia in grado di visualizzare gli elementi e di spuntarli & Non implementato \\ \hline
		TVFO10 & Viene verificato che se un utente inoltra una lista della spesa qust'ultima venga inoltrata in forma testuale bloccando quindi ogni forma di interazione con l'utente & Non implementato \\ \hline
		TVVO11 & Viene  verificato che tutti i file che contengono codice scritto in javascript o coffe script passino i test di linting & Non implementato \\ \hline
	\end{longtable}
\end{center}
