\subsubsection{monolith::bubble::BaseBubble}

\label{monolith::bubble::BaseBubble}
\begin{figure}[ht]
	\centering
	\includegraphics[scale=0.5]{Sezioni/SottosezioniST/img/BaseBubble.png}
	\caption{monolith::bubble::BaseBubble}
\end{figure}

\begin{itemize}
\item \textbf{Descrizione:} Classe base astratta che rappresenta le bolle di Monolith.
\item \textbf{Utilizzo:} Classe base astratta utilizzata ed estesa ogni qualvolta uno sviluppatore intende creare nuove bolle.
\item \textbf{Attributi:} 
\begin{itemize}
\item \textit{protected layout:VerticalLayoutView}\\
Oggetto che rappresenta il layout verticale di default della bolla.
\end{itemize}
\item \textbf{Metodi:}
\begin{itemize}
	\item \textit{public addComponent(component:BaseComponent):void}\\
Aggiunge un widget o layout alla bolla.
		  \\ \textbf{Parametri}: \begin{itemize}
				\item \textit{component:BaseComponent}\\
					Oggetto che rappresenta il componente che si desidera aggiungere alla bolla.
			\end{itemize}
	\item \textit{public renderView():string}\\
Genera il codice HTML, CSS e JavaScript necessario per visualizzare bolle.
\end{itemize}
\end{itemize}

\subsubsection{monolith::bubble::todolist::ToDoListBubble}

\label{monolith::bubble::todolist::ToDoListBubble}
\begin{figure}[ht]
	\centering
	\includegraphics[scale=0.5]{Sezioni/SottosezioniST/img/ToDoListBubble.png}
	\caption{monolith::bubble::todolist::ToDoListBubble}
\end{figure}

\begin{itemize}
\item \textbf{Descrizione:} Classe concreta che estende BaseBubble, destinata alla creazione di bolle lista di Monolith.
\item \textbf{Utilizzo:} Classe utilizzata ogni qualvolta uno sviluppatore intende creare nuove bolle lista.
\item \textbf{Attributi:}
\begin{itemize}
\item \textit{private textView:TextWidgetView}\\
Oggetto che rappresenta il widget contenente il testo della bolla lista.
\item \textit{private checklist:CheckListView}\\
Oggetto che rappresenta il widget contenente la lista di oggetti che si possono spuntare.
\end{itemize}
\item \textbf{Metodi:}
\begin{itemize}
\item \textit{public ToDoListBubble():ToDoListBubble}\\
Il costruttore della classe ToDoListBubble.
\item \textit{public addItem(item:string):void}\\
Aggiunge un elemento con il nome definito alla bolla lista.
\ \textbf{Parametri}: \begin{itemize}
\item \textit{item:string}\\
Valore che rappresenta il nome dell'elemento che si vuole aggiungere alla lista.
\end{itemize}
\end{itemize}
\end{itemize}

\subsubsection{monolith::bubble::markdown::MarkdownBubble}

\label{monolith::bubble::markdown::MarkdownBubble}
\begin{figure}[ht]
	\centering
	\includegraphics[scale=0.5]{Sezioni/SottosezioniST/img/MarkdownBubble.png}
	\caption{monolith::bubble::markdown::MarkdownBubble}
\end{figure}

\begin{itemize}
\item \textbf{Descrizione:} Classe concreta che estende BaseBubble, destinata alla creazione di bolle testo markdown di Monolith.
\item \textbf{Utilizzo:} Classe utilizzata ogni qualvolta uno sviluppatore intende creare nuove bolle testo markdown.
\item \textbf{Attributi:}
\begin{itemize}
\item \textit{private textview:TextWidgetView}\\
Oggetto che rappresenta il widget contenente il testo della bolla testo markdown.
\end{itemize}
\item \textbf{Metodi:}
\begin{itemize}
\item \textit{public MarkdownBubble():MarkdownBubble}\\
Il costruttore della classe MarkdownBubble.
\item \textit{public setText(text:string):void}\\
Imposta il testo della bolla con il valore definito.
\\ \textbf{Parametri}: \begin{itemize}
\item \textit{text:string}\\
Valore che rappresenta il testo che si vuole inserire nella bolla testo markdown.
\end{itemize}
\end{itemize}
\end{itemize}

\subsubsection{monolith::bubble::alert::AlertBubble}

\label{monolith::bubble::alert::AlertBubble}
\begin{figure}[ht]
	\centering
	\includegraphics[scale=0.5]{Sezioni/SottosezioniST/img/AlertBubble.png}
	\caption{monolith::bubble::alert::AlertBubble}
\end{figure}

\begin{itemize}
\item \textbf{Descrizione:} Classe concreta che estende BaseBubble, destinata alla creazione di bolle avviso di Monolith.
\item \textbf{Utilizzo:} Classe utilizzata ogni qualvolta uno sviluppatore intende creare nuove bolle avviso.
\item \textbf{Attributi:} 
\begin{itemize}
\item \textit{private titleView:TextWidgetView}\\
Campo che rappresenta e contiene il titolo della bolla avviso.
\item \textit{private messageView:TextWidgetView}\\
Campo che rappresenta e contiene il messaggio della bolla avviso.
\end{itemize}
\item \textbf{Metodi:}
\begin{itemize}
\item \textit{public AlertBubble():AlertBubble}\\
Il costruttore della classe AlertBubble.
\item \textit{public setTitle(title:string):void}\\
Imposta il titolo della bolla avviso con il valore title.
\\ \textbf{Parametri}: \begin{itemize}
\item \textit{title:string}\\
Valore che rappresenta il titolo che si vuole impostare alla bolla avviso.
\end{itemize}
\item \textit{public setMessage(message:string):void}\\
Imposta il messaggio della bolla avviso con il valore message.
\\ \textbf{Parametri}: \begin{itemize}
\item \textit{message:string}\\
Valore che rappresenta il messaggio testuale che si vuole impostare alla bolla avviso.
\end{itemize}
\end{itemize}
\end{itemize}