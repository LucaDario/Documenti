\section{Use requirements}

% Requisiti generali
\subsection{Requirements}
\subsubsection{\progettoShort}
To use \progettoShort, the developer needs an internet connection and some kind of knowledge on how to setup a \termine{Meteor.js} project.

% Requisiti dell'app
\subsubsection{\app}
In order to use \app the user needs access to the internet either using a laptop or desktop computer, or using a mobile cellphone or tablet. Other than that, the device which will be used to use \app\ must have a browser which supports \termine{JavaScript} and has it enabled.

% Requisiti da desktop o laptop
\paragraph{Desktop requirements}
To access \app\ using a desktop or laptop computer, one of the following browsers is required:
\begin{itemize}
	\item Microsoft Internet Edge 13 or above;
	\item Mozilla Firefox 45 or above;
	\item Google Chrome 56 or above;
	\item Opera 43 or above;
	\item Apple Safari 43 or above.
\end{itemize}

% Requisiti da mobile
\paragraph{Mobile requirements}
To access \app\ using a mobile phone or table, one of the following requirement needs to be satisfied:
\begin{itemize}
	\item If the device has Android as its operative system, it needs to have Google Chrome 56 or above installed;
	\item If the device has iOS as its operative system, it needs to have either iOS 10 or above installed, or have Google Chrome 56 or above as browser;
	\item If none of the above requirement is satisfied, the \app\ application can also be accessed from any mobile device using the \termine{Rocket.chat}'s mobile application.
\end{itemize}

% Sezione su come abilitare JS
\paragraph{Javascript}
In order to enable \termine{JavaScript} inside the different browsers' versions the next steps must be followed:
% Scrivere come abilitare JavaScript


%Sezione dedicata a come installare i prodotti
\subsection{Installation}
\subsubsection{\progettoShort}
To use \progettoShort\ inside his project, the user needs to follow the following steps:
\begin{enumerate}
	\item Create a \termine{Meteor.js} project;
	\item Open a prompt shell inside the project's root and type the following commands:
	\begin{lstlisting}
	>> meteor install monolith
	>> meteor reset
	>> meteor npm install
	>> meteor run
	\end{lstlisting}
	Note that this procedure will erase all the temporary files that might have been created during previous project runs, but it will ensure that \progettoShort\ is properly installed
\end{enumerate}

\subsubsection{\app}
\app\ installation is borne by the supplier.

\subsection{Access to the application}
\subsubsection{\progettoShort}
The only way to access \progettoShort\ is by writing code which uses its classes. Major details are given below.

\subsubsection{\app}
To access \app\, the only thing required is to connect to the following \termine{Rocket.chat} server from a device which has access to the internet:
\begin{lstlisting}
Inserire credenziali del nostro server Rocket.chat dove e' possibile usare 
l'applicazione demo.
\end{lstlisting}


\newpage