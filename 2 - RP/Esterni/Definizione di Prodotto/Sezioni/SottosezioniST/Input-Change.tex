\subsubsection{InputListInfoViewImpl}
\begin{itemize}
\item \textbf{Descrizione}: Classe dedicata alla visualizzazione del popup che permette all'utente di inserire, rimuovere o modificare i dati relativi alla lista.
\item \textbf{Utilizzo}: Classe utilizzata per visualizzare popup che permette all'utente di inserire, rimuovere o modificare tutti i dati relativi alla lista.
\item \textbf{Attributi}: 
	\begin{itemize}
	\item \textit{private presenter:InputListInfoViewPresenter}\\
		Il presenter associato alla view del popup, al quale questa classe delega la gestione del comportamento della view stessa.
	\end{itemize}
\item \textbf{Metodi}:
	\begin{itemize}	
	\item \textit{public renderView():string}\\
			Genera il codice HTML CSS JS necessario per visualizzare la view.
	\item \textit{public InputListInfoViewImpl():InputListInfoViewImpl}\\
	Il costruttore della classe InputListInfoViewImpl.
	\item \textit{public createViewForListWithId(listId:string):void}\\
		Metodo che permette di creare una una view per visualizzarli.
			\item{\textbf{Parametri}: \begin{itemize}
			\item \textit{listId:string}\\
			Id della lista.
			\end{itemize}}
	\item \textit{public emitOnSaveEvent(list:ListData):void}\\
	Evento che notifica tutti gli oggetti in ascolto che è stata richiesta il salvataggio della lista.
			\item{\textbf{Parametri}: \begin{itemize}
			\item \textit{list:ListData}\\
				Insieme di tutti i dati e le informazioni di una lista.
			\end{itemize}}
	\end{itemize}
\item{Eventi}:
\end{itemize}

\subsubsection{InputListInfoViewPresenter}
\begin{itemize}
\item \textbf{Descrizione}: Classe che rappresenta il presenter dedicato alla view per l'inserimento dei dati della lista.
\item \textbf{Utilizzo}: Il presenter fa da tramite tra l'implementazione della view e la parte logica dell'applicazione, formattando i dati che verranno visualizzati nella view e manipolando gli input dell'utente per eseguire le operazioni predisposte.
\item \textbf{Attributi}: 	
	\begin{itemize}
	\item \textit{private view:InputListInfoView}\\
		La view associata al presenter.
	\item \textit{private getListInfoUseCases:GetListInfoUseCases}\\
		Oggetto che rappresenta tutti gli elementi presenti nella lista e le loro informazioni.
	\item \textit{private showPopupUseCases:ShowPopupUseCases}\\
		Oggetto che permette la creazione di un popup per l'immissione dei dati della lista da creare.	
		\end{itemize}
\item \textbf{Metodi}:
	\begin{itemize}	
	\item \textit{public InputListInfoViewPresenter(view:InputListInfoView,infoUseCase:GetListInfoUseCase,popupUseCase:ShowPopupUseCase):InputListInfoViewPresenter}\\
	Il costruttore della classe InputListInfoViewPresenter.
			\item{\textbf{Parametri}: \begin{itemize}
			\item \textit{view:InputListInfoView}\\
			La view necessaria alla costruzione del presenter.
			\item \textit{infoUseCase:GetListInfoUseCase}\\
				Oggetto che rappresenta tutti gli elementi presenti nella lista e le loro informazioni.
			\item \textit{popupUseCase:ShowPopupUseCase}\\
				Oggetto che permette la creazione di un popup per l'immissione dei dati della lista da creare.	
			\end{itemize}}
	\item \textit{private createListData():ListData}\\
	 	Metodo dedicato alla creazione dell'oggetto rappresentante l'insieme dei dati, nuovi o modificati, che compongono della lista.
	\item \textit{public renderView():string}\\
	Genera il codice HTML CSS JS necessario per visualizzare la view.
	\item \textit{public createViewForListWithId(listId:string):void}\\
		Metodo che permette di creare una una view per visualizzarli.
			\item{\textbf{Parametri}: \begin{itemize}
			\item \textit{listId:string}\\
			Parametro che rappresenta l'id della lista di cui si vuole creare in una view.
			\end{itemize}}
	\end{itemize}
\item{Eventi}:
\end{itemize}

\subsubsection{ChangeListInfoView}
\begin{itemize}
\item \textbf{Descrizione}: Interfaccia che una volta implementata permette al presenter e allo sviluppatore di modificare la view dedicata alla modifica delle informazioni di una lista.
\item \textbf{Utilizzo}: L'interfaccia viene utilizzata per disaccoppiare presenter e implementazione della vista, e visualizza i dati che gli vengono passati dal presenter.
\item \textbf{Attributi}: 
\item \textbf{Metodi}:
\item{Eventi}:
	\begin{itemize}	
	\item \textit{public onChangeListInfoClicked():void}\\
		Evento che notifica tutti gli oggetti in ascolto che è stato cliccato il pulsante relativo alla modifica della lista.
	\end{itemize}
\end{itemize}

\subsubsection{ChangeListInfoViewImpl}
\begin{itemize}
\item \textbf{Descrizione}: Classe dedicata alla modifica dei dati e informazioni relativi alla lista.
\item \textbf{Utilizzo}: Il presenter fa da tramite tra l'implementazione della view e la parte logica dell'applicazione, formattando i dati che verranno visualizzati nella view e manipolando gli input dell'utente per eseguire le operazioni predisposte.
\item \textbf{Attributi}: 
	\begin{itemize}
	\item \textit{private presenter:ChangeListInfoViewPresenter}\\
	Il presenter associato alla view dedicata alla modifica dei dati relativi alla lista, al quale questa classe delega la gestione del comportamento della view stessa.
	\end{itemize}
\item \textbf{Metodi}:
	\begin{itemize}
	\item \textit{public ChangeListInfoViewImpl():ChangeListInfoViewImpl}\\
	Il costruttore della classe ChangeListInfoViewImpl.	
	\item \textit{public renderView():string}\\
	Genera il codice HTML CSS JS necessario per visualizzare la view.
	\end{itemize}
\item{Eventi}:
\end{itemize}

\subsubsection{ChangeListInfoViewPresenter}
\begin{itemize}
\item \textbf{Descrizione}: Classe che rappresenta il presenter dedicato alla modifica delle informazioni di una lista.
\item \textbf{Utilizzo}: Il presenter fa da tramite tra l'implementazione della view e la parte logica dell'applicazione, formattando i dati che verranno visualizzati nella view e manipolando gli input dell'utente per eseguire le operazioni predisposte.
\item \textbf{Attributi}:
	\begin{itemize}
	\item \textit{private changeListView:ChangeListInfoView}\\
		La view associata al presenter.
	\item \textit{private inputView:InputListInfoView}\\
		Oggetto che rappresenta l'insieme dei dati immessi dall'utente durante la modifica della lista.
	\item \textit{private manageList:ManageListsUseCase}\\
Oggetto dedicato alla gestione dei dati relativi alla lista all'interno del database.
	\item \textit{private showPopup:ShowPopupUseCase}\\
		Oggetto che permette la creazione di un popup per l'immissione dei dati della lista da creare.	
	\end{itemize} 
\item \textbf{Metodi}:
	\begin{itemize}
	\item \textit{public ChangeListInfoViewPresenter(listInfoView:ChangeListInfoView,inputView:InputItemInfoView,manageList:ManageListsUseCase,showpopup:ShowPopupUseCase):ChangeListInfoViewPresenter}\\
Il costruttore della classe ChangeListInfoViewPresenter.
			\item{\textbf{Parametri}: \begin{itemize}
			\item \textit{listInfoView:ChangeListInfoView}\\
			La view necessaria alla costruzione del presenter.
			\item \textit{inputView:InputItemInfoView}\\
			  Oggetto che rappresenta l'insieme dei dati immessi dall'utente durante la modifica della lista.
			\item \textit{manageList:ManageListsUseCase}\\
				Oggetto dedicato alla gestione dei dati relativi alla lista all'interno del database.
			\item \textit{showpopup:ShowPopupUseCase}\\
				Oggetto che permette la creazione di un popup per l'immissione dei dati della lista da creare.	
			\end{itemize}}
	\item \textit{private showInputListInfoView():void}\\
		Metodo che permette di visualizzare il popup che conterrà la vista dove l'utente modificherà i dati della lista.
	\item \textit{private createViewForListWithId(list ListData:int):void}\\
		Metodo che permette di creare una una view per visualizzarli.
			\item{\textbf{Parametri}: \begin{itemize}
			\item \textit{list:ListData}\\
			Insieme di tutti i dati e le informazioni di una lista.
			\end{itemize}}
	\item \textit{public renderView():string}\\
	Genera il codice HTML CSS JS necessario per visualizzare la view.
	\end{itemize} 
\item{Eventi}:
\end{itemize}

\subsubsection{GetListInfoUseCase}
\begin{itemize}
\item \textbf{Descrizione}: 
\item \textbf{Utilizzo}:
\item \textbf{Attributi}: 
	\begin{itemize}
	\item \textit{private databaseSource:DatabaseSource}\\
		Riferimento al database.
	\end{itemize}
\item \textbf{Metodi}:
	\begin{itemize}
	\item \textit{GetListInfoUseCase(source:DatabaseSource):GetListInfoUseCase}\\
	Il costruttore della classe GetListInfoUseCase.
			\item{\textbf{Parametri}: \begin{itemize}
			\item \textit{source:DatabaseSource}\\
		Riferimento al database.
			\end{itemize}}
	\item \textit{public getListData(listId:string):ListData}\\
	Motodo che restituisce l'insieme dei dati e informazioni relativi a una lista.
				\item{\textbf{Parametri}: \begin{itemize}
				\item \textit{listId:string}\\
				Parametro che rappresenta l'id della lista di cui si vuole recuperare i dati e le informazioni.
				\end{itemize}}
	\end{itemize}
\item{Eventi}:
\end{itemize}

\subsubsection{ModifyListUseCase}
\begin{itemize}
\item \textbf{Descrizione}: 
\item \textbf{Utilizzo}:
\item \textbf{Attributi}: 
	\begin{itemize}
	\item \textit{private databaseSource:DatabaseSource}\\
			Riferimento al database.
	\end{itemize}
\item \textbf{Metodi}:
	\begin{itemize}
	\item \textit{ModifyListUseCase(source:DatabaseSource):ModifyListUseCase}\\
	Il costruttore della classe ModifyListUseCase.
			\item{\textbf{Parametri}: \begin{itemize}
			\item \textit{source:DatabaseSource}\\
						Riferimento al database.
			\end{itemize}}
	\item \textit{public changeListInfo(listId:string,newList:ListData):void}\\
	Metodo che permette di cambiare i dati e le informazioni di una determinata lista.
			\item{\textbf{Parametri}: \begin{itemize}
			\item \textit{listId:string}\\
			Parametro che rappresenta l'id della lista di cui si vuole modificare i dati e le informazioni.
			\item \textit{newList:ListData}\\
			Insieme di tutti i dati e le informazioni modificate di una lista.
			\end{itemize}}
	\item \textit{public addItemToList(listId:string,item:ListItem):void}\\
	Metodo che permette aggiunta di un elemento alla lista.
			\item{\textbf{Parametri}: \begin{itemize}
			\item \textit{listId:string}\\
			Parametro che rappresenta l'id della lista a cui si vuole aggiungere un elemento.
			\item \textit{item:ListItem}\\
			Elemento da inserire nella lista.
			\end{itemize}}
	\item \textit{public removeItemFromList(listId:string,item:ListItem):void}\\
	Metodo che permette l'eliminazione di un elemento da una lista.
			\item{\textbf{Parametri}: \begin{itemize}
			\item \textit{listId:string}\\
			Parametro che rappresenta l'id della lista a cui si vuole rimuovere un elemento.
			\item \textit{item:ListItem}\\
			Elemento da rimuovere dalla lista.
			\end{itemize}}	
	\item \textit{public updateItemInsideList(listId:string,item:ListItem):void}\\
	Metodo che permette la modifica di un elemento della lista.
			\item{\textbf{Parametri}: \begin{itemize}
			\item \textit{listId:string}\\
			Parametro che rappresenta l'id della lista a cui si vuoe aggiornare un elemento.
			\item \textit{item:ListItem}\\
			Elemento da modificare della lista.
			\end{itemize}}				
	\end{itemize}
\item{Eventi}:
\end{itemize}