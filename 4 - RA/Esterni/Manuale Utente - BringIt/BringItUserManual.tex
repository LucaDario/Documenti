% Questo file definisce lo stile che verrà applicato
% ad ogni pagina di contenuto
\documentclass[a4paper,11pt]{article}

\usepackage{ifthen}
\usepackage[
 a4paper,
 top=2.5cm,
 bottom=2.5cm,
 left=1.5cm,
 right=1.5cm,
 head=30pt
]{geometry}
\usepackage[italian]{babel}
\usepackage[utf8x]{inputenc}
\usepackage[T1]{fontenc}
\usepackage{fancyhdr}
\usepackage[colorlinks=true, urlcolor=black, citecolor=black, linkcolor=black]{hyperref}
\usepackage{tabularx}
\usepackage{multirow}
\usepackage{booktabs}
\usepackage{color}
\usepackage{graphicx}
\usepackage{eurosym}
\usepackage{amsmath}
\usepackage{relsize}

\usepackage[multidot]{grffile}
\usepackage{xcolor,colortbl}
\definecolor{lightblue}{HTML}{56B4E6}
\definecolor{blue}{HTML}{2953A1}
\definecolor{darkblue}{HTML}{1E396E}

\usepackage[toc,page]{appendix}
\renewcommand\appendixtocname{Appendice}
\renewcommand{\appendixpagename}{Appendice}

\newcommand\pagenumberingnoreset[1]{\gdef\thepage{\csname @#1\endcsname\c@page}}

% Cambia il font 
\renewcommand*\rmdefault{qhv}

% ***STILE PAGINA***
\pagestyle{fancy}
\fancyhf{}
\setlength{\headheight}{1cm} 
% No indentazione paragrafo
\setlength{\parindent}{0pt}

% ***INTESTAZIONE***
\newcommand\textline[4][t]{%
  \noindent\parbox[#1]{.333\textwidth}{\raisebox{-0.40\height}{#2}}%
  \parbox[#1]{.333\textwidth}{\centering #3}%
  \parbox[#1]{.333\textwidth}{\raggedleft #4}%
}

\lhead{
	\textline[t]{\includegraphics[width=1cm, keepaspectratio=true]{../../../Template/Logo/Logo.png}}{\progettoShort}{\documento}
}

\renewcommand{\headrulewidth}{0.4pt}  %Linea sotto l'intestazione

% ***PIÈ DI PAGINA***
\lfoot{\textit{\gruppoLink}\\ \footnotesize{\email}}
\rfoot{\thepage} %per le prime pagine: mostra solo il numero romano
\cfoot{}
\renewcommand{\footrulewidth}{0.4pt}   %Linea sopra il piè di pagina


% Ridefinisce command \paragraph{} andando a capo ogni dopo la parola dentro le parentesi ed ha la possibiltà di enumerazione fino a n cifre modificando il numero dentro "secnumdepth"
\usepackage{titlesec}

\setcounter{secnumdepth}{7}
\setcounter{tocdepth}{7}
%
%
%\titleformat{\paragraph}
%{\normalfont\normalsize\bfseries}{\theparagraph}{1em}{}
%\titlespacing*{\paragraph}
%{0pt}{3.25ex plus 1ex minus .2ex}{1.5ex plus .2ex}
%
%
%\titleclass{\subsubparagraph}{straight}[\subparagraph]
%\newcounter{subsubparagraph}
%
%\titleformat{\subsubparagraph}[display]
%  {\normalfont\normalsize\bf}
%  {\thesubsubparagraph.}
%  {.5em}
%  {}
%\renewcommand\thesubsubparagraph\textbf{\roman{subsubparagraph}}
%\titlespacing*{\subsubparagraph} {0pt}{4pt}{6pt}


%***LA SOTTOSEZIONE PARAGRAPH VIENE VISUALIZZATA COME UNA SECTION
\titleformat{\paragraph}{\normalfont\normalsize\bfseries}{\theparagraph}{1em}{}
\titlespacing*{\paragraph}{0pt}{3.25ex plus 1ex minus .2ex}{1.5ex plus .2ex}

\titleformat{\subparagraph}{\normalfont\normalsize\bfseries}{\thesubparagraph}{1em}{}
\titlespacing*{\subparagraph}{0pt}{3.25ex plus 1ex minus .2ex}{1.5ex plus .2ex}

\makeatletter
\newcounter{subsubparagraph}[subparagraph]
\renewcommand\thesubsubparagraph{%
  \thesubparagraph.\@arabic\c@subsubparagraph}
\newcommand\subsubparagraph{%
  \@startsection{subsubparagraph}    % counter
    {6}                              % level
    {\parindent}                     % indent
    {3.25ex \@plus 1ex \@minus .2ex} % beforeskip
    {0.75em}                           % afterskip
    {\normalfont\normalsize\bfseries}}
\newcommand\l@subsubparagraph{\@dottedtocline{6}{13em}{5.5em}} %gestione dell'indice
\newcommand{\subsubparagraphmark}[1]{}
\makeatother

\makeatletter
\newcounter{subsubsubparagraph}[subsubparagraph]
\renewcommand\thesubsubsubparagraph{%
  \thesubsubparagraph.\@arabic\c@subsubsubparagraph}
\newcommand\subsubsubparagraph{%
  \@startsection{subsubsubparagraph}    % counter
    {7}                              % level
    {\parindent}                     % indent
    {3.25ex \@plus 1ex \@minus .2ex} % beforeskip
    {0.75em}                           % afterskip
    {\normalfont\normalsize\bfseries}}
\newcommand\l@subsubsubparagraph{\@dottedtocline{7}{16em}{6.5em}} %gestione dell'indice
\newcommand{\subsubsubparagraphmark}[1]{}
\makeatother

%Generali
\newcommand{\capitolato}{C5 - Monolith: An interactive bubble provider}
\newcommand{\progettoShort}{Monolith}
\newcommand{\progetto}{Monolith: An interactive bubble provider}
\newcommand{\gruppo}{NPE Developers}
\newcommand{\gruppoLink}{\href{https://gitlab.com/npe-developers}{NpeDevelopers}}
\newcommand{\email}{\href{mailto:npe.developers@gmail.com}{\textcolor{blue}{npe.developers@gmail.com}}}
\newcommand{\password}{NP3Devel0pers}
\newcommand{\myincludegraphics}[2][]{%
	\setbox0=\hbox{\phantom{X}}%
	\vtop{
		\hbox{\phantom{X}}
		\vskip-\ht0
		\hbox{\includegraphics[#1]{#2}}}
}




%Componenti del gruppo
\newcommand{\RM}{Riccardo Montagnin}
\newcommand{\MT}{Manuel Turetta}
\newcommand{\FB}{Francesco Bazzerla}
\newcommand{\SL}{Stefano Lia}
\newcommand{\LD}{Luca Dario}
\newcommand{\DC}{Diego Cavestro}
\newcommand{\ND}{Nicolò Dovico}

%Ruoli
\newcommand{\Pm}{Project Manager}
\newcommand{\Am}{Amministratore}
\newcommand{\AmP}{Amministratori}
\newcommand{\An}{Analista}
\newcommand{\AnP}{Analisti}
\newcommand{\Dev}{Sviluppatore}
\newcommand{\DevP}{Sviluppatori}
\newcommand{\Ver}{Verificatore}
\newcommand{\VerP}{Verificatori}
\newcommand{\Progr}{Programmatore}
\newcommand{\ProgrP}{Programmatori}
\newcommand{\Prog}{Progettista}
\newcommand{\ProgP}{Progettisti}



%Firme
\newcommand{\RMFirma}{\myincludegraphics[scale = 0.5]{../../../Template/Firme/RM.png}}
\newcommand{\MTFirma}{\myincludegraphics[scale = 0.5]{../../../Template/Firme/MT.png}}
\newcommand{\FBFirma}{\myincludegraphics[scale = 0.5]{../../../Template/Firme/FB.png}}
\newcommand{\SLFirma}{\myincludegraphics[scale = 0.5]{../../../Template/Firme/SL.png}}
\newcommand{\LDFirma}{\myincludegraphics[scale = 0.5]{../../../Template/Firme/LD.png}}
\newcommand{\DCFirma}{\myincludegraphics[scale = 0.5]{../../../Template/Firme/DC.png}}
\newcommand{\NDFirma}{\myincludegraphics[scale = 0.5]{../../../Template/Firme/ND.png}}

%Professori e proponente
\newcommand{\TV}{Prof. Tullio Vardanega}
\newcommand{\RC}{Prof. Riccardo Cardin}
\newcommand{\RB}{Red Babel}
\newcommand{\proponente}{Red Babel}

%Documenti
\newcommand{\Gl}{Glossario}
\newcommand{\glossario}{\textit{\Gl\_v.1.0.0.pdf}}
\newcommand{\AdR}{Analisi dei Requisiti}
\newcommand{\analisiDeiRequisiti}{\textit{\AdR\_v.1.0.0.pdf}}
\newcommand{\AdRvDue}{AnalisiDeiRequisiti}
\newcommand{\NdP}{Norme di Progetto}
\newcommand{\normeDiProgetto}{\textit{\NdP\_v.1.0.0.pdf}}
\newcommand{\PdP}{Piano di Progetto}
\newcommand{\pianoDiProgetto}{\textit{\PdP\_v.1.0.0.pdf}}
\newcommand{\SdF}{Studio di Fattibilità}
\newcommand{\studioDiFattibilita}{\textit{\SdF\_v.1.0.0.pdf}}
\newcommand{\PdQ}{Piano di Qualifica}
\newcommand{\pianoDiQualifica}{\textit{\PdQ\_v.1.0.0.pdf}}
\newcommand{\VI}{Verbale Interno}
\newcommand{\VE}{Verbale Esterno}
\newcommand{\ST}{Specifica Tecnica}
\newcommand{\MU}{Manuale Utente}
\newcommand{\DDP}{Definizione di Prodotto}

%Periodo di progetto
\newcommand{\ARM}{Analisi dei Requisiti di Massima}
\newcommand{\ARD}{Analisi dei Requisiti in Dettaglio}
\newcommand{\PA}{Progettazione Architetturale}
\newcommand{\PD}{Progettazione di Dettaglio}
\newcommand{\COD}{Codifica}
\newcommand{\VV}{Verifica e Validazione Finale}

%Consegne
\newcommand{\RR}{Revisione dei Requisiti}
\newcommand{\RP}{Revisione di Progettazione}
\newcommand{\RQ}{Revisione di Qualifica}
\newcommand{\RA}{Revisione di Accettazione}


%Formattazione
\newcommand{\termine}[1]{\textit{#1}\small{$_G$}}
\newcommand{\link}[1]{\href{#1}{\textcolor{blue}{\texttt{#1}}}} 

% Testi ricorrenti
\newcommand{\scopoProdotto}{L'obiettivo di questo progetto è la realizzazione di un \termine{SDK} che permetta la creazione di bolle interattive, le quali, successivamente, verranno utilizzate all'interno dell'applicazione di messaggistica istantanea open source \termine{Rocket.chat}. \\
Dopo la realizzazione di tale \termine{SDK}, è proposto lo sviluppo di un'applicazione in grado di sfruttare l'\termine{SDK} per implementare un uso originale di tali bolle.
}
\newcommand{\descrizioneGlossario}{Al fine di mantenere questo documento compatto e di facile lettura è stato realizzato un glossario esterno contenente tutte le definizioni dei termini che più comunemente verranno presentati al lettore.  
Tale glossario si ritrova all'interno del file \glossario, e contiene tutti e soli i termini che vengono marcati con una \textit{G} a pedice.
}
\newcommand{\riferimentiNormativi}{
	\begin{itemize}
		\item \textbf{Norme di Progetto}: \normeDiProgetto
		\item \textbf{\termine{Capitolato} d'appalto C5: Monolith - An Interactive bubble provider} \\
			  \link{http://www.math.unipd.it/~tullio/IS-1/2016/Progetto/C5.pdf}
	\end{itemize}
}

% Comandi per generare l'intro
\newcommand{\documento}{BringIt User Manual}
\newcommand{\versione}{2.0.0}
\newcommand{\redatori}{\DC}
\newcommand{\revisori}{\RM}
\newcommand{\dataApprovazione}{20 june 2017}
\newcommand{\approvazione}{\MT}
\newcommand{\statoapprovazione}{Approved}
\newcommand{\uso}{External}
\newcommand{\destinatari}{\RB\\ & \TV\\ & \RC}

\newcommand{\sommario}{This document contains the user manual of \app, the application that has been created in order to show what a developer can do with \progettoShort, a project by \RB}
\usepackage{graphicx}
\usepackage{placeins}
\usepackage{ltablex}
\usepackage{float}
\usepackage{verbatim}


\newcommand{\modifiche}{
	1.0.0 & Approved & \ND & \Pm & 06/05/2017 \\\midrule
	0.1.0 & Verified the whole document & \LD & \ProgrEn & 06/05/2017 \\\midrule
	0.0.3 & Added missing images & \RM & \ProgEn & 02/05/2017 \\\midrule
	0.0.2 & Written the whole manual & \RM & \ProgEn & 30/04/2017 \\\midrule
	0.0.1 & Wrote the template & \RM & \ProgEn & 29/04/2017 \\\midrule
}

\begin{document}


% Questo file contiene il layout della prima pagina
\pagenumbering{gobble}

\title{\includegraphics[width=8cm, keepaspectratio=true]{../../../Template/Logo/Logo.png} \\
	\documento \\
	Version \versione
}
\date{\dataApprovazione}

\maketitle

\begin{center}

\begin{tabular}{ r | l }
  \textbf{Role} & \textbf{Component} \\
  Redaction & \redatori \\
  Revision & \revisori \\
  Approval & \approvazione \\
  \\
  Condition & \statoapprovazione \\
  Usage & \uso \\
  Recipients & \destinatari
\end{tabular}
\end{center}

\begin{center}
\textbf{Summary\\}
\sommario \\
\vspace{1.5cm}\email
\end{center}

\clearpage

\pagenumbering{arabic}
%Questo file si occupa di generare la tabella delle modifiche
\pagenumbering{Roman}

\begin{center}
    \Large{\textbf{Change log}}
    	\\\vspace{0.5cm}
    	\normalsize
    \begin{tabularx}{\textwidth}{cXXcc}
        \textbf{Version} & \textbf{Changes - Motivation} & \textbf{Author} & \textbf{Role} & \textbf{Date} \\\toprule
        \modifiche
    \end{tabularx}
\end{center}

\newpage




% Renames the index to the english version
\renewcommand{\contentsname}{Table of Contents}
\input{../../../Template/Indice.tex}

\renewcommand{\listfigurename}{List of figures}
\renewcommand{\figurename}{Fig.}
\listoffigures
\newpage

\section{Introduction}

\subsection{Purpose of the document}
The purpose of this document is supplying a detailed guide to the user of \app, the demo application that has been created by \gruppo\ to show some of the possible uses of \progetto, the \termine{SDK} that the same group has created. \\
This document is so intended for the application's users, which will use the demo application as-it-is without modifying its code.

\subsection{Purpose of the product}
This project is divided in two parts with different purposes. \\
The first part is an \termine{SDK}, called \progettoShort, which allows a developer to create interactive bubbles easily which have to be able to work inside the \termine{Rocket.chat} environment. \\
The second part is a demo application developed using the above mentioned \progettoShort\ and which uses the provided bubbles. Our application is called \app, and it allows you to create an interactive and sharable to-do or shopping list.

\subsection{Glossary}
To avoid misunderstandings with technical terms of this document, words that require a detailed explanation will be marked with a \textit{G} and then the word will be inserted in the respective section of the glossary.

\newpage
\section{Use requirements}

% Requisiti generali
\subsection{Requirements}
In order to use \app\ the user needs to access the internet either using a laptop or desktop computer, or using a mobile cellphone or tablet. Other than that, the device which will be used to use \app\ must have a browser which supports \termine{JavaScript} and has it enabled.

% Requisiti da desktop o laptop
\subsubsection{Desktop requirements}
To access \app\ using a desktop or laptop computer, one of the following browsers is required:
\begin{itemize}
	\item Microsoft Internet Edge 13 or above;
	\item Mozilla Firefox 45 or above;
	\item Google Chrome 56 or above;
	\item Opera 43 or above;
	\item Apple Safari 43 or above.
\end{itemize}

% Requisiti da mobile
\subsubsection{Mobile requirements}
To access \app\ using a mobile phone or table, one of the following requirement needs to be satisfied:
\begin{itemize}
	\item If the device has Android as its operative system, it needs to have Google Chrome 56 or above installed;
	\item If the device has iOS as its operative system, it needs to have either iOS 10 or above installed, or it needs to have Google Chrome 56 or above as browser.
\end{itemize}

% Sezione su come abilitare JS
\subsubsection{Javascript}
In order to enable \termine{JavaScript} inside the different browsers' versions the next steps must be followed:
% Scrivere come abilitare JavaScript


%Sezione dedicata a come installare i prodotti
\subsection{Installation}
\app\ installation is borne by the supplier.

\subsection{Access to the application}
To access \app\, the only things required are the following:
\begin{enumerate}
	\item Connect to the following \termine{Rocket.chat} server from a device which has access to the 		
		  internet:
			\begin{lstlisting}
			54.135.165.325 
			\end{lstlisting}
	\item Either:
		  \begin{enumerate}
		  	\item Register a new account if you do not have one.
		  		\begin{enumerate}
		  			\item Click on "Register a new account";
		  			\item Fill all the required fields with your information;
		  			\item Click on "Register a new account".
		  		\end{enumerate}
		  		
		  	\item Login with you credentials.
		  		\begin{enumerate}
		  			\item Insert your credentials inside the required fields
		  			\item Click on "Login"
		  		\end{enumerate}
		  \end{enumerate}
\end{enumerate}

Once logged in, you will be able to use \app\ as a fully-implemented feature of our server.

\newpage
\section{Features Guide}
Inside this section there will be some examples of code that show how to use \progettoShort, followed by a guide explaining how to use the \app application.

\subsection{SDK}
This guide is dedicated to a generic developer which wants to use \progettoShort\ to integrate its feature inside his \termine{Meteor.js} project. 

\subsection{Widgets}
\subsubsection{TextWidget}
\begin{lstlisting}[language=JavaScript]
// Create a TextWidget
let textWidget = new Monolith.Widget.TextWidget;

// Hide the widget
textWidget.setVisibility(false); // Default is true, which will show is

// Set the text. Markdown notation is also supported
textWidget.setText("Foo");
textWidget.setText("Markdown __is supported__ **too**");

// Set the text color using HEX notation (http://www.color-hex.com/)
textWidget.setTextColor("#C61A10");

// Set the text size in pixel
textWidget.setTextSize(15);

// Set the URL highlighting color
textWidget.setUrlHighligthColor("#EE42F4");

// Enable or disable the text formatting, this includes also URL highlighting
textWidget.setFormatText(true);
textWidget.setFormatText(false);
\end{lstlisting}
~\\~\\

\subsubsection{ImageWidget}
\begin{lstlisting}[language=JavaScript]
// Create the ImageWidget
let imageWidget = new Monolith.Widget.ImageWidget;

// Hide the widget
imageWidget.setVisibility(false); // Default is true, which will show is

// Set the image associated with the widget
imageWidget.setImage("path/to/image.png");

// Set the image dimensions
imageWidget.setWidth(200);
imageWidget.setHeight(50);
\end{lstlisting}

\newpage
\subsubsection{ButtonWidget}
\begin{lstlisting}[language=JavaScript]
// Create a ButtonWidget
let buttonWidget = new Monolith.Widget.ButtonWidget;

// Set the dimensions of the button
buttonWidget.setWidth(100);
buttonWidget.setHeight(50);

// Set the color of the button
buttonWidget.setBackgroundColor("#41F492");

// Set the action associated with the button
buttonWidget.setOnClickAction(function(){
    alert("The button has been clicked");
});

// Set the action associated with the button when the user long-clicks it
buttonWidget.setOnLongClickAction(function(){
    alert("The button has been long clicked");
});

// Set the milliseconds that need to pass before a click is considered a long click
buttonWidget.setOnLongClickActionTimer(500);
\end{lstlisting}
~\\~\\

\subsubsection{ListWidget}
\begin{lstlisting}[language=JavaScript]
// Create the ListWidget
let listWidget = new Monolith.Widget.ListWidget;

// Add items to the list
listWidget.addItem("First");
listWidget.addItem("Second");
listWidget.addItem("Third");

// Set the indicator of the list
listWidget.setCharacterNumber(); // Numbered list
listWidget.setCharacterCircle(); // Unnumbered list

// Set the indicator color
listWidget. setColor("#292929");
\end{lstlisting}

\newpage
\subsubsection{CheckListItemWidget}
\begin{lstlisting}[language=JavaScript]
// Create a new CheckListItemWidget
let checkListItemWidget = new Monolith.Widget.CheckListItemWidget;

// Set the text associated with the item
checkListItemWidget.setText("Click me!");

// Customize the check appereance
// Color the check box instead of using a check tick
checkListItemWidget.setUseSelectionMark(true); 
// Set the color that will be used to color the check box
checkListItemWidget.setSelectionColor("#AAAAAA"); 
// Use a check tick
checkListItemWidget.setUseSelectionMark(false); 
// Set the character used as check tick
checkListItemWidget.setSelectionCharacter("X"); 

// Check or un-check the option
checkListItemWidget.setChecked(true); // Checked
checkListItemWidget.setChecked(false); // Un-checked

// Know it the option is checked or not
let isChecked = checkListItemWidget.isChecked();
if (isChecked){
    // The option is checked
} else {
    // The option is not checked
}


// Set the action to perform on click
checkListItemWidget.setOnClick(function(item){
    // The item parameter represents the view of the item that has been clicked
    item.setText("New text after click");
});

// Set the action to perform after a long click (1000 ms)
checkListItemWidget.setOnLongClick(function(){
    // The item parameter represents the view of the item that has been clicked
   item.setText("New text after long click");
});


// Delete the item
checkListItemWidget.removeOption();
\end{lstlisting}

\newpage
\subsubsection{Create a custom widget}
In order to create a new custom widget and add it to \termine{Monolith} so than you can use it like the default ones, you have to do as follows.
\begin{enumerate}

	\item Create a new class which extends from \texttt{BaseWidget}
\begin{lstlisting}[language=JavaScript]
export class MyWidget extends Monolith.Widget.BaseWidget {

    constructor(){
        super(); // You need to call this to create the above hierarchy
        
        // Initialize your widget here
    }
    
    renderView(){
        // Renders the view of the widget and returns a DOMElement object
    }

    performOperation(){
        // Perform some operation
    }

}
\end{lstlisting}

	\item Use your widget wherever you want
\begin{lstlisting}[language=JavaScript]
// Import the widget
import {MyWidget} from '/path/to/MyWidget.js';

// Istantiate the widget
let myWidget = new MyWidget();

// Perform operations with the widget
myWidget.performOperation();

// Render the widget's view
myWidget.renderView();
\end{lstlisting}
  
\end{enumerate}  
  
\textbf{Note} \\ 
The default widget's behaviour does \textbf{not} let the user use a single widget without a bubble container that holds it. \\
If you plan to render the widget's view inside a Rocket.chat room, please create a bubble and add your widget to the bubble, so that the bubble will render it and show it to the user.

\newpage


\subsubsection{Layouts}
Inside \progettoShort\ there are two classes which gives the possibility of arranging the different widgets in two main orientations:
\begin{itemize}
	\item \texttt{HorizontaLayout}: it allows the widgets that are added to it to be displayed each one on the right of the previous ones;
	\item \texttt{VerticalLayout}: it allows the widgets to be displayed one below the previous ones.
\end{itemize}

\paragraph{Using a layout}
To use one of the given layout classes, all that needs to be done is the following:

% Immagine layout prima dell'aggiunta di un widget
\begin{lstlisting}[language=JavaScript, frame=single]
// Needed to be able to create the new layout
import {HorizontalLayout} from {BOH}

let widget = new TextWidget();

let layout = new HorizontalLayout();
layout.addComponent(widget);
\end{lstlisting}
% Immagine layout dopo l'aggiunta del widget

\paragraph{Creating a new layout}
In order to create a new layout class, the only thing that needs to be done is creating a new class which extends from \texttt{BaseLayout} and implements its abstract methods:
\begin{lstlisting}[language=JavaScript, frame=single]
// Needed to be able to create the new layout
import {BaseLayout} from {BOH}

export class MyLayout extends BaseLayout {
    constructor(){
        // Needed in order to create the base class instance
        super();
    }
    
    renderView(){
        // Renders the layout view usually calling its components' 
        // renderView() method and returns the HTML code
    }

}
\end{lstlisting}
\subsection{Bubbles}
\subsubsection{MarkdownBubble}
\begin{lstlisting}[language=JavaScript]
// Create the bubble
let markdownBubble = new Monolith.Bubble.MarkdownBubble("Optional text here");

// Change the bubble text
markdownBubble.setText("Text with **markdown** __inside__");

// Render the bubble
markdownBubble.renderView();
\end{lstlisting}
~\\~\\

\subsubsection{AlertBubble}
\begin{lstlisting}[language=JavaScript]
// Create the bubble
let alertBubble = new Monolith.Bubble.AlertBubble;

// Set the alert title
alertBubble.setTitle("Warning");

// Set the alert message
alertBubble.setMessage("Please check your data");

// Build the view
alertBubble.renderView();
\end{lstlisting}

\newpage
\subsubsection{ToDoListBubble}
\begin{lstlisting}[language=JavaScript]
// Create the ToDoListBubble
let toDoListBubble = new Monolith.Bubble.ToDoListBubble;

// --- GENERIC OPERATIONS ---

// Get the id of the bubble
let id = toDoListBubble.getId();

// Set the text associated with the bubble
toDoListBubble.setText("This bubble contains a lot of items that can be checked");

// Set the color of the text
toDoListBubble.setTextColor("#1A5418");

// Tell the bubble to format the text and so support the markdown notation
toDoListBubble.setFormatText(true);

// Disable text formatting
toDoListBubble.setFormatText(false);

// Set the highlight color for URLs
toDoListBubble.setUrlHighlightColor("#891C15");

// Set the bubble's text size in pixel
toDoListBubble.setTextSize(15);

// Set the message to show upon completion 
toDoListBubble.setCompletionMessage("The list has all been checked.")


// --- ITEMS OPERATIONS ---

// Add an item to the list
toDoListBubble.addItem("First item");

// You can specify a second parameter which tells if the item that is being added
// should be initially checked or not
toDoListBubble.addItem("Second item", true); 

// Set the text of an item
toDoListBubble.setItemText("New text", 0); // Changes the text of the first item

// Check an item
toDoListBubble.setChecked(true, 0); // Checks the first item

// Remove an item specifying the index of that item. Indexes start from 0
toDoListBubble.removeItem(1);


// Tell the bubble to use selection marks when ticking the options
toDoListBubble.setUseSelectionMark(true);

// Set the character to use when ticking an option
toDoListBubble.setSelectionCharacter("X");

// Tell the bubble to color the check box when ticking the options
toDoListBubble.setUseSelectionMark(false);

// Set the color used to color the check box
toDoListBubble.setSelectionColor("#1D2565");

// Set the function to call when clicking an item
// This function will be called after the click of any of the items.
// The parameter indicates the view of the item that has been clicked
toDoListBubble.setOnItemClick(function(item){
    item.setText("New text after click");
});

// Set the function to call when long-clicking an item
// This function will be called after the long click of any of the items.
// The parameter indicates the view of the item that has been long clicked
toDoListBubble.setOnItemClick(function(item){
    item.setText("New text after long click");
});

\end{lstlisting}

\newpage
\subsubsection{Create a custom bubble}
In order to create a new bubble type you have to do as follows.

\begin{enumerate}
	\item Create a new class which extends from `Monolith.Bubble.BaseBubble`.
\begin{lstlisting}[language=JavaScript]
export class CustomBubble extends Monolith.Bubble.BaseBubble {
    
    constructor(params){
        super(); // Always remember to call super!
        
        // Do something with the params
        // Setup the bubble
    }
    
    customOperation(){
        // Perform a custom operation
    }
    
    renderView(){
        super.renderView(); // Again, necessary call
        
        // Return a DOMElement object
    }
    
}
\end{lstlisting}

	\item Put the following code wherever you want. Thi can be done even inside the same file that contains the class definition, outside the definition itself.
\begin{lstlisting}[language=JavaScript]
Monolith.Bubble.addBubble("key", function(message){
    // Create the bubble
    let customBubble = new CustomBubble(params);
    
    // Perform the operations you want
    customBubble.customOperation();
    
    // Return the setup bubble
    return customBubble;
});
\end{lstlisting}
Please note that this function takes two parameters:
	\begin{enumerate}
		\item A \texttt{string} which defines the unique key that identifies your custom bubble.   \\
   		A good naming conventions for this key would be using the \url{reverse domain name notation}{https://en.wikipedia.org/wiki/Reverse_domain_name_notation} (e.g. \texttt{com.mycompany.bubble.custom}) which allows you to identify your custom bubble type among all the other bubbles types.
		\item A \texttt{function} which takes as parameter a \texttt{message} object that identifies a \url{Rocket.chat message}{https://rocket.chat/docs/developer-guides/realtime-api/the-message-object/}.  \\
   This function is the one that creates the bubble, taking data from the parameter object, performs operation on it if there's the need, and then return it.
   \end{enumerate}
\end{enumerate}   
   
\textbf{Note} \\
All of the written above should be made \textbf{only} inside the \texttt{client} directory of your \termine{Meteor} project, otherwise it will make your application crash.

\newpage
\subsubsection{Composition of layouts or widgets}
Using the \texttt{BaseLayout}'s or \texttt{BaseBubble}'s \texttt{addComponent(component : BaseComponent)}  method, you can add to a layout or bubble either a widget or a sub-layout. \\

\textbf{Notes}. 
\begin{enumerate}
	\item The default bubble's layout will always be a \texttt{VerticalLayout} and cannot be changed.
	\item The below examples will use a bubble as the containers, but the same code applies also to simple layouts.
\end{enumerate}

\paragraph{Adding a widget}
In order to add a widget, the only thing there's need to do is create the widget instance and then call the \texttt{addComponent} method passing it as parameter.

% Immagine bolla prima del'aggiunta di un widget
\begin{lstlisting}[language=JavaScript, frame=single]
const bubble = new ToDoListBubble();
const widget = new TextWidget();
widget.setText("This is a new widget which has been added to the bubble");
bubble.addComponent(widget);
\end{lstlisting}
% Immagine bolla dopo l'aggiunta del widget

\paragraph{Adding a Layout with some widgets}
To add a layout to a bubble, the method which needs to be follow is the same as the one presented for the widget. The layout needs to be created and then \texttt{addComponent} must be called giving the layout as parameter.

% Immagine bolla prima dell'aggiunta del layout
\begin{lstlisting}[language=JavaScript, frame=single]
// Needed to be able to extend from the BaseBubble class
const bubble = new ToDoListBubble();

const widget = new TextWidget();
widget.setText("First text widget");
const widget 2 = new TextWidget();
widget2.setText("Second text widget");

const layout = new HorizzontalLayout();
layout.addComponent(widget1);
layout.addComponent(widget2);

bubble.addComponent(layout);
\end{lstlisting}
% Immagine della bolla dopo l'aggiunta del layout



\newpage
\section{Glossary}
% Questo file definisce lo stile che verrà applicato
% ad ogni pagina di contenuto
\documentclass[a4paper,11pt]{article}

\usepackage{ifthen}
\usepackage[
 a4paper,
 top=2.5cm,
 bottom=2.5cm,
 left=1.5cm,
 right=1.5cm,
 head=30pt
]{geometry}
\usepackage[italian]{babel}
\usepackage[utf8x]{inputenc}
\usepackage[T1]{fontenc}
\usepackage{fancyhdr}
\usepackage[colorlinks=true, urlcolor=black, citecolor=black, linkcolor=black]{hyperref}
\usepackage{tabularx}
\usepackage{multirow}
\usepackage{booktabs}
\usepackage{color}
\usepackage{graphicx}
\usepackage{eurosym}
\usepackage{amsmath}
\usepackage{relsize}

\usepackage[multidot]{grffile}
\usepackage{xcolor,colortbl}
\definecolor{lightblue}{HTML}{56B4E6}
\definecolor{blue}{HTML}{2953A1}
\definecolor{darkblue}{HTML}{1E396E}

\usepackage[toc,page]{appendix}
\renewcommand\appendixtocname{Appendice}
\renewcommand{\appendixpagename}{Appendice}

\newcommand\pagenumberingnoreset[1]{\gdef\thepage{\csname @#1\endcsname\c@page}}

% Cambia il font 
\renewcommand*\rmdefault{qhv}

% ***STILE PAGINA***
\pagestyle{fancy}
\fancyhf{}
\setlength{\headheight}{1cm} 
% No indentazione paragrafo
\setlength{\parindent}{0pt}

% ***INTESTAZIONE***
\newcommand\textline[4][t]{%
  \noindent\parbox[#1]{.333\textwidth}{\raisebox{-0.40\height}{#2}}%
  \parbox[#1]{.333\textwidth}{\centering #3}%
  \parbox[#1]{.333\textwidth}{\raggedleft #4}%
}

\lhead{
	\textline[t]{\includegraphics[width=1cm, keepaspectratio=true]{../../../Template/Logo/Logo.png}}{\progettoShort}{\documento}
}

\renewcommand{\headrulewidth}{0.4pt}  %Linea sotto l'intestazione

% ***PIÈ DI PAGINA***
\lfoot{\textit{\gruppoLink}\\ \footnotesize{\email}}
\rfoot{\thepage} %per le prime pagine: mostra solo il numero romano
\cfoot{}
\renewcommand{\footrulewidth}{0.4pt}   %Linea sopra il piè di pagina


% Ridefinisce command \paragraph{} andando a capo ogni dopo la parola dentro le parentesi ed ha la possibiltà di enumerazione fino a n cifre modificando il numero dentro "secnumdepth"
\usepackage{titlesec}

\setcounter{secnumdepth}{7}
\setcounter{tocdepth}{7}
%
%
%\titleformat{\paragraph}
%{\normalfont\normalsize\bfseries}{\theparagraph}{1em}{}
%\titlespacing*{\paragraph}
%{0pt}{3.25ex plus 1ex minus .2ex}{1.5ex plus .2ex}
%
%
%\titleclass{\subsubparagraph}{straight}[\subparagraph]
%\newcounter{subsubparagraph}
%
%\titleformat{\subsubparagraph}[display]
%  {\normalfont\normalsize\bf}
%  {\thesubsubparagraph.}
%  {.5em}
%  {}
%\renewcommand\thesubsubparagraph\textbf{\roman{subsubparagraph}}
%\titlespacing*{\subsubparagraph} {0pt}{4pt}{6pt}


%***LA SOTTOSEZIONE PARAGRAPH VIENE VISUALIZZATA COME UNA SECTION
\titleformat{\paragraph}{\normalfont\normalsize\bfseries}{\theparagraph}{1em}{}
\titlespacing*{\paragraph}{0pt}{3.25ex plus 1ex minus .2ex}{1.5ex plus .2ex}

\titleformat{\subparagraph}{\normalfont\normalsize\bfseries}{\thesubparagraph}{1em}{}
\titlespacing*{\subparagraph}{0pt}{3.25ex plus 1ex minus .2ex}{1.5ex plus .2ex}

\makeatletter
\newcounter{subsubparagraph}[subparagraph]
\renewcommand\thesubsubparagraph{%
  \thesubparagraph.\@arabic\c@subsubparagraph}
\newcommand\subsubparagraph{%
  \@startsection{subsubparagraph}    % counter
    {6}                              % level
    {\parindent}                     % indent
    {3.25ex \@plus 1ex \@minus .2ex} % beforeskip
    {0.75em}                           % afterskip
    {\normalfont\normalsize\bfseries}}
\newcommand\l@subsubparagraph{\@dottedtocline{6}{13em}{5.5em}} %gestione dell'indice
\newcommand{\subsubparagraphmark}[1]{}
\makeatother

\makeatletter
\newcounter{subsubsubparagraph}[subsubparagraph]
\renewcommand\thesubsubsubparagraph{%
  \thesubsubparagraph.\@arabic\c@subsubsubparagraph}
\newcommand\subsubsubparagraph{%
  \@startsection{subsubsubparagraph}    % counter
    {7}                              % level
    {\parindent}                     % indent
    {3.25ex \@plus 1ex \@minus .2ex} % beforeskip
    {0.75em}                           % afterskip
    {\normalfont\normalsize\bfseries}}
\newcommand\l@subsubsubparagraph{\@dottedtocline{7}{16em}{6.5em}} %gestione dell'indice
\newcommand{\subsubsubparagraphmark}[1]{}
\makeatother

%Generali
\newcommand{\capitolato}{C5 - Monolith: An interactive bubble provider}
\newcommand{\progettoShort}{Monolith}
\newcommand{\progetto}{Monolith: An interactive bubble provider}
\newcommand{\gruppo}{NPE Developers}
\newcommand{\gruppoLink}{\href{https://gitlab.com/npe-developers}{NpeDevelopers}}
\newcommand{\email}{\href{mailto:npe.developers@gmail.com}{\textcolor{blue}{npe.developers@gmail.com}}}
\newcommand{\password}{NP3Devel0pers}
\newcommand{\myincludegraphics}[2][]{%
	\setbox0=\hbox{\phantom{X}}%
	\vtop{
		\hbox{\phantom{X}}
		\vskip-\ht0
		\hbox{\includegraphics[#1]{#2}}}
}




%Componenti del gruppo
\newcommand{\RM}{Riccardo Montagnin}
\newcommand{\MT}{Manuel Turetta}
\newcommand{\FB}{Francesco Bazzerla}
\newcommand{\SL}{Stefano Lia}
\newcommand{\LD}{Luca Dario}
\newcommand{\DC}{Diego Cavestro}
\newcommand{\ND}{Nicolò Dovico}

%Ruoli
\newcommand{\Pm}{Project Manager}
\newcommand{\Am}{Amministratore}
\newcommand{\AmP}{Amministratori}
\newcommand{\An}{Analista}
\newcommand{\AnP}{Analisti}
\newcommand{\Dev}{Sviluppatore}
\newcommand{\DevP}{Sviluppatori}
\newcommand{\Ver}{Verificatore}
\newcommand{\VerP}{Verificatori}
\newcommand{\Progr}{Programmatore}
\newcommand{\ProgrP}{Programmatori}
\newcommand{\Prog}{Progettista}
\newcommand{\ProgP}{Progettisti}



%Firme
\newcommand{\RMFirma}{\myincludegraphics[scale = 0.5]{../../../Template/Firme/RM.png}}
\newcommand{\MTFirma}{\myincludegraphics[scale = 0.5]{../../../Template/Firme/MT.png}}
\newcommand{\FBFirma}{\myincludegraphics[scale = 0.5]{../../../Template/Firme/FB.png}}
\newcommand{\SLFirma}{\myincludegraphics[scale = 0.5]{../../../Template/Firme/SL.png}}
\newcommand{\LDFirma}{\myincludegraphics[scale = 0.5]{../../../Template/Firme/LD.png}}
\newcommand{\DCFirma}{\myincludegraphics[scale = 0.5]{../../../Template/Firme/DC.png}}
\newcommand{\NDFirma}{\myincludegraphics[scale = 0.5]{../../../Template/Firme/ND.png}}

%Professori e proponente
\newcommand{\TV}{Prof. Tullio Vardanega}
\newcommand{\RC}{Prof. Riccardo Cardin}
\newcommand{\RB}{Red Babel}
\newcommand{\proponente}{Red Babel}

%Documenti
\newcommand{\Gl}{Glossario}
\newcommand{\glossario}{\textit{\Gl\_v.1.0.0.pdf}}
\newcommand{\AdR}{Analisi dei Requisiti}
\newcommand{\analisiDeiRequisiti}{\textit{\AdR\_v.1.0.0.pdf}}
\newcommand{\AdRvDue}{AnalisiDeiRequisiti}
\newcommand{\NdP}{Norme di Progetto}
\newcommand{\normeDiProgetto}{\textit{\NdP\_v.1.0.0.pdf}}
\newcommand{\PdP}{Piano di Progetto}
\newcommand{\pianoDiProgetto}{\textit{\PdP\_v.1.0.0.pdf}}
\newcommand{\SdF}{Studio di Fattibilità}
\newcommand{\studioDiFattibilita}{\textit{\SdF\_v.1.0.0.pdf}}
\newcommand{\PdQ}{Piano di Qualifica}
\newcommand{\pianoDiQualifica}{\textit{\PdQ\_v.1.0.0.pdf}}
\newcommand{\VI}{Verbale Interno}
\newcommand{\VE}{Verbale Esterno}
\newcommand{\ST}{Specifica Tecnica}
\newcommand{\MU}{Manuale Utente}
\newcommand{\DDP}{Definizione di Prodotto}

%Periodo di progetto
\newcommand{\ARM}{Analisi dei Requisiti di Massima}
\newcommand{\ARD}{Analisi dei Requisiti in Dettaglio}
\newcommand{\PA}{Progettazione Architetturale}
\newcommand{\PD}{Progettazione di Dettaglio}
\newcommand{\COD}{Codifica}
\newcommand{\VV}{Verifica e Validazione Finale}

%Consegne
\newcommand{\RR}{Revisione dei Requisiti}
\newcommand{\RP}{Revisione di Progettazione}
\newcommand{\RQ}{Revisione di Qualifica}
\newcommand{\RA}{Revisione di Accettazione}


%Formattazione
\newcommand{\termine}[1]{\textit{#1}\small{$_G$}}
\newcommand{\link}[1]{\href{#1}{\textcolor{blue}{\texttt{#1}}}} 

% Testi ricorrenti
\newcommand{\scopoProdotto}{L'obiettivo di questo progetto è la realizzazione di un \termine{SDK} che permetta la creazione di bolle interattive, le quali, successivamente, verranno utilizzate all'interno dell'applicazione di messaggistica istantanea open source \termine{Rocket.chat}. \\
Dopo la realizzazione di tale \termine{SDK}, è proposto lo sviluppo di un'applicazione in grado di sfruttare l'\termine{SDK} per implementare un uso originale di tali bolle.
}
\newcommand{\descrizioneGlossario}{Al fine di mantenere questo documento compatto e di facile lettura è stato realizzato un glossario esterno contenente tutte le definizioni dei termini che più comunemente verranno presentati al lettore.  
Tale glossario si ritrova all'interno del file \glossario, e contiene tutti e soli i termini che vengono marcati con una \textit{G} a pedice.
}
\newcommand{\riferimentiNormativi}{
	\begin{itemize}
		\item \textbf{Norme di Progetto}: \normeDiProgetto
		\item \textbf{\termine{Capitolato} d'appalto C5: Monolith - An Interactive bubble provider} \\
			  \link{http://www.math.unipd.it/~tullio/IS-1/2016/Progetto/C5.pdf}
	\end{itemize}
}

% Comandi per generare l'intro
\newcommand{\documento}{\Gl}
\newcommand{\versione}{1.0.1}
\newcommand{\redatori}{\MT}
\newcommand{\revisori}{\DC}
\newcommand{\approvazione}{\LD}
\newcommand{\statoapprovazione}{Da approvare}
% Quando il documento sarà approvato, inserire all'interno del comando seguente la data nel formato GG mese AAAA dove GG è il giorno a due cifre, mese è il mese scritto per esteso con la prima lettera minuscola, e AAAA è l'anno a quattro cifre
\newcommand{\dataApprovazione}{}
\newcommand{\uso}{Esterno}
\newcommand{\destinatari}{\TV\\ & \RC\\ & \RB}

\newcommand{\sommario}{Questo documento si prefigge di chiarire le possibili ambiguità tra i vari termini utilizzati all'interno dei documenti redatti dal gruppo \gruppo.
}

\newcommand{\modifiche}{
	Verifica sezioni 1 e 4 & \RM & \Prog & 06/03/2017 & 0.1.0 \\\midrule
	Stesura appendice 1 & \RM & \Prog & 03/03/2017 & 0.0.5 \\\midrule
	Stesura sezione 4 & \DC & \Prog & 28/02/2017 & 0.0.4 \\\midrule
	Stesura sezione 4 & \FB & \Prog & 28/02/2017 & 0.0.3 \\\midrule
	Stesura sezione 1 & \FB & \Prog & 28/02/2017 & 0.0.2 \\\midrule
    Creazione del template & \FB & \Prog & 28/02/2017 & 0.0.1 \\\midrule
}

\begin{document}

\input{../../../Template/Intro.tex}
%Questo file si occupa di generare la tabella delle modifiche
\pagenumbering{Roman}

\begin{center}
    \Large{\textbf{Registro delle modifiche}}
    	\\\vspace{0.5cm}
    	\normalsize
    \begin{tabularx}{\textwidth}{cXXcc}
        \textbf{Versione} & \textbf{Modifica - Motivazione} & \textbf{Autore} & \textbf{Ruolo} & \textbf{Data} \\\toprule
        \modifiche
    \end{tabularx}
\end{center}

\newpage



\input{../../../Template/Indice.tex}

\section*{1}
\addcontentsline{toc}{section}{1}
\begin{itemize}
	\item
	\textbf{12 Factors App Guidelines}: Insieme di \termine{Best Practices} per lo sviluppo di \termine{web app}.
\end{itemize}
\newpage
\section*{A}
\addcontentsline{toc}{section}{A}
\begin{itemize}
	\item
	\textbf{Accoppiamento Afferente}: Numero di classi esterne ad un \termine{package} che dipendono da una classe interna ad esso.
	\item
	\textbf{Accoppiamento Efferente}: Numero di classi interne ad un \termine{package} che dipendono da una classe esterna ad esso.
	\item
	\textbf{Ambienti Di Sviluppo}: Insieme di software atti a suportare il programmatore durante la scrittura del codice sorgente di un software.
	\item
	\textbf{Analisi}: Metodo di indagine basato sulla scomposizione di ciò che si presenta unitario nei suoi elementi costitutivi.
	\item
	\textbf{Api}: Insieme di procedure disponibili al programmatore, di solito raggruppate a formare un set di strumenti specifici per l'espletamento di un determinato compito all'interno di un certo programma.
	\item
	\textbf{Atmosphere}: Sito web da cui è possibile cercare librerie compatibili con il \termine{framework} \termine{Meteor.js}.
\end{itemize}
\newpage
\section{B}
\begin{itemize}
	\item
	\textbf{Best Practice}: Insieme delle attività che, organizzate in modo sistematico, possono essere prese come riferimento e riprodotte per favorire il raggiungimento dei risultati migliori.
	\item
	\textbf{Bitbucket}: Servizio di hosting per progetti software.
	\item
	\textbf{Bolla}: Ogni messaggio presente all'interno di \termine{Monolith} e, più in generale di \termine{Rocket.chat}, che l'utente può visualizzare all'interno di una qualsiasi chat.
	\item
	\textbf{Bolle Interattive}: Messagio in grado di cambiare il proprio contenuto dopo essere stato inviato.
	\item
	\textbf{Bot}: Programma che accede alla rete attraverso lo stesso tipo di canali utilizzati dagli utenti umani, e che nel nostro caso invia messaggi all'interno di una chat offrendo servizi utili agli utenti.
	\item
	\textbf{Bottone Semplice}: Oggetto che genera eventi alla sua pressione.
	\item
	\textbf{Business}: Termine inglese che identifica in generale un'attività economica e approssimativamente può essere tradotto con il termine italiano \textit{affari}.
\end{itemize}
\newpage
\section{C}
\begin{itemize}
	\item
	\textbf{Callback}: In programmazione, un callback (o, in italiano, richiamo) è, in genere, una funzione, o un "blocco di codice" che viene passata come parametro ad un'altra funzione.
	\item
	\textbf{Camel Case}: Pratica di scrivere parole composte o frasi unendo le parole tra loro ma lasciando le loro iniziali maiuscole es: (Ciao mondo = CiaoMondo).
	\item
	\textbf{Capability Maturity Model}: Modello di riferimento costituito da pratiche consolidate in una disciplina specifica.
	\item
	\textbf{Capability Maturity Model Integration -- Cmmi}: Approccio al miglioramento dei processi il cui obiettivo è di aiutare un'organizzazione a migliorare le sue prestazioni.
	\item
	\textbf{Capitolato}: Atto allegato a un contratto d'appalto che intercorre tra il cliente ed una ditta appaltatrice in cui vengono indicate modalità, costi e tempi di realizzazione dell'opera oggetto del contratto.
	\item
	\textbf{Caso D'Uso}: Tecnica usata nei processi di ingegneria del software per effettuare in maniera esaustiva e non ambigua, la raccolta dei requisiti al fine di produrre software di qualità.
	\item
	\textbf{Chai}: E' una libreria per verificare che lo stato di una variabile o oggetto sia quello desiderato.
	\item
	\textbf{Checkbutton}: Componenete che ha due stati: selezionato e non selezionato.
	\item
	\textbf{Ciclo Di Deming}: Metodo di gestione in quattro fasi, utilizzato in attività per il controllo e il miglioramento continuo dei processi e dei prodotti.
	\item
	\textbf{Codifica}: Periodo durante il quale uno o più programmatori scrivono del codice (\textit{codificano}) al fine di creare un prodotto software.
	\item
	\textbf{Commento A Blocco}: Pratica che consente di scrivere un commento separandolo su più linee.
	\item
	\textbf{Commit}: Comando del software \termine{git} attraverso il quale è possibile salvare localmente una o più modifiche eseguite all'interno di uno o più file, spesso allegando un messaggio identificativo di tali modifiche.
	\item
	\textbf{Committente}: Figura che commissiona un lavoro, indipendentemente dall'entità o dall'importo.
	\item
	\textbf{Complessità Ciclomatica}: Metrica software utilizzata per misurare la complessità di un programma.
	\item
	\textbf{Conformità}: Situazione in cui il software prodotto risulta essere coerente con gli obiettivi stabiliti in precedenza alla sua realizzazione.
	\item
	\textbf{Css}: \textit{eng. Cascading Style Sheets}, in italiano \textit{fogli di stile a cascata}. Linguaggio usato per definire la formattazione di documenti HTML, XHTML e XML ad esempio i siti web e relative pagine web.
	\item
	\textbf{Css3}: Versione 3 del linguaggio \termine{CSS}.
	\item
	\textbf{Csshint}: Software per rendere lo stile di scrittura dei documenti css conforme alle regole stabilite dal gruppo.
\end{itemize}
\newpage
\section{D}
\begin{itemize}
	\item
	\textbf{Dbms}: \textit{eng. Database Management System}. Sistema software progettato per consentire la creazione, la manipolazione e l'interrogazione efficiente di database.
	\item
	\textbf{Debugger}: Programma/software specificatamente progettato per l'analisi e l'eliminazione dei bug (\textit{debugging}), ovvero errori di programmazione interni al codice di altri programmi.
	\item
	\textbf{Default}: Stato o risposta di un sistema qualunque in assenza (per difetto, cioè in mancanza) di interventi espliciti (ad esempio input o configurazioni dell'utente), ovverosia \textit{predefinito}.
	\item
	\textbf{Demo}: Campione dimostrativo della produzione dei programmatori. Specificatamente, risultato dell'utilizzo dell'\termine{SDK} che ne mostra alcune (se non tutte) le funzionalità.
	\item
	\textbf{Design-Pattern}: Soluzione progettuale generale ad un problema ricorrente.
	\item
	\textbf{Diagramma Di Gantt}: Strumento di supporto alla gestione dei progetti si rappresenta sotto forma di un grafico avente un asse orizzontale in cui vi è la rappresentazione dell'arco temporale totale del progetto, suddiviso in fasi incrementali (ad esempio, giorni, settimane, mesi) e da un asse verticale che mostra la rappresentazione delle mansioni o attività che costituiscono il progetto.
\end{itemize}
\newpage
\section{E}
\begin{itemize}
	\item
	\textbf{Editing Bubble}: Bolla avente contenuto modificabile.
	\item
	\textbf{Efficacia}: Capacità di produrre l'effetto e i risultati voluti o sperati.
	\item
	\textbf{Efficienza}: Capacità di azione o di produzione con il minimo di scarto, di spesa, di risorse e di tempo impiegati.
\end{itemize}
\newpage
\section*{F}
\addcontentsline{toc}{section}{F}
\begin{itemize}
	\item
	\textbf{Font}: \textit{eng. Tipo di carattere}. Insieme di caratteri tipografici caratterizzati e accomunati da un certo stile grafico o intesi per svolgere una data funzione.
	\item
	\textbf{Framework}: Un framework in informatica e specificatamente nello sviluppo software è un'architettura logica di supporto su cui un software può essere progettato e realizzato, spesso facilitandone lo sviluppo da parte del programmatore.
	\item
	\textbf{Framework Frontend}: \termine{Framework} atto allo sviluppo di una interfaccia grafica.
\end{itemize}
\newpage
\section*{G}
\addcontentsline{toc}{section}{G}
\begin{itemize}
	\item
	\textbf{Ganttchart}: \textit{eng. Diagramma di Gantt} Strumento di supporto alla gestione dei progetti usato principalmente nelle attività di project management e costruito partendo da un asse orizzontale -- che rappresenta dell'arco temporale totale del progetto, suddiviso in fasi incrementali (ad esempio, giorni, settimane, mesi) -- e da un asse verticale -- che rappresenta delle mansioni o attività che costituiscono il progetto.
	\item
	\textbf{Gesture}: Gesto a cui è associata una azione.
	\item
	\textbf{Git}: Software di controllo di versione distribuito utilizzabile da interfaccia a riga di comando.
	\item
	\textbf{Github}: Servizio di hosting per progetti software.
	\item
	\textbf{Gitlab}: Servizio di hosting per progetti software.
	\item
	\textbf{Google Drive}: Servizio web di archiviazione di file.
	\item
	\textbf{Grafo Di Controllo Di Flusso}: Grafo che mostra il flusso di esecuzione di un programma, usato per calcolare la \termine{Complessità Ciclomatica}.
	\item
	\textbf{Gruppo}: Insieme delle persone fisiche Diego Cavestro, Francesco Bazzerla, Luca Dario, Manuel Turetta, Nicolò Dovico, Stefano Lia e Riccardo Montagnin.
	\item
	\textbf{Gummi}: Editor per LaTeX disponibile per Linux e Windows.
\end{itemize}
\newpage
\section*{H}
\addcontentsline{toc}{section}{H}
\begin{itemize}
	\item
	\textbf{Heroku}: Heroku è un Platform as a service (PaaS) su cloud che supporta diversi linguaggi di programmazione. Un PaaS è un'attività economica che consiste nel servizio di messa a disposizione di piattaforme di elaborazione (Computing platform) e di solution stack. Gli elementi del PaaS permettono di sviluppare, sottoporre a test, implementare e gestire le applicazioni aziendali senza i costi e la complessità associati all'acquisto, alla configurazione, all'ottimizzazione e alla gestione dell'hardware e del software di base.
	\item
	\textbf{Hosting}: Un servizio di rete che consiste nell'allocare su un server web le pagine web di un sito web o un'applicazione web, rendendolo così accessibile dalla rete Internet e ai suoi utenti.
	\item
	\textbf{Html}: \textit{eng. HyperText Markup Language, linguaggio a marcatori per ipertesti}. Linguaggio di \termine{markup} solitamente usato per la formattazione e impaginazione di documenti ipertestuali disponibili nel World Wide Web sotto forma di pagine web, nato assieme al web 1.0.
	\item
	\textbf{Html5}: Versione 5 di \termine{HTML}, pubblicato come W3C Recommendation da ottobre 2014.
\end{itemize}
\newpage
\section*{I}
\addcontentsline{toc}{section}{I}
\begin{itemize}
	\item
	\textbf{Ide}: \textit{eng. Integrated Development Environment, ambiente di sviluppo integrato}. Software che, in fase di programmazione, aiuta i programmatori nello sviluppo del codice sorgente di un programma.
	\item
	\textbf{Indice Di Manutenibilità}: Valore che indica quanto il codice risulta manutenibile.
	\item
	\textbf{Inline}: Mantenere una sequenza di caratteri nella stessa riga.
	\item
	\textbf{Inspection}: \textit{eng. Ispezione}.
	\item
	\textbf{Instant Messaging}: E' una categoria di sistemi di comunicazione in tempo reale in rete, tipicamente Internet o una rete locale, che permette ai suoi utilizzatori lo scambio di messaggi.
	\item
	\textbf{Intellij Idea}: Un ambiente di sviluppo integrato \termine{IDE}.
	\item
	\textbf{Inversion Of Control}: Pattern per cui un componente di livello applicativo riceve il controllo da un componente appartenente a un libreria riusabile.
	\item
	\textbf{Iso}: \textit{eng. International Organization for Standardization, Organizzazione internazionale per la normazione}. La più importante organizzazione a livello mondiale per la definizione di norme tecniche.
	\item
	\textbf{Issue}: \textit{eng. Problema}. All'interno del mondo \termine{Git}, una \termine{issue} è una metodologia utilizzata per far presente agli sviluppatori al lavoro su una determinata \termine{repository} che vi è un problema con il codice scritto in essa.
\end{itemize}
\newpage
\section*{J}
\addcontentsline{toc}{section}{J}
\begin{itemize}
	\item
	\textbf{Javascript}: Linguaggio di scripting orientato agli oggetti e agli eventi, comunemente utilizzato nella programmazione Web lato client per la creazione, in siti web e applicazioni web, di effetti dinamici interattivi tramite funzioni di script invocate da eventi innescati a loro volta in vari modi dall'utente sulla pagina web in uso.
	\item
	\textbf{Jenkins}: Strumento open source di continuous integration, scritto in linguaggio Java e che fornisce dei servizi di integrazione continua per lo sviluppo del software.
	\item
	\textbf{Jolie}: Linguaggio di programmazione orientato ai micro servizi.
	\item
	\textbf{Jquery}: Libreria JavaScript per applicazioni web. Nasce con l'obiettivo di semplificare la selezione, la manipolazione, la gestione degli eventi e l'animazione di elementi DOM in pagine HTML, nonché per implementare funzionalità AJAX.
	\item
	\textbf{Jshint}: Strumento di analisi statica del codice volto al linguaggio \termine{JavaScript}.
\end{itemize}
\newpage
\section{L}
\begin{itemize}
	\item
	\textbf{Latex}: \termine{Linguaggio di markup} usato per la preparazione di testi basato sul programma di composizione tipografica \LaTeX.
	\item
	\textbf{Librerie}: Insieme di utility software.
	\item
	\textbf{Linguaggio Di Markup}: E' un insieme di regole che descrivono i meccanismi di rappresentazione (strutturali, semantici o presentazionali) di un testo che, utilizzando convenzioni standardizzate, sono utilizzabili su più supporti.
	\item
	\textbf{Log}: File o insieme di file su cui vengono memorizzate sequenzialmente e cronologicamente delle operazioni effettuate.
\end{itemize}
\newpage
\section*{M}
\addcontentsline{toc}{section}{M}
\begin{itemize}
	\item
	\textbf{Manuale Utente}: Documento destinato all'utilizzatore finale del prodotto, contenente la documentazione del prodotto.
	\item
	\textbf{Markdown}: Linguaggio di markup con una sintassi del testo semplice progettata in modo che possa essere convertita in HTML e in molti altri formati usando uno strumento omonimo.
	\item
	\textbf{Markup}: Sequenza di caratteri con cui si marcano gli elementi di un file di testo per assegnare loro determinate caratteristiche o funzioni.
	\item
	\textbf{Meteor.Js}: \termine{Framework} per creare web app usando il linguaggio di programmazione \termine{javascript}.
	\item
	\textbf{Microservizio}: Parte di un sistema che abbraccia l'approccio di implementazione a micro-servizi, ha la caratteristica di essere indipendente dagli altri moduli in modo tale da riuscire a sviluppare in modo agile sistemi distribuiti.
	\item
	\textbf{Milestone}: \textit{eng. Pietra miliare}. Indica importanti traguardi intermedi nello svolgimento del progetto.
	\item
	\textbf{Mocha}: \termine{Framework} per eseguire test automatici su codice \termine{Javascript}.
	\item
	\textbf{Mock}: Serve per indicare che una o più componenti software sono state simulate durante l'esecuzione di test automatici.
	\item
	\textbf{Model-View-Controller}: Pattern architetturale che gestisce direttamente i dati, la logica e le regole dell'applicazione.
	\item
	\textbf{Model-View-Presenter}: Pattern architetturale che permette la separazione concettuale tra parte logica e grafica di una applicazione, mediante l'utilizzo di una classe intermedia tra le due.
	\item
	\textbf{Modello Incrementale}: Per modello incrementale o modello iterativo si intende, nell'ambito dell'ingegneria informatica, un modello di sviluppo di un progetto software basato sulla successione dei seguenti passi principali: pianificazione, analisi dei requisiti, progetto, implementazione, prove, valutazione.
	\item
	\textbf{Mongodb}: \termine{DBMS} non relazionale orientato ai documenti. Classificato come un database di tipo NoSQL, \termine{MongoDB} si allontana dalla struttura tradizionale basata su tabelle dei database relazionali in favore di documenti in stile JSON con schema dinamico (BSON), rendendo l'integrazione di dati di alcuni tipi di applicazioni più facile e veloce.
	\item
	\textbf{Monolith}: \termine{SDK} che permette agli sviluppatori che operano su piattaforma \termine{Rocket.chat} di creare varie tipologie di bolle differenti oppure di utilizzare quelle preesistenti.
\end{itemize}
\newpage
\section*{O}
\addcontentsline{toc}{section}{O}
\begin{itemize}
	\item
	\textbf{Open Source}: Software per il quale viene reso pubblico anche il suo codice sorgente.
	\item
	\textbf{Organigramma}: Rappresentazione grafica di una struttura organizzativa corrente o in un certo momento storico.
\end{itemize}
\newpage
\section*{P}
\addcontentsline{toc}{section}{P}
\begin{itemize}
	\item
	\textbf{Pacchetto Stand-Alone}: Pacchetto che non dipende da librerie esterne.
	\item
	\textbf{Package}: \textit{eng. Pacchetto}. Collezione di classi e interfacce correlate.
	\item
	\textbf{Package Management}: Collezione di strumenti che automatizzano il processo di installazione, aggiornamento, configurazione e rimozione dei pacchetti software.
	\item
	\textbf{Package Manager}: Software che gestisce e organizza dei pacchetti software.
	\item
	\textbf{Pascalcase}: Tecnica tipografica che prevede la scrittura di parole composte come una unica parola all'interno delle quali l'iniziale di ciascuna parola inizia con una lettera maiuscola.
	\item
	\textbf{Pdca}: \textit{eng. Plan Do Check Act, sin. Ciclo di Deming}. Metodo di gestione in quattro fasi iterativo, utilizzato in attività per il controllo e il miglioramento continuo dei processi e dei prodotti.
	\item
	\textbf{Pdf}: \textit{eng. Portable Document Format, Formato di documenti portatile}. Formato di file basato su un linguaggio di descrizione di pagina sviluppato da Adobe Systems nel 1993, per rappresentare documenti in modo indipendente dall'hardware e dal software utilizzati per generarli o per visualizzarli.
	\item
	\textbf{Plugin}: Programma non autonomo che interagisce con un altro programma per ampliarne o estenderne le funzionalità originarie.
	\item
	\textbf{Png}: \textit{eng. Portable Network Graphics}. Formato di file per memorizzare immagini.
	\item
	\textbf{Prodotto}: Nel campo del project management il prodotto rilasciato, indicato solitamente con il termine \textit{deliverable} nella letteratura tecnologica, indica un oggetto materiale o immateriale realizzato (fornito/consegnato) come risultato di una attività del progetto.
	\item
	\textbf{Progettazione Architetturale}: Progettazione "ad altissimo livello", in cui si definisce solo la struttura complessiva del sistema in termini dei principali moduli di cui esso è composto e delle relazioni macroscopiche fra di essi.
	\item
	\textbf{Progettazione Di Dettaglio}: Rappresenta una descrizione del sistema molto vicina alla codifica, ovvero che la vincola in maniera sostanziale.
	\item
	\textbf{Promise Centric Approach}: E un approccio centrato sul uso delle promesse, cioè oggetti che rappresentano il risultato di una chiamata di funzione asincrona, cioe una promessa che un risultato verrà fornito non appena disponibile.
	\item
	\textbf{Proponente}: Persona fisica che presenta una idea di progetto da realizzare.
	\item
	\textbf{Pull}: \textit{eng. Tirare}. Atto con il quale, nel mondo \termine{Git}, viene indicata l'azione di prelevare uno o più \termine{commit} da una \termine{repository} remota.
	\item
	\textbf{Push}: \textit{eng. Spingere}. Atto con il quale, nel mondo \termine{Git}, viene indicata l'azione di caricare uno o più \termine{commit} su una \termine{repository} remota.
\end{itemize}
\newpage
\section*{Q}
\addcontentsline{toc}{section}{Q}
\begin{itemize}
	\item
	\textbf{Quality Assurance}: Insieme delle iniziative e delle attività rivolte a predefinire e a garantire che un prodotto oppure un servizio corrisponda a prestabiliti requisiti di qualità.
	\item
	\textbf{Qualità Di Affidabilità}: Insieme delle iniziative e delle attività rivolte ad assicurare la capacità di rispettare le specifiche tecniche ed il loro pieno soddisfacimento.
	\item
	\textbf{Qualità Di Efficienza}: Insieme delle iniziative e delle attività rivolte ad assicurare un utilizzo minimo di risorse di spesa e di tempo impiegato.
	\item
	\textbf{Qualità Di Manutenibilità}: Insieme delle iniziative e delle attività rivolte ad assicurare una buona manutenibilità nel tempo del software.
	\item
	\textbf{Qualità Di Usabilità}: Insieme delle iniziative e delle attività rivolte ad assicurare un facile uso del software.
	\item
	\textbf{Qualità Funzionale}: Insieme delle iniziative e delle attività rivolte ad assicurare usabilità, affidabilità e performance del software.
\end{itemize}
\newpage
\section*{R}
\addcontentsline{toc}{section}{R}
\begin{itemize}
	\item
	\textbf{Repository}: Ambiente di un sistema informativo (ad es. di tipo ERP), in cui vengono gestiti i metadati, attraverso tabelle relazionali; l'insieme di tabelle, regole e motori di calcolo tramite cui si gestiscono i metadati prende il nome di \textit{metabase}. Nell'ambiente \termine{Git} viene indicato con questo termine il luogo dove è possibile salvare in remoto del codice, eseguire dei \termine{pull}, dei \termine{push} e aprire delle \termine{issue}.
	\item
	\textbf{Rest}: \textit{eng. REpresentational State Transfer}. Interfaccia che usa il protocollo HTTP per invio, ricezione ed eliminazione dei dati.
	\item
	\textbf{Rich Media Bubble}: \termine{Bolla} che mostra delle anteprime mediante immagini di contenuti multimediali.
	\item
	\textbf{Rocket.Chat}: Applicazione di \termine{Instant Messaging}.
\end{itemize}
\newpage
\subsection*{S}
\begin{itemize}
	\item
	\textbf{Sdk}: A software development kit (SDK or devkit) is typically a set of software development tools that allows the creation of applications for a certain software package, software framework, hardware platform, computer system, video game console, operating system, or similar development platform..
\end{itemize}
\newpage
\section{T}
\begin{itemize}
	\item
	\textbf{Tab}: \textit{abbr. Tabulazione}.
	\item
	\textbf{Tailoring}: Significa adattare i requisiti e le specifiche di un progetto alle attuali esigenze operative di un'organizzazione attraverso la revisione, la modifica e l'integrazione di nuovi dati al progetto.
	\item
	\textbf{Task}: \textit{eng. Compito}.
	\item
	\textbf{Team}: \textit{eng. \termine{Gruppo}}.
	\item
	\textbf{Telegram}: Servizio di messaggistica istantanea basato su cloud ed erogato senza fini di lucro dalla società \textit{Telegram LLC}.
	\item
	\textbf{Template}: Documento o programma nel quale, come in un foglio semicompilato cartaceo, su una struttura generica o standard esistono spazi temporaneamente "bianchi" da riempire successivamente.
	\item
	\textbf{Tempo Di Slack}: Periodo di tempo durante il quale un'attività può essere ritardata senza ritardare l'intero progetto di cui fa parte.
	\item
	\textbf{Testing}: Fase del ciclo di vita di un software all'interno della quale il codice prodotto dalla fase di programmazione viene sottoposto a dei controllo che ne assicurano l'\termine{efficacia} e l'\termine{efficienza}.
	\item
	\textbf{Twitter Bootstrap}: \termine{Framework} per sviluppare interfacce web compatibili con i dispositivi mobili.
\end{itemize}
\newpage
\section*{U}
\addcontentsline{toc}{section}{U}
\begin{itemize}
	\item
	\textbf{Uml}: \textit{eng. Unified Modeling Language, Linguaggio di modellizzazione unificato}. Linguaggio di modellizzazione e specifica basato sul paradigma orientato agli oggetti.
	\item
	\textbf{Uniformità}: Costanza di aspetto o di comportamento.
	\item
	\textbf{Upper\textunderscore Case}: Tecnica che consiste nel scrivere ogni lettera di una frase in maiuscolo rimpiazzando gli spazi con un underscore.
	\item
	\textbf{Use Case}: Tecnica usata nei processi di ingegneria del software per effettuare in maniera esaustiva e non ambigua, la raccolta dei requisiti al fine di produrre software di qualità.
\end{itemize}
\newpage
\section*{V}
\addcontentsline{toc}{section}{V}
\begin{itemize}
	\item
	\textbf{Validazione}: Fase del ciclo di vita di un prodotto software, all'interno della quale si vuole andare a confermare la sua \termine{efficacia}.
	\item
	\textbf{Verifica}: Fase del ciclo di vita di un prodotto software, all'interno della quale si vuole andare a confermare la sua \termine{efficienza} e correttezza di esecuzione.
	\item
	\textbf{View}: Parte grafica di un'applicazione.
	\item
	\textbf{Visual Paradigm}: Software per la realizzazione di diagrammi \termine{UML}.
	\item
	\textbf{Void}: Tipo di ritorno di una funzione. Segnala che la funzione in questione non ritorna alcun tipo di parametro.
	\item
	\textbf{Vue.Js}: Framework JavaScript che facilità l'implementazione del Model-View-Presenter.
\end{itemize}
\newpage
\subsection*{W}
\begin{itemize}
	\item
	\textbf{Widget}: Inside Monolith, a widget is everything that can be inserted inside a bubble, such as a button, a text view, a list, or other custom widget that the developer can create..
\end{itemize}
\newpage


\end{document}

\end{document}
