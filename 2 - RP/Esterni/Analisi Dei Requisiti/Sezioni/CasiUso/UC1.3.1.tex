\paragraph{Caso d'uso UC 1.3.1: Personalizzazione dell'elenco puntato}
\label{UC 1.3.1: Personalizzare l'elenco puntato}
\begin{figure}[ht]
	\centering
	\includegraphics{Usecases/img/{UC1.3.1}.png}
	\caption{Caso d'uso UC 1.3.1: Personalizzazione dell'elenco puntato}
\end{figure}

\FloatBarrier
\begin{itemize}
\item\textbf{Attori}: Sviluppatore.
\item\textbf{Descrizione}: Lo sviluppatore vuole personalizzare lo stile di visualizzazione della checklist.
\item\textbf{Precondizione}: Lo sviluppatore utilizza una bolla di tipo checklist e vuole personalizzarne lo stile di visualizzazione della checklist.
\item\textbf{Postcondizione}: Lo sviluppatore ha personalizzato lo stile di visualizzazione della checklist con i parametri scelti.
\item\textbf{Scenario principale}:
	\begin{itemize}
		\item Lo sviluppatore vuole scegliere una visualizzazione come elenco numerato (UC 1.3.1.1).
		\item Lo sviluppatore vuole scegliere una visualizzazione come elenco non numerato con indicatore "pallino" (UC 1.3.1.2).
		\item Lo sviluppatore vuole scegliere una visualizzazione come elenco non numerato con indicatore "trattino" (UC 1.3.1.3).
	\end{itemize}
\end{itemize}
