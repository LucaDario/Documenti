\section{Processi di supporto}
\subsection{Processo di documentazione}
In questa sezione verranno descritti tutti i metodi ed i procedimenti utilizzati per registrare le informazioni prodotte dal ciclo di vita di un processo o di una attività.

\subsubsection{Descrizione}
Tratteremo di seguito tutte le convenzioni adottate dal \termine{team} per produrre i documenti dove registrare le informazioni sopra dette, e tutti i metodi per standardizzare la loro  stesura, verifica e approvazione. \\
I documenti vengono classificati come:
\begin{itemize}
  \item \textit{Interni} se il loro utilizzo rimane interno al \termine{team};
  \item \textit{Esterni}: se la loro distribuzione avviene anche alle componenti esterne al gruppo, come ad esempio il committente e/o il proponente.
\end{itemize}

I documenti \textit{approvati} dal \textit{\Pm} devono avere un nome strutturato nel seguente modo:
\begin{itemize}
  \item La prima lettera del documento deve essere maiuscola;
  \item Il nome non deve contenere spazi e deve essere scritto in \termine{camel case} per identificare; 
  \item La versione del documento deve essere indicata nella parte finale del nome in forma numerica
  \begin{center}
  \textit{NomeDelDocumento\_v.1.0.0}
  \end{center}\
\end{itemize}
Inoltre nel documento deve essere specificato il suo scopo (facilmente comprensibile dal suo titolo) e  nel diario delle modifiche dovranno essere elencati i soggetti che hanno preso parte alla stesura, modifica, verifica e approvazione del documento stesso.

\subsubsection{Strumenti}
Per la stesura dell'intera documentazione è stato scelto di utilizzare \textit{\termine{LaTeX}} in quanto questo \termine{linguaggio di markup} permette di avere uno
standard comune ed evitare possibili conflitti e incompatibilità derivanti
dall'utilizzo di software differente.

\subsubsection{Struttura del documento}
\paragraph{Prima pagina}
Ogni documento deve avere nella prima pagina le seguenti informazioni:
\begin{itemize}
  \item Nome del gruppo;
  \item Logo del progetto;
  \item Nome del progetto;
  \item Nome del documento;
  \item Versione del documento;
  \item Data di creazione del documento;
  \item Data di ultima modifica del documento;
  \item Stato del documento;
  \item Nome e cognome del redattore del documento;
  \item Nome e cognome del verificatore del documento;
  \item Nome e cognome del responsabile approvatore del documento;
  \item Uso del documento;
  %\item Lista di distribuzione del documento;
  \item Destinatari del documento;
  \item Un sommario contenente una breve descrizione del documento.
\end{itemize}

\paragraph{Diario delle modifiche}
La seconda pagina di ogni documento deve contenere il diario delle modifiche. In questa tabella vengono inserite tutte le modifiche effettuate dai vari redattori del documento. Ogni
riga della tabella deve contenere le seguenti informazioni:
\begin{itemize}
  \item \textbf{Descrizione}: descrizione della modifica apportata;
  \item \textbf{Autore e Ruolo}: autore della modifica e ruolo che esso ricopre;
  \item \textbf{Data}: data della modifica apportata;
  \item \textbf{Versione}: versione del documento dopo la modifica.
\end{itemize}

\paragraph{Indice}
In ogni documento, dopo il diario delle modifiche, deve essere presente un
indice di tutte le sezioni. In presenza di tabelle e/o immagini queste devono
essere indicate con i relativi codici identificati.

\paragraph{Formattazione generale delle pagine}
La formattazione della pagina, oltre al contenuto, prevede un'intestazione e un
piè di pagina. \\
L'intestazione della pagina contiene:
\begin{itemize}
  \item Nome del progetto;
  \item Nome del documento;
  \item Logo del gruppo.
\end{itemize}
Il piè di pagina contiene:
\begin{itemize}
  \item Il nome del gruppo;
  \item Email del gruppo;
  \item Numero della pagina corrente.
\end{itemize}

\subsubsection{Norme tipografiche}
Le seguenti norme tipografiche indicano i criteri riguardanti
l'ortografia e la tipografia di tutti i documenti.

\paragraph{Stili di testo}
\begin{itemize}
  \item \textbf{Grassetto}: Il grassetto deve essere utilizzato per evidenziare parole
  particolarmente importanti, negli elenchi puntati o nelle frasi;
  \item \textbf{Corsivo}: Il corsivo deve essere utilizzato nelle seguenti
  situazioni:
  \begin{itemize}
    \item Ruoli: ogni riferimento a ruoli di progetto va scritto in corsivo;
    \item Documenti: ogni riferimento a un documento va scritto in corsivo;
    \item Stati del documento: ogni stato del documento va scritto in corsivo;
    \item Citazioni: ogni citazione va scritta in corsivo;
    \item Glossario: ogni parola presente nel glossario, oltre ad avere un pedice, deve
    essere scritta in corsivo.
  \end{itemize}
\end{itemize}

\paragraph{Composizione del testo}
\begin{itemize}
   \item \textbf{Glossario}: il pedice \termine{} verrà utilizzato in corrispondenza di vocaboli presenti nel \textit{\glossario}.
\end{itemize}

\paragraph{Formati}
\begin{itemize}
   \item \textbf{Date}: le date presenti nei documenti devono seguire lo standard seguente:
   \begin{center}
     GG-MM-AA
   \end{center}
   dove:
   \begin{itemize}
   		\item GG rappresenta il giorno.
     	\item MM rappresenta il mese;
     	\item AA rappresenta l'anno;
   \end{itemize}
   \item \textbf{Ore}: le ore presenti nei documenti devono seguire lo standard seguente (a 24 ore):
   \begin{center}
     hh:mm
   \end{center}
   dove:
   \begin{itemize}
     \item hh: rappresentano le ore;
     \item mm: rappresentano i minuti.
   \end{itemize}
   \item \textbf{Nome del documento}: per riferirsi al nome del documento si
   dovrà utilizzare il comando \termine{LaTeX} \\ \verb|\Nome del documento| garantendo in questo modo la corretta sintassi;
   \item \textbf{Nome del gruppo}: per riferirsi al nome del gruppo si dovrà
   utilizzare il comando \termine{LaTeX} \verb|\gruppo| garantendo in questo modo la corretta sintassi;
   \item \textbf{Nome del progetto}: per riferirsi al nome del progetto si dovrà
   utilizzare il comando \termine{LaTeX} \verb|\progetto| garantendo in questo modo la corretta sintassi;
   \item \textbf{Link sito del gruppo}: per riferirsi al link del sito del gruppo si dovrà
   utilizzare il comando \termine{LaTeX} \verb|\gruppoLink| garantendo in questo modo la corretta sintassi;
   \item \textbf{Email del gruppo}: per riferirsi all'indirizzo email del gruppo si dovrà
   utilizzare il comando \termine{LaTeX} \verb|\email| garantendo in questo modo la corretta sintassi;
   \item \textbf{Nome del proponente}: per riferirsi al nome del proponente, ovvero \proponente, si dovrà
   utilizzare il comando \termine{LaTeX}  \verb|\proponente| garantendo in questo modo la corretta
   sintassi.
\end{itemize}

\subsubsection{Componenti grafiche}
\paragraph{Tabelle}
Tutte le tabelle presenti all'interno del documento devono avere una didascalia
ed un codice identificativo univoco.

\paragraph{Immagini}
Le immagini inserite nel documento devono essere in formato \termine{PNG}.

\subsubsection{Composizione email}
Di seguito vengono descritte le norme da applicare nella composizione delle email.

\paragraph{Destinatario}
\begin{itemize}
  \item Interno: l'indirizzo da utilizzare è \email;
  \item Esterno: l'indirizzo del destinatario varia a seconda si tratti del Prof. Tullio Vardanega \\ (\link{tullio.vardanega@math.unipd.it}), del Prof. Riccardo Cardin (\link{rcardin@math.unipd.it}) o i proponenti del progetto.
\end{itemize}

\paragraph{Mittente}
\begin{itemize}
  \item Interno: l'indirizzo è di colui che scrive la email;
  \item Esterno: l'indirizzo da utilizzare è \email e può usarlo unicamente il \Pm.
\end{itemize}

\paragraph{Oggetto}
L'oggetto della mail deve essere esplicito, comprensibile e conciso in modo da rendere
immediato il riconoscimento del contenuto.

\paragraph{Corpo}
Il testo del corpo della mail deve essere esaustivo. All'interno del
testo ci si potrà riferire a un ruolo preciso all'interno del team o a un destinatario univoco.

\paragraph{Allegati}
È sconsigliato l'invio di allegati tramite mail in quanto è preferibile infatti condividere file all'interno del gruppo tramite strumenti più adatti, come \termine{Google Drive}.

\subsubsection{Produzione di documenti}
I documenti prodotti come sopra detto verranno innanzitutto creati con il linguaggio di markup \termine{LaTeX}, e successivamente salvati in formato \termine{PDF}. I tipi di documento prodotti sono i seguenti.

\paragraph{\SdF}
Questo documento riporta le motivazioni che hanno portato l'intero
\termine{team} all'accettazione dello sviluppo del progetto scelto. E' un documento interno e la lista di distribuzione riguarda i committenti.

\paragraph{\NdP}
Questo documento riporta le convenzioni, strumenti e
norme che il \termine{team} ha scelto di adottare durante  lo sviluppo del progetto. Questo
documento è interno e la lista di distribuzione riguarda i committenti.

\paragraph{\PdP}
Questo documento descrive come il \termine{team} gestisce le risorse
temporali e umane a disposizione e come sono stati valutati e trattati i rischi. Questo documento è esterno e la lista di distribuzione comprende sia i committenti che i proponenti.

\paragraph{\PdQ}
Questo documento descrive in che maniera l'intero \termine{team} cerca di soddisfare gli obiettivi di qualità del progetto. Questo documento è esterno e la lista di distribuzione comprende sia i committenti che i proponenti.

\paragraph{\AdR}
Questo documento dà una visione di insieme dei casi d'uso del progetto e dei requisiti stabiliti da esso. Contiene inoltre i diagrammi delle attività di interazione tra utente e sistema sviluppato e i servizi che il progetto una volta concluso andrà ad erogare. È un documento esterno e la lista di distribuzione comprende committenti che i proponenti.

\paragraph{\ST}
L'intento del suddetto documento è dare una visione generale del prodotto e delle sue richieste. Più nello specifico verrà mostrata una progettazione ad alto livello, basata su diagrammi dei \termine{package} per la descrizione delle componenti, diagrammi di sequenza per descrivere gli scenari d'uso e diagrammi di attività. Verranno inoltre elencati i software che saranno utilizzati per la realizzazione del progetto. Questo è un documento esterno e la sua lista di distribuzione comprende  sia i committenti che i proponenti.

\paragraph{\DDP}
L'intento di tale documento è fornire una progettazione dettagliata del
prodotto. In esso verranno forniti tutti i dettagli implementativi del prodotto, fondamentali in fase di codifica, che comprenderanno diagrammi \termine{UML} delle classi e i relativi metodi. Questo è un documento esterno e la sua lista di distribuzione comprende  sia i committenti che i proponenti.

\paragraph{\glossario}
Questo documento viene prodotto al fine di dare una spiegazione dettagliata dei termini tecnici e degli acronimi acronimi utilizzati nell'intera documentazione. La sua lista di distribuzione comprende i committenti che i proponenti ed è quindi un documento esterno.

\paragraph{\MU}
Questo documento fornisce all'utente una guida delle funzionalità e dei servizi offerti dal prodotto. Essendo una guida destinata all'utente è un documento esterno e la sua lista di distribuzione comprende i sopracitati utenti.

\paragraph{Verbale}
Questo documento ha lo scopo di riassumere in modo formale le discussioni effettuate e le decisioni prese durante le riunioni. I verbali, come i documenti, sono classificati in:
\textbf{interni} ed \textbf{esterni}. In particolare i verbali esterni, essendo ufficiali, devono essere redatti dal \Pm. \\
Ogni verbale dovrà essere denominato nel seguente modo:
\begin{center}
  \textit{{Verbale}\_{Numero del verbale}\_{Tipo di verbale}\_{Data del verbale}}
\end{center}
dove:
\begin{itemize}
  \item \textbf{Numero del verbale} è numero identificativo univoco del verbale;
  \item \textbf{Tipo di verbale} identifica se si tratta di un \textit{\VI} oppure di un \textit{\VE};
  \item \textbf{Data del verbale} identifica la data nella quale si è svolta la
  riunione corrispondente al verbale. Il formato è il seguente :
  \begin{center}
  YYYY-MM-DD
  \end{center}
\end{itemize}
Cosi come per i documenti anche per i verbali si è cercato di standardizzare la loro struttura, e pertanto la  parte introduttiva deve essere correlata con le seguenti informazioni:
\begin{itemize}
  \item \textbf{Data incontro}: data in cui si è svolta la riunione;
  \item \textbf{Ora inizio incontro}: orario di inizio della riunione;
  \item \textbf{Ora termine incontro}: orario di terminazione della riunione;
  \item \textbf{Luogo incontro}: luogo in cui si è svolta la riunione;
  \item \textbf{Durata}: durata della riunione;
  \item \textbf{Oggetto}: argomento della riunione;
  \item \textbf{Segretario}: cognome e nome del membro incaricato a redigere il
  verbale;
  \item \textbf{Partecipanti}: cognome e nome di tutti i membri partecipanti
  alla riunione.
\end{itemize}

\subsubsection{Ciclo di vita di un documento}
 I documenti possono avere 3 stati diversi:
\begin{itemize}
  \item Documenti \textbf{in lavorazione};
  \item Documenti \textbf{da verificare};
  \item Documenti \textbf{approvati}.
\end{itemize}
I documenti \textbf{in lavorazione} sono quelli in fase di stesura da parte del relativo redattore. Ultimata la loro realizzazione questi documenti vengono segnati come \textbf{da verificare} e passano in mano al relativo \Ver. Infine i documenti \textbf{verificati} vengono consegnati al \Pm che avrà il compito di approvarli definitivamente.

\subsubsection{Versionamento}
Ogni documento prodotto deve essere identificato, oltre che dal nome, dal numero
di versione nel seguente modo:
\begin{center}
  \_v.X.Y.Z
\end{center}
dove:

\begin{itemize}
  \item \textbf{X}: indica il numero di uscite formali del documento e viene
  incrementato in seguito all'approvazione finale da parte del \textit{\Pm}.
  L'incremento dell'indice \textbf{X} comporta l'azzeramento degli indici
  \textbf{Y} e \textbf{Z};
  \item \textbf{Y}: indica il numero crescente delle verifiche. L'incremento viene 	eseguito dal \textit{\Ver} e comporta l'azzeramento dell'indice \textbf{Z};
  \item \textbf{Z}: indica il numero di modifiche minori apportate al documento
  prima della sua verifica e viene aumentato ad ogni modifica apportata.
  \end{itemize}
A ogni modifica del documento anche il nome del file fisico deve essere
modificato, seguendo lo stesso schema precedentemente proposto:
\begin{center}
  nomeDocumento\_v.X.Y.Z.pdf
\end{center}

\subsection{Configurazione e gestione dei processi}
\subsubsection{Implementazione dei processi}
È stato steso un piano generale per l'implementazione dei processi. Questo piano si divide in 5 punti fondamentali: RR (Revisione dei Requisiti), RP (Revisione di Progetto), RQ (Revisione di Qualifica), RA (Revisione di Accettazione). \\
Per ogni punto fondamentale ci sono diversi task assegnati ai diversi componenti del \termine{team} e delle milestone da raggiungere. Per assegnare i task e determinare le date di ciascuna \termine{milestone} è stato usato un strumento di pianificazione, \termine{Wrike}, come descritto in dettaglio nella sezione 1.3.4 di questo documento.

\subsubsection{Gestione delle configurazioni dei processi software}
Per controllare le versioni e le modifiche applicate al software viene utilizzata una \termine{repository} su \termine{GitHub}. All'interno di essa ogni partecipante può verificare l'integrità e la correttezza delle modifiche applicate. Inoltre, per ogni modifica effettuata, viene fatto un \termine{commit} con lo scopo di tenere traccia di ogni variazione effettuata nel ciclo di sviluppo. Ogni volta che deve venire accettata una modifica bisogna effettuare un 	\termine{push} del software modificato che risiede nella \termine{repository} (con il relativo \termine{commit}), e prima di essere accettato saranno presenti dei testi di integrazione automatici sviluppati con i framework \termine{SonarQube} e \termine{Jenkins} implementati nella \termine{repository} di \termine{GitHub} tramite \termine{webhook}.

\subsection{Garanzia di qualità del processo}
Per garantire la qualità del prodotto vengono adottato varie metodologie tra le quali ritroviamo i processi di verifica e di validazione successivamente illustrati. \\
Nel caso in cui i task assegnati al team non siano soddisfatti per eventuali problematiche, questi devono essere riportati nella lista di problemi menzionata nel \textit{processo di soluzione di problemi}. \\
Un ulteriore garanzia di qualità è data dall'assicurarsi che i requisiti principali del progetto siano soddisfatti alla fine dello sviluppo del software. Ci si può assicurare di questo attraverso una documentazione valida ed un incontro con i committenti ed i proponenti che dovranno accertarsi della validità del prodotto finale. \\
Infine l'organizzazione più naturale dei contenuti di questo documento è per processi (quelli adottati e adattati a partire dallo standard ISO/IEC 12207).


\subsection{Processo di verifica}
\subsubsection{Descrizione}
Il processo di verifica ha il compito di controllare che ogni documento e software prodotto dal \termine{team} non presenti eventuali anomalie o errori e che esso rispecchi quanto previsto dai requisiti ad esso associati. In generale esistono due tipi di analisi nel mondo informatico, l'analisi statica e quella dinamica.

\subsubsection{Analisi}
\paragraph{Analisi statica}
L'analisi statica può essere impiegata in due modi diversi:
\begin{itemize}
  \item \textbf{\termine{Walkthrough}}: questa tecnica  consiste nella
  lettura a largo spettro del documento o del codice, al fine di trovare all'interno di esso eventuali anomalie. Essa non viene applicata per trovare un errore preciso ma risulta
  molto utile nella fase iniziale dello sviluppo del prodotto data la scarsa
  esperienza dei membri del \termine{team}. Questa tecnica è utilizzata dai \textit{verificatori}
  che avranno il compito di stilare una lista contenente gli errori rilevati più
  spesso al interno dei documenti. Una volta che la lista ha preso una certa consistenza essa sarà allegata al relativo documento, e sarà così possibile passare alla tecnica di \termine{inspection};
  \item \textbf{\termine{Inspection}}: questa tecnica consiste in una
  lettura molto più dettagliata e più mirata dei documenti (o del codice) rispetto alla tecnica precedente, ed utilizza come supporto la lista di controllo creata tramite \termine{walkthrough}.
\end{itemize}

\paragraph{Analisi dinamica}
L'analisi dinamica viene applicata solamente al software prodotto in quanto consiste nell'esecuzione di test mirati a verificare la correttezza del comportamento dello stesso software.

\subsubsection{Test}
Posiamo individuare delle tipologie standard di test.

\paragraph{Test di unità}
I test di unità verificano che ogni singola componente del software funzioni correttamente. Effettuando questi test di riduce al minimo la presenza di errori di tutte le componenti di base. I test di unità sono identificati dalla seguente sintassi:
\begin{center}
  TU[Codice Test]
\end{center}

\paragraph{Test di integrazione}
I test di integrazione verificano che più unità, validate singolarmente, funzionino
correttamente una volta assemblate. I test di integrazione sono identificati dalla seguente sintassi:
\begin{center}
  TI[Codice Test]
\end{center}

\paragraph{Test di sistema}
I test di sistema vengono eseguiti sul prodotto che si ritiene essere giunto ad
una versione definitiva e serve a verificare che tutti i requisiti siano soddisfatti. I test di sistema sono identificati dalla seguente sintassi:
\begin{center}
  TS[Codice Requisito]
\end{center}

\paragraph{Test di regressione}
I test di regressione consistono nella riesecuzione di tutti i test relativi ad una
componente del software prodotto che ha subito delle modifiche. I test di regressione sono identificati dalla seguente sintassi:
\begin{center}
  TR[Codice Test]
\end{center}


\subsubsection{Strumenti per l'analisi statica}
\begin{itemize}
  \item \textbf{\termine{SonarQube}}: strumento che esegue una serie di ispezioni sul codice in modo statico, andando a misurare un insieme di parametri per determinare poi se il codice sorgente soddisfa o meno i requisiti di qualità scelti dal \termine{team}. Tra i vari parametri misurati si ritrovano:
  \begin{itemize}
    \item \textbf{\termine{Complessità ciclomatica}}: misura la complessità delle classi, dei metodi e delle funzioni del programma;
    \item Rapporto linee di commento su linee di codice: calcola il rapporto tra
    le righe di codice e le righe di commento scritte;
    \item Dipendenze: restituisce le dipendenze interne o esterne con altre
    classi;
    \item \textbf{\termine{Indice di manutenibilità}}: calcola un valore che indica quanto il
    codice risulta manutenibile.
    \item E molti altri.
  \end{itemize}
\end{itemize}

\subsubsection{Strumenti per l'analisi dinamica}
\begin{itemize}
  \item \textbf{\termine{Jenkins}}: software che permette l'integrazione di vari altri software per l'analisi statica e dinamica, fornendo un luogo dove lo sviluppatore può controllare facilmente il report dei vari test eseguiti;
  \item \textbf{\termine{Mocha}}: \termine{framework} che permette di eseguire test di unità e di integrazione andando a creare dei \termine{mock} che simulino il comportamento delle varie parti, con lo scopo di testare ogni dettaglio in modo indipendente dal resto;
  \item \textbf{\termine{Chai}}: libreria per formulare asserzioni sull'output atteso;
  \item \textbf{\termine{Sinon}}: libreria per simulare delle risposte di richieste a server remoti.

\end{itemize}

\subsection{Processi di validazione}
Per assicurarsi la validità del codice scritto, colui che assegna i task deve assicurarsi, una volta conclusi, il pieno soddisfacimento dei requisiti che esso proponeva. \\
I metodi usati per un'adeguata validazione finale vengono trattati più approfonditamente all'interno del documento \pianoDiQualifica.

\subsubsection{Test di validazione}
Il test di validazione coincide con il collaudo del software in presenza del \textit{proponente} e determina, in caso di esito positivo, il rilascio del software. I test di validazione sono identificati dalla seguente sintassi:
\begin{center}
  TV[Codice requisito]
\end{center}

\subsection{Processi per la risoluzione dei problemi}
Per ogni problema trovato all'interno del progetto (sia esso di qualunque natura e importanza), esso viene inserito all'interno di un documento non ufficiale presente nella \termine{repository} di documenti. \\
All'interno del documento sarà presente uno schema concettuale per dividere i problemi a seconda della loro priorità e natura. Il problema dovrà essere descritto in maniera precisa e chiara, inoltre sarà presente un campo dove verrà inserito se il problema in questione è in uno dei seguenti tre stati
\begin{itemize}
	\item \textit{Identificato}: il problema è stato trovato ma non è ancora stato preso in carica da nessun componente del \termine{team};
	\item\textit{In lavorazione}: il problema è stato già analizzato ed è stato preso in carica da uno più componenti del \termine{team} per la sua risoluzione; una volta che entra in questo stato devono essere indicati anche i componenti del \termine{team} che stanno lavorando su di esso;
	\item \textit{Risolto}: il problema è già stato risolto.
\end{itemize}

\newpage