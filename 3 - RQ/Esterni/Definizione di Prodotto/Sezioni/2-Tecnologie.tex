\section{Tecnologie utilizzate}

\subsection{Vue.js}
\textit{Vue.js} è un \termine{framework} per lo sviluppo di interfacce utente. Esso verrà utilizzato per aggiornare le view non appena verranno fatte modifiche nel model e viceversa. In altre parole tale \termine{framework} verrà utilizzato per la gestione della comunicazione tra le componenti del pattern \termine{Model View Presenter}.

\subsection{Bootstrap}

\subsection{Meteor}
Meteor è una piattaforma Javascript per lo sviluppo di applicazioni web e mobile. Essa viene utilizzata come ambiente di sviluppo per l'applicazione e per l'\termine{SDK} che verranno integrate come pacchetti stand-alone in \termine{Rocket.Chat}.
Meteor permette un'ottima gestione delle componenti permettendo lo sviluppo sia la parte server, la parte client e la parte per la comunicazione tra esse in un unico progetto. Per fare ciò Meteor include un set di tecnologie da utilizzare, tra le quali: Node.js, MongoDB e tutte quelle disponibili all'interno del \termine{package manager} \termine{Atmosphere}.
\subsection{MongoDB}
MongoDB è un \termine{DBMS} non relazionale, orientato ai documenti. Classificato come un database di tipo NoSQL, si allontana dalla
struttura tradizionale basata su tabelle dei database relazionali in favore di documenti in stile
JSON con schema dinamico. MongoDB è utilizzato come \termine{storage} da \termine{Rocket.Chat}, così come per qualsiasi progetto meteor. Dunque, il suo utilizzo è derivato principalmente per questo motivo.
