% Questo file definisce lo stile che verrà applicato
% ad ogni pagina di contenuto
\documentclass[a4paper,11pt]{article}

\usepackage{ifthen}
\usepackage[
 a4paper,
 top=2.5cm,
 bottom=2.5cm,
 left=1.5cm,
 right=1.5cm,
 head=30pt
]{geometry}
\usepackage[italian]{babel}
\usepackage[utf8x]{inputenc}
\usepackage[T1]{fontenc}
\usepackage{fancyhdr}
\usepackage[colorlinks=true, urlcolor=black, citecolor=black, linkcolor=black]{hyperref}
\usepackage{tabularx}
\usepackage{multirow}
\usepackage{booktabs}
\usepackage{color}
\usepackage{graphicx}
\usepackage{eurosym}
\usepackage{amsmath}
\usepackage{relsize}

\usepackage[multidot]{grffile}
\usepackage{xcolor,colortbl}
\definecolor{lightblue}{HTML}{56B4E6}
\definecolor{blue}{HTML}{2953A1}
\definecolor{darkblue}{HTML}{1E396E}

\usepackage[toc,page]{appendix}
\renewcommand\appendixtocname{Appendice}
\renewcommand{\appendixpagename}{Appendice}

\newcommand\pagenumberingnoreset[1]{\gdef\thepage{\csname @#1\endcsname\c@page}}

% Cambia il font 
\renewcommand*\rmdefault{qhv}

% ***STILE PAGINA***
\pagestyle{fancy}
\fancyhf{}
\setlength{\headheight}{1cm} 
% No indentazione paragrafo
\setlength{\parindent}{0pt}

% ***INTESTAZIONE***
\newcommand\textline[4][t]{%
  \noindent\parbox[#1]{.333\textwidth}{\raisebox{-0.40\height}{#2}}%
  \parbox[#1]{.333\textwidth}{\centering #3}%
  \parbox[#1]{.333\textwidth}{\raggedleft #4}%
}

\lhead{
	\textline[t]{\includegraphics[width=1cm, keepaspectratio=true]{../../../Template/Logo/Logo.png}}{\progettoShort}{\documento}
}

\renewcommand{\headrulewidth}{0.4pt}  %Linea sotto l'intestazione

% ***PIÈ DI PAGINA***
\lfoot{\textit{\gruppoLink}\\ \footnotesize{\email}}
\rfoot{\thepage} %per le prime pagine: mostra solo il numero romano
\cfoot{}
\renewcommand{\footrulewidth}{0.4pt}   %Linea sopra il piè di pagina


% Ridefinisce command \paragraph{} andando a capo ogni dopo la parola dentro le parentesi ed ha la possibiltà di enumerazione fino a n cifre modificando il numero dentro "secnumdepth"
\usepackage{titlesec}

\setcounter{secnumdepth}{7}
\setcounter{tocdepth}{7}
%
%
%\titleformat{\paragraph}
%{\normalfont\normalsize\bfseries}{\theparagraph}{1em}{}
%\titlespacing*{\paragraph}
%{0pt}{3.25ex plus 1ex minus .2ex}{1.5ex plus .2ex}
%
%
%\titleclass{\subsubparagraph}{straight}[\subparagraph]
%\newcounter{subsubparagraph}
%
%\titleformat{\subsubparagraph}[display]
%  {\normalfont\normalsize\bf}
%  {\thesubsubparagraph.}
%  {.5em}
%  {}
%\renewcommand\thesubsubparagraph\textbf{\roman{subsubparagraph}}
%\titlespacing*{\subsubparagraph} {0pt}{4pt}{6pt}


%***LA SOTTOSEZIONE PARAGRAPH VIENE VISUALIZZATA COME UNA SECTION
\titleformat{\paragraph}{\normalfont\normalsize\bfseries}{\theparagraph}{1em}{}
\titlespacing*{\paragraph}{0pt}{3.25ex plus 1ex minus .2ex}{1.5ex plus .2ex}

\titleformat{\subparagraph}{\normalfont\normalsize\bfseries}{\thesubparagraph}{1em}{}
\titlespacing*{\subparagraph}{0pt}{3.25ex plus 1ex minus .2ex}{1.5ex plus .2ex}

\makeatletter
\newcounter{subsubparagraph}[subparagraph]
\renewcommand\thesubsubparagraph{%
  \thesubparagraph.\@arabic\c@subsubparagraph}
\newcommand\subsubparagraph{%
  \@startsection{subsubparagraph}    % counter
    {6}                              % level
    {\parindent}                     % indent
    {3.25ex \@plus 1ex \@minus .2ex} % beforeskip
    {0.75em}                           % afterskip
    {\normalfont\normalsize\bfseries}}
\newcommand\l@subsubparagraph{\@dottedtocline{6}{13em}{5.5em}} %gestione dell'indice
\newcommand{\subsubparagraphmark}[1]{}
\makeatother

\makeatletter
\newcounter{subsubsubparagraph}[subsubparagraph]
\renewcommand\thesubsubsubparagraph{%
  \thesubsubparagraph.\@arabic\c@subsubsubparagraph}
\newcommand\subsubsubparagraph{%
  \@startsection{subsubsubparagraph}    % counter
    {7}                              % level
    {\parindent}                     % indent
    {3.25ex \@plus 1ex \@minus .2ex} % beforeskip
    {0.75em}                           % afterskip
    {\normalfont\normalsize\bfseries}}
\newcommand\l@subsubsubparagraph{\@dottedtocline{7}{16em}{6.5em}} %gestione dell'indice
\newcommand{\subsubsubparagraphmark}[1]{}
\makeatother

%Generali
\newcommand{\capitolato}{C5 - Monolith: An interactive bubble provider}
\newcommand{\progettoShort}{Monolith}
\newcommand{\progetto}{Monolith: An interactive bubble provider}
\newcommand{\gruppo}{NPE Developers}
\newcommand{\gruppoLink}{\href{https://gitlab.com/npe-developers}{NpeDevelopers}}
\newcommand{\email}{\href{mailto:npe.developers@gmail.com}{\textcolor{blue}{npe.developers@gmail.com}}}
\newcommand{\password}{NP3Devel0pers}
\newcommand{\myincludegraphics}[2][]{%
	\setbox0=\hbox{\phantom{X}}%
	\vtop{
		\hbox{\phantom{X}}
		\vskip-\ht0
		\hbox{\includegraphics[#1]{#2}}}
}




%Componenti del gruppo
\newcommand{\RM}{Riccardo Montagnin}
\newcommand{\MT}{Manuel Turetta}
\newcommand{\FB}{Francesco Bazzerla}
\newcommand{\SL}{Stefano Lia}
\newcommand{\LD}{Luca Dario}
\newcommand{\DC}{Diego Cavestro}
\newcommand{\ND}{Nicolò Dovico}

%Ruoli
\newcommand{\Pm}{Project Manager}
\newcommand{\Am}{Amministratore}
\newcommand{\AmP}{Amministratori}
\newcommand{\An}{Analista}
\newcommand{\AnP}{Analisti}
\newcommand{\Dev}{Sviluppatore}
\newcommand{\DevP}{Sviluppatori}
\newcommand{\Ver}{Verificatore}
\newcommand{\VerP}{Verificatori}
\newcommand{\Progr}{Programmatore}
\newcommand{\ProgrP}{Programmatori}
\newcommand{\Prog}{Progettista}
\newcommand{\ProgP}{Progettisti}



%Firme
\newcommand{\RMFirma}{\myincludegraphics[scale = 0.5]{../../../Template/Firme/RM.png}}
\newcommand{\MTFirma}{\myincludegraphics[scale = 0.5]{../../../Template/Firme/MT.png}}
\newcommand{\FBFirma}{\myincludegraphics[scale = 0.5]{../../../Template/Firme/FB.png}}
\newcommand{\SLFirma}{\myincludegraphics[scale = 0.5]{../../../Template/Firme/SL.png}}
\newcommand{\LDFirma}{\myincludegraphics[scale = 0.5]{../../../Template/Firme/LD.png}}
\newcommand{\DCFirma}{\myincludegraphics[scale = 0.5]{../../../Template/Firme/DC.png}}
\newcommand{\NDFirma}{\myincludegraphics[scale = 0.5]{../../../Template/Firme/ND.png}}

%Professori e proponente
\newcommand{\TV}{Prof. Tullio Vardanega}
\newcommand{\RC}{Prof. Riccardo Cardin}
\newcommand{\RB}{Red Babel}
\newcommand{\proponente}{Red Babel}

%Documenti
\newcommand{\Gl}{Glossario}
\newcommand{\glossario}{\textit{\Gl\_v.1.0.0.pdf}}
\newcommand{\AdR}{Analisi dei Requisiti}
\newcommand{\analisiDeiRequisiti}{\textit{\AdR\_v.1.0.0.pdf}}
\newcommand{\AdRvDue}{AnalisiDeiRequisiti}
\newcommand{\NdP}{Norme di Progetto}
\newcommand{\normeDiProgetto}{\textit{\NdP\_v.1.0.0.pdf}}
\newcommand{\PdP}{Piano di Progetto}
\newcommand{\pianoDiProgetto}{\textit{\PdP\_v.1.0.0.pdf}}
\newcommand{\SdF}{Studio di Fattibilità}
\newcommand{\studioDiFattibilita}{\textit{\SdF\_v.1.0.0.pdf}}
\newcommand{\PdQ}{Piano di Qualifica}
\newcommand{\pianoDiQualifica}{\textit{\PdQ\_v.1.0.0.pdf}}
\newcommand{\VI}{Verbale Interno}
\newcommand{\VE}{Verbale Esterno}
\newcommand{\ST}{Specifica Tecnica}
\newcommand{\MU}{Manuale Utente}
\newcommand{\DDP}{Definizione di Prodotto}

%Periodo di progetto
\newcommand{\ARM}{Analisi dei Requisiti di Massima}
\newcommand{\ARD}{Analisi dei Requisiti in Dettaglio}
\newcommand{\PA}{Progettazione Architetturale}
\newcommand{\PD}{Progettazione di Dettaglio}
\newcommand{\COD}{Codifica}
\newcommand{\VV}{Verifica e Validazione Finale}

%Consegne
\newcommand{\RR}{Revisione dei Requisiti}
\newcommand{\RP}{Revisione di Progettazione}
\newcommand{\RQ}{Revisione di Qualifica}
\newcommand{\RA}{Revisione di Accettazione}


%Formattazione
\newcommand{\termine}[1]{\textit{#1}\small{$_G$}}
\newcommand{\link}[1]{\href{#1}{\textcolor{blue}{\texttt{#1}}}} 

% Testi ricorrenti
\newcommand{\scopoProdotto}{L'obiettivo di questo progetto è la realizzazione di un \termine{SDK} che permetta la creazione di bolle interattive, le quali, successivamente, verranno utilizzate all'interno dell'applicazione di messaggistica istantanea open source \termine{Rocket.chat}. \\
Dopo la realizzazione di tale \termine{SDK}, è proposto lo sviluppo di un'applicazione in grado di sfruttare l'\termine{SDK} per implementare un uso originale di tali bolle.
}
\newcommand{\descrizioneGlossario}{Al fine di mantenere questo documento compatto e di facile lettura è stato realizzato un glossario esterno contenente tutte le definizioni dei termini che più comunemente verranno presentati al lettore.  
Tale glossario si ritrova all'interno del file \glossario, e contiene tutti e soli i termini che vengono marcati con una \textit{G} a pedice.
}
\newcommand{\riferimentiNormativi}{
	\begin{itemize}
		\item \textbf{Norme di Progetto}: \normeDiProgetto
		\item \textbf{\termine{Capitolato} d'appalto C5: Monolith - An Interactive bubble provider} \\
			  \link{http://www.math.unipd.it/~tullio/IS-1/2016/Progetto/C5.pdf}
	\end{itemize}
}

% Comandi per generare l'intro
\newcommand{\documento}{User Manual}
\newcommand{\versione}{0.1.2}
\newcommand{\redatori}{\RM}
\newcommand{\revisori}{\DC}
\newcommand{\dataApprovazione}{05 may 2017}
\newcommand{\approvazione}{\ND}
\newcommand{\statoapprovazione}{Non approvato}
\newcommand{\uso}{Esterno}
\newcommand{\destinatari}{\RB\\ & \TV\\ & \RC}

\newcommand{\sommario}{This document describes the User Manual for the project \progettoShort by \RB
}
\usepackage{graphicx}
\usepackage{placeins}
\usepackage{ltablex}
\usepackage{float}
\usepackage{verbatim}


\newcommand{\modifiche}{
	1.0.0 & Approved & \ND & \Pm & 06/05/2017 \\\midrule
	0.1.0 & Verified the whole document & \LD & \ProgrEn & 06/05/2017 \\\midrule
	0.0.3 & Added missing images & \RM & \ProgEn & 02/05/2017 \\\midrule
	0.0.2 & Written the whole manual & \RM & \ProgEn & 30/04/2017 \\\midrule
	0.0.1 & Wrote the template & \RM & \ProgEn & 29/04/2017 \\\midrule
}

\begin{document}


% Questo file contiene il layout della prima pagina
\pagenumbering{gobble}

\title{\includegraphics[width=8cm, keepaspectratio=true]{../../../Template/Logo/Logo.png} \\
	\documento \\
	Version \versione
}
\date{\dataApprovazione}

\maketitle

\begin{center}

\begin{tabular}{ r | l }
  \textbf{Role} & \textbf{Component} \\
  Redaction & \redatori \\
  Revision & \revisori \\
  Approval & \approvazione \\
  \\
  Condition & \statoapprovazione \\
  Usage & \uso \\
  Recipients & \destinatari
\end{tabular}
\end{center}

\begin{center}
\textbf{Summary\\}
\sommario \\
\vspace{1.5cm}\email
\end{center}

\clearpage

\pagenumbering{arabic}
%Questo file si occupa di generare la tabella delle modifiche
\pagenumbering{Roman}

\begin{center}
    \Large{\textbf{Change log}}
    	\\\vspace{0.5cm}
    	\normalsize
    \begin{tabularx}{\textwidth}{cXXcc}
        \textbf{Version} & \textbf{Changes - Motivation} & \textbf{Author} & \textbf{Role} & \textbf{Date} \\\toprule
        \modifiche
    \end{tabularx}
\end{center}

\newpage




% Renames the index to the english version
\renewcommand{\contentsname}{Table of Contents}
\input{../../../Template/Indice.tex}

\renewcommand{\listfigurename}{List of figures}
\renewcommand{\figurename}{Fig.}
\listoffigures
\newpage

\section{Introduction}

\subsection{Purpose of the document}
The purpose of this document is supplying a detailed guide to the user of \app, the demo application that has been created by \gruppo\ to show some of the possible uses of \progetto, the \termine{SDK} that the same group has created. \\
This document is so intended for the application's users, which will use the demo application as-it-is without modifying its code.

\subsection{Purpose of the product}
This project is divided in two parts with different purposes. \\
The first part is an \termine{SDK}, called \progettoShort, which allows a developer to create interactive bubbles easily which have to be able to work inside the \termine{Rocket.chat} environment. \\
The second part is a demo application developed using the above mentioned \progettoShort\ and which uses the provided bubbles. Our application is called \app, and it allows you to create an interactive and sharable to-do or shopping list.

\subsection{Glossary}
To avoid misunderstandings with technical terms of this document, words that require a detailed explanation will be marked with a \textit{G} and then the word will be inserted in the respective section of the glossary.

\newpage
\section{Use requirements}

% Requisiti generali
\subsection{Requirements}
In order to use \app\ the user needs to access the internet either using a laptop or desktop computer, or using a mobile cellphone or tablet. Other than that, the device which will be used to use \app\ must have a browser which supports \termine{JavaScript} and has it enabled.

% Requisiti da desktop o laptop
\subsubsection{Desktop requirements}
To access \app\ using a desktop or laptop computer, one of the following browsers is required:
\begin{itemize}
	\item Microsoft Internet Edge 13 or above;
	\item Mozilla Firefox 45 or above;
	\item Google Chrome 56 or above;
	\item Opera 43 or above;
	\item Apple Safari 43 or above.
\end{itemize}

% Requisiti da mobile
\subsubsection{Mobile requirements}
To access \app\ using a mobile phone or table, one of the following requirement needs to be satisfied:
\begin{itemize}
	\item If the device has Android as its operative system, it needs to have Google Chrome 56 or above installed;
	\item If the device has iOS as its operative system, it needs to have either iOS 10 or above installed, or it needs to have Google Chrome 56 or above as browser.
\end{itemize}

% Sezione su come abilitare JS
\subsubsection{Javascript}
In order to enable \termine{JavaScript} inside the different browsers' versions the next steps must be followed:
% Scrivere come abilitare JavaScript


%Sezione dedicata a come installare i prodotti
\subsection{Installation}
\app\ installation is borne by the supplier.

\subsection{Access to the application}
To access \app\, the only things required are the following:
\begin{enumerate}
	\item Connect to the following \termine{Rocket.chat} server from a device which has access to the 		
		  internet:
			\begin{lstlisting}
			54.135.165.325 
			\end{lstlisting}
	\item Either:
		  \begin{enumerate}
		  	\item Register a new account if you do not have one.
		  		\begin{enumerate}
		  			\item Click on "Register a new account";
		  			\item Fill all the required fields with your information;
		  			\item Click on "Register a new account".
		  		\end{enumerate}
		  		
		  	\item Login with you credentials.
		  		\begin{enumerate}
		  			\item Insert your credentials inside the required fields
		  			\item Click on "Login"
		  		\end{enumerate}
		  \end{enumerate}
\end{enumerate}

Once logged in, you will be able to use \app\ as a fully-implemented feature of our server.

\newpage
\section{Features Guide}
Inside this section there will be some examples of code that show how to use \progettoShort, followed by a guide explaining how to use the \app application.

\subsection{SDK}
This guide is dedicated to a generic developer which wants to use \progettoShort\ to integrate its feature inside his \termine{Meteor.js} project. 

\subsection{Widgets}
\subsubsection{TextWidget}
\begin{lstlisting}[language=JavaScript]
// Create a TextWidget
let textWidget = new Monolith.Widget.TextWidget;

// Hide the widget
textWidget.setVisibility(false); // Default is true, which will show is

// Set the text. Markdown notation is also supported
textWidget.setText("Foo");
textWidget.setText("Markdown __is supported__ **too**");

// Set the text color using HEX notation (http://www.color-hex.com/)
textWidget.setTextColor("#C61A10");

// Set the text size in pixel
textWidget.setTextSize(15);

// Set the URL highlighting color
textWidget.setUrlHighligthColor("#EE42F4");

// Enable or disable the text formatting, this includes also URL highlighting
textWidget.setFormatText(true);
textWidget.setFormatText(false);
\end{lstlisting}
~\\~\\

\subsubsection{ImageWidget}
\begin{lstlisting}[language=JavaScript]
// Create the ImageWidget
let imageWidget = new Monolith.Widget.ImageWidget;

// Hide the widget
imageWidget.setVisibility(false); // Default is true, which will show is

// Set the image associated with the widget
imageWidget.setImage("path/to/image.png");

// Set the image dimensions
imageWidget.setWidth(200);
imageWidget.setHeight(50);
\end{lstlisting}

\newpage
\subsubsection{ButtonWidget}
\begin{lstlisting}[language=JavaScript]
// Create a ButtonWidget
let buttonWidget = new Monolith.Widget.ButtonWidget;

// Set the dimensions of the button
buttonWidget.setWidth(100);
buttonWidget.setHeight(50);

// Set the color of the button
buttonWidget.setBackgroundColor("#41F492");

// Set the action associated with the button
buttonWidget.setOnClickAction(function(){
    alert("The button has been clicked");
});

// Set the action associated with the button when the user long-clicks it
buttonWidget.setOnLongClickAction(function(){
    alert("The button has been long clicked");
});

// Set the milliseconds that need to pass before a click is considered a long click
buttonWidget.setOnLongClickActionTimer(500);
\end{lstlisting}
~\\~\\

\subsubsection{ListWidget}
\begin{lstlisting}[language=JavaScript]
// Create the ListWidget
let listWidget = new Monolith.Widget.ListWidget;

// Add items to the list
listWidget.addItem("First");
listWidget.addItem("Second");
listWidget.addItem("Third");

// Set the indicator of the list
listWidget.setCharacterNumber(); // Numbered list
listWidget.setCharacterCircle(); // Unnumbered list

// Set the indicator color
listWidget. setColor("#292929");
\end{lstlisting}

\newpage
\subsubsection{CheckListItemWidget}
\begin{lstlisting}[language=JavaScript]
// Create a new CheckListItemWidget
let checkListItemWidget = new Monolith.Widget.CheckListItemWidget;

// Set the text associated with the item
checkListItemWidget.setText("Click me!");

// Customize the check appereance
// Color the check box instead of using a check tick
checkListItemWidget.setUseSelectionMark(true); 
// Set the color that will be used to color the check box
checkListItemWidget.setSelectionColor("#AAAAAA"); 
// Use a check tick
checkListItemWidget.setUseSelectionMark(false); 
// Set the character used as check tick
checkListItemWidget.setSelectionCharacter("X"); 

// Check or un-check the option
checkListItemWidget.setChecked(true); // Checked
checkListItemWidget.setChecked(false); // Un-checked

// Know it the option is checked or not
let isChecked = checkListItemWidget.isChecked();
if (isChecked){
    // The option is checked
} else {
    // The option is not checked
}


// Set the action to perform on click
checkListItemWidget.setOnClick(function(item){
    // The item parameter represents the view of the item that has been clicked
    item.setText("New text after click");
});

// Set the action to perform after a long click (1000 ms)
checkListItemWidget.setOnLongClick(function(){
    // The item parameter represents the view of the item that has been clicked
   item.setText("New text after long click");
});


// Delete the item
checkListItemWidget.removeOption();
\end{lstlisting}

\newpage
\subsubsection{Create a custom widget}
In order to create a new custom widget and add it to \termine{Monolith} so than you can use it like the default ones, you have to do as follows.
\begin{enumerate}

	\item Create a new class which extends from \texttt{BaseWidget}
\begin{lstlisting}[language=JavaScript]
export class MyWidget extends Monolith.Widget.BaseWidget {

    constructor(){
        super(); // You need to call this to create the above hierarchy
        
        // Initialize your widget here
    }
    
    renderView(){
        // Renders the view of the widget and returns a DOMElement object
    }

    performOperation(){
        // Perform some operation
    }

}
\end{lstlisting}

	\item Use your widget wherever you want
\begin{lstlisting}[language=JavaScript]
// Import the widget
import {MyWidget} from '/path/to/MyWidget.js';

// Istantiate the widget
let myWidget = new MyWidget();

// Perform operations with the widget
myWidget.performOperation();

// Render the widget's view
myWidget.renderView();
\end{lstlisting}
  
\end{enumerate}  
  
\textbf{Note} \\ 
The default widget's behaviour does \textbf{not} let the user use a single widget without a bubble container that holds it. \\
If you plan to render the widget's view inside a Rocket.chat room, please create a bubble and add your widget to the bubble, so that the bubble will render it and show it to the user.

\newpage


\subsubsection{Layouts}
Inside \progettoShort\ there are two classes which gives the possibility of arranging the different widgets in two main orientations:
\begin{itemize}
	\item \texttt{HorizontaLayout}: it allows the widgets that are added to it to be displayed each one on the right of the previous ones;
	\item \texttt{VerticalLayout}: it allows the widgets to be displayed one below the previous ones.
\end{itemize}

\paragraph{Using a layout}
To use one of the given layout classes, all that needs to be done is the following:

% Immagine layout prima dell'aggiunta di un widget
\begin{lstlisting}[language=JavaScript, frame=single]
// Needed to be able to create the new layout
import {HorizontalLayout} from {BOH}

let widget = new TextWidget();

let layout = new HorizontalLayout();
layout.addComponent(widget);
\end{lstlisting}
% Immagine layout dopo l'aggiunta del widget

\paragraph{Creating a new layout}
In order to create a new layout class, the only thing that needs to be done is creating a new class which extends from \texttt{BaseLayout} and implements its abstract methods:
\begin{lstlisting}[language=JavaScript, frame=single]
// Needed to be able to create the new layout
import {BaseLayout} from {BOH}

export class MyLayout extends BaseLayout {
    constructor(){
        // Needed in order to create the base class instance
        super();
    }
    
    renderView(){
        // Renders the layout view usually calling its components' 
        // renderView() method and returns the HTML code
    }

}
\end{lstlisting}
\subsection{Bubbles}
\subsubsection{MarkdownBubble}
\begin{lstlisting}[language=JavaScript]
// Create the bubble
let markdownBubble = new Monolith.Bubble.MarkdownBubble("Optional text here");

// Change the bubble text
markdownBubble.setText("Text with **markdown** __inside__");

// Render the bubble
markdownBubble.renderView();
\end{lstlisting}
~\\~\\

\subsubsection{AlertBubble}
\begin{lstlisting}[language=JavaScript]
// Create the bubble
let alertBubble = new Monolith.Bubble.AlertBubble;

// Set the alert title
alertBubble.setTitle("Warning");

// Set the alert message
alertBubble.setMessage("Please check your data");

// Build the view
alertBubble.renderView();
\end{lstlisting}

\newpage
\subsubsection{ToDoListBubble}
\begin{lstlisting}[language=JavaScript]
// Create the ToDoListBubble
let toDoListBubble = new Monolith.Bubble.ToDoListBubble;

// --- GENERIC OPERATIONS ---

// Get the id of the bubble
let id = toDoListBubble.getId();

// Set the text associated with the bubble
toDoListBubble.setText("This bubble contains a lot of items that can be checked");

// Set the color of the text
toDoListBubble.setTextColor("#1A5418");

// Tell the bubble to format the text and so support the markdown notation
toDoListBubble.setFormatText(true);

// Disable text formatting
toDoListBubble.setFormatText(false);

// Set the highlight color for URLs
toDoListBubble.setUrlHighlightColor("#891C15");

// Set the bubble's text size in pixel
toDoListBubble.setTextSize(15);

// Set the message to show upon completion 
toDoListBubble.setCompletionMessage("The list has all been checked.")


// --- ITEMS OPERATIONS ---

// Add an item to the list
toDoListBubble.addItem("First item");

// You can specify a second parameter which tells if the item that is being added
// should be initially checked or not
toDoListBubble.addItem("Second item", true); 

// Set the text of an item
toDoListBubble.setItemText("New text", 0); // Changes the text of the first item

// Check an item
toDoListBubble.setChecked(true, 0); // Checks the first item

// Remove an item specifying the index of that item. Indexes start from 0
toDoListBubble.removeItem(1);


// Tell the bubble to use selection marks when ticking the options
toDoListBubble.setUseSelectionMark(true);

// Set the character to use when ticking an option
toDoListBubble.setSelectionCharacter("X");

// Tell the bubble to color the check box when ticking the options
toDoListBubble.setUseSelectionMark(false);

// Set the color used to color the check box
toDoListBubble.setSelectionColor("#1D2565");

// Set the function to call when clicking an item
// This function will be called after the click of any of the items.
// The parameter indicates the view of the item that has been clicked
toDoListBubble.setOnItemClick(function(item){
    item.setText("New text after click");
});

// Set the function to call when long-clicking an item
// This function will be called after the long click of any of the items.
// The parameter indicates the view of the item that has been long clicked
toDoListBubble.setOnItemClick(function(item){
    item.setText("New text after long click");
});

\end{lstlisting}

\newpage
\subsubsection{Create a custom bubble}
In order to create a new bubble type you have to do as follows.

\begin{enumerate}
	\item Create a new class which extends from `Monolith.Bubble.BaseBubble`.
\begin{lstlisting}[language=JavaScript]
export class CustomBubble extends Monolith.Bubble.BaseBubble {
    
    constructor(params){
        super(); // Always remember to call super!
        
        // Do something with the params
        // Setup the bubble
    }
    
    customOperation(){
        // Perform a custom operation
    }
    
    renderView(){
        super.renderView(); // Again, necessary call
        
        // Return a DOMElement object
    }
    
}
\end{lstlisting}

	\item Put the following code wherever you want. Thi can be done even inside the same file that contains the class definition, outside the definition itself.
\begin{lstlisting}[language=JavaScript]
Monolith.Bubble.addBubble("key", function(message){
    // Create the bubble
    let customBubble = new CustomBubble(params);
    
    // Perform the operations you want
    customBubble.customOperation();
    
    // Return the setup bubble
    return customBubble;
});
\end{lstlisting}
Please note that this function takes two parameters:
	\begin{enumerate}
		\item A \texttt{string} which defines the unique key that identifies your custom bubble.   \\
   		A good naming conventions for this key would be using the \url{reverse domain name notation}{https://en.wikipedia.org/wiki/Reverse_domain_name_notation} (e.g. \texttt{com.mycompany.bubble.custom}) which allows you to identify your custom bubble type among all the other bubbles types.
		\item A \texttt{function} which takes as parameter a \texttt{message} object that identifies a \url{Rocket.chat message}{https://rocket.chat/docs/developer-guides/realtime-api/the-message-object/}.  \\
   This function is the one that creates the bubble, taking data from the parameter object, performs operation on it if there's the need, and then return it.
   \end{enumerate}
\end{enumerate}   
   
\textbf{Note} \\
All of the written above should be made \textbf{only} inside the \texttt{client} directory of your \termine{Meteor} project, otherwise it will make your application crash.

\newpage
\subsubsection{Composition of layouts or widgets}
Using the \texttt{BaseLayout}'s or \texttt{BaseBubble}'s \texttt{addComponent(component : BaseComponent)}  method, you can add to a layout or bubble either a widget or a sub-layout. \\

\textbf{Notes}. 
\begin{enumerate}
	\item The default bubble's layout will always be a \texttt{VerticalLayout} and cannot be changed.
	\item The below examples will use a bubble as the containers, but the same code applies also to simple layouts.
\end{enumerate}

\paragraph{Adding a widget}
In order to add a widget, the only thing there's need to do is create the widget instance and then call the \texttt{addComponent} method passing it as parameter.

% Immagine bolla prima del'aggiunta di un widget
\begin{lstlisting}[language=JavaScript, frame=single]
const bubble = new ToDoListBubble();
const widget = new TextWidget();
widget.setText("This is a new widget which has been added to the bubble");
bubble.addComponent(widget);
\end{lstlisting}
% Immagine bolla dopo l'aggiunta del widget

\paragraph{Adding a Layout with some widgets}
To add a layout to a bubble, the method which needs to be follow is the same as the one presented for the widget. The layout needs to be created and then \texttt{addComponent} must be called giving the layout as parameter.

% Immagine bolla prima dell'aggiunta del layout
\begin{lstlisting}[language=JavaScript, frame=single]
// Needed to be able to extend from the BaseBubble class
const bubble = new ToDoListBubble();

const widget = new TextWidget();
widget.setText("First text widget");
const widget 2 = new TextWidget();
widget2.setText("Second text widget");

const layout = new HorizzontalLayout();
layout.addComponent(widget1);
layout.addComponent(widget2);

bubble.addComponent(layout);
\end{lstlisting}
% Immagine della bolla dopo l'aggiunta del layout



\end{document}
