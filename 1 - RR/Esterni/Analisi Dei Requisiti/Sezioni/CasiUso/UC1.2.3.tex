\newpage
\paragraph{Caso d'uso UC 1.2.3: Impostazione della reazione ad un evento del bottone}
\label{UC 1.2.3: Impostazione della reazione ad un evento del bottone}
\begin{figure}[ht]
	\centering
	\includegraphics[scale=0.54]{Usecases/img/{UC1.2.3}.png}
	\caption{Caso d'uso UC 1.2.3: Impostazione della reazione ad un evento del bottone}
\end{figure}
\FloatBarrier
\begin{itemize}
\item\textbf{Attori}: Sviluppatore.
\item\textbf{Descrizione}: Lo sviluppatore vuole collegare una reazione ad un qualsiasi evento associato al bottone.
\item\textbf{Precondizione}: Lo sviluppatore utilizza un bottone e vuole associarci un evento.
\item\textbf{Postcondizione}: Lo sviluppatore ha associato un evento al bottone oppure non l'ha associato.
\item\textbf{Scenario principale}:
	\begin{itemize}
		\item Lo sviluppatore vuole collegare una reazione all'evento "click normale" del bottone (ossia un semplice click sul bottone) (UC 1.2.3.1).
		\item Lo sviluppatore vuole collegare una reazione all'evento "click prolungato" del bottone (ossia un click che ha una certa rilevanza di tempo sul bottone) (UC1.2.3.2).
	\end{itemize}
\end{itemize}
