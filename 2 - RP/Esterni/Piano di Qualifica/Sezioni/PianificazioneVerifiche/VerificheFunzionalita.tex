\newpage
\subsubsection{Verifiche di funzionalità}
In questa sezione verranno descritte come e quando effettuare le metriche incentrate sulla funzionalità del prodotto.

\paragraph{Completezza dell’implementazione funzionale}
\begin{itemize}
\item \textbf{Chi:} Jenkins.
\item \textbf{Quando:} Prima del inclusione del codice di una o più unità nel ramo di produzione.
\item \textbf{Come:} \termine{Jenkins} eseguira i test automatici predisposti dai \ProgrP, se il test non viene superato il codice non verrà integrato nel ramo di produzione.
\end{itemize}

\paragraph{Accuratezza rispetto alle attese}
\begin{itemize}
\item \textbf{Chi:} Jenkins.
\item \textbf{Quando:} Dopo ogni push sulla \termine{repository}.
\item \textbf{Come:} Questo test viene effettuato automaticamente da Jenkins, se il test non viene superato il codice non verrà integrato nel ramo di produzione.
\end{itemize}

\paragraph{Controllo degli accessi}
\begin{itemize}
\item \textbf{Chi:} Jenkins.
\item \textbf{Quando:} Dopo ogni push sulla \termine{repository}.
\item \textbf{Come:} Questo test viene effettuato automaticamente da Jenkins, se il test non viene superato il codice non verrà integrato nel ramo di produzione.
\end{itemize}