% Questo file definisce lo stile che verrà applicato alle lettere
\documentclass[
  SN,% lco file for swiss letters (SN 010 130))
  refline=dateleft,
  firstfoot=false,
  enlargefirstpage,
  backaddress=false,
  foldmarks=off,
  parskip=half,
  headheight=3cm,
  ]{scrlttr2}

\usepackage{ifthen}
\usepackage[italian]{babel}
\usepackage[utf8x]{inputenc}
\usepackage[T1]{fontenc}
\usepackage{fancyhdr}
\usepackage[colorlinks=true, urlcolor=black, citecolor=black, linkcolor=black]{hyperref}
\usepackage{tabularx}
\usepackage{multirow}
\usepackage{booktabs}
\usepackage{color}
\usepackage{graphicx}
\usepackage{eurosym}
\usepackage{amsmath}
\usepackage{relsize}
\usepackage{url}

\usepackage[multidot]{grffile}
\usepackage{xcolor,colortbl}
\definecolor{lightblue}{HTML}{56B4E6}
\definecolor{blue}{HTML}{2953A1}
\definecolor{darkblue}{HTML}{1E396E}

% Cambia il font 
\renewcommand*\rmdefault{qhv}

% ***STILE PAGINA***
\pagestyle{fancy}
\fancyhf{}
\renewcommand{\headrulewidth}{0pt}


\setkomavar{firsthead}{
	\begin{minipage}[c]{5cm}
	\begin{tabular}{c}
	\usekomavar{fromlogo} \\
	\scriptsize \usekomavar{fromemail}
	\end{tabular}
	\end{minipage}
}

\setkomavar{fromemail}{npe.developers@gmail.com}
\setkomavar{fromlogo}{\includegraphics[height=3.5cm, keepaspectratio=true]{\logo}}
\setkomavar{date}{\data}
\setkomavar{subject}{Oggetto{\normalfont: \oggetto}}
\setkomafont{title}{\bfseries\normalsize\raggedright}


\makeatletter
  \@setplength{firstheadwidth}{\textwidth}
\makeatother
\renewcommand\raggedsignature{\raggedright}

\setkomavar{signature}{\mittente \\ \includegraphics[scale = 0.5]{\firma}}

%Generali
\newcommand{\capitolato}{C5 - Monolith: An interactive bubble provider}
\newcommand{\progettoShort}{Monolith}
\newcommand{\progetto}{Monolith: An interactive bubble provider}
\newcommand{\gruppo}{NPE Developers}
\newcommand{\gruppoLink}{\href{https://gitlab.com/npe-developers}{NpeDevelopers}}
\newcommand{\email}{\href{mailto:npe.developers@gmail.com}{\textcolor{blue}{npe.developers@gmail.com}}}
\newcommand{\password}{NP3Devel0pers}
\newcommand{\myincludegraphics}[2][]{%
	\setbox0=\hbox{\phantom{X}}%
	\vtop{
		\hbox{\phantom{X}}
		\vskip-\ht0
		\hbox{\includegraphics[#1]{#2}}}
}




%Componenti del gruppo
\newcommand{\RM}{Riccardo Montagnin}
\newcommand{\MT}{Manuel Turetta}
\newcommand{\FB}{Francesco Bazzerla}
\newcommand{\SL}{Stefano Lia}
\newcommand{\LD}{Luca Dario}
\newcommand{\DC}{Diego Cavestro}
\newcommand{\ND}{Nicolò Dovico}

%Ruoli
\newcommand{\Pm}{Project Manager}
\newcommand{\Am}{Amministratore}
\newcommand{\AmP}{Amministratori}
\newcommand{\An}{Analista}
\newcommand{\AnP}{Analisti}
\newcommand{\Dev}{Sviluppatore}
\newcommand{\DevP}{Sviluppatori}
\newcommand{\Ver}{Verificatore}
\newcommand{\VerP}{Verificatori}
\newcommand{\Progr}{Programmatore}
\newcommand{\ProgrP}{Programmatori}
\newcommand{\Prog}{Progettista}
\newcommand{\ProgP}{Progettisti}



%Firme
\newcommand{\RMFirma}{\myincludegraphics[scale = 0.5]{../../../Template/Firme/RM.png}}
\newcommand{\MTFirma}{\myincludegraphics[scale = 0.5]{../../../Template/Firme/MT.png}}
\newcommand{\FBFirma}{\myincludegraphics[scale = 0.5]{../../../Template/Firme/FB.png}}
\newcommand{\SLFirma}{\myincludegraphics[scale = 0.5]{../../../Template/Firme/SL.png}}
\newcommand{\LDFirma}{\myincludegraphics[scale = 0.5]{../../../Template/Firme/LD.png}}
\newcommand{\DCFirma}{\myincludegraphics[scale = 0.5]{../../../Template/Firme/DC.png}}
\newcommand{\NDFirma}{\myincludegraphics[scale = 0.5]{../../../Template/Firme/ND.png}}

%Professori e proponente
\newcommand{\TV}{Prof. Tullio Vardanega}
\newcommand{\RC}{Prof. Riccardo Cardin}
\newcommand{\RB}{Red Babel}
\newcommand{\proponente}{Red Babel}

%Documenti
\newcommand{\Gl}{Glossario}
\newcommand{\glossario}{\textit{\Gl\_v.1.0.0.pdf}}
\newcommand{\AdR}{Analisi dei Requisiti}
\newcommand{\analisiDeiRequisiti}{\textit{\AdR\_v.1.0.0.pdf}}
\newcommand{\AdRvDue}{AnalisiDeiRequisiti}
\newcommand{\NdP}{Norme di Progetto}
\newcommand{\normeDiProgetto}{\textit{\NdP\_v.1.0.0.pdf}}
\newcommand{\PdP}{Piano di Progetto}
\newcommand{\pianoDiProgetto}{\textit{\PdP\_v.1.0.0.pdf}}
\newcommand{\SdF}{Studio di Fattibilità}
\newcommand{\studioDiFattibilita}{\textit{\SdF\_v.1.0.0.pdf}}
\newcommand{\PdQ}{Piano di Qualifica}
\newcommand{\pianoDiQualifica}{\textit{\PdQ\_v.1.0.0.pdf}}
\newcommand{\VI}{Verbale Interno}
\newcommand{\VE}{Verbale Esterno}
\newcommand{\ST}{Specifica Tecnica}
\newcommand{\MU}{Manuale Utente}
\newcommand{\DDP}{Definizione di Prodotto}

%Periodo di progetto
\newcommand{\ARM}{Analisi dei Requisiti di Massima}
\newcommand{\ARD}{Analisi dei Requisiti in Dettaglio}
\newcommand{\PA}{Progettazione Architetturale}
\newcommand{\PD}{Progettazione di Dettaglio}
\newcommand{\COD}{Codifica}
\newcommand{\VV}{Verifica e Validazione Finale}

%Consegne
\newcommand{\RR}{Revisione dei Requisiti}
\newcommand{\RP}{Revisione di Progettazione}
\newcommand{\RQ}{Revisione di Qualifica}
\newcommand{\RA}{Revisione di Accettazione}


%Formattazione
\newcommand{\termine}[1]{\textit{#1}\small{$_G$}}
\newcommand{\link}[1]{\href{#1}{\textcolor{blue}{\texttt{#1}}}} 

% Testi ricorrenti
\newcommand{\scopoProdotto}{L'obiettivo di questo progetto è la realizzazione di un \termine{SDK} che permetta la creazione di bolle interattive, le quali, successivamente, verranno utilizzate all'interno dell'applicazione di messaggistica istantanea open source \termine{Rocket.chat}. \\
Dopo la realizzazione di tale \termine{SDK}, è proposto lo sviluppo di un'applicazione in grado di sfruttare l'\termine{SDK} per implementare un uso originale di tali bolle.
}
\newcommand{\descrizioneGlossario}{Al fine di mantenere questo documento compatto e di facile lettura è stato realizzato un glossario esterno contenente tutte le definizioni dei termini che più comunemente verranno presentati al lettore.  
Tale glossario si ritrova all'interno del file \glossario, e contiene tutti e soli i termini che vengono marcati con una \textit{G} a pedice.
}
\newcommand{\riferimentiNormativi}{
	\begin{itemize}
		\item \textbf{Norme di Progetto}: \normeDiProgetto
		\item \textbf{\termine{Capitolato} d'appalto C5: Monolith - An Interactive bubble provider} \\
			  \link{http://www.math.unipd.it/~tullio/IS-1/2016/Progetto/C5.pdf}
	\end{itemize}
}

\newcommand{\logo}{../../Template/Logo/Logo.png}
\newcommand{\data}{\today}
\newcommand{\oggetto}{Partecipazione alla Revisione di Accettazione.}
\newcommand{\mittente}{Riccardo Montagnin \\ Responsabile NPE Developers}
\newcommand{\firma}{../../Template/Firme/RM.png}



\begin{document}

\begin{letter}{Alla cortese attenzione di:  \\
 Prof. Vardanega Tullio \\
 Prof. Cardin Riccardo \\
 Università degli Studi di Padova \\
 Dipartimento di Matematica \\
 Via Trieste, 63 \\
 35121 - Padova (PD)}

\opening{Egregio Professor Vardanega Tullio,} 

con la presente il gruppo \gruppo\ intende comunicarLe ufficialmente la partecipazione alla \textbf{Revisione di Accettazione} per la realizzazione del prodotto da Lei commissionato, denominato:
\begin{center}
\textbf{Monolith} - An interactive bubble provider
\end{center}
L'offerta è corredata dai seguenti documenti allegati alla presente lettera:
\begin{itemize}
	\item Analisi dei Requisiti v3.0.0 (Esterni/\analisiDeiRequisiti);
	\item BringIt User Manual v2.0.0 (Esterni/\manualeUtenteApp);
	\item Definizione di Prodotto v2.0.0 (Esterni/\definizioneDiProdotto);
	\item Glossario v3.0.0 (Esterni/\glossario);
	\item Monolith User Manual v2.0.0 (Esterni/\manualeUtente);
	\item Piano di Progetto v4.0.0 (Esterni/\pianoDiProgetto);
	\item Piano di Qualifica v4.0.0 (Esterni/\pianoDiQualifica);	
	\item Norme di Progetto v4.0.0 (Interni/\normeDiProgetto);	
	
\end{itemize}

I dettagli di progettazione architetturale, di pianificazione e di qualità sono trattati in maniera approfondita nei documenti allegati. \\
Viene inoltre fornito il codice sorgente del prodotto \progettoShort\, della sua versione per progetti \termine{Rocket.chat} \progettoShort-rc, del prodotto \app, e della versione dal gruppo modificata di \termine{Rocket.chat} stesso. Tale codice sorgente potrà essere ritrovato all'interno della cartella \textit{Codice prodotto} oppure all'interno delle seguenti repository GitHub:
\begin{itemize}
	\item \progettoShort\ v.1.0.1: \link{https://github.com/NPE-Developers/Monolith}
	\item \progettoShort-rc v1.0.0: \link{https://github.com/NPE-Developers/Monolith-rc}
	\item \app\ v.1.0.0: \link{https://github.com/NPE-Developers/BringIt}
	\item \termine{Rocket.chat}: \link{https://github.com/NPE-Developers/Rocket.Chat}
\end{itemize}


\textbf{Il costo finale del progetto è di 13.639,00 \euro}\\
Di seguito viene presentato l'organigramma del team:

\begin{center}

	\begin{tabular}{>{\centering\color{white}}m{4cm} >{\centering\color{white}}m{1.8cm} >{\centering\arraybackslash}m{0pt}@{}}
	\rowcolor{darkblue} \textbf{Nominativo} & \textbf{Matricola} & \\[1ex]
	\rowcolor{blue} Diego Cavestro & 1094301 & \\[1ex]	
	\rowcolor{lightblue} Francesco Bazzerla & 1097417 & \\[1ex]
	\rowcolor{blue}  Luca Dario & 1097935 & \\[1ex]
	\rowcolor{lightblue} Manuel Turetta & 1103106 & \\[1ex] 
	\rowcolor{blue} Nicolò Dovico & 1102846 & \\[1ex]
	\rowcolor{lightblue} Riccardo Montagnin & 1100577  &\\[1ex]
	\rowcolor{blue} Stefano Lia & 1097641 & \\[1ex]
	\end{tabular}

\end{center}

Rimango a Sua completa disposizione per ogni ulteriore chiarimento. \\
La ringrazio anticipatamente per la Sua attenzione. 
\closing{}

\end{letter}
 
\end{document}