\newpage
\subsection{Requisiti Funzionali}
\normalsize
\begingroup
\renewcommand\arraystretch{2}
\begin{longtable}{|c|>{\centering}m{7cm}|c|}
\hline
\textbf{Id Requisito} & \textbf{Descrizione} & \textbf{Fonti}\\
\hline
\endhead
			R1F1 & Deve essere possibile utilizzare alcune tipologie di bolle già definite all'interno dell'\termine{SDK} & Interno, UC1 \\
			\hline
			R1F1.1 & All'interno dell'\termine{SDK} deve essere presente una bolla di tipo testo formattato & Interno, UC1.1 \\
			\hline
			R1F1.2 & All'interno dell'\termine{SDK} deve essere presente una bolla di tipo bottone & Interno, UC1.2 \\ 
			\hline
			R1F1.3 & All'interno dell'\termine{SDK} deve essere presente una bolla di tipo checklist & Interno, UC1.3 \\ 
			\hline
			R1F1.4 & All'interno dell'\termine{SDK} deve essere presente una bolla di tipo immagine & Interno, UC1.4 \\ 
			\hline
			R1F1.5 & Deve essere possibile modificare una istanza di una bolla già creata & Interno, UC1.5 \\ 
			\hline
			R1F1.6 & All'interno dell'\termine{SDK} deve essere presente una bolla vuota & Interno, UC1.6 \\ 
			\hline
			R2F1.1.1 & Deve essere possibile cambiare la dimensione del carattere con il quale è scritto il testo all'interno di una bolla di testo formattato & Interno, UC1.1.1 \\
			\hline
			R1F1.1.2 & Se non viene inserita una grandezza del font personalizzata, oppure se ne viene inserita una non valida, deve essere impostata una grandezza di default & Interno, UC1.1.2\\
			\hline
			R2F1.1.3 & Deve essere possibile formattare parte del testo di una bolla di testo formattato in corsivo & Interno, UC1.1.3 \\
			\hline
			R1F1.1.4 & Deve essere possibile inserire uno o più link cliccabili all'interno di una bolla di testo formattato & Interno, UC1.1.4 \\
			\hline
			R1F1.1.5 & Deve essere possibile impostare il colore del link cliccabile all'interno della bolla di testo formattato & Interno, UC1.1.5 \\
			\hline
			R1F1.1.6 & Deve essere previsto un colore di default per i link cliccabili all'interno di una bolla di testo formattato & Interno, UC1.1.6 \\
			\hline
			R2F1.1.7 & Deve essere possibile formattare parte del testo di una bolla di testo formattato in grassetto & Interno, UC1.1.7 \\
			\hline
			R1F1.1.8 & Deve essere possibile impostare un colore personalizzato per parte del testo di una bolla di testo formattato & Interno, UC1.1.8 \\
			\hline
			R1F1.1.9 & Deve essere previsto un colore di default per il testo di una bolla di testo formattato & Interno, UC1.1.9 \\
			\hline
			R1F1.1.10 & Deve essere possibile aggiungere del testo ad una bolla di tipo testo formattato & Interno, UC1.1.10 \\
			\hline
			R1F1.1.11 & Deve essere previsto un messaggio di errore dettagliato nell'evenienza in cui venga aggiunto del testo non valido ad una bolla di tipo testo formattato.	& Interno, UC1.1.11 \\
			\hline
			R1F1.2.1 & Deve essere possibile creare una bolla che rappresenti un bottone semplice & Interno, UC1.2.1 \\
			\hline
			R1F1.2.2 & Deve essere possibile creare una bolla che rappresenti un check button & Interno, UC1.2.2 \\
			\hline
			R1F1.2.3 & Deve essere possibile impostare una azione personalizzata per ogni evento di un bottone & Interno, UC1.2.3 \\
			\hline
			R1F1.2.1.1 & Deve essere possibile impostare il testo presente all'interno di un bottone semplice & Interno, UC1.2.1.1 \\
			\hline
			R1F1.2.1.2 & Se viene inserito un testo non valido oppure vuoto come testo interno ad un bottone semplice, deve essere visualizzato un messaggio di errore & Interno, UC1.2.1.2 \\
			\hline
			R2F1.2.1.3 & Deve essere possibile impostare le dimensioni di un bottone presente all'interno di una bolla & Interno, UC1.2.1.3 \\
			\hline
			R1F1.2.1.4 & Le dimensioni del bottone devono essere impostate a valori di default nel qual caso lo sviluppatore inserisca delle dimensioni personalizzate non valide & Interno, UC1.2.1.4 \\
			\hline
			R3F1.2.1.5 & Deve essere possibile impostare il colore di sfondo di un bottone & Interno, UC1.2.1.5 \\
			\hline
			R1F1.2.1.6 & Deve essere previsto un colore di sfondo di default nel qual caso lo sviluppatore non lo personalizzi oppure tenti di utilizzarne uno non valido & Interno, UC1.2.1.6 \\
			\hline
			R1F1.2.2.1 & Deve essere possibile impostare lo stato iniziale di un check button & Interno, UC1.2.2.1 \\
		\hline
		R1F1.2.2.2 & Deve essere previsto uno stato di default per i bottoni di check nel qual caso il loro stato non venga impostato ad un valore personalizzato & Interno, UC1.2.2.2 \\
		\hline
		R1F1.2.2.3 & Deve essere possibile personalizzare l'aspetto del bottone di check una volta che viene selezionato & Interno, UC1.2.2.3 \\
		\hline
		R2F1.2.2.3.1 & Deve essere possibile impostare la visualizzazione della selezione di un bottone di check mediante una "X" & Interno, UC1.2.2.3.1 \\ 
		\hline
		R1F1.2.2.3.2 & Deve essere possibile impostare la visualizzazione della selezione di un bottone di check mediante una spunta & Interno, UC1.2.2.3.2 \\ 
		\hline
		R2F1.2.2.3.3 & Deve essere possibile impostare la visualizzazione della selezione di un bottone di check mediante la colorazione stessa del bottone & Interno, UC1.2.2.3.3 \\ 
		\hline
		R3F1.2.2.3.4 & Deve essere possibile scegliere un colore personalizzato da utilizzare per indicare la selezione di un bottone di check & Interno, UC1.2.2.3.4 \\ 
		\hline
		R2F1.2.2.3.5 & Se non viene personalizzato, il colore di default utilizzato per l'indicazione della selezione di un bottone di check deve essere il verde & Interno, UC1.2.2.3.5 \\
		\hline
		R1F1.2.3.1 & Deve essere possibile impostare una azione da eseguire in seguito al click breve di un bottone & Interno, UC1.2.3.1\\ 
		\hline
		R2F1.2.3.2 & Deve essere possibile impostare una azione da eseguire in seguito al click prolungato di un bottone & Interno, UC1.2.3.2\\ 
		\hline
		R1F1.2.3.3 & Deve essere possibile impostare il tempo che deve intercorrere tra l'inizio della pressione e l'avvio dell'azione conseguente al click prolungato di un bottone & Interno, UC1.2.3.3\\ 
		\hline
		R3F1.2.3.4 & Deve essere previsto un valore di default per il tempo che deve intercorrere tra l'inizio del click prolungato di un bottone e l'avvio dell'azione associata & Interno, UC1.2.3.4\\ 
		\hline
		R1F1.3.1 & Deve essere possibile personalizzare l'elenco puntato di una bolla di tipo checklist & Interno, UC1.3.1 \\
		\hline
		R1F1.3.2 & Deve essere possibile impostare un messaggio di completamento da visualizzare una volta che tutti gli elementi di una checklist sono stati segnati come selezionati & Interno, UC1.3.2 \\
		\hline
		R1F1.3.3 & Deve essere previsto un messaggio di errore da visualizzare nel qual caso si cerchi di impostare un messaggio di completamento non valido per una bolla di checklist & Interno, UC1.3.3 \\
		\hline	
		R3F1.3.1.1 & Deve essere possibile visualizzare la lista di una bolla di tipo checklist sotto forma di elenco numerato & Interno, UC1.3.1.1 \\
		\hline
		R1F1.3.1.2 & Deve essere possibile visualizzare la lista di una bolla di tipo checklist sotto forma di elenco non numerato con indicatore un pallino & Interno, UC1.3.1.2 \\
		\hline
		R2F1.3.1.3 & Deve essere possibile visualizzare la lista di una bolla di tipo checklist sotto forma di elenco non numerato con indicatore un trattino & Interno, UC1.3.1.3 \\
		\hline
		R1F1.4.1 & Deve essere possibile associare una immagine ad una bolla di tipo immagine & Interno, UC1.4.1 \\
		\hline
		R1F1.4.2 & Deve essere previsto un messaggio di errore da visualizzare nel qual caso lo sviluppatore non associ una immagine ad una bolla immagine oppure tenti di associarne una non supportata & Interno, UC1.4.1 \\
		\hline
		R2F1.4.3 & Deve essere possibile impostare delle dimensioni personalizzate all'immagine presente all'interno di una bolla di tipo immagine & Interno, UC1.4.3 \\
		\hline
		R1F1.4.4 & Devono essere previste delle dimensioni di default nel qual caso lo sviluppatore tenti di personalizzare le dimensioni di una immagine all'interno di una bolla immagine usando valori non validi & Interno,UC1.4.4 \\
			\hline
		R1F1.5.1 & Deve essere possibile modificare il testo già contenuto all'interno di una bolla testo & Interno, UC1.5.1 \\
		\hline
		R1F1.5.2 & Deve essere possibile modificare l'immagine associata ad una bolla di tipo immagine & Interno, UC1.5.2 \\
		\hline
		R1F1.5.3 & Deve essere previsto un messaggio d'errore nel qual caso lo sviluppatore tenti di modificare una immagine già esistente utilizzandone una non valida & Interno, UC1.5.3 \\
		\hline
		R1F1.5.1.1 & Deve essere possibile modificare parte del testo precedentemente formattato in corsivo all'interno di una bolla di testo formattato & Interno, UC1.5.1.1.1 \\
		\hline
		R1F1.5.1.2 & Deve essere possibile modificare parte del testo precedentemente formattato in grassetto all'interno di una bolla di testo formattato & Interno, UC1.5.1.2 \\
		\hline
		R1F1.5.1.3 & Deve essere possibile modificare la grandezza del carattere di parte del testo presente all'interno di una bolla di testo formattato precedentemente creata & Interno, UC1.5.1.1.3 \\
		\hline
		R1F1.5.1.4 & Deve essere prevista una grandezza del font di default da utilizzare nel qual caso la modifica della grandezza del font di una bolla precedentemente venga utilizzata in modo scorretto & Interno, UC1.5.1.1.3 \\
		\hline
		R1F1.5.1.5 & Deve essere possibile modificare i link cliccabili precedentemente inseriti all'interno di una bolla di testo formattato & Interno, UC1.5.1.1.5 \\
		\hline
		R2F1.5.1.6 & Deve essere possibile cambiare il colore dei link cliccabili presenti all'interno di una bolla di testo formattato precedentemente creata & UC1.5.1.1.6 \\
		\hline
		R1F1.5.1.7 & Nel qual caso si tenti di modificare il colore dei link cliccabili presenti all'interno di una bolla di testo formattato utilizzando un colore non valido, il colore di tali link deve essere impostato a blu & Interno, UC1.5.1.1.7 \\
		\hline
		R1F1.5.1.8 & Deve essere possibile modificare il colore precedentemente utilizzato per colorare parte del testo di una bolla di tipo testo formattato & Interno, UC1.5.1.1.8 \\
		\hline
		R1F1.5.1.9 & Deve essere previsto un colore di default nel qual caso si tenti di modificare il colore di parte del testo di una bolla di testo formattato utilizzando un valore non valido & Interno, UC1.5.1.1.9 \\
		\hline
		R1F1.5.1.10 & Deve essere possibile cambiare il testo presente all'interno di una bolla precedentemente creata & Interno, UC1.5.1.1.10 \\
		\hline
		R1F1.5.1.11 & Deve essere previsto un messaggio di errore nel qual caso il testo che si vuole inserire in una bolla non sia valido & Interno, UC1.5.1.1.11 \\
		\hline
			R1F2 & Deve essere possibile creare delle nuove bolle a partire da quelle già esistenti all'interno dell'\termine{SDK} & Interno, UC2 \\
			\hline
			R1F2.1 & Deve essere possibile aggiungere dei metodi ad una bolla & Interno, UC2.1 \\
			\hline
			R1F2.2 & Deve essere possibile aggiungere campi dati ad una bolla & Interno, UC2.2 \\
			\hline
			R1F2.3 & Deve essere possibile sovrascrivere/ridefinire un metodo giù esistente all'interno di una bolla & Interno, UC2.3 \\
			\hline
			R1F3 & Deve essere possibile accedere all'\termine{SDK} & Capitolato \\
			\hline
			R1F4 & Deve essere possibile creare facilmente delle bolle nuove & Capitolato \\
			\hline				
\caption[Requisiti Funzionali]{Requisiti Funzionali}
\label{tabella: Requisiti Funzionali}
\end{longtable}
\endgroup
\clearpage