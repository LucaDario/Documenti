\newpage

\section{Qualità di processo}
Per garantire la qualità del prodotto è necessario perseguire la qualità dei processi che lo definiscono. Per fare ciò si è deciso di adottare lo standard ISO/IEC 15504 descritto nelle \NdP. Il gruppo \gruppo{} si impegnerà a seconda dei periodi coinvolti di migliorare ed incrementare eventuali processi con rispettive metriche al fine di garantire una buona qualità.


\subsection{Pianificazione, Controllo e Valutazione}
Il macro-processo (derivante dall’unione dei processi di Pianificazione, Controllo e Valutazione) ha lo scopo di produrre dei piani di sviluppo per il progetto, comprendenti
la scelta del modello di ciclo di vita del prodotto, descrizioni delle attività e dei compiti
da svolgere, pianificazione temporale del lavoro e dei costi da sostenere, allocazione di compiti e
responsabilità, e misurazioni per rilevare lo stato del progetto rispetto alle pianificazioni prodotte.

\subsubsection{Obiettivi di Qualità}
L'intero sviluppo del progetto dovrà seguire la pianificazione prodotta, in particolare:

\begin{itemize}
\item Ogni attività assegnata tramite l'utilizzo dello strumento \termine{Wrike} dovrà essere eseguita nel tempo previsto e da colui a cui è stata assegnata.
\item Il costo preventivato per il periodo descritto nel \PdP{} non dovrà eccedere, ma dovrà rimanere al di sotto della soglia prevista.
\end{itemize}

\subsubsection{Strategie}
La pianificazione dovrà essere costantemente aggiornata durante lo sviluppo del progetto per essere coerente alla situazione corrente. Al fine di garantire una pianificazione corretta e conforme alle attese qualsiasi valore negativo rilevato a livello di Schedule Variance o Budget Variance rilevato in una fase di lavoro dovrà essere
assolutamente compensato entro la fine dell’attività di progetto, in quanto non è assolutamente
ammesso eccedere le ore di lavoro finali e il preventivo dei costi finale indicato nella pianificazione.

\subsubsection{Metriche}

\paragraph{Schedule Variance}
Indica se si è in linea, in anticipo o in ritardo rispetto la pianificazione temporale prevista dal \PdP.

\begin{itemize}
\item \textbf{Misurazione}: $SV = BCWP - BCWS$, dove $BCWP$ sono le attività completate ad un certo momento e $BCWS$ le attività che, secondo la pianificazione, dovrebbero essere state completate a quel momento.
\end{itemize}

\begin{center}

		\begin{tabular}{|P{2.5cm}|P{2.5cm}|P{6cm}|}
		\hline
			\textbf{Valore di accettazione}	& \textbf{Valore ottimale} & \textbf{Motivazione} \\
			\hline
			$\geq 0$ & $\geq 0$ & Garantire una buona pianificazione dei tempi in modo da evitare ritardi di consegna. \\
			\hline
			\end{tabular}
\captionof{table}{Schedule Variance}
\end{center}


\paragraph{Budget Variance}

\begin{itemize}
\item \textbf{Misurazione}: $BV = BCWS - ACWP$, dove $BCWS$ è il costo pianificato per realizzare le attività di progetto alla data corrente e $ACWP$ è il costo effettivamente sostenuto alla data corrente.
\end{itemize}

\begin{center}
	\begin{tabular}{|P{2.5cm}|P{2.5cm}|P{6cm}|}
		\hline
			\textbf{Valore di accettazione}	& \textbf{Valore ottimale} & \textbf{Motivazione} \\
			\hline
			$\geq 0$ & $\geq 0$ & Garantire una buona pianificazione dei costi in modo da rimanere entro i limiti del budget prefissati. \\
			\hline
			\end{tabular}
\captionof{table}{Budget Variance}
\end{center}

\subsection{Processo di gestione dei rischi}
L'obiettivo del processo è quello di identificare, analizzare, trattare e monitorare continuamente i rischi che possono insorgere durante l'intera attività di progetto.

\subsubsection{Obiettivi di qualità}
Il gruppo dovrà gestire e monitorare tutti i possibili rischi ed in particolare:

\begin{itemize}
\item all'inizio dell'attività di progetto, verranno individuati i principali fattori di rischio che si possono verificare in ogni attività.
\item all'inizio di ogni periodo, il gruppo dovrà analizzare eventuali rischi non preventivati ed in caso trovarne altri associati.
\item i rischi analizzati che si paleseranno saranno trattati secondo le strategie individuate e descritte nel \PdP.
\end{itemize}

\subsubsection{Metriche}

\paragraph{Rischi non preventivati}

\begin{itemize}
\item \textbf{Misurazione}: indice numerico che viene incrementato nel momento in cui si manifesta un rischio non individuato nell'attività di analisi dei rischi;
\end{itemize}

\begin{center}
	\begin{tabular}{|P{2.5cm}|P{2.5cm}|P{6cm}|}
		\hline
			\textbf{Valore di accettazione}	& \textbf{Valore ottimale} & \textbf{Motivazione} \\
			\hline
			[$0 - 5$] & $0$ & Garantire una buona gestione dei rischi, in modo da evitare che se ne verifichino troppi rischi non preventivati. \\
			\hline
			\end{tabular}
\captionof{table}{Rischi non preventivati}
\end{center}


\subsection{Processo di analisi dei requisiti software e di sistema}

Il processo punta a trasformare i requisiti individuati dalle fonti in un set di requisiti tecnici che fungerà da linea guida nella progettazione del sistema.

\subsubsection{Obiettivi di qualità}
I requisiti dovranno essere gestiti dal gruppo nel seguente modo:

\begin{itemize}
\item ogni requisito sarà dotato di un codice univoco che lo identifica.
\item per ogni requisito verrà tenuta traccia della fonte da cui è stato ricavato.
\item per ogni requisito dovrà essere possibile indicare dei test da effettuare con rispettivo risultato.
\item nessun requisito dovrà risultare superfluo o ambiguo.
\end{itemize}

\subsubsection{Strategie}
Tutti i requisiti individuati saranno riportati in un \termine{file json} che ne manterrà il tracciamento.

\subsubsection{Metriche}
\paragraph{Requisiti obbligatori soddisfatti}

\begin{itemize}
\item \textbf{Misurazione}: 
		$$C=\left(1-\mathlarger{\frac{N_{FM}}{N_{FI}}}\right) \cdot 100$$ 
	dove $N_{FM}$ è il numero di funzionalità mancanti nell'implementazione e $N_{FI}$ è il numero di funzionalità individuate nell'attività di analisi. 
\end{itemize}
	
\begin{center}
	\begin{tabular}{|P{2.5cm}|P{2.5cm}|P{6cm}|}
		\hline
			\textbf{Valore di accettazione}	& \textbf{Valore ottimale} & \textbf{Motivazione} \\
			\hline
			$100$ & $100$ & Garantire il pieno soddisfacimento dei requisiti obbligatori individuati nel documento \AdR. \\
			\hline
			\end{tabular}
\captionof{table}{Requisiti obbligatori soddisfatti}
\end{center}	

\subsection{Processo di sviluppo dell'architettura di sistema}

Il processo si pone come obiettivo quello di identificare una corrispondenza fra requisiti di sistema ed elementi del sistema.
\subsubsection{Obiettivi di qualità}
Durante lo svolgimento delle attività previste da questo processo, il gruppo si impegnerà a trovare una buona architettura di sistema nel seguente modo:
\begin{itemize}
\item ogni componente progettato come parte del sistema risulterà essere necessario per il funzionamento del prodotto e, quindi, costantemente tracciabile ai requisiti che soddisfa.
\item il sistema dovrà presentare basso accoppiamento ed alta coesione, tenendo conto che \termine{RocketChat} implementa il pattern \termine{Observer} per la gestione degli eventi. Il suddetto pattern aumenta l'accoppiamento per cui il \termine{team} dovrà gestire in maniera ottimale il problema.
\item ogni componente dovrà essere progettato puntando su incapsulamento, modularizzazione e riuso di codice.
\end{itemize}
\newpage

\subsubsection{Strategie}
Nel corso dell'attività di progettazione, sia ad alto livello che di dettaglio, le componenti verranno inserite in un \termine{file json} dedicato, il quale si occuperà del tracciamento con i requisiti trovati nell'attività di analisi.


\subsubsection{Metriche}

\paragraph{Livello di stabilità}

\begin{itemize}
\item \textbf{Misurazione:} 
\begin{displaymath}
{\text{Accoppiamento Efferente}}\over{\text{Accoppiamento Afferente} + \text{Accoppiamento Efferente}}
\end{displaymath} 
\end{itemize}

\begin{center}
		\begin{tabular}{|P{2.5cm}|P{2.5cm}|P{6cm}|}
		\hline
			\textbf{Valore di accettazione}	& \textbf{Valore ottimale} & \textbf{Motivazione} \\
			\hline
			[0.0 −- 1] & [0.0 −- 0.6] &	Garantire una buona indipendenza e stabilità delle parti del software. \\
			\hline
			\end{tabular}
\captionof{table}{Livello di stabilità}
\end{center}

\paragraph{Astrattezza}

\begin{itemize}
\item \textbf{Misurazione:}
\begin{displaymath}
{\text{Numero classi astratte e interfacce}}\over{\text{Numero totale classi}}
\end{displaymath}
\end{itemize}

\begin{center}
		\begin{tabular}{|P{2.5cm}|P{2.5cm}|P{6cm}|}
		\hline
			\textbf{Valore di accettazione}	& \textbf{Valore ottimale} & \textbf{Motivazione} \\
			\hline
			[0.0 −- 0.8] & [0.0 −- 0.3] &	Garantire un sufficiente polimorfismo e riuso delle componenti. \\
			\hline
			\end{tabular}
\captionof{table}{Astrattezza}
\end{center}

\paragraph{Distanza dalla sequenza principale}

\begin{itemize}
\item \textbf{Misurazione:}
\begin{displaymath}
{|\text{Astratezza} + \text{Livello di stabilità} - 1|}
\end{displaymath}
\end{itemize}

\begin{center}
		\begin{tabular}{|P{2.5cm}|P{2.5cm}|P{6cm}|}
		\hline
			\textbf{Valore di accettazione}	& \textbf{Valore ottimale} & \textbf{Motivazione} \\
			\hline
			[0.0 −- 1] & [0.0 −- 0.4] &	Garantire un buon bilanciamento tra l’astrattezza e la stabilità del \termine{package} da noi sviluppati.\\
			\hline
			\end{tabular}
\captionof{table}{Distanza dalla sequenza principale}
\end{center}

\subsection{Processo di sviluppo dell'architettura di sistema in dettaglio}
Lo scopo del processo è fornire una progettazione di dettaglio del prodotto che andrà ad implementare i requisiti individuati.
\subsubsection{Obiettivi di qualità}
Le attività svolte dovranno raggiungere i seguenti obiettivi:
\begin{itemize}
\item il livello di dettaglio della progettazione dovrà essere tale da guidare codifica e testing senza bisogno di informazioni aggiuntive, indicando metodi con i relativi parametri e campi dati forniti da ciascuna componente.
\item la struttura dell'architettura sia ad alto che a basso livello verrà definita nel documento \DDP
\item oltre alle unità software individuate, le attività permetteranno di definire dettagliatamente le interfacce fra esse costituite.
\end{itemize}

\subsubsection{Strategie}

Sarà necessario effettuare un'analisi dettagliata delle componenti individuate in progettazione architetturale, suddividendole in unità che siano facilmente codificabili e testabili per le attività successive.

\subsubsection{Metriche}

\subsubsection{Numero di attributi per classe}

\begin{itemize}
\item \textbf{Misurazione}: indice numerico che indica il numero di attributi per classe.
\end{itemize}

\begin{center}
		\begin{tabular}{|P{2.5cm}|P{2.5cm}|P{6cm}|}
		\hline
			\textbf{Valore di accettazione}	& \textbf{Valore ottimale} & \textbf{Motivazione} \\
			\hline
			[0 −- 11] & [0 −- 7] &	Garantire un giusto equilibrio tra le classi, in modo da garantire una sufficiente coesione. \\
			\hline
			\end{tabular}
\captionof{table}{Numero di attributi per classe}
\end{center}

\paragraph{Numero di metodi per classe}

\begin{itemize}
\item \textbf{Misurazione}: indice numerico che indica il numero di metodi definiti in una classe.
\end{itemize}


\begin{center}
\begin{tabular}{|P{2.5cm}|P{2.5cm}|P{6cm}|}
		\hline
			\textbf{Valore di accettazione}	& \textbf{Valore ottimale} & \textbf{Motivazione} \\
			\hline
			[$1 - 10$] & [$1 - 7$] &	Garantire un giusto equilibrio tra funzionalità e coesione tra le classi. \\
			\hline
			\end{tabular}
\captionof{table}{Numero di metodi per classe}
\end{center}

\subsection{Processo di codifica del Software}
Il processo definisce le attività principali volte alla produzione di unità software eseguibili che riflettano quanto progettato.

\subsubsection{Obiettivi di qualità}
Le unità software prodotte dovranno risultare di qualità; a questo fine il \textit{team\ped{G}} si è posto i seguenti obiettivi:
\begin{itemize}
\item l'implementazione delle classi e dei metodi definiti in progettazione dovrà puntare a produrre codice a bassa complessità, in modo tale che quanto prodotto risulti facilmente comprensibile e testabile.
\item il codice prodotto dovrà risultare facilmente manutenibile.
\end{itemize}
\subsubsection{Strategie}
Durante l'attività di codifica, il \textit{\Progr} dovrà attenersi a quanto indicato nel documento \textit{\DDP}, concentrandosi (in particolare) nel limitare la complessità del codice prodotto. Sarà necessario inoltre procedere con la codifica dei test individuati nell'attività di progettazione, in modo tale da consentire la verifica del corretto funzionamento delle varie unità prodotte. Per i suddetti test verranno utilizzati gli strumenti descritti nelle \NdP.
\subsubsection{Metriche}

\subsubsection{Complessità ciclomatica}

\begin{itemize}
\item \textbf{Misurazione}: indice numerico che indica il numero cammini percorribili nel grafo di controllo di flusso di un metodo.
\end{itemize}

\begin{center}
		\begin{tabular}{|P{2.5cm}|P{2.5cm}|P{6cm}|}
		\hline
			\textbf{Valore di accettazione}	& \textbf{Valore ottimale} & \textbf{Motivazione} \\
			\hline
			[1 -− 15] & [1 -− 10] &	Garantire una minore complessità durante la fase di \COD. \\
			\hline
			\end{tabular}
\captionof{table}{Complessità ciclomatica}
\end{center}


\subsubsection{Linee di commento per linee di codice}

\begin{itemize}
\item \textbf{Misurazione}: $P=\frac{N_{C}}{N_{SLOC}} \cdot 100$, dove $N_{C}$ è il numero di linee di commento presenti nel codice e $N_{SLOC}$ è il numero di Source Lines Of Code prodotte.
\end{itemize}

\begin{center}
		\begin{tabular}{|P{2.5cm}|P{2.5cm}|P{6cm}|}
		\hline
			\textbf{Valore di accettazione}	& \textbf{Valore ottimale} & \textbf{Motivazione} \\
			\hline
			[> 0.15] & [> 0.20] &	Garantire una buona comprensibilità del codice, in modo da facilitarne la manutenibilità. \\
			\hline
			\end{tabular}
\captionof{table}{Linee di commento per linee di codice}
\end{center}

\subsubsection{Numero di livelli di annidamento per metodo}

\begin{itemize}
\item \textbf{Misurazione}: indice numerico che indica il numero di chiamate a funzioni o procedure presenti all'interno di un metodo.
\end{itemize}

\begin{center}
		\begin{tabular}{|P{2.5cm}|P{2.5cm}|P{6cm}|}
		\hline
			\textbf{Valore di accettazione}	& \textbf{Valore ottimale} & \textbf{Motivazione} \\
			\hline
			[1 -- 5] & [1 −- 3] &	Garantire una minore complessità del metodo e garantire una sua facile comprensione. \\
			\hline
			\end{tabular}
\captionof{table}{Numero di livelli di annidamento per metodo}
\end{center}


\subsubsection{Lint}

Per garantire una buona qualità anche per il processo di Codifica verrà utilizzato lo strumento \termine{Lint}. Esso garantisce il controllo di alcuni elementi nel codice già descritti nelle \NdP.



\subsection{Processo di integrazione Software}
Il processo si occupa di integrare fra loro gli elementi del sistema, rispettando quanto stabilito nell'attività di progettazione. L'obiettivo di questo processo è di produrre un prodotto completo che soddisfi quanto espresso dai requisiti identificati.

\subsubsection{Obiettivi di qualità}
Le attività previste da questo processo dovranno puntare a raggiungere un alto livello di automazione, in particolare:
\begin{itemize}
\item l'integrazione delle varie parti del sistema sarà completamente automatizzata utilizzando lo strumento di continuous integration Jenkins;
\item il livello di integrazione raggiunto del sistema sarà sempre consultabile grazie all'utilizzo dello strumento di continuous integration \termine{Jenkins}.
\end{itemize}

\subsubsection{Strategie}
Sarà necessario configurare accuratamente lo strumento di \termine {continuous integration} \termine{Jenkins} affinché esegua dei test di integrazione di quanto prodotto prima che le ultime modifiche diventino parte del sistema.

\subsubsection{Metriche}
\paragraph{Componenti integrate}

\begin{itemize}
\item \textbf{Misurazione}: $I=\frac{N_{CI}}{N_{CP}} \cdot 100$, dove $N_{CI}$ è il numero di componenti attualmente integrate nel sistema e $N_{CP}$ è il numero di componenti delineate nell'attività di progettazione;
\end{itemize}

\begin{center}
		\begin{tabular}{|P{2.5cm}|P{2.5cm}|P{6cm}|}
		\hline
			\textbf{Valore di accettazione}	& \textbf{Valore ottimale} & \textbf{Motivazione} \\
			\hline
			$100$ & $100$ &	Garantire un buon soddisfacimento dei requisiti. \\
			\hline
			\end{tabular}
\captionof{table}{Numero di livelli di annidamento per metodo}
\end{center}

\subsection{Processo di Validazione}
Lo scopo del processo è quello di assicurare che ogni requisito individuato sia stato implementato nel prodotto.

\subsubsection{Obiettivi di qualità}
Durante lo svolgimento delle attività, ci si impegnerà affinché:
\begin{itemize}
\item le attività di test previste dal processo verranno svolte su un sistema le cui componenti sono verificate e correttamente integrate fra loro.
\item il sistema dovrà implementare tutti i requisiti obbligatori individuati nell'attività di analisi.
\end{itemize}

\subsubsection{Strategie}
Bisognerà cercare di implementare il maggior livello possibile di automazione nell'esecuzione dei test di sistema, in modo tale che la loro esecuzione non richieda costi eccessivi (soprattutto in termini temporali).

\subsubsection{Metriche}

\paragraph{Test di Unità eseguiti}
Indica la percentuale di test di unità eseguiti.
\begin{itemize}
\item \textbf{Misurazione}: $UE=\frac{N_{TUE}}{N_{TUP}} \cdot 100$, dove $N_{TUE}$ è il numero di test di unità eseguiti e $N_{TUP}$ è il numero di test di unità pianificati.
\end{itemize}

\begin{center}
		\begin{tabular}{|P{2.5cm}|P{2.5cm}|P{6cm}|}
		\hline
			\textbf{Valore di accettazione}	& \textbf{Valore ottimale} & \textbf{Motivazione} \\
			\hline
			[$90 - 100$] & $100$ &	Garantire che tutti o la maggior parte dei test di Unità siano eseguiti con successo. \\
			\hline
			\end{tabular}
\captionof{table}{Test di Unità eseguiti}
\end{center}


\paragraph{Test di Integrazione eseguiti}
Indica la percentuale di test di integrazione eseguiti.
\begin{itemize}
\item \textbf{Misurazione}: $IE=\frac{N_{TIE}}{N_{TIP}} \cdot 100$, dove $N_{TIE}$ è il numero di test di integrazione eseguiti e $N_{TIP}$ è il numero di test di Integrazione pianificati.
\end{itemize}

\begin{center}
		\begin{tabular}{|P{2.5cm}|P{2.5cm}|P{6cm}|}
		\hline
			\textbf{Valore di accettazione}	& \textbf{Valore ottimale} & \textbf{Motivazione} \\
			\hline
			[$60 - 100$] & [$70 - 100$] &	Garantire che tutti o la maggior parte dei test di Integrazione siano eseguiti con successo. \\
			\hline
			\end{tabular}
\captionof{table}{Test di Integrazione eseguiti}
\end{center}


\paragraph{Test di Sistema eseguiti}
Indica la percentuale di test di sistema eseguiti in modo automatico.
\begin{itemize}
\item \textbf{Misurazione}: $SE=\frac{N_{TSE}}{N_{TSP}} \cdot 100$, dove $N_{TSE}$ è il numero di test di sistema eseguiti e $N_{TSP}$ è il numero di test di sistema pianificati.
\end{itemize}

\begin{center}
		\begin{tabular}{|P{2.5cm}|P{2.5cm}|P{6cm}|}
		\hline
			\textbf{Valore di accettazione}	& \textbf{Valore ottimale} & \textbf{Motivazione} \\
			\hline
			[$70 - 100$] & [$80 - 100$] &	Garantire che tutti o la maggior parte dei test di Sistema siano eseguiti con successo. \\
			\hline
			\end{tabular}
\captionof{table}{Test di Sistema eseguiti}
\end{center}

\paragraph{Test di Validazione eseguiti}
Indica la percentuale di test di validazione eseguiti manualmente.
\begin{itemize}
\item \textbf{Misurazione}: $VE=\frac{N_{TVE}}{N_{TVP}} \cdot 100$, dove $N_{TVE}$ è il numero di test di validazione eseguiti e $N_{TVP}$ è il numero di test di Validazione pianificati.
\end{itemize}

\begin{center}
		\begin{tabular}{|P{2.5cm}|P{2.5cm}|P{6cm}|}
		\hline
			\textbf{Valore di accettazione}	& \textbf{Valore ottimale} & \textbf{Motivazione} \\
			\hline
			$100$ & $100$ &	Garantire che tutti o la maggior parte dei test di Validazione siano eseguiti con successo. \\
			\hline
			\end{tabular}
\captionof{table}{Test di Validazione eseguiti}
\end{center}

\paragraph{Test superati}
Indica la percentuale di test superati.
\begin{itemize}
\item \textbf{Misurazione}: $S=\frac{N_{TS}}{N_{TE}} \cdot 100$, dove $N_{TS}$ è il numero di test superati e $N_{TE}$ è il numero di test eseguiti;
\end{itemize}

\begin{center}
		\begin{tabular}{|P{2.5cm}|P{2.5cm}|P{6cm}|}
		\hline
			\textbf{Valore di accettazione}	& \textbf{Valore ottimale} & \textbf{Motivazione} \\
			\hline
			[$90 - 100$] & $100$ &	Garantire che il numero di test totali eseguiti sia uguale o quasi a quelli preventivati per garantire tutte le funzionalità previste. \\
			\hline
			\end{tabular}
\captionof{table}{Test di Unità eseguiti}
\end{center}



\subsection{Processo di documentazione}

Il processo punta a produrre e manutenere le informazioni sul software prodotte dai processi attuati.
\subsubsection{Obiettivi di qualità}
Il processo di documentazione dovrà perseguire le seguenti direttive:
\begin{itemize}
\item la documentazione prodotta dovrà essere chiara e comprensibile a tutti i committenti e sarà resa disponibile a chiunque la necessiti.
\item ogni forma di ambiguità sul significato di un termine utilizzato verrà eliminata grazie al \glossario.
\item la documentazione prodotta sarà sempre aggiornata ed allineata allo stato attuale del processo di sviluppo del prodotto.
\item i documenti saranno verificati ed approvati solo se soddisfacenti tutte le metriche per la qualità. Inoltre tramite le tecniche di analisi statica dovranno essere individuati il maggior numero di errori sintattici.
\end{itemize}
\subsubsection{Strategie}
Durante la stesura della documentazione, ogni termine con significato ambiguo deve essere indicato (corredato di definizione) nel \glossario\ ; gli script automatici provvederanno alla sua segnalazione all'interno del documento.\\
Ogni documento sarà dotato di numero di versione e corredato da un diario delle modifiche che consente di prendere visione di tutte le azioni effettuate sul testo in oggetto.

\subsubsection{Metriche}

\paragraph{Indice di Gulpease}

\begin{itemize}
\item \textbf{Misurazione:} Per misurare l'indice di Gulpease verrà utilizzato uno script in python che calcolerà il valore del documento.
\end{itemize}

\begin{center}

		\begin{tabular}{|P{2.5cm}|P{2.5cm}|P{6cm}|}
		\hline
			\textbf{Valore di accettazione}	& \textbf{Valore ottimale} & \textbf{Motivazione} \\
			\hline
			[40 -- 100] & [50--100] &	Per garantire una buona leggibilità dei documenti. I documenti accettati nel valore di accettazione dovranno essere migliorati per la successiva consegna fino al raggiungimento del valore ottimale. \\
			\hline
			\end{tabular}
\captionof{table}{Indice di Gulpease}
\end{center}