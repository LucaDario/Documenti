\subparagraph{Caso d'uso UC 1.3.2.3: Personalizzazione del bottone}
\label{UC 1.3.2.3: Personalizzazione del bottone}
\begin{figure}[ht]
	\centering
	\includegraphics[scale=0.60]{Usecases/img/{UC1.3.2.3}.png}
	\caption{Caso d'uso UC 1.3.2.3: Personalizzazione del bottone}
\end{figure}
\FloatBarrier
\begin{itemize}
\item\textbf{Attori}: Sviluppatore.
\item\textbf{Descrizione}: Lo sviluppatore vuole personalizzare la modalità in cui il bottone viene visualizzato.
\item\textbf{Precondizione}: Lo sviluppatore utilizza un widget checkbutton.
\item\textbf{Postcondizione}: Lo sviluppatore ha personalizzato la modalità di visualizzazione del checkbutton.
\item\textbf{Scenario principale}:
	\begin{itemize}
		\item Visualizzazione del "check" mediante una "x" (UC 1.3.2.3.1).
		\item Visualizzazione del "check" mediante una "v" (UC1.3.2.3.2).
		\item Visualizzazione del "check" mediante una colorazione del bottone (UC1.3.2.3.3).
	\end{itemize}
	\item\textbf{Scenario alternativo}:
	\begin{itemize}
		\item Lo sviluppatore non imposta il colore di check oppure lo imposta in modo non valido (UC 1.3.2.3.5).
	\end{itemize}
\end{itemize}
