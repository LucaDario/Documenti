\subsubsection{Composition of layouts or widgets}
Using the \texttt{BaseLayout}'s or \texttt{BaseBubble}'s \texttt{addComponent(component : BaseComponent)}  method, you can add to a layout or bubble either a widget or a sub-layout. \\

\textbf{Notes}. 
\begin{enumerate}
	\item The default bubble's layout will always be a \texttt{VerticalLayout} and cannot be changed.
	\item The below examples will use a bubble as the containers, but the same code applies also to simple layouts.
\end{enumerate}

\paragraph{Adding a widget}
In order to add a widget, the only thing there's need to do is create the widget instance and then call the \texttt{addComponent} method passing it as parameter.

% Immagine bolla prima del'aggiunta di un widget
\begin{lstlisting}[language=JavaScript, frame=single]
const bubble = new ToDoListBubble();
const widget = new TextWidget();
widget.setText("This is a new widget which has been added to the bubble");
bubble.addComponent(widget);
\end{lstlisting}
% Immagine bolla dopo l'aggiunta del widget

\paragraph{Adding a Layout with some widgets}
To add a layout to a bubble, the method which needs to be follow is the same as the one presented for the widget. The layout needs to be created and then \texttt{addComponent} must be called giving the layout as parameter.

% Immagine bolla prima dell'aggiunta del layout
\begin{lstlisting}[language=JavaScript, frame=single]
// Needed to be able to extend from the BaseBubble class
const bubble = new ToDoListBubble();

const widget = new TextWidget();
widget.setText("First text widget");
const widget 2 = new TextWidget();
widget2.setText("Second text widget");

const layout = new HorizzontalLayout();
layout.addComponent(widget1);
layout.addComponent(widget2);

bubble.addComponent(layout);
\end{lstlisting}
% Immagine della bolla dopo l'aggiunta del layout