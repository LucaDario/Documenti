\subsubsection{Layouts}
Inside \progettoShort\ there are two classes which gives the possibility of arranging the different widgets in two main orientations:
\begin{itemize}
	\item \texttt{HorizontaLayout}: it allows the widgets that are added to it to be displayed each one on the right of the previous ones;
	\item \texttt{VerticalLayout}: it allows the widgets to be displayed one below the previous ones.
\end{itemize}

\paragraph{Using a layout}
To use one of the given layout classes, all that needs to be done is the following:

% Immagine layout prima dell'aggiunta di un widget
\begin{lstlisting}[language=JavaScript, frame=single]
// Needed to be able to create the new layout
import {HorizontalLayout} from {BOH}

let widget = new TextWidget();

let layout = new HorizontalLayout();
layout.addComponent(widget);
\end{lstlisting}
% Immagine layout dopo l'aggiunta del widget

\paragraph{Creating a new layout}
In order to create a new layout class, the only thing that needs to be done is creating a new class which extends from \texttt{BaseLayout} and implements its abstract methods:
\begin{lstlisting}[language=JavaScript, frame=single]
// Needed to be able to create the new layout
import {BaseLayout} from {BOH}

export class MyLayout extends BaseLayout {
    constructor(){
        // Needed in order to create the base class instance
        super();
    }
    
    renderView(){
        // Renders the layout view usually calling its components' 
        // renderView() method and returns the HTML code
    }

}
\end{lstlisting}