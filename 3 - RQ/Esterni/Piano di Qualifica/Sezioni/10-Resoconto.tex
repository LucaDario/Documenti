\section{Resoconto delle attività di verifica}

In questa sezione vengono inserite tutte le misurazioni delle metriche trovate dal gruppo \gruppo.
Il team si impegna a garantire almeno il soddisfacimento del range di accettazione per ogni metrica.


\subsection{Revisione dei Requisiti}
In questa sezione vengono inseriti i risultati relativi al periodo di Revisione dei Requisiti e le metriche relative ad esso.

\subsubsection{Analisi statica dei documenti}
L'analisi statica dei documenti è stata fatta mediante Walkthrough ed ha portato all'individuazione di alcuni errori. Tra gli errori individuati quelli più frequenti sono stati:
		\begin{itemize}
			\item Errori nei concetti esposti.
			\item Aggettivi o verbi utilizzati in modo scorretto.
			\item Periodi troppo lunghi o complessi da capire ed interpretare.
		\end{itemize}
		
\subsubsection{Indice di Gulpease}

\begin{table}[h]
	\begin{center}
		\begin{tabular}{|c|c|c|c|}
			\hline
			\textbf{Documento}	& \textbf{Risultato} & \textbf{Esito} & \textbf{Valore} \\
			\hline
		 \termine{Analisi} dei Requisiti v1.0.0 & 90 & Superato & Ottimale	\\
			\hline
			Glossario v1.0.0 & 56 & Superato & Ottimale	\\
			\hline
			Norme di Progetto v1.0.0 & 47 & Superato & Accettabile \\
			\hline
			Piano di Progetto v1.0.0 & 48 & Superato & Accettabile\\
			\hline
			Piano di Qualifica v1.0.0	& 48 & Superato & Accettabile\\
			\hline
			Studio di Fattibilità v1.0.0	& 47 & Superato & Accettabile\\
			\hline
			Verbale 23-12-2016 v1.0.0	& 51 & Superato & Ottimale	\\
			\hline
		\end{tabular}
	\end{center}
	\caption{RR - Risultato indice di Gulpease}
\end{table}



\newpage

\subsection{Revisione di Progettazione}

\subsubsection{Analisi statica dei documenti}
L'analisi statica dei documenti è stata fatta mediante Walkthrough ed ha portato all'individuazione di alcuni errori. Tra gli errori individuati quelli più frequenti sono stati:
		\begin{itemize}
			\item Errori ortografici.
			\item Parole con lettere mancanti o invertite.
			\item Periodi troppo lunghi o complessi da capire ed interpretare.
		\end{itemize}

\paragraph{Densità di failure}

Nessuna operazione di testing si è conclusa in modo fallimentare.

\subsection{Metriche per i documenti}

\subsubsection{Indice di Gulpease}

\begin{table}[h]
	\begin{center}
		\begin{tabular}{|c|c|c|c|}
			\hline
			\textbf{Documento}	& \textbf{Risultato} & \textbf{Esito} & \textbf{Valore}\\
			\hline
		 \termine{Analisi} dei Requisiti v2.0.0 &	90 & Superato & Ottimale\\
			\hline
			Glossario v2.0.0 &	54 & Superato & Ottimale\\
			\hline
			Norme di Progetto v2.0.0 &	48 & Superato & Accettabile\\
			\hline
			Piano di Progetto v2.0.0	&	52 & Superato & Ottimale\\
			\hline
			Piano di Qualifica v2.0.0	&	46 & Superato & Accettabile\\
			\hline
			Definizione di \termine{Prodotto} v1.0.0	&	64 & Superato & Ottimale\\
			\hline
			Verbale 22-02-2017 v1.0.0	&	54 & Superato & Ottimale\\
			\hline
			Verbale 24-02-2017 v1.0.0	&	53 & Superato & Ottimale\\
			\hline
			Verbale 26-02-2017 v1.0.0	&	51 & Superato & Ottimale\\
			\hline
						Verbale 28-02-2017 v1.0.0	&	52 & Superato & Ottimale\\
			\hline
		\end{tabular}
	\end{center}
	\caption{RP - Risultato indice di Gulpease}
\end{table}

\subsection{Numero di parametri per metodo}

Tutti i metodi soddisfano il livello ottimale per questa metrica.

\subsection{Numero di attributi per classe}

Tutte le classi hanno un numero inferiore o uguale di attributi a sette.

\subsubsection{Livello di stabilità}

\begin{table}[h]
\begin{center}
	\begin{tabular}{|c|c|c|c|}
	\hline
	\textbf{Parte coinvolta} & \textbf{Risultato massimo} & \textbf{Esito} & \textbf{Valore} \\
	\hline
	SDK & 1 & Superato & Accettabile \\
	\hline
	Applicazione & 1 & Superato & Accettabile \\
	\hline
	\end{tabular}
\end{center}
\caption{RP - Livello di Stabilità}
\end{table}

\newpage

\subsection{Astrattezza}

\begin{table}[h]
\begin{center}
	\begin{tabular}{|c|c|c|c|}
	\hline
	\textbf{Parte del progetto} & \textbf{Risultato massimo} & \textbf{Esito} & \textbf{Valore} \\
	\hline
	SDK & 0.8 & Superato & Accettabile \\
	\hline
	Applicazione & 0.5 & Superato & Ottimale \\
	\hline
	\end{tabular}
\end{center}
\caption{RP - Astrattezza}
\end{table}

\subsection{Distanza dalla sequenza principale}

\begin{table}[h]
\begin{center}
	\begin{tabular}{|c|c|c|c|}
	\hline
	\textbf{Parte del progetto} & \textbf{Risultato} & \textbf{Esito} & \textbf{Valore} \\
	\hline
	SDK & [0 -- 1] & Superato & Accettabile\\
	\hline
	Applicazione & [0 -- 0.7] & Superato & Ottimale \\
	\hline
	\end{tabular}
\end{center}
\caption{RP - Distanza dalla sequenza principale}
\end{table}

Il risultato delle altre metriche non è stato inserito poiché, per farlo, è necessario essere entrati nella fase di \COD.
