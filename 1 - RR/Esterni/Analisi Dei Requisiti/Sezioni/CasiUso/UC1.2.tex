\subsubsection{Caso d'uso UC 1.2: Utilizzo di una bolla di tipo bottone}
\label{Caso d'uso UC 1.2: Utilizzo di una bolla di tipo bottone}
\begin{figure}[ht]
	\centering
	\includegraphics[scale=0.80]{Usecases/img/{UC1.2}.png}
	\caption{Caso d'uso UC 1.2: Utilizzo di una bolla di tipo bottone}
\end{figure}
\FloatBarrier
\begin{itemize}
\item\textbf{Attori}: Sviluppatore.
\item\textbf{Descrizione}: Lo sviluppatore vuole utilizzare una bolla di tipo bottone.
\item\textbf{Precondizione}: Lo sviluppatore vuole utilizzare una bolla di tipo bottone.
\item\textbf{Postcondizione}: Lo sviluppatore crea codice eseguibile per creare una bolla di tipo bottone.
\item\textbf{Scenario principale}:
	\begin{itemize}
		\item Lo sviluppatore vuole utilizzare una bolla di tipo \termine{bottone semplice}(UC 1.2.1).
		\item Lo sviluppatore vuole utilizzare una bolla di tipo \termine{checkbutton}(UC 1.2.2).
		\item Lo sviluppatore vuole impostare una reazione ad un evento del bottone(UC 1.2.3).
	\end{itemize}

\end{itemize}

