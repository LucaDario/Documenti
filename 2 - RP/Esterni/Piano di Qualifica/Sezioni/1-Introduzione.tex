\section{Introduzione}
\subsection{Scopo del documento}
Questo documento vuole definire le strategie che il \termine{team} ha deciso di adottare per perseguire gli obiettivi di qualità di processo e di prodotto ricercati. A tal fine è necessaria una costante attività di verifica e validazione del lavoro svolto in modo da poter rilevare e correggere le anomalie che potrebbero nascere.

\subsection{Scopo del prodotto}
\scopoProdotto

\subsection{Struttura del documento}
Il \termine{team} ha strutturato questo documento nel seguente modo:
\begin{itemize}
\item Elenco delle fonti, con relativa spiegazione, che sono state ritenute opportune come metro di giudizio per il prodotto software. Ogni fonte contiene, al suo interno, una sottosezione nella quale vengono spiegate le scelte del \termine{team} in merito ad essa.
\item Elenco delle metriche con rispettivo indice di accettazione e di ottimalità individuate dal \termine{team} per garantire qualità e conformità.
\item Elenco dei test utilizzati per garantire il corretto funzionamento del prodotto software.
\item Pianificazione delle verifiche. Ovvero il piano relativo a come e quando, la verifica delle metriche e la realizzazione dei test, debbano essere effettuate.
\end{itemize}

\subsection{Glossario}
\descrizioneGlossario

\subsection{Riferimenti}
\subsubsection{Normativi}
\riferimentiNormativi

\subsubsection{Informativi}
\begin{itemize}
	\item \textbf{\AdR}: \analisiDeiRequisiti;
	\item \textbf{\PdP}: \pianoDiProgetto;
	\item \textbf{\textit{Slide} dell'insegnamento di Ingegneria del Software}: \\
		  \link{http://www.math.unipd.it/~tullio/IS-1/2016/}
	\item \textbf{\textit{Standard} ISO/IEC 9126}: Product quality \\
	 	  \link{https://en.wikipedia.org/wiki/ISO/IEC\_9126}
	\item \textbf{\textit{Standard} tecnici ISO/IEC 15504}: Software process assessment \\
		  \link{https://en.wikipedia.org/wiki/ISO/IEC\_15504}
	\item \textbf{Ciclo di Deming (\termine{PDCA})}: Miglioramento dei processi \\
		  \link{https://en.wikipedia.org/wiki/PDCA}
\end{itemize}

\newpage