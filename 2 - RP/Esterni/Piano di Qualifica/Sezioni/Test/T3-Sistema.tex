\subsubsection{Test di Sistema}
Con questa tipologia di test si vuole verificare il corretto funzionamento delle componenti architetturali.\\
I test di sistema saranno descritti nel modo seguente:
\begin{center}
\textbf{TS}[\textit{TipoRequisito}][\textit{ImportanzaRequisito}][\textit{IdRequisito}]
\end{center}
dove:
\begin{itemize}
\item \textbf{TipoRequisito} può assumere valori tra:
\begin{itemize}
\item \textit{F} per i requisiti funzionali;
\item \textit{Q} per i requisiti di qualità;
\item \textit{V} per i requisiti di vincolo;
\item \textit{P} per i requisiti prestazionali.
\end{itemize}
\item \textbf{ImportanzaRequisito} può assumere valori tra:
\begin{itemize}
\item \textit{D} per i requisiti desiderabili;
\item \textit{O} per i requisiti di obbligatori;
\item \textit{F} per i requisiti di facoltativi.
\end{itemize}
\item \textbf{IdRequisito} può assumere un valore gerarchico che identifica il singolo requisito.
\end{itemize}

\begin{center}
	\begin{longtable}{|c|>{\centering}m{10cm}|c|}\hline
		Id & Descrizione & stato \\ \hline
		TSFO1 & Viene verificato che sia possibile creare e aggiungere ad una bolla un widget per tipo, tra quelli presenti nell'SDK & Implementato \\ \hline
		TSFO2 & Viene verificato che sia possibile creare e aggiungere ad una bolla un layout per tipo, tra quelli presenti nell'SDK & Implementato \\ \hline
		TSFO3 & Viene verificato che sia possibile impostare le variabili di un widget testo formattato & Implementato \\ \hline
		TSFO4 & Viene verificato che un widget testo formattato faccia visualizzare un messaggio d'errore in caso venga impostata una sua variabile in maniera errata & Implementato \\ \hline
		TSFO5 & Viene verificato che sia possibile impostare le variabili di un widget immagine & Implementato \\ \hline
		TSFO6 & Viene verificato che un widget immagine faccia visualizzare un messaggio d'errore in caso venga impostata una sua variabile in maniera errata & Implementato \\ \hline
		TSFO7 & Viene verificato che sia possibile impostare le variabili di un widget bottone & Implementato \\ \hline
		TSFO8 & Viene verificato che un widget bottone faccia visualizzare un messaggio d'errore in caso venga impostata una sua variabile in maniera errata & Implementato \\ \hline
		TSFO9 & Viene verificato che sia possibile impostare le variabili di un widget checklistitem & Implementato \\ \hline
		TSFO10 & Viene verificato che un widget checklistitem faccia visualizzare un messaggio d'errore in caso venga impostata una sua variabile in maniera errata & Implementato \\ \hline
		TSFO11 & Viene verificato che sia possibile impostare le variabili di un widget lista & Implementato \\ \hline
		TSFO12 & Viene verificato che un widget lista faccia visualizzare un messaggio d'errore in caso venga impostata una sua variabile in maniera errata & Implementato \\ \hline
		TSFO13 & Viene verificato che sia possibile aggiungere un widget all'interno di un VerticalLayoutView & Implementato \\ \hline
		TSFO14 & Viene verificato che sia possibile aggiungere un widget all'interno di un HorizzontalLayoutView & Implementato \\ \hline
		TSFO15 & Viene verificato che sia possibile istanziare una bolla avviso e sia possibile utilizzare tutti i suoi metodi & Implementato \\ \hline
		TSFO16 & Viene verificato che sia possibile istanziare una bolla markdown e sia possibile utilizzare tutti i suoi metodi & Implementato \\ \hline
		TSFO17 & Viene verificato che sia possibile istanziare una bolla lista e sia possibile utilizzare tutti i suoi metodi & Implementato \\ \hline
		TSFO18 & Viene verificato che sia possibile creare una bolla "lista della spesa" & Non implementato \\ \hline
		TSFO19 & Viene verificato che l'aggiunta di un nuovo prodotto alla lista della spesa venga eseguita correttamente & Non implementato \\ \hline
		TSFO20 & Viene verificato che sia possibile rimuovere un elemento ad una bolla "lista della spesa" & Non implementato \\ \hline
		TSFO21 & Viene verificato che sia possibile spuntare un elemento di una bolla "lista della spesa" & Non implementato \\ \hline
		TSFO22 & Viene verificato che l'inoltro a un utente di una bolla "lista della spesa" avvenga in modo corretto & Non implementato \\ \hline
		TSFO23 & Viene verificato che l'inoltro a un gruppo di una bolla "lista della spesa" avvenga in modo corretto & Non implementato \\ \hline
		TSFO24 & Viene verificato che la pubblicazione a un utente di una bolla "lista della spesa" avvenga in modo corretto & Non implementato \\ \hline
		TSFO25 & Viene verificato che la pubblicazione a un gruppo di una bolla "lista della spesa" avvenga in modo corretto & Non implementato \\ \hline
	\end{longtable}
\end{center}
