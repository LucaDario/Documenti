\section{Tecnologie utilizzate}

\subsection{Rocket.Chat}
\termine{Rocket.Chat} è una Web Chat Server, sviluppata in JavaScript, utilizzando il \termine{framework} Meteor. Lo scopo principale del capitolato è quello di sviluppare un'applicazione integrabile con il \termine{package manager} \termine{Atmosphere} nel progetto Meteor \termine{Rocket.Chat}. L'applicazione, denominata \textit{Bring-It}, userà l'\termine{SDK} integrata nello stesso metodo. Una volta compilato il tutto, il risultato sarà la versione di \termine{Rocket.Chat} del gruppo \gruppo\ che avrà incluso le funzionalità previste, sottoforma di pacchetti.

\subsection{Meteor.js}
Meteor è una piattaforma Javascript per lo sviluppo di applicazioni web e mobile. Essa viene utilizzata come ambiente di sviluppo per l'applicazione e per l'\termine{SDK} che verranno integrate come pacchetti stand-alone in \termine{Rocket.Chat}.
Meteor permette un'ottima gestione delle componenti, permettendo lo sviluppo della parte server, della parte client e della parte per la comunicazione tra esse in un unico progetto. Per fare ciò Meteor include un set di tecnologie da utilizzare, tra le quali: Node.js, MongoDB e tutte quelle disponibili all'interno del \termine{package manager} \termine{Atmosphere}.

\subsection{MongoDB}
MongoDB è un \termine{DBMS} non relazionale, orientato ai documenti. Classificato come un database di tipo NoSQL, si allontana dalla
struttura tradizionale basata su tabelle dei database relazionali in favore di documenti in stile
JSON con schema dinamico. MongoDB è utilizzato come \termine{storage} da \termine{Rocket.Chat}, così come per qualsiasi progetto meteor. Dunque, il suo utilizzo è derivato principalmente per questo motivo.

\subsection{Node.js}
Per lo sviluppo della parte Back-End dell’applicazione si è deciso di utilizzare la piattaforma \termine{event-driven} Node.js, basata sul motore JavaScript V8. Esso permette di realizzare applicazioni
Web utilizzando il linguaggio JavaScript, tipicamente client-side, per
la scrittura server-side. 
Node.js è essenziale per lo sviluppo di Monolith e per la compilazione di \termine{Rocket.Chat}. Infatti l'installazione di Meteor comporta anche l'installazione di Node.js essenziale per il corretto funzionamento dei progetti Meteor. Il principale vantaggio dell'utilizzo di tale tecnologia è che permette di accedere alle risorse del sistema operativo in modalità \termine{event-driven} non sfruttando il modello basato sui thread concorrenti, utilizzato dai classici web servers.

\subsection{Atmosphere e npm}
\textit{Atmosphere} e \textit{npm} sono dei \termine{packages managers} di Meteor e serve per aggiungere pacchetti \termine{stand-alone} all'interno di un generico progetto, come appunto \termine{Monolith} e relativa applicazione. Viene tenuta traccia dei pacchetti integrati nel progetto tramite la directory \textit{packages/} che si crea all'interno del progetto.

\subsection{Can.js}
\textit{Can.js} è una collezione di \termine{librerie} lato client per lo sviluppo di applicazioni web. In particolare il gruppo ha utilizzato la \termine{libreria} \textit{can-stache} utilizzata per costruire dei \textit{template} HTML ed aggiornarli in modo semplice e garantire un \termine{binding} dinamico.

\subsection{Bootstrap}
\textit{Bootstrap} è un \termine{framework} per la gestione di HTML, CSS e JavaScript. Viene utilizzato per semplicità di utilizzo e ridurre il tempo necessario per l'implementazione e l'integrazione dei file CSS, HTML e CSS dell'applicazione. Inoltre, tale \termine{framework} è già integrato dentro \termine{Rocket.Chat} e viene, dunque, utilizzato anche per retrocompatibilità.

\subsection{jQuery}
\textit{jQuery} è una libreria JavaScript. Essa permette di modificare facilmente il file HTML basandosi sul \termine{paradigma DOM}. Esso viene utilizzato nel caso di operazioni semplici (come una semplice animazione) oppure per implementare o gestire funzionalità che non si possono fare tramite i \termine{frameworks} scelti.

\subsection{Ecmascript 6}
\textit{Ecmascript 6} è un linguaggio di programmazione standardizzato e mantenuto da Ecma International nell'ECMA-262 ed ISO/IEC 16262. Le implementazioni più conosciute di questo linguaggio (spesso definite come dialetti) sono JavaScript, JScript e ActionScript che sono entrati largamente in uso, inizialmente, come linguaggi client-side nel web development. L versione 6 implementa significanti cambiamenti sintattici per scrivere applicazioni più complesse, incluse le classi e i moduli. Ciò ci permette di creare una struttura ben definita e di poter utilizzarla per implementare una struttura modulare.

\subsection{Dependency-injection-es6}
\textit{Dependency-injection-es6} è una \termine{libreria} per \textit{Node.js} e l'ambiente \textit{JavaScript} dove \textit{EcmaScript 6} è supportato. Grazie a quest'ultima caratteristica, tale \termine{libreria} è stata scelta dal team per implementare il \termine{design-pattern} della \termine{dependency-injection}.

\subsection{Es6-event-emitter}
\textit{Es6-event-emitter} è una \termine{libreria} che serve per emettere e catturare eventi, generati dopo particolari azioni.

\subsection{Sass}
\textit{Sass} è preprocessore CSS. Esso serve a definire fogli di stile con una forma più semplice, completa e potente rispetto ai \termine{CSS} e a generare file \termine{CSS} ottimizzati, aggregando le strutture definite anche in modo complesso. Esso è utilizzato per generare alcuni fogli di stile particolarmente complicati e per semplificare la loro gestione.

\subsection{Marked}
\textit{Marked} è una \termine{libreria} che trasforma del testo che rispetta la sintassi \termine{Markdown} in \termine{HTML}. Tale \termine{libreria} viene utilizzata per la generazione dell'\termine{HTML} del widget Markdown per semplificarne la gestione. 

\subsection{BlazeJS}
\textit{BlazeJS} è una potente \termine{libreria} per la creazione di interfacce utente tramite template \termine{HTML}. Essa elimina l'esigenza di aggiornare la parte logica dell'applicazione che è in ascolto dei cambiamenti e di eventuali manipolazioni del \termine{DOM}.